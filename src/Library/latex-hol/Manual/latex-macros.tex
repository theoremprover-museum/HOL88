\chapter{\LaTeX\ macros}

This chapter documents all the \LaTeX\ macros defined in the file {\tt
holmacs.tex}. They are required to format the output produced by the
functions of this library. The macros can be divided into two groups:
special symbols and environments. They can be changed to suit the
document style in use. Care should be taken not to abuse them. All
special symbols are listed in the table below and this is followed by
the description of the environment macros.
\section*{Special symbols}
\begin{center}
\begin{tabular}{|l|c||l|c||l|c|}
\hline
\sc NAME & \sc SYMBOL & \sc NAME & \sc SYMBOL & \sc NAME & \sc SYMBOL
\\ \hline \hline
\verb|\US| & \US & \verb|\SH| & \SH & \verb|\AM| & \AM \\ \hline
\verb|\PC| & \PC & \verb|\DO| & \DO & \verb|\BS| & \BS \\ \hline
\verb|\PR| & \PR & \verb|\TI| & \TI & \verb|\AS| & \AS \\ \hline
\verb|\LE| & \LE & \verb|\BA| & \BA & \verb|\GR| & \GR \\ \hline
\verb|\LB| & \LB & \verb|\RB| & \RB & \verb|\CI| & \CI \\ \hline
\verb|\LC| & \LC & \verb|\RC| & \RC &    &   \\ \hline \hline
\verb|\THM| & \THM & \verb|\AND| & \AND & \verb|\OR| & \OR \\ \hline
\verb|\IMP| & \IMP & \verb|\LONG| & \LONG & \verb|\IFF| & \IFF \\ \hline
\verb|\LEE| & \LEE & \verb|\GEE| & \GEE & \verb|\EXISTSUNIQUE| & \EXISTSUNIQUE \\ \hline
\verb|\LES| & \LES & \verb|\GRE| & \GRE & \verb|\MUL| & \MUL \\ \hline
\verb|\NOT| & \NOT & \verb|\FORALL| & \FORALL & \verb|\EXISTS| & \EXISTS \\ \hline
\verb|\SELECT| & \SELECT & \verb|\FUNCOM| & \FUNCOM & \verb|\LAMBDA| & \LAMBDA \\ \hline
\verb|\DOT| & \DOT & \verb|\NIL| & \NIL & \verb|\EMPTYSET| & \EMPTYSET \\ \hline
\verb|\BEGINSET| & $\BEGINSET\right.$ & \verb|\ENDSET| &$\left. \ENDSET$ & \verb|\SUCHTHAT| & \SUCHTHAT \\ \hline
\end{tabular}
\end{center}

\section*{Commands}
\begin{description}
\item[{\tt\BS CONST}]  The argument to this is a logical constant. It
	is set in the font specified by \verb|\constfont|.
\item[{\tt\BS KEYWD}]  The argument to this is a keyword. It
	is set in the font specified by \verb|\keyfont|.
\item[{\tt\BS sec}]  This begins a section within a theory. It produces
	an unnumbered section heading.
\item[{\tt\BS theory}]  This begins a theory. It produces a numbered
	section heading.
\item[{\tt\BS endthy}]  This ends a theory. It produces a rule across
	the page with the nam of the theory at the center of it.
\item[{\tt\BS typefont}]  This font is used to typeset logical
	types and ML identifiers. By default, it it \verb|\tt|.
\item[{\tt\BS constfont}]  This font is used to typeset logical
	constants. By default, it it \verb|\sf|.
\item[{\tt\BS keyfont}]  This font is used to typeset keywords.
	By default, it it \verb|\bf|.
\end{description}

\section*{Environments}

\begin{description}
\item[{\tt ttlist}] The {\it parents\/} and {\it types\/} sections are
	set in this environment.
\item[{\tt typelist}] The {\it constants\/} and {\it infixes\/}
	sections are set in this environment. It is a labeled list with the
	names of the constants as the labels.
\item[{\tt thmlist}] The {\it definitions\/} and {\it theorems\/}
	sections are set in this environment. It is a labeled list
	with the names of the definitions or theorems as the labels.
\end{description}