
\chapter{The pretty-printing algorithm\label{algorithm}}


\countdef\ppboxhlessone=255
\countdef\ppboxwlessio=254
\countdef\ppboxwlessfo=253
\countdef\ppboxx=252
\countdef\ppboxy=251
\countdef\ppboxscratch=250

\countdef\pplinen=249
\countdef\pplinesubl=248
\countdef\pplinehalfsubl=247
\countdef\pplinex=246
\countdef\ppliney=245
\countdef\pplinedeltax=244
\countdef\pplinedeltay=243
\countdef\pplinescratch=242

\def\ppline(#1,#2)(#3,#4)#5{\put(#1,#2){\line(#3,#4){#5}}}

\def\ppdashline(#1,#2)(#3,#4)#5{
   \ifnum#5>0
   \pplinen=#5
   \divide\pplinen by 5
   \ifodd\pplinen \advance\pplinen by 1 \fi
   \pplinesubl=#5
   \divide\pplinesubl by \pplinen
   \pplinehalfsubl=\pplinesubl
   \divide\pplinehalfsubl by 2
   \put(#1,#2){\line(#3,#4){\pplinehalfsubl}}
   \pplinescratch=\pplinesubl
   \advance\pplinescratch by \pplinehalfsubl
   \pplinex=\pplinescratch
   \ppliney=\pplinescratch
   \multiply\pplinex by #3
   \multiply\ppliney by #4
   \advance\pplinex by #1
   \advance\ppliney by #2
   \pplinescratch=\pplinesubl
   \multiply\pplinescratch by 2
   \pplinedeltax=\pplinescratch
   \pplinedeltay=\pplinescratch
   \multiply\pplinedeltax by #3
   \multiply\pplinedeltay by #4
   \divide\pplinen by 2
   \advance\pplinen by -1
   \multiput(\pplinex,\ppliney)(\pplinedeltax,\pplinedeltay){\pplinen}
      {\line(#3,#4){\pplinesubl}}
   \pplinex=#3
   \ppliney=#4
   \multiply\pplinex by #5
   \multiply\ppliney by #5
   \advance\pplinex by #1
   \advance\ppliney by #2
   \put(\pplinex,\ppliney){\line(-#3,-#4){\pplinehalfsubl}}
   \fi
   }

\def\ppbox(#1,#2)((#3,#4,#5),#6)#7#8{
   \ppboxhlessone=#6
   \advance\ppboxhlessone by -1
   \ppboxwlessio=#3
   \multiply\ppboxwlessio by -1
   \advance\ppboxwlessio by #4
   \ppboxwlessfo=#5
   \multiply\ppboxwlessfo by -1
   \advance\ppboxwlessfo by #4

   \ppboxx=#1
   \ppboxy=#2
   \ppboxscratch=#7
   \divide\ppboxscratch by 5
   \put(\ppboxx,\ppboxy){\circle*{\ppboxscratch}}
   \ppboxscratch=\ppboxwlessio
   \multiply\ppboxscratch by #7
   \csname pp#8\endcsname(\ppboxx,\ppboxy)(1,0){\ppboxscratch}
   \csname pp#8\endcsname(\ppboxx,\ppboxy)(0,-1){#7}

   \advance\ppboxy by -#7
   \ppboxscratch=#3
   \multiply\ppboxscratch by #7
   \advance\ppboxx by -\ppboxscratch
   \csname pp#8\endcsname(\ppboxx,\ppboxy)(1,0){\ppboxscratch}
   \ppboxscratch=\ppboxhlessone
   \multiply\ppboxscratch by #7
   \csname pp#8\endcsname(\ppboxx,\ppboxy)(0,-1){\ppboxscratch}

   \advance\ppboxy by -\ppboxscratch
   \ppboxscratch=#5
   \multiply\ppboxscratch by #7
   \advance\ppboxx by \ppboxscratch
   \csname pp#8\endcsname(\ppboxx,\ppboxy)(-1,0){\ppboxscratch}
   \csname pp#8\endcsname(\ppboxx,\ppboxy)(0,1){#7}

   \advance\ppboxy by #7
   \ppboxscratch=\ppboxwlessfo
   \multiply\ppboxscratch by #7
   \advance\ppboxx by \ppboxscratch
   \csname pp#8\endcsname(\ppboxx,\ppboxy)(-1,0){\ppboxscratch}
   \ppboxscratch=\ppboxhlessone
   \multiply\ppboxscratch by #7
   \csname pp#8\endcsname(\ppboxx,\ppboxy)(0,1){\ppboxscratch}
   }

\def\ppboxplain(#1,#2)((#3,#4,#5),#6)#7{
   \ppbox(#1,#2)((#3,#4,#5),#6){#7}{line}}

\def\ppboxdashed(#1,#2)((#3,#4,#5),#6)#7{
   \ppbox(#1,#2)((#3,#4,#5),#6){#7}{dashline}}

{\setlength{\unitlength}{1pt}

\section{Requirements}

The development of the pretty-printer was motivated by the embedding of
languages within the logic of the \HOL\ system. Typically, a construct of the
language is represented by a new \HOL\ constant, which may be of a function
type. The new constant is defined to have a logical property which expresses
the meaning of the language construct. The appearance of the constant within
a \HOL\ term is often too unlike the syntax of the language to be readable.
The pretty-printer aims to solve this problem, by displaying the new \HOL\
constant as the language construct the constant is representing.

Consider an example. In set theory, the set:

\begin{small}\begin{verbatim}
   {1,2,3}
\end{verbatim}\end{small}

\noindent
might be represented in \HOL\ as:

\begin{small}\begin{verbatim}
   "INSERT 1 (INSERT 2 (INSERT 3 EMPTY))"
\end{verbatim}\end{small}

\noindent
where {\small\verb%EMPTY%} and {\small\verb%INSERT%} are new \HOL\ constants.
It would be nice to have the term appear in the set notation.

The \HOL\ term to be pretty-printed in a special way may be contained within a
larger term, the rest of which is to be pretty-printed in the standard way for
\HOL\ terms. Similarly, the term to be pretty-printed specially may itself
contain sub-terms which are to be pretty-printed in the normal way. For
example, a list of sets of lists:

\begin{small}\begin{verbatim}
   "[{[1;2],[3],[4;5;6]}; {}; {[8;9],[]}]"
\end{verbatim}\end{small}

\noindent
To make use of the standard \HOL\ pretty-printer for most of the term, and
only use a special pretty-printer for those parts which require it, would need
a two-way flow of information between the two pretty-printers. In the example
above, when the set theory constants are encountered, the standard \HOL\
pretty-printer would have to invoke the special pretty-printer. This in turn
would need to invoke the standard pretty-printer to deal with the elements of
the sets, which are lists. With the standard pretty-printer written in Lisp,
this did not seem to be a practical solution. It is unlikely that the code
produced would be significantly cleaner than using {\it ad hoc\/} Lisp hacks
to provide the required pretty-printing, and one of the main aims of the
new pretty-printer is to provide a cleaner way of doing things.

The alternative to a bidirectional link between the standard and
special-purpose pretty-printers is to completely re-implement the \HOL\
pretty-printer, so that no distinction is made between pretty-printing
built-in \HOL\ constants and pretty-printing user-defined constants. This
approach is the one that has been taken. It has meant extra work in providing
a pretty-printer for standard \HOL\ constants. However, it has allowed the
pretty-printer to be general-purpose; it can be used for pretty-printing
almost any \ML\ datatype as well as \HOL\ terms.

A number of pretty-printing algorithms were investigated. It was decided to
base the design on \PPML~\cite{PPML}. This system has the particular
advantage that it pretty-prints from a tree structure rather than from text
-- a useful feature when dealing with \HOL\ terms. The trees used by \PPML\ are
parse-trees. Since \HOL\ terms are similar, but not identical to the
parse-trees for the text they are to appear as, the pretty-printing language
has been modified to cope with the differences.


\section{Boxes}

The fundamental object for pretty-printing is a {\it box\/} of text. A box
may extend over several lines of output, and can have initial and final
indentations.

\begin{center}
\begin{picture}(190,100)
\ppboxplain(40,90)((2,8,5),3){20}
\put(15,92){{\it io}}
\put(14,95){\vector(-1,0){14}}
\put(26,95){\vector(1,0){14}}
\put(45,20){{\it fo}}
\put(44,23){\vector(-1,0){44}}
\put(56,23){\vector(1,0){44}}
\put(67,10){{\it width}}
\put(64,13){\vector(-1,0){64}}
\put(96,13){\vector(1,0){64}}
\put(165,56){{\it height}}
\put(180,51){\vector(0,-1){21}}
\put(180,69){\vector(0,1){21}}
\end{picture}
\end{center}

\noindent
{\it io\/} and {\it fo\/} are the initial and final offsets respectively. The
dot indicates the position of the `origin' of the box. Normally {\it io\/}
will be zero. A non-zero value allows text on the second and subsequent lines
to be negatively indented. The last line of the text does not have to extend
as far as preceding lines. This is indicated by a value of {\it fo\/} which is
less than the value of {\it width}.

Two boxes of text may be composed to give a compound box of the same shape.
The composition can be either horizontal or vertical. For horizontal
composition the first line of the second box begins on the same line as the
last line of the first box. The separation is given by {\it dx}:

\begin{center}
\begin{picture}(200,100)
\ppboxplain(40,90)((2,8,5),3){20}
\ppboxplain(140,50)((4,7,5),2){20}
\ppboxdashed(40,90)((2,10,8),4){20}
\put(114,37){{\it dx}}
\put(114,40){\vector(-1,0){14}}
\put(126,40){\vector(1,0){14}}
\end{picture}
\end{center}

\noindent
The vertical composition is similar, but there are two parameters. {\it di\/}
is the horizontal offset of the second box relative to the first. This offset
is measured between the origins of the two boxes. {\it dh\/} is the vertical
separation between the two boxes.

\begin{center}
\begin{picture}(200,140)
\ppboxplain(40,130)((2,8,7),3){20}
\ppboxplain(80,50)((1,7,5),2){20}
\ppboxdashed(40,130)((2,10,8),6){20}
\put(55,57){{\it di}}
\put(54,60){\vector(-1,0){14}}
\put(66,60){\vector(1,0){14}}
\put(103,56){{\it dh}}
\put(110,66){\vector(0,1){4}}
\put(110,54){\vector(0,-1){4}}
\end{picture}
\end{center}

\noindent
The parameters of the composite box cannot be computed using simple formulae.
The values are dependent on whether the first box protrudes further to the
right than the second box, and on which box protrudes further to the left. It
is helpful to compute two values, {\it lo\/} and {\it ro}, as shown in the
diagram below.

\begin{center}
\begin{picture}(200,80)
\ppboxplain(40,70)((2,8,5),2){20}
\ppboxplain(140,50)((4,7,6),2){20}
\ppboxdashed(40,70)((2,10,9),3){20}
\put(114,37){{\it dx}}
\put(114,40){\vector(-1,0){14}}
\put(126,40){\vector(1,0){14}}
\put(25,17){{\it lo}}
\put(24,20){\vector(-1,0){24}}
\put(36,20){\vector(1,0){24}}
\put(175,57){{\it ro}}
\put(174,60){\vector(-1,0){14}}
\put(186,60){\vector(1,0){14}}
\end{picture}
\end{center}

\noindent
As shown in the diagram, {\it lo\/} and {\it ro\/} are taken to have positive
values. Either or both can also be negative. In the next diagram, both values
are negative.

\begin{center}
\begin{picture}(200,80)
\ppboxplain(60,70)((1,8,2),2){20}
\ppboxplain(120,50)((6,8,7),2){20}
\ppboxdashed(60,70)((3,10,7),3){20}
\put(94,37){{\it dx}}
\put(94,40){\vector(-1,0){14}}
\put(106,40){\vector(1,0){14}}
\put(15,37){{\it lo}}
\put(14,40){\vector(-1,0){14}}
\put(26,40){\vector(1,0){14}}
\put(175,37){{\it ro}}
\put(174,40){\vector(-1,0){14}}
\put(186,40){\vector(1,0){14}}
\end{picture}
\end{center}

\noindent
The equations for the parameters of the composite box are given by:

\begin{small}\begin{verbatim}
   lo = (fo1 + dx) - io2
   ro = (w2 - io2) - (w1 - (fo1 + dx))
   io = if (lo < 0) then (io1 - lo) else io1
   w  = if (lo < 0)
        then if (ro < 0) then (w1 - lo) else w2
        else if (ro < 0) then w1 else (w2 + lo)
   fo = if (lo < 0) then fo2 else (fo2 + lo)
   h  = h1 + h2 - 1
\end{verbatim}\end{small}

\noindent
where identifiers ending with a {\small\verb%1%} are parameters of the first
box to be composed and identifiers ending with a {\small\verb%2%} are
parameters of the second box. {\it dx\/} should be checked to make sure it is
not negative.

The vertical composition of boxes is similar. The equations have to be changed
as follows:

\begin{small}\begin{verbatim}
   lo = (io1 + di) - io2
   ro = (w2 - io2) - (w1 - (io1 + di))
   h  = h1 + h2 + dh
\end{verbatim}\end{small}

\noindent
{\it dh\/} should be checked to make sure it is not negative.


\subsection{Horizontal boxes}

The two operations described above are the primitive compositions. Lists of
boxes can be composed using these operations. There are four ways of doing
this: horizontal, vertical, horizontal/vertical, and horizontal-or-vertical.
A list of boxes composed horizontally cannot be broken between lines. Nothing
can be done about a composite box that does not fit in the output width
available. Here is an example:

\begin{center}
\begin{picture}(380,140)
\ppboxplain(40,130)((2,8,5),3){20}
\ppboxplain(120,90)((3,7,5),2){20}
\ppboxplain(160,70)((1,12,5),3){20}
\ppboxdashed(40,130)((2,19,12),6){20}
\end{picture}
\end{center}

\noindent
The restriction of not being able to break the composite box between lines
makes horizontal composition the most difficult. In order to fit the boxes
into the available width, any flexibility in their construction should be
exploited. So, it is desirable to know how the boxes are to be composed before
those boxes are themselves built from sub-boxes. Unfortunately, the way in
which the boxes are composed may in turn depend on the size and shape of the
boxes. So, iteration is necessary to find an optimal solution. Iteration
would not be necessary if worst cases were always assumed, but this leads to
unnecessarily unpleasant formatting.

Iteration could be tricky because a tiny change in a sub-box can have a very
large effect on an enclosing box. More importantly perhaps, performing
iteration would make the pretty-printing algorithm too slow. There is
however a non-iterative solution for horizontal composition which seems at
first sight to make unreasonable assumptions about the boxes to be composed.
However, analysis of the use of horizontal composition reveals that the
assumptions are justified in a large proportion of cases. The assumption is
that horizontal boxes are made up of sub-boxes, all but one of which are of
width one character. A very common case is the horizontal composition of three
boxes, the first of which is a left parenthesis; the second is a large unit,
and the third is a right parenthesis. For example:

\begin{center}
\begin{picture}(180,80)
\ppboxplain(0,70)((0,1,1),1){20}
\ppboxplain(20,70)((0,8,5),3){20}
\ppboxplain(120,30)((0,1,1),1){20}
\ppboxdashed(0,70)((0,9,7),3){20}
\put(2,55){{\LARGE\verb|(|}}
\put(22,55){{\LARGE\verb|( a  *  b )  +|}}
\put(22,35){{\LARGE\verb|( c  *  d  *  e )|}}
\put(169,35){{\LARGE\verb|+|}}
\put(22,15){{\LARGE\verb|( f  *  g )|}}
\put(130,15){{\LARGE\verb|)|}}
\end{picture}
\end{center}

\noindent
It is worth noting that a box may take up zero or negative space in a
composite box, even though it has a positive width. For example, consider the
following box:

\begin{center}
\begin{picture}(160,80)
\ppboxplain(100,70)((5,8,2),3){20}
\end{picture}
\end{center}

\noindent
The end of the box has a smaller horizontal offset than the start. So any box
composed horizontally onto the end of the box can begin at a lower horizontal
offset. This fact indicates that there are two notions of box width. One is
the difference between the leftmost and rightmost edges. The other is the
effective width when used in a composite box.

An informal description of the horizontal composition algorithm is as follows.
The total separation between the sub-boxes is computed. To this is added the
number of sub-boxes (less the first). The available width is then reduced by
this total to give the new available width. This is an attempt to guess how
much space to leave on the line for the remainder of the sub-boxes. The
effective width of each sub-box is assumed to be one. As we have seen, it
could be any value, even negative.

As each sub-box is built, the gap between it and the previous sub-box, plus
one, is added back on to the available width, and the indentation from the
left margin is changed by the genuine amount. In fact, the indentation is
computed each time from the original indentation, the effective width of the
box built so far, and the effective width of the latest sub-box.


\subsection{Vertical boxes}

Vertical composition of boxes does not allow any flexibility. However, unlike
horizontal composition, it is a near to optimal solution. By specifying a
large relative indentation between the sub-boxes, it is possible to cause the
display width to be exceeded when not at all necessary, but for typical
(i.e.~small) indentation values, vertical composition is the method least
likely to cause the display width to be exceeded.

The relative indentation can either be between a sub-box and its predecessor,
or between a sub-box and the first box of those to be composed:

\begin{center}
\begin{picture}(400,100)
\ppboxplain(0,90)((0,5,5),1){20}
\ppboxplain(40,70)((0,5,5),1){20}
\ppboxplain(80,50)((0,5,5),1){20}
\ppboxplain(120,30)((0,5,5),1){20}
\put(15,57){{\it di}}
\put(14,60){\vector(-1,0){14}}
\put(26,60){\vector(1,0){14}}
\put(55,37){{\it di}}
\put(54,40){\vector(-1,0){14}}
\put(66,40){\vector(1,0){14}}
\put(95,17){{\it di}}
\put(94,20){\vector(-1,0){14}}
\put(106,20){\vector(1,0){14}}
\ppboxplain(260,90)((0,5,5),1){20}
\ppboxplain(300,70)((0,5,5),1){20}
\ppboxplain(300,50)((0,5,5),1){20}
\ppboxplain(300,30)((0,5,5),1){20}
\put(275,57){{\it di}}
\put(274,60){\vector(-1,0){14}}
\put(286,60){\vector(1,0){14}}
\put(275,37){{\it di}}
\put(274,40){\vector(-1,0){14}}
\put(286,40){\vector(1,0){14}}
\put(275,17){{\it di}}
\put(274,20){\vector(-1,0){14}}
\put(286,20){\vector(1,0){14}}
\end{picture}
\end{center}

\noindent
It is worth noting that the order in which the sub-boxes are composed affects
the meaning of the indentation value {\it di}. If the first two boxes are
composed first, and the resulting box composed with the third box, all using
{\it di\/} for the indentation, the indentation is relative to the first box.
For example:

\begin{center}
\begin{picture}(140,80)
\ppboxplain(0,70)((0,5,5),1){20}
\ppboxplain(40,50)((0,5,5),1){20}
\ppboxdashed(0,70)((0,7,7),2){20}
\ppboxplain(40,30)((0,5,5),1){20}
\put(15,37){{\it di}}
\put(14,40){\vector(-1,0){14}}
\put(26,40){\vector(1,0){14}}
\put(15,17){{\it di}}
\put(14,20){\vector(-1,0){14}}
\put(26,20){\vector(1,0){14}}
\end{picture}
\end{center}

\noindent
If the last two boxes are composed first, followed by a composition of the
third-to-last box with the resulting box, the indentation is relative to the
previous box:

\begin{center}
\begin{picture}(180,80)
\ppboxplain(0,70)((0,5,5),1){20}
\ppboxplain(40,50)((0,5,5),1){20}
\ppboxplain(80,30)((0,5,5),1){20}
\ppboxdashed(40,50)((0,7,7),2){20}
\put(15,37){{\it di}}
\put(14,40){\vector(-1,0){14}}
\put(26,40){\vector(1,0){14}}
\put(55,17){{\it di}}
\put(54,20){\vector(-1,0){14}}
\put(66,20){\vector(1,0){14}}
\end{picture}
\end{center}


\subsection{Horizontal/vertical boxes\label{hvboxes}}

With horizontal/vertical composition, boxes are composed horizontally until
the display width is exceeded. At this point a new horizontal box is begun. A
new horizontal box is begun whenever the current one exceeds the display
width. When all of the original boxes have been used up, there is a list of
composite horizontal boxes. These are then composed vertically. An example:

\begin{center}
\begin{picture}(300,120)
\ppboxplain(0,110)((0,5,5),1){20}
\ppboxplain(100,110)((0,5,3),2){20}
\ppboxplain(160,90)((0,6,6),1){20}
\ppboxdashed(0,110)((0,14,14),2){20}
\ppboxplain(40,70)((0,11,10),2){20}
\ppboxplain(80,30)((0,6,6),1){20}
\ppboxplain(200,30)((0,5,5),1){20}
\ppboxdashed(0,110)((0,15,15),5){20}
\end{picture}
\end{center}

\noindent
When trying to add a sub-box to the current horizontal box, the algorithm
evaluates by how much the offset from the left margin will be increased if a
line-break is not used. If this is less than or equal to the increase that
will occur with a line-break, the sub-box is added to the horizontal box
regardless.

Two criteria are used for determining when to break. If the width of the
composite box less its initial offset exceeds the difference between the
space available and the initial indentation, then the right-hand edge of the
box will exceed the right margin. A break should therefore occur. This
criteria does not take into account that the initial offset of the box may be
greater than the initial indentation. If that is the case, the left-hand edge
of the box may exceed the left margin. To check for this, a second criteria is
used, namely whether the width of the box exceeds the space available.

Since the indentation can be negative, a false result may be obtained for the
first criteria. For this reason negative indentations are taken to be zero.

The vertical composition parameters of the first sub-box of a horizontal box
are remembered when the horizontal box is begun. This is so that they can be
used as the parameters for the composite horizontal box.


\subsection{Horizontal-or-vertical boxes}

Boxes to be composed horizontal-or-vertically are first composed horizontally.
If the composite box is too big, the boxes are composed vertically. The
narrower of the two compositions is used. The criteria for a box being too big
are specified in Section~\ref{hvboxes}.


\subsection{Building sub-boxes}

For horizontal/vertical and horizontal-or-vertical boxes, the sub-boxes are
built under the assumption that they will have to begin on a new line. This
approach leads to line-breaking occurring at a high level, i.e.~a block of
text is not usually split if the text could also be made to fit by making a
split between a box containing the block of text and some other box. An
example of low-level breaking is:

\begin{center}
\begin{picture}(360,80)
\ppboxdashed(0,70)((0,4,4),1){20}
\ppboxdashed(80,70)((0,4,4),1){20}
\ppboxdashed(160,70)((0,4,4),1){20}
\ppboxplain(0,70)((0,12,12),1){20}
\ppboxplain(240,70)((0,4,4),1){20}
\ppboxplain(260,50)((0,4,4),1){20}
\ppboxplain(280,30)((0,4,4),1){20}
\ppboxdashed(240,70)((0,6,6),3){20}
\end{picture}
\end{center}

\noindent
The high-level breaking version normally looks much better:

\begin{center}
\begin{picture}(360,60)
\ppboxdashed(0,50)((0,4,4),1){20}
\ppboxdashed(80,50)((0,4,4),1){20}
\ppboxdashed(160,50)((0,4,4),1){20}
\ppboxplain(0,50)((0,12,12),1){20}
\ppboxdashed(40,30)((0,4,4),1){20}
\ppboxdashed(120,30)((0,4,4),1){20}
\ppboxdashed(200,30)((0,4,4),1){20}
\ppboxplain(40,30)((0,12,12),1){20}
\end{picture}
\end{center}


\section{Limitations}

Certain kinds of pretty-printing are not possible with the algorithm described
here. In particular, suppose one wanted to pretty-print a {\small\verb%while%}
loop as either:

\begin{small}\begin{verbatim}
   while <cond> do <statement>
\end{verbatim}\end{small}

\noindent
or as:

\begin{small}\begin{verbatim}
   while <cond>
   do    <statement>
\end{verbatim}\end{small}

\noindent
depending on available space. It is possible to do the first version in all
situations and it is possible to do the second version in all cases. It is not
however possible to do one or the other depending on available space.

The spacing between {\small\verb%do%} and {\small\verb%<statement>%} varies
according to whether the enclosing box is broken. To achieve this it would be
necessary to build the enclosing box before deciding how to build the
{\small\verb%do <statement>%} box. However, it is not known how to build the
enclosing box until the inner boxes have been built.

Is it not possible to assume the worst case? Suppose the
{\small\verb%do <statement>%} box is built with the larger spacing. Can the
decision as to how to break the enclosing box then be made safely? Assuming it
can, it is then known how to build the {\small\verb%do <statement>%} box
properly, and if the worst case build is not the correct way, it can be
rebuilt. The enclosing box can then be built.

The problem with this technique is that it applies at all levels, so it is
unclear that rebuilding the {\small\verb%do <statement>%} box with the
non-worst-case parameters will be guaranteed to produce a `narrower' box. This
becomes of particular concern when one considers that a small change in a
sub-box can have a large effect on the enclosing box. Further work is required
to decide whether or not the technique is valid.

}
