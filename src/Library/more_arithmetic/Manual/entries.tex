\chapter{ML Functions in the more\_arithmetic Library}
This chapter provides documentation on some of the \ML\ functions and values
that are made available in \HOL\ when the \ml{trs} library is loaded. Those
functions and values not documented are internal and not intended for use.
This documentation is also available online via the \ml{help} facility.



\DOC{GEN\_INDUCT\_RULE}

\TYPE {\small\verb%GEN_INDUCT_RULE : (thm -> thm -> thm)%}\egroup

\SYNOPSIS
Performs a proof by general mathematical induction on the natural numbers.

\DESCRIBE
The derived inference rule {\small\verb%GEN_INDUCT_RULE%} implements the rule of general
mathematical induction:
{\par\samepage\setseps\small
\begin{verbatim}
    A1 |- P[0/n]      A2 |- !n. (!m. m < n ==> P[m/n]) ==> P
   ----------------------------------------------------------  GEN_INDUCT_RULE
                        A1 u A2 |- !n. P
\end{verbatim}
}
\noindent When supplied with a theorem {\small\verb%A1 |- P[0/n]%}, which asserts the base
case of a proof of the proposition {\small\verb%P%} by general induction on {\small\verb%n%}, and the
 theorem
{\par\samepage\setseps\small
\begin{verbatim}
   A2 |- !n. (!m. m < n ==> P[m/n]) ==> P
\end{verbatim}
}

\noindent which asserts the step case in the induction on
{\small\verb%n%}, the inference rule {\small\verb%GEN_INDUCT_RULE%} returns {\small\verb%A1 u A2 |- !n. P%}.

\FAILURE
A call to {\small\verb%GEN_INDUCT_RULE th1 th2%} fails if the theorems {\small\verb%th1%} and {\small\verb%th2%} do
 not have the forms
{\small\verb%A1 |- P[0/n]%} and {\small\verb%A2 |- !n. (!m. m < n ==> P[m/n]) ==> P%} respectively.

\SEEALSO
GEN_INDUCT_TAC, INDUCT.

\ENDDOC
\DOC{GEN\_INDUCT\_TAC}

\TYPE {\small\verb%GEN_INDUCT_TAC : tactic%}\egroup

\SYNOPSIS
Performs tactical proof by general mathematical induction on the natural numbers.

\DESCRIBE
{\small\verb%GEN_INDUCT_TAC%} reduces a goal {\small\verb%!n.P%}, where {\small\verb%n%} has type {\small\verb%num%}, to two
subgoals corresponding to the base and step cases in a proof by general mathematical induction on {\small\verb%n%}. The induction hypothesis appears among the assumptions of the
subgoal for the step case.  The specification of {\small\verb%GEN_INDUCT_TAC%} is:
{\par\samepage\setseps\small
\begin{verbatim}
                          A ?- !n. P
   ========================================================  GEN_INDUCT_TAC
    A ?- P[0/n]     A u {!m. m < n' ==> P[m/n]} ?- P[n'/n]
\end{verbatim}
}
\noindent where {\small\verb%n'%} is a primed variant of {\small\verb%n%} that does not appear free in
the assumptions {\small\verb%A%} (usually, {\small\verb%n'%} just equals {\small\verb%n%}). When {\small\verb%GEN_INDUCT_TAC%} is
applied to a goal of the form {\small\verb%!n.P%}, where {\small\verb%n%} does not appear free in {\small\verb%P%},
the subgoals are just {\small\verb%A ?- P%} and {\small\verb%A u {!m. m < n' ==> P} ?- P%}.

\FAILURE
{\small\verb%GEN_INDUCT_TAC g%} fails unless the conclusion of the goal {\small\verb%g%} has the form
{\small\verb%!n.t%},
where the variable {\small\verb%n%} has type {\small\verb%num%}.

\SEEALSO
GEN_INDUCT_RULE, INDUCT_TAC.

\ENDDOC
\DOC{NUM\_EQ\_PLUS\_CONV}

\TYPE {\small\verb%NUM_EQ_PLUS_CONV : (term -> conv)%}\egroup

\SYNOPSIS
Adds a given expression of type {\small\verb%:num%} to both sides of an equality between
natural numbers.

\DESCRIBE
When applied to terms of the form {\small\verb%"n"%} and {\small\verb%"p = q"%}, where
{\small\verb%n%}, {\small\verb%p%} and {\small\verb%q%} have type {\small\verb%:num%}, the conversion
{\small\verb%NUM_EQ_PLUS_CONV%} returns the theorem:
{\par\samepage\setseps\small
\begin{verbatim}
   |-  (p = q) = (p + n = q + n)
\end{verbatim}
}

\FAILURE
Fails if the first term does not have type {\small\verb%:num%} or the second does not have
the form {\small\verb%"p = q"%}, where {\small\verb%p%} and {\small\verb%q%} are terms of type {\small\verb%:num%}.

\SEEALSO
NUM_LESS_EQ_PLUS_CONV, NUM_LESS_PLUS_CONV.

\ENDDOC
\DOC{NUM\_LESS\_EQ\_PLUS\_CONV}

\TYPE {\small\verb%NUM_LESS_EQ_PLUS_CONV : (term -> conv)%}\egroup

\SYNOPSIS
Adds a given expression of type {\small\verb%:num%} to both sides of a less-than-or-equal
expression. 

\DESCRIBE
When applied to terms of the form {\small\verb%"n"%} and {\small\verb%"p <= q"%}, where
{\small\verb%n%}, {\small\verb%p%} and {\small\verb%q%} have type {\small\verb%:num%}, the conversion
{\small\verb%NUM_LESS_EQ_PLUS_CONV%} returns the theorem:
{\par\samepage\setseps\small
\begin{verbatim}
   |-  p <= q = (p + n) <= (q + n)
\end{verbatim}
}

\FAILURE
Fails if the first term does not have type {\small\verb%:num%} or the second does not have
the form {\small\verb%"p <= q"%}.

\SEEALSO
NUM_EQ_PLUS_CONV, NUM_LESS_PLUS_CONV.

\ENDDOC
\DOC{NUM\_LESS\_PLUS\_CONV}

\TYPE {\small\verb%NUM_LESS_PLUS_CONV : (term -> conv)%}\egroup

\SYNOPSIS
Adds a given expression of type {\small\verb%:num%} to both sides of a less expression.

\DESCRIBE
When applied to terms of the form {\small\verb%"n"%} and {\small\verb%"p < q"%}, where
{\small\verb%n%}, {\small\verb%p%} and {\small\verb%q%} have type {\small\verb%:num%}, the conversion
{\small\verb%NUM_LESS_PLUS_CONV%} returns the theorem:
{\par\samepage\setseps\small
\begin{verbatim}
   |-  p < q = (p + n) < (q + n)
\end{verbatim}
}

\FAILURE
Fails if the first term does not have type {\small\verb%:num%} or the second does not have
the form {\small\verb%"p < q"%}.

\SEEALSO
NUM_EQ_PLUS_CONV, NUM_LESS_EQ_PLUS_CONV.

\ENDDOC
