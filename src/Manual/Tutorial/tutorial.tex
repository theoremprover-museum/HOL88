% =====================================================================
% HOL Manual LaTeX Source: tutorial
% =====================================================================

\documentstyle[12pt,fleqn,../LaTeX/alltt,../LaTeX/layout]{book}

% ---------------------------------------------------------------------
% Input defined macros and commands
% ---------------------------------------------------------------------
\input{../LaTeX/commands}
\input{../LaTeX/pic-macros}
% =====================================================================
% Macros for typesetting hol reference manual entries
% =====================================================================

% ---------------------------------------------------------------------
% boolean flag for verbose printing of reference manual typesetting
% ---------------------------------------------------------------------

\newif\ifverboseref
\verbosereffalse			  % don't be verbose

% ---------------------------------------------------------------------
% \DOC{<object>}  : start a manual entry for <object> (to be used when
%	            <object> is an UPPER-CASE ML identifier.
% ---------------------------------------------------------------------
\newcommand{\DOC}[1]%
{\bigskip
 {\ifverboseref{\def\_{\string_}\typeout{Typesetting: #1}}\fi}
 \markright{{\protect\small\bf #1}}
 \autoindex{#1@{\tt #1}}
 \begin{flushleft}
 \begin{tabular}{|c|}\hline
 \begin{minipage}{\minipagewidth}
 \bigskip
 {\LARGE\tt #1}
 \bigskip
 \end{minipage}\\ \hline
 \end{tabular}
 \end{flushleft}
 \bigskip
}

% ---------------------------------------------------------------------
% \LDOC{<object>}  : start a manual entry for <object> (to be used when
%		     <object> is a lower-case ML identifier.
% ---------------------------------------------------------------------

\newcommand{\LDOC}[1]%
{\bigskip
 {\ifverboseref{\def\_{\string_}\typeout{Typesetting: #1}}\fi}
 \markright{\bf #1}
 \autoindex{#1@{\tt #1}}
 \begin{flushleft}
 \begin{tabular}{|c|}\hline
 \begin{minipage}{\minipagewidth}
 \bigskip
 {\LARGE\tt #1}
 \bigskip
 \end{minipage}\\ \hline
 \end{tabular}
 \end{flushleft}
 \bigskip
}

% ---------------------------------------------------------------------
% Commands for parts of a \DOC:
%    \SYNOPSIS 
%    \DESCRIBE
%    \FAILURE
%    \EXAMPLE
%    \USES
%    \SEEALSO
% ---------------------------------------------------------------------

\newcommand{\SYNOPSIS}%
{\bigskip{\noindent\large\bf Synopsis}\newline\mbox{}}

\newcommand{\CATEGORIES}%
{\bigskip{\noindent\large\bf Categories}\newline\mbox{}}

\newcommand{\DESCRIBE}%
{\bigskip{\noindent\large\bf Description}\newline\mbox{}}

\newcommand{\FAILURE}%
{\bigskip{\noindent\large\bf Failure}\newline\mbox{}}

\newcommand{\EXAMPLE}%
{\bigskip{\noindent\large\bf Example}\newline\mbox{}}

\newcommand{\USES}%
{\bigskip{\noindent\large\bf Uses}\newline\mbox{}}

\newcommand{\SEEALSO}%
{\bigskip{\noindent\large\bf See also}}

% ---------------------------------------------------------------------
% \ENDDOC = do nothing
% ---------------------------------------------------------------------

\newcommand{\ENDDOC}{}

\makeatletter

\begingroup \catcode `|=0 \catcode `[= 1
\catcode`]=2 \catcode `\{=12 \catcode `\}=12              
\catcode`\\=12 |gdef|@xboxverb#1\ENDTHEOREM[#1|ENDTHEOREM]
|endgroup                                                 

\def\@boxverb{\bgroup\leftskip=5mm\parindent\z@
\parfillskip=\@flushglue\parskip\z@
\obeylines\tt \catcode``=13 \@noligs \let\do\@makeother \dospecials}

\def\boxverb{\@boxverb \frenchspacing\@vobeyspaces \@xboxverb}

\def\ENDTHEOREM{\egroup\filbreak}

\def\THEOREM #1 #2 {
 \autoindex{#1@{\tt #1}}
   \vspace{4mm plus2mm minus1mm}
   \noindent {\tt #1}\quad ({\tt #2}) \par \boxverb 
}

\makeatother

\def\none{{\it none}}



% Counter Peano used in logic.tex 
\newcounter{Peano} 
\setcounter{Peano}{1}

%\includeonly{title,contents,preface,../LaTeX/ack,intro,ml,logic,
% proof,references,Studies/preface,Studies/parity/parity}
%             Studies/microprocessor/all,
%\includeonly{Studies/microprocessor/all}
%\includeonly{Studies/int_mod/mod_arith_study/tutorial}
%\includeonly{binomial}
%\includeonly{parity}


\begin{document}

   \setlength{\unitlength}{1mm}		  % unit of length = 1mm
   \setlength{\baselineskip}{16pt}        % line spacing = 16pt

   % ---------------------------------------------------------------------
   % prelims
   % ---------------------------------------------------------------------

   \pagenumbering{roman}	          % roman page numbers for prelims
   \setcounter{page}{1}		          % start at page 1

   \begin{titlepage}

\setcounter{page}{1}

\begin{center}
{\Huge{\bf HOL CASE STUDY}}\\
\vskip .3in
{\Large\bf Microprocessor Systems}\\
\end{center}
\vskip .2in

\begin{inset}{Author}
Jeff Joyce\newline
Dept. of Computer Science\newline
University of British Columbia\newline
Vancouver, British Columbia\newline
Canada V6T 1W5\newline
{\bf Telephone:} (604) 228-3061\newline
{\bf Email:} \verb"joyce@ca.ubc.cs"
\end{inset}

\begin{inset}{Concepts illustrated}
Multi-level specification and verification,
abstract description and specification of hardware,
parameterization of theories,
proof techniques based on symbolic execution,
embedding other calculi in higher-order logic.
\end{inset}

\begin{inset}{Prerequisites}
Familiarity with \HOL\ language syntax
as given in {\it Getting Started with HOL}.
Familiarity with the \HOL\ system is not essential.
\end{inset}

\begin{inset}{Supporting files}
In the directory {\verb"hol/Training/studies/microprocessor"}.
\end{inset}

\begin{inset}{Abstract}
The multi-level specification and
verification of a very simple microprocessor is described.
A relationship between specifications for its implementation,
the external environment, and
its intended behaviour is rigorously established by formal
proof using the \HOL\ system.
This case study gives a detailed description of
how the \HOL\ language can be
used to specify hardware structure and behaviour.
It also outlines a strategy for using the \HOL\ system to verify a
set of correctness
theorems which relate different levels of specification.
The main proof technique is the use of inference rules
to symbolically execute microinstructions and microcode.
A form of temporal logic is embedded in the \HOL\ logic
for the special purpose of reasoning
about asynchronous interactions between the microprocessor
and external memory.
\end{inset}

\end{titlepage}

\newpage

			  % tutorial title page
   \chapter*{Preface}\markboth{Preface}{Preface}

This volume provides documentation for the user-contributed libraries
distributed with the \HOL\ system.  It should be read in conjunction with the
\HOL\ system documentation, which consists of three volumes:

\begin{myenumerate}
\item \TUTORIAL: a tutorial introduction to \HOL, with case studies.
\item \DESCRIPTION: a description of higher order logic,
the \ML\ programming language, and theorem proving methods in the \HOL\ system;
\item \REFERENCE: the reference documentation of the tools available in \HOL.
\end{myenumerate}

\noindent These documents are be referred to by the short names (in small
slanted capitals) given above.

The present volume, \LIBRARIES, contains a collection independent sections, one
for each documented library distributed with the system. This documentation,
which is written by the contributors of the corresponding \HOL\ system
libraries, is not, strictly speaking, part of the \HOL\ manual set, which
consists of the three volumes listed above.  The \LIBRARIES\ documentation is,
however, typeset and distributed with the system in a form consistent with the
\HOL\ system documentation.


		          % preface to entire tutorial
   \vfill
\chapter*{Authors}

This library is the combination of the arithmetic theories of the following people:

\begin{quote}
Mike Benjamin (British Aerospace Sowerby Research Centre, Bristol, U.K.), \\
R. J. Boulton (University of Cambridge, Computer Laboratory, U.K.),	 \\
Albert J. Camilleri (Hewlett-Packard Laboratories, Bristol, U.K.), \\
Rachel Cardell-Oliver (University of Cambridge, Computer Laboratory, U.K.), \\
Paul Curzon (University of Cambridge, Computer Laboratory, U.K.), \\
Elsa L. Gunter (AT\&T Bell Labs, USA) , \\
Jeff Joyce (University of British Columbia, Canada), \\
Philippe Leveilley (Ecole Polytechnique, Paris, France), \\
Wim Ploegaerts   (Imec vzw. Leuven, Belgium), \\
Wai Wong (University of Cambridge, Computer Laboratory, U.K.). \\
\end{quote}

\noindent Wim Ploegaerts compiled the theorems which came from other libraries
in the \HOL\ system and made many useful suggestions about the organization of
the library.



\newpage
		  % global acknowledgements
   \include{contents}		          % table of contents   

   \pagenumbering{arabic}		 % arabic page numbers
   \setcounter{page}{1}		         % start at page 1

   \chapter{The more\_arithmetic Library}

This document describes the facilities provided by the \ml{more\_arithmetic}
library. The 
library contains an \adhoc\ collection of definitions 
and theorems about arithmetic which extends the standard theory
\ml{arithmetic}. Included 
are many general purpose theorems about the arithmetic operators defined in the
theory \ml{arithmetic}. In addition, definitions of new operators are given with
associated 
theorems. Several conversions related to arithmetic are also
provided.  
The library is split into several theories, the ancestry of which is shown
below. 
{\samepage
\begin{center}
\begin{picture}(135,65)(-10,10)

\thicklines

% -----------------------------------------------------------
% Lines in theory hierarchy graph
% -----------------------------------------------------------

 \put(70,25){\line(0,1){5}}	 % more_arithmetic --> sub


 \put(10,27.5){\line(0,1){2.5}}  % more_arithmetic --> odd_even
 \put(90,27.5){\line(0,1){2.5}}	 % more_arithmetic --> pre
 \put(110,27.5){\line(0,1){2.5}} % more_arithmetic --> mult
 \put(-10,27.5){\line(0,1){2.5}} % more_arithmetic --> min_max
 \put(-10,27.5){\line(1,0){140}} % more_arithmetic --> min_max
 \put(130,27.5){\line(0,1){2.5}} % more_arithmetic --> div_mod
 \put(30,27.5){\line(0,1){2.5}}    % more_arithmetic --> ineq

 \put(70,35){\line(0,1){5}}	 % sub --> add

 \put(50,37.5){\line(0,1){2.5}}	 % sub --> zero_ineq
 \put(50,37.5){\line(1,0){20}}	 % sub --> zero_ineq

 \put(70,45){\line(0,1){5}}	 % add --> suc

 \put(-10,57.5){\line(1,0){140}} % arithmetic --> min_max

 \put(70,55){\line(0,1){5}}      % suc --> arithmetic
 \put(-10,35){\line(0,1){22.5}}	 % min_max --> arithmetic
 \put(10,35){\line(0,1){22.5}}	 % odd_even --> arithmetic
 \put(90,35){\line(0,1){22.5}}	 % pre --> arithmetic
 \put(110,35){\line(0,1){22.5}}	 % mult --> arithmetic
 \put(130,35){\line(0,1){22.5}}	 % div_mod --> arithmetic

 \put(50,45){\line(0,1){12.5}}	 % zero_ineq --> arithmetic
 \put(30,35){\line(0,1){22.5}}	 % ineq --> arithmetic


% -----------------------------------------------------------
% Theory names:
% -----------------------------------------------------------

\put(70,22.5){\makebox(0,0){\verb!more\_arithmetic!}}
\put(90,32.5){\makebox(0,0){\verb!pre!}}
\put(50,42.5){\makebox(0,0){\verb!zero\_ineq!}}
\put(10,32.5){\makebox(0,0){\verb!odd\_even!}}
\put(-10,32.5){\makebox(0,0){\verb!min\_max!}}
\put(110,32.5){\makebox(0,0){\verb!mult!}}
\put(130,32.5){\makebox(0,0){\verb!div\_mod!}}
\put(30,32.5){\makebox(0,0){\verb!ineq!}}
\put(70,52.5){\makebox(0,0){\verb!suc!}}
\put(70,32.5){\makebox(0,0){\verb!sub!}}
\put(70,42.5){\makebox(0,0){\verb!add!}}
\put(70,62.5){\makebox(0,0){\verb!arithmetic!}}
\end{picture}
\end{center}
\nopagebreak

\noindent
A complete list of the arithmetic theorems which are available in the library
 is given in Chapter \ref{thms}. Due
to the large number of theorems, users of the library may find the theorem
retrieval library of help. The above hierarchy should be kept in mind when
searching for theorems.
 All definitions and theorems from the ancestor theories are
automatically loaded, if possible, when their names are first mentioned during
a \HOL\ session. 
}

\section{The Theory {\tt more\_arithmetic}}

The theory \ml{more\_arithmetic}\index{more_arithmetic, theory@{\ptt more\_arithmetic}, theory}
is the descendant theory of all the theories in 
the library. It contains no theorems or definitions itself.

			 % intro: getting and installing hol
   % =====================================================================
% HOL Course Slides: introduction to ML		      (c) T melham 1990
% =====================================================================

\documentstyle[alltt,12pt,layout]{article}

% ---------------------------------------------------------------------
% Preliminary settings.
% ---------------------------------------------------------------------

\renewcommand{\textfraction}{0.01}	  % 0.01 of the page must contain text
\setcounter{totalnumber}{10}	 	  % max of 10 figures per page
\flushbottom				  % text extends right to the bottom
\pagestyle{slides}			  % slides page style
\setlength{\unitlength}{1mm}		  % unit = 1 mm

% ---------------------------------------------------------------------
% load macros
% ---------------------------------------------------------------------


\input{holmacs}
\input{tokmac}

%\newtokmac{ter}{\tt}
%\newtokmac{nter}{\tt}
\newtokmac{mlname}{\tt}
\newtokmac{CONST}{\constfont}


\newenvironment{hproof}{\begin{center}
 \begin{tabular*}{5.25in}{r>{$}l<{$}@{\extracolsep{0pt plus 1fill}}>{[}r<{]}}}
 {\end{tabular*}\end{center}}

\makeatletter

%% \verbatimnumbered{<file>}
%% read a file and format the contains in a verbatim environment with
%% line number.
%\newcounter{VerbatimLineNo}
%\def\verbatimnumbered#1{\begingroup
%  \setcounter{VerbatimLineNo}{0}%
%  \def\verbatim@processline{%
%    \addtocounter{VerbatimLineNo}{1}%
%    \leavevmode
%    \llap{\theVerbatimLineNo
%          \ \hskip\@totalleftmargin}%
%    \the\verbatim@line\par}%
%  \verbatiminput{#1}\endgroup}

%% Redefine the theindex environment
%%
%\renewenvironment{theindex}{\begin{multicols}{2}[\section*{\indexname}]%
% \columnseprule \z@ \columnsep 35\p@
% \parindent\z@ \parskip\z@ plus.3\p@\relax\let\item\@idxitem}{\end{multicols}}
%
%\def\@idxitem{\par\hangindent 40\p@}
%
%\def\subitem{\par\hangindent 40\p@ \hspace*{20\p@}}
%
%\def\subsubitem{\par\hangindent 40\p@ \hspace*{30\p@}}
%
%\def\indexspace{\par \vskip 10\p@ plus5\p@ minus3\p@\relax}

\makeatother

%% macros for special words
%%
%\def\HOL{{\sc  hol}}
%\def\CHOL{{\sc hol88}}

%% Redfine constfont. The font cmssc12 is a vertual font based on
%% cmss12. The only difference is that the character at '137 becomes
%% an underline in cmssc12.
\font\sfc=cmssc12 \def\constfont{\sfc}
\def\ul#1{$\underline{#1}$}



% ---------------------------------------------------------------------
% set caption at the foot of pages for this series of slides
% ---------------------------------------------------------------------
\ftext{Introduction to ML}{4}

% ---------------------------------------------------------------------
% Slides
% ---------------------------------------------------------------------
\begin{document}

% ---------------------------------------------------------------------
% Title page for this series of slides
% ---------------------------------------------------------------------

\bsectitle
A Brief\\
Introduction to ML
\esectitle

% =====================================================================
\slide{The ML Programming Language}

\point{Functional programming language.}
\point{Normally used interactively.}
\point{Strongly typed, with:}
   \subpoint{type inference,}
   \subpoint{abstract data types, and}
   \subpoint{polymorphic types.}
\point{Has exception-handling mechanisms.}

\vskip5mm

\point{ML is the meta-language for HOL.}
\point{Note: the ML of HOL88 is \underline{not} SML.}


% =====================================================================
\slide{Evaluation and Bindings}

\point{Evaluation:}

\begin{session}\begin{verbatim}
#2 + 3;;   
5 : int

#rev [1;2;3;4;5];;
[5; 4; 3; 2; 1] : int list
\end{verbatim}\end{session}

\point{Simple bindings:}

\begin{session}\begin{verbatim}
#let n = 8 * 2 + 5;;
n = 21 : int

#n * 2;;
42 : int
\end{verbatim}\end{session}

\point{The special identifier `{\tt it}':}

\begin{session}\begin{verbatim}
#it;;
42 : int
\end{verbatim}\end{session}

% =====================================================================
\slide{Multiple and Local Bindings}

\point{Multiple bindings:}

\begin{session}\begin{verbatim}
#let one = 1 and two = 2;;
one = 1 : int
two = 2 : int
\end{verbatim}\end{session}

\point{Local bindings:}

\begin{session}\begin{verbatim}
#let n = 0;;
n = 0 : int

#let n = 12 in n/6;;
2 : int

#n;;
0 : int
\end{verbatim}\end{session}

\point{Multiple local bindings:}

\begin{session}\begin{verbatim}
#let n = 5 and m = 6 in n + m;;
11 : int
\end{verbatim}\end{session}


% =====================================================================
\slide{Integers and Booleans}

\point{Operations on integers:}

\begin{session}\begin{verbatim}
#let n = 2 + (3 * 4);;
n = 14 : int

#let n = (10 / 2) - 7;;
n = -2 : int
\end{verbatim}\end{session}


\point{Operations on booleans:}

\begin{session}\begin{verbatim}
#let b1 = true and b2 = false;;
b1 = true : bool
b2 = false : bool

#1 = (1 + 1);;
false : bool

#not (b1 or b2);;
false : bool

#(7 < 3) & (false or 2 > 0);;
true : bool
\end{verbatim}\end{session}


% =====================================================================
\slide{Defining Functions}

\point{One argument:}

\begin{session}\begin{verbatim}
#let f n = n + n;;
f = - : (int -> int)

#f 22;;
44 : int
\end{verbatim}\end{session}

\point{Two or more arguments:}

\begin{session}\begin{verbatim}
#let plus(n,m) = n + m;;
plus = - : ((int # int) -> int)

#plus (1,2);;
3 : int
\end{verbatim}\end{session}


% =====================================================================
\slide{Curried Functions}

\point{Curried addition:}

\begin{session}\begin{verbatim}
#let plus n m = n + m;;
plus = - : (int -> int -> int)

#plus 1 2;;
3 : int
\end{verbatim}\end{session}

\point{Partial application:}

\begin{session}\begin{verbatim}
#let inc = plus 1;;
inc = - : (int -> int)

#inc 7;;
8 : int
\end{verbatim}\end{session}

\point{Higher-order functions:}

\begin{session}\begin{verbatim}
#let foo f n = (f(n+1)) / 2 ;;
foo = - : ((int -> int) -> int -> int)

#foo inc 3;;
2 : int
\end{verbatim}\end{session}


% =====================================================================
\slide{Defining Recursive Functions}

\point{{\tt let} won't define recursive functions:}

\begin{session}\begin{verbatim}
#let f n = n + n;;
f = - : (int -> int)

#let f n = if (n=0) then 1 else n * f(n-1);;
f = - : (int -> int)

#f 3;;
12 : int
\end{verbatim}\end{session}

\point{Defining recursive functions, {\tt letrec}:}

\begin{session}\begin{verbatim}
#letrec f n = if (n=0) then 1 else n * f(n-1);;
f = - : (int -> int)

#f 3;;
6 : int
\end{verbatim}\end{session}

% =====================================================================
\slide{Lambda Abstractions}

\point{The increment function:}

\begin{session}\begin{verbatim}
#\x. x + 1;;
- : (int -> int)

#(\x. x + 1) 2;;
3 : int
\end{verbatim}\end{session}

\point{Curried multiplication:}

\begin{session}\begin{verbatim}
#\x. \y. x * y;;
- : (int -> int -> int)

#let double = (\x. \y. x * y) 2;;
double = - : (int -> int)

#double 22;;
44 : int
\end{verbatim}\end{session}

\point{Abbreviation:}
\begin{session}\begin{verbatim}
#\x y. x * y;;
- : (int -> int -> int)
\end{verbatim}\end{session}


% =====================================================================
\slide{Other Data Types}

\point{Strings:}
\begin{session}\begin{verbatim}
#`abc`;;
`abc` : string

#ascii;;
- : (int -> string)

#ascii 97;;
`a` : string
\end{verbatim}\end{session}

\point{Void:}

\begin{session}\begin{verbatim}
#();;
() : void

#quit;;
- : (void -> void)
\end{verbatim}\end{session}

% =====================================================================
\slide{Lists}

\point{Head and tail:}

\begin{session}\begin{verbatim}
#let l = [2;3;2+3];;
l = [2; 3; 5] : int list

#hd l;;
2 : int

#tl l;;
[3; 5] : int list
\end{verbatim}\end{session}

\point{Cons and concatenation:}
\begin{session}\begin{verbatim}
#9 . l;;
[9; 2; 3; 5] : int list

#[true;false] @ [false;true];;
[true; false; false; true] : bool list
\end{verbatim}\end{session}

\point{Empty lists:}
\begin{session}\begin{verbatim}
#null l;;
false : bool

#null [];;
true : bool
\end{verbatim}\end{session}

% =====================================================================
\slide{Pairs}

\point{Notation:}

\begin{session}\begin{verbatim}
#let p = (2,3);;
p = (2, 3) : (int # int)

#fst p;;
2 : int

#snd p;;
3 : int
\end{verbatim}\end{session}

\point{Right associativity:}

\begin{session}\begin{verbatim}
#fst (2,3,4);;
2 : int

#snd(2,3,4);;
(3, 4) : (int # int)
\end{verbatim}\end{session}

\point{Example:}
\begin{session}\begin{verbatim}
#let swap p = (snd p, fst p);;
rev = - : ((* # **) -> (** # *))

#rev p;;
(3, 2) : (int # int)
\end{verbatim}\end{session}


% =====================================================================
\slide{Polymorphism}

\point{Example: the function {\tt hd}}

\begin{session}\begin{verbatim}
#hd [2;3];;
2 : int

#hd [true;false];;
true : bool
\end{verbatim}\end{session}
\vskip1mm
\bpindent\LARGE\bf
What is the type of {\tt hd}:
\epindent
\vskip7mm
\bspindent\LARGE
\bf \verb!int list -> int! or  \verb!bool list -> bool!?
\espindent
\vskip5mm

\point{The function {\tt hd} has both these types:}

\begin{session}\begin{verbatim}
#hd;;
- : (* list -> *)
\end{verbatim}\end{session}
\vskip1mm
\bpindent\LARGE\bf
The `{\tt *}' here is a type variable.
\epindent

\vskip5mm


\point{That is, {\tt hd} can have any type of the form}
\vskip7mm
\bspindent\LARGE
 {\tt $\sigma$ list -> $\sigma$} 
\espindent 
\vskip7mm
\bpindent\LARGE\bf
where $\sigma$ is an ML type. 
\epindent


% =====================================================================
\slide{Type Inference}

\point{ML infers the most general type.}

\vskip4mm

\point{For example, the mapping function:}
\begin{session}\begin{verbatim}
#letrec map f l =
        if (null l)
           then []
           else f(hd l).(map f (tl l));;
map = - : ((* -> **) -> * list -> ** list)

#map (\x.0);;
- : (* list -> int list)
\end{verbatim}\end{session}

\point{Function composition:}
\begin{session}\begin{verbatim}
#let comp f g x = f(g x);;
comp = - : ((* -> **) -> (*** -> *) -> *** -> **)

#comp null (map \y.y+1);;
- : (int list -> bool)
\end{verbatim}\end{session}

% =====================================================================
\slide{Exception Handling}

\point{Failure:}

\begin{session}\begin{verbatim}
#hd [];;
evaluation failed     hd

#1 / 0;;
evaluation failed     div
\end{verbatim}\end{session}

\point{Explicitly generating failure}
\begin{session}\begin{verbatim}
#failwith `foo`;;
evaluation failed     foo

#letrec fact n =
    if (n<0)
       then failwith `negative argument to fact`
       else if (n=0) then 1 else n * fact(n-1);;
fact = - : (int -> int)

#fact (-1);;
evaluation failed     negative argument to fact
\end{verbatim}\end{session}

\point{Trapping failure:}
\begin{session}\begin{verbatim}
#hd [] ? 99;;
99 : int
\end{verbatim}\end{session}

% =====================================================================
\slide{Common Errors}

\point{Unbound variable:}
\begin{session}\begin{verbatim}
#foo;;

unbound or non-assignable variable foo
1 error in typing
typecheck failed     
\end{verbatim}\end{session}

\point{Type violation:}
\begin{session}\begin{verbatim}
#hd 2;;

ill-typed phrase: 2
has an instance of type  int
which should match type  * list
1 error in typing
typecheck failed     
\end{verbatim}\end{session}

\point{Mismatched parentheses}
\begin{session}\begin{verbatim}
#let f x = (x + 2;;
bad paren balance
skipping: 2 ;; parse failed     

#let f x = x + 2);;
non top level decln must have IN clause
skipping: 2 ) ;; parse failed     
\end{verbatim}\end{session}


% =====================================================================
\slide{Some System Functions}


\point{Load a file called {\tt foo.ml}:}

\begin{session}\begin{alltt}
#load;;
- : ((string # bool) -> void)

#load(`foo`, false);;
......() : void

#load(`foo`,true);;
      \(\vdots\)   
   {\it (output from evaluating contents of{\tt foo})}
File foo loaded
() : void
\end{alltt}\end{session}

\point{Get online help:}
\begin{session}\begin{alltt}
#help;;
- : (string -> void)

#help `item`;;
      \(\vdots\)
   {\it (prints information about{\tt item})}
() : void
\end{alltt}\end{session}

\point{Terminate the session:}

\begin{session}\begin{alltt}
#quit();;
\end{alltt}\end{session}


% =====================================================================
\slide{Summary}

\def\bk{{\tt\char`\\ }}
\def\_{\char'137}

\point{Bindings}

  \subpoint{{\tt let $x_1$ = $e_1$ and $\dots$ and $x_n$ = $e_n$;;}}
  \subpoint{{\tt let $x_1$ = $e_1$ and $\dots$ and $x_n$ = $e_n$ in $e$;;}}

\point{Defining functions}

  \subpoint{{\tt let f $v_1$ $\dots$ $v_n$ = $e$;;}}
  \subpoint{{\tt letrec f $v_1$ $\dots$ $v_n$ = $e$;;}}

\point{Lambda abstractions}

  \subpoint{{\tt \bk $v_1$ $\dots$ $v_n$ .\ $e$}}

\point{Data types}

  \subpoint{{\tt bool}, {\tt int}, {\tt string}, {\tt void}}
  \subpoint{{\tt * list}, {\tt * \# **}, {\tt * -> **}}

\point{Exception handling}

\subpoint{{\tt $e_1$ ?\ $e_2$}, {\tt failwith `{\it string}`}}

\point{Useful system functions}

  \subpoint{{\tt load}, {\tt help}, {\tt quit}}


\end{document}

				 % intro to ml
   \include{logic}			 % the HOL logic
   \include{proof}			 % intro to proof in HOL
   \include{parity}                      % parity example
   \include{tool}                        % conjunction canonicalization tool
   \include{binomial}                    % Andy Gordon's Binomial Thm in HOL
   \begin{thebibliography}{99}

\bibitem{shapiro} % OK
L. Sterling and E. Shapiro, {\it The Art of Prolog: Advanced Programming
Techniques}, {\small MIT} Press, 1986.

\bibitem{fischer} % OK
C.N. Fischer and R.J. LeBlanc, Jr., {\it Crafting a Compiler},
Benjamin/Cummings, 1988.

\end{thebibliography}
			 % references

\end{document}


