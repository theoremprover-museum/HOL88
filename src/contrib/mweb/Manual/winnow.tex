%% To format this section only, uncomment those line starting with a single %
%% Document Type: LaTeX
%\documentstyle[12pt,array,verbatim,vrbinput]{article}

%\begin{document}

%

\input{holmacs}
\input{tokmac}

%\newtokmac{ter}{\tt}
%\newtokmac{nter}{\tt}
\newtokmac{mlname}{\tt}
\newtokmac{CONST}{\constfont}


\newenvironment{hproof}{\begin{center}
 \begin{tabular*}{5.25in}{r>{$}l<{$}@{\extracolsep{0pt plus 1fill}}>{[}r<{]}}}
 {\end{tabular*}\end{center}}

\makeatletter

%% \verbatimnumbered{<file>}
%% read a file and format the contains in a verbatim environment with
%% line number.
%\newcounter{VerbatimLineNo}
%\def\verbatimnumbered#1{\begingroup
%  \setcounter{VerbatimLineNo}{0}%
%  \def\verbatim@processline{%
%    \addtocounter{VerbatimLineNo}{1}%
%    \leavevmode
%    \llap{\theVerbatimLineNo
%          \ \hskip\@totalleftmargin}%
%    \the\verbatim@line\par}%
%  \verbatiminput{#1}\endgroup}

%% Redefine the theindex environment
%%
%\renewenvironment{theindex}{\begin{multicols}{2}[\section*{\indexname}]%
% \columnseprule \z@ \columnsep 35\p@
% \parindent\z@ \parskip\z@ plus.3\p@\relax\let\item\@idxitem}{\end{multicols}}
%
%\def\@idxitem{\par\hangindent 40\p@}
%
%\def\subitem{\par\hangindent 40\p@ \hspace*{20\p@}}
%
%\def\subsubitem{\par\hangindent 40\p@ \hspace*{30\p@}}
%
%\def\indexspace{\par \vskip 10\p@ plus5\p@ minus3\p@\relax}

\makeatother

%% macros for special words
%%
%\def\HOL{{\sc  hol}}
%\def\CHOL{{\sc hol88}}

%% Redfine constfont. The font cmssc12 is a vertual font based on
%% cmss12. The only difference is that the character at '137 becomes
%% an underline in cmssc12.
\font\sfc=cmssc12 \def\constfont{\sfc}
\def\ul#1{$\underline{#1}$}



\section{An Extension to the {\tt mweb} suite}\label{sec-winnow}

The {\bf winnow} and {\bf mmerge} programs were added to the suite of
\mweb\ tools to permit the creation of a ``master'' format file from a
fairly ordinary \LaTeX\ source in which interactive \HOL\ sessions were
included.  The circumstances included a very large document with many
formatting-introduced errors, and generally datedness as the \HOL\ 
system evolved during the lengthy preparation of the material.
The intent was to generate a file which could be used with the \mweb\ 
suite of tools to (nearly automatically) update and maintain the
interactive \HOL\ material in the document. 

By providing a filter from a \LaTeX\ source, there is flexibility in
chosing which file to edit.  Both the \LaTeX\ source and the master
files are possible, depending entirely on the inclination of the user.
Using the \mweb\ suite to update the files with newly run interactive
HOL sessions when required greatly reduces the overhead of maintaining
the currency of a document.

In order to make the process as automatic as possible, each file
needed to contain all of the ML source code required for execution
in \HOL.  In some cases this required material which was not to appear
in the final document to be included in the source.  A ``hidden hol''
environment was created for this purpose. 

The modified processing sequence is shown in Figure~\ref{fig:new-proc}.
Observe that the two instances of \LaTeX\ files in the diagram can be
interpreted as a cycle in the file processing.


\begin{figure}
\begin{center}
\setlength{\unitlength}{0.009000in}%
\small
\begin{picture}(340,700)(40,260)
\thicklines
\put(240,860){\oval(80,40)}
\put(340,380){\oval(80,40)}
\put(340,540){\oval(80,40)}
\put(140,700){\oval(80,40)}
\put(140,540){\oval(80,40)}
\put(200,380){\oval(80,40)}
\put( 40,440){\framebox(80,40){}}
\put(160,440){\framebox(80,40){}}
\put(100,600){\framebox(80,40){}}
\put(200,920){\framebox(80,40){}}
\put(200,760){\framebox(80,40){}}
\put(300,440){\framebox(80,40){}}
\put(300,280){\framebox(80,40){}}
\put(160,280){\framebox(80,40){}}
\put(240,920){\vector( 0,-1){ 40}}
\put(240,840){\vector( 0,-1){ 40}}
\put(140,680){\vector( 0,-1){ 40}}
\put(140,600){\vector( 0,-1){ 40}}
\put(340,520){\vector( 0,-1){ 40}}
\put(340,440){\vector( 0,-1){ 40}}
\put(340,360){\vector( 0,-1){ 40}}
\put(200,440){\vector( 0,-1){ 40}}
\put(200,360){\vector( 0,-1){ 40}}
\put(240,760){\vector(-2,-1){ 80}}
\put(240,760){\line  ( 5,-2){100}}
\put(340,720){\vector( 0,-1){160}}
\put(140,520){\vector(-3,-2){ 60}}
\put(140,520){\vector( 3,-2){ 60}}
\put(200,280){\line( 0,-1){ 20}}
\put(200,260){\line( 1, 0){ 85}}
\put(285,260){\line( 0, 1){280}}
\put(285,540){\vector( 1, 0){ 15}}
\put(180,620){\line( 1, 0){ 75}}
\put(255,620){\line( 0,-1){240}}
\put(255,380){\vector(-1, 0){ 15}}
\put(140,615){\makebox(0,0)[b]{\raisebox{0pt}[0pt][0pt]{ML file}}}
\put(240,935){\makebox(0,0)[b]{\raisebox{0pt}[0pt][0pt]{\LaTeX\ file}}}
\put(240,855){\makebox(0,0)[b]{\raisebox{0pt}[0pt][0pt]{\winnow}}}
\put(240,775){\makebox(0,0)[b]{\raisebox{0pt}[0pt][0pt]{MASTER}}}
\put(140,695){\makebox(0,0)[b]{\raisebox{0pt}[0pt][0pt]{\tangle}}}
\put(140,535){\makebox(0,0)[b]{\raisebox{0pt}[0pt][0pt]{HOL}}}
\put( 80,455){\makebox(0,0)[b]{\raisebox{0pt}[0pt][0pt]{THEORY}}}
\put(200,455){\makebox(0,0)[b]{\raisebox{0pt}[0pt][0pt]{LOG file}}}
\put(340,455){\makebox(0,0)[b]{\raisebox{0pt}[0pt][0pt]{\LaTeX\ file}}}
\put(340,535){\makebox(0,0)[b]{\raisebox{0pt}[0pt][0pt]{\weave}}}
\put(340,375){\makebox(0,0)[b]{\raisebox{0pt}[0pt][0pt]{\LaTeX}}}
\put(340,295){\makebox(0,0)[b]{\raisebox{0pt}[0pt][0pt]{Document}}}
\put(200,375){\makebox(0,0)[b]{\raisebox{0pt}[0pt][0pt]{\merge}}}
\put(200,295){\makebox(0,0)[b]{\raisebox{0pt}[0pt][0pt]{TAG file}}}
\end{picture}

\end{center}
\caption{Extended operation of \mweb\label{fig:new-proc}}
\end{figure}

\subsection{Using {\tt winnow}}

An informal specification for the function of the winnow utility is
given in Figure~\ref{fig:winnow-io}.  There are two envionments in the
\LaTeX\ document which are modified by the filter.  The first is the
environment which contains displayed interactive \HOL\ sessions,
delimited by \verb+\beginhol+ and \verb+\endhol+.  These commands are
\LaTeX\ macros which define how the material will be displayed (in
this case, a framebox the allowed textwidth size encloses the
verbatim-displayed material.)   

\begin{figure}
\begin{center}\it
\begin{tabular}{|l|c|l|}
\multicolumn{1}{c}{\rm input } & \multicolumn{1}{c}{} &
 \multicolumn{1}{c}{\rm \winnow\ output}\\
 \cline{1-1}\cline{3-3}
 & & \\
\verb+\beginhol+ \hspace*{7em} & \hspace*{1em} & \verb+\beginhol+ \hspace*{2em} \\
                               & & \para{@ \{<infilename>_<seccount>\}} \\
                               & & \para{@t\{<infilename>_<seccount>\}} \\
                               & & \para{@N} \\
\verb+%+ hol comment \verb+%+  & & \verb+%+ hol comment \verb+%+ \\
\verb+#+command line\verb+;;+  & & command line \verb+;;+ \\
returned value                 & & \\
                               & & \\
\verb+%+ comment line 1        & & \verb+%+ comment line 1 \\
  comment line 2 \verb+%+      & &   comment line 2 \verb+%+ \\
\verb+#+command line 1         & & command line 1 \\
 command line 2\verb+;;+       & & command line 2\verb+;;+ \\
returned value/feedback        & & \\
 \ldots                        & & \\
\verb+\endhol+                 & & \para{@-}\\
                               & & \verb+\endhol+\\
other input                    & & other input\\
 \ldots                        & & \ldots \\
\verb+\begin{hh}+              & & \verb+\begin{hh}+ \\
                               & & \para{@ \{\$<infilename>_<seccount>\}} \\
                               & & \para{@t\{\$<infilename>_<seccount>\}} \\
                               & & \para{@N} \\
command 1\verb+;;+             & & command 1\verb+;;+ \\
 \ldots                        & &  \ldots \\
command n\verb+;;+             & & command n\verb+;;+ \\
\verb+\end{hh}+                & & \para{@-} \\
                               & & \verb+\end{hh}+ \\
\cline{1-1}\cline{3-3}
\end{tabular}
\end{center}
\caption{Input format and corresponding output from \winnow\label{fig:winnow-io}}
\end{figure}


The other environment contains the \HOL\ commands which will not
appear in the final document, but which are necessary for the
extracted ML file to be executable.  The definition of this macro
follows.

\vspace*{12pt}
\noindent
\begin{boxedminipage}{\textwidth}
\begin{center}
\begin{verbatim}

\makeatletter
\def\hh{\@bsphack
             \let\do\@makeother\dospecials\catcode`\^^M\active
             \let\verbatim@startline\relax
             \let\verbatim@addtoline\@gobble
             \let\verbatim@processline\relax
             \let\verbatim@finish\relax
             \verbatim@}
\let\endhh=\@esphack
\makeatother

\end{verbatim}
\end{center}
\end{boxedminipage}
\vspace*{12pt}

A number of formatting requirements and assumptions exist.  
\begin{enumerate}
\item All displayed \HOL\ commands (ie.~those enclosed by the commands\linebreak
\verb+\beginhol+ and \verb+\endhol+) are preceded by a single prompt \verb+#+,
regardless of how many lines the command occupies.  This prompt is the
first character on the line on which it occurs.  The program expects
that the character sequence beginning after the prompt and continuing
until the {\bf end} of the line on which the characters \verb+;;+
appear, will comprise an executable \HOL\ command.  Thus, comments may
be included within a command for example.
\item Comments begin with the ML comment character \verb+%+ as the
first character on a line, and continue up to the next occurrance of a
comment character.  Multiline comments are permitted, but
``supercomments'' (delimited by \verb+%<+ and \verb+>%+) are not
supported to enclose simple comments. 
\item The commands \verb+\beginhol+, \verb+\endhol+, \verb+\begin{hh}+,
and \verb+\end{hh}+ must all be at the beginning of a line.  No
indentation is permitted.
\item Hidden \HOL\ commands (ie.~those enclosed by \verb+\begin{hh}+
and \verb+\end{hh}+) must not have a prompt; everything thus enclosed
is expected to be suitable for feeding directly to \HOL\ without
modification or stripping prompt characters.  All commands must be
terminated with \verb+;;+.
\end{enumerate}

The program shares the same front end as the other utilities in the
\mweb\ suite, so the command line argments, etc.\ have the same
format.  There are different defaults for some parameters, as well as
a few added parameters.  These are listed below.

The most visible difference is that all commands in the \LaTeX\ source
file begin with the \verb+\+ character since they are also commands to
\LaTeX.  As with the other \mweb\ tools, the exact command strings and
output strings can be customized by changing the values of a set of
parameters. 

\noindent\cmdrule

\begin{describecmd}{begin_section}{char_begin_sec}{\none}
{\verb|\\beginhol|}
\describe
This command begins a section of displayed interactive \HOL\ material.
\winnowaction: pass the command unchanged to output.  Then output
three \mweb\ macros as follows: 
\para{out_char_cmd} \para{out_begin_sec_cmd} with an argument made up of the 
current values of \para{infilename} and \para{seccount}, enclosed
by \para{mac_begin_arg} and \para{mac_end_arg};
\para{out_char_cmd} \para{char_begin_tag} with an argument made up of the 
current values of \para{infilename} and \para{seccount}, enclosed
by \para{mac_begin_arg} and \para{mac_end_arg};
\para{out_char_cmd} \para{out_char_new_cmd}.  Then, copy all \HOL\ commands
and comments to the output until the command \cmd{char_end_sec} is
found.  \HOL\ commands begin with a \para{char_hol_prompt} and end
with \para{char_end_hol_in}.  Comments are enclosed by
\para{char_hol_begin_comment} and \para{char_hol_end_comment}.
\end{describecmd}

\begin{describecmd}{end_section}{char_end_sec}{\none}
{\verb|\\endhol|}
\describe
This command terminates a section of displayed interactive \HOL\ material.
\winnowaction: Output \mweb\ macro
\para{out_char_cmd} \para{out_end_sec_cmd} followed by the command itself unchanged.
\end{describecmd}

\begin{describecmd}{begin_ml_section}{char_ml_sec}{\none}
{\verb|\\begin{hh}|}
\describe
This command begins a `hidden \HOL' section. 
\winnowaction: pass the command unchanged to output.  Then output
three \mweb\ macros as follows: 
\para{out_char_cmd} \para{out_begin_sec_cmd} with an argument made up of
\para{char_hide_indicator} and the 
current values of \para{infilename} and \para{seccount}, enclosed
by \para{mac_begin_arg} and \para{mac_end_arg};
\para{out_char_cmd} \para{char_begin_tag} with an argument made up of
\para{char_hide_indicator} and the current values of \para{infilename} and
\para{seccount}, enclosed by \para{mac_begin_arg} and \para{mac_end_arg};
\para{out_char_cmd} \para{out_char_new_cmd}.  
Then, copy all \HOL\ commands and comments to the output until the
command \cmd{char_end_hide} is found. 
\end{describecmd}

\begin{describecmd}{end_hide_hol_section}{char_end_hide}{\none}
{\verb|\\end{hh}|}
\describe
This command terminates a section of displayed interactive \HOL\ material.
\winnowaction: Output \mweb\ macro
\para{out_char_cmd} \para{out_end_sec_cmd} followed by the command itself unchanged.
\end{describecmd}
\vspace*{8pt}

The parameters understood and used by \winnow\ in a different way or
in addition to those of the rest of \mweb\ are described in the
following.  All the values are strings.  The values of the parameters
may be changed by the user using the \cmd{para_def} command or the
command line option {\tt -D}\index{command line options!-D@{\tt -D}}.
First we present the parameters which
are used differently or have different defaults.

\noindent\cmdrule

\begin{describepara}{char_begin_sec}{\verb|beginhol|}\describe
This parameter associates with the \cmd{begin_section} command
which begins a section of visible interactive \HOL. 
\end{describepara}
\begin{describepara}{char_cmd}{\verb|\\|}\describe
This parameter is the command escape character.
\end{describepara}
\begin{describepara}{char_end_sec}{\verb|endhol|}\describe
This parameter associates with the \cmd{end_section} command
which terminates a section of visible interactive \HOL. 
\end{describepara}
\begin{describepara}{char_begin_star_sec}{\verb|\n|}\describe
This parameter is an unused entry in an internal procedure to process
a section of text.
\end{describepara}
\begin{describepara}{insuffix}{\verb*|.tex|}\describe
The value of this parameter is the default suffix of the input file.
If the name of input file given on the command line has no suffix,
this string is appended to it. The \winnow\ program attempts to open a
file of this resulting name for input.
\end{describepara}
\begin{describepara}{char_ml_sec}{\verb|begin{hh}|}\describe
This parameter associates with the \cmd{begin_ml_section} command
which begins a `hidden \HOL' section. 
\end{describepara}
\begin{describepara}{filter_mode}{true}\describe
The value of this parameter is boolean. When it is true, the filter
mode is active.  It must be set to true for \winnow.
\end{describepara}

\vspace*{12pt}

Added parameters follow.

\noindent\cmdrule

{\def\describ@para#1{\endgroup
 \vspace*{5pt}\noindent
 \begin{minipage}{\textwidth}
 \noindent\index{#1@\string\idxmlname{#1}|ul}%
 \begin{tabular}{>{\bf}ll>{\bf}ll}
 {NAME:} & \hbox to 13pc{\tt#1\hss} &
  {DEFAULT:}&\relax }

\begin{describepara}{out_char_cmd}{\verb|@|}\describe
This is the output command escape character.
\end{describepara}
\begin{describepara}{out_begin_sec_cmd}{\verb*| |}\describe
This is the output \para{begin_section} command string.
\end{describepara}
\begin{describepara}{out_end_sec_cmd}{\verb|-|}\describe
This is the output \para{char_end_sec} command string.
\end{describepara}
\begin{describepara}{out_char_new_cmd}{\verb|N|}\describe
This is the output \para{native_part} command string.
\end{describepara}
\begin{describepara}{out_ml_sec}{\verb|M|}\describe
This is the output \para{begin_ml_section} command string.
\end{describepara}
\begin{describepara}{char_hol_prompt}{\verb|#|}\describe
This string identifies the start of a \HOL\ command.
\end{describepara}
\begin{describepara}{char_end_hol_in}{\verb|endhol|}\describe
This string identifies the end of a \HOL\ command.
\end{describepara}
\begin{describepara}{char_hol_begin_comment}{\verb|%|}\describe
This string identifies the start of a \HOL\ comment.
\end{describepara}
\begin{describepara}{char_hol_end_comment}{\verb|%|}\describe
This string identifies the end of a \HOL\ comment.
\end{describepara}
\begin{describepara}{char_end_hide}{\verb|end{hh}|}\describe
This parameter associates with the \cmd{end_hide_hol_section}
\end{describepara}
\begin{describepara}{char_hide_indicator}{\verb|$|}\describe
This parameter marks a section as containing hidden hol commands.  
\end{describepara}
}
\vspace*{12pt}

\subsection{Using {\tt mmerge} to prepare the tag file}

Once the ``master'' file has been created, an ML source file can be
extracted using the command {\tt mtangle <filename>}.  Given the
presence of a master file named ``{\tt <filename>.m}'', this will
generate an ML file named ``{\tt <filename>.ml}''.  A log of executing
the source file should be saved in a file named {\tt <filename>.log}.
This file can be combined with the original ML file to generate the
``tag'' file, which will contain segments of (slightly modified)
interactive sessions of \HOL\ to be reinserted into a \LaTeX\ document.

The \merge\ utility is independent of the \mweb\ system.  Its purpose is
to create the ``tag'' files.  In the presence of files {\tt
<filename>.log} and {\tt <filename>.ml} the proper usage is given as follows.
\begin{center}
\verb+mmerge [-p prompt] [-hH] <filename>+
\end{center}
The {\tt p} option permits recognizing other than the default \HOL\
prompt \verb+#+.  The {\tt h} and {\tt H} options cause a usage help
message to be printed to standard output.  A file labelled {\tt
<filename>.tag} will be generated if the program successfully
executes.

The source and log files are merged together such that:
\begin{itemize}
\item Imbedded comments are passed to the output untouched, with no
preceding prompt.
\item Every visible command is passed to the output file, preceded by a single
prompt character \verb+#+.  Hidden commands are passed to the output
file, but without a prompt.
\item The system response to visible commands is passed to the output.
\item Imbedded comments placed by the \tangle\ program identify the
extent of ``sections'', which were delineated by the commands
\verb|\beginhol|, \verb|\endhol| and \verb|\begin{hh}|,
\verb|\end{hh}| in the original file.
The corresponding tag sections
are delineated by commands of the form
\verb|@t{<filename>_<seccount>}| and \verb|@e| for visible sections,
and \verb|@t{$<filename>_<seccount>}| and \verb|@e| for hidden sections,
\end{itemize}

It has been the experience of the author that some manual editing of
the tag files may be necessary to cope with unusual demands on the
program.  Examples include eliminating the listing of a long series of
proved subgoals when a goal is finally solved (an imbedded comment
that this is being done serves both to inform the reader, and remind
the person processing the text), splitting a section
into two if the original \LaTeX\ file had split the system feedback
so (the empty section serves as a useful reminder in this case), and
reformatting the output to limit line lengths if the \HOL\ pretty
printer was not configured appropriately.  A list of all such manual
changes should be kept to ease repeating of the process without introducing
random errors.


\subsection{Using {\tt mweave} to put it all together}

For the most part the standard \mweb\ utilities when applied to the
output of the \winnow\ program are run as described in the main
documentation.  The only exception to this is the \weave\ utility,
which requires the setting of particular parameters.  Filter mode must
be turned on by using the command option \verb+-f+.  In addition, the
following parameters must be set as shown, typically in an include file.

\begin{verbatim}
@D"doc_only_mode"="true"
@D"char_begin_sec"="$"
@D"char_anon_sec"=" "
@D"mac_before_tag"=""
@D"mac_after_tag"=""
@D"mac_end_sec"=""
\end{verbatim}

%\end{document}
