% Document Type: LaTeX
\section{A tutorial introduction}

\subsection{What is {\tt mweb}?}

\mweb\ is a suite of utility programs for documenting and managing \HOL\ proof
scripts. It is inspired by the idea of {\em literate
programming\/} of D.~Knuth\cite{Knuth:CJ-27-2-97}. The core of this idea
is to treat programs as {\it works of literature\/}. When programming
in this method, one creates a master file containing both the document
and the program text. This master file is processed by utility
programs to generate a program file and a document file separately. The
program file is then processed by a compiler or interpreter whilst the
document file is processed by a text formatter to produce a human
readable document. The programs developed by Knuth which embodied this
idea are known as the {\tt WEB} system. It has been used to develop one
of the most popular text formatters \TeX.\footnote{The whole program of
\TeX\ has been published as a book \cite{Knuth:ct-b} which
contains the program text and detail documentation in one volume.} The
main advantage of keeping both the program 
source and the document in a single file is that it is much easier to
keep them both up-to-date in respect to each other and to produce a more
accurate and significantly better document.

\mweb\ is a suite of such utility programs principally for developing
proof scripts in \HOL, and essentially for programming in ML.  It is
simple to use. It has a number of special features, such as multiple
languages, including tag files, special operation modes and
user-definable parameter values. Probably the most important feature
of \mweb\ is its highly programmable input syntax and output
format. This allows \mweb\ to be easily customised to work with
different specification languages, programming languages and text
formatters, ant to suit different applications, requirements and
individual styles. Section~\ref{sec-winnow} describes such an
extension to the basic \mweb\ suite.

The basic \mweb\ utility consists of two programs: \tangle\ and \weave.
\tangle\ processes the master file to generate an ML file which can
then be loaded into \HOL\ to perform the proof. \weave\ processes the
master file to generate a \LaTeX\ file which can then be formatted
using \LaTeX\ to produce a document. \weave\ also accepts extra files,
known as the include files or {\it tag\/} files. These files usually
contains the results from running the script through \HOL. This
process is illustrated in Figure~\ref{fig-web-proc}.
\begin{figure}
\begin{center}
\setlength{\unitlength}{0.012500in}%
\begin{picture}(320,520)(40,280)
\thicklines
\put(320,380){\oval(80,40)}
\put(320,540){\oval(80,40)}
\put(140,700){\oval(80,40)}
\put(140,540){\oval(80,40)}
\put( 40,440){\framebox(80,40){}}
\put(160,440){\framebox(80,40){}}
\put(100,600){\framebox(80,40){}}
\put(200,760){\framebox(80,40){}}
\put(280,440){\framebox(80,40){}}
\put(280,280){\framebox(80,40){}}
\put(140,680){\vector( 0,-1){ 40}}
\put(140,600){\vector( 0,-1){ 40}}
\put(320,520){\vector( 0,-1){ 40}}
\put(320,440){\vector( 0,-1){ 40}}
\put(320,360){\vector( 0,-1){ 40}}
\put(240,760){\vector(-2,-1){ 80}}
\put(240,760){\line( 2,-1){ 80}}
\put(320,720){\vector( 0,-1){160}}
\put(140,520){\vector(-3,-2){ 60}}
\put(140,520){\vector( 3,-2){ 60}}
\put(200,440){\line( 0,-1){ 30}}
\put(200,410){\line( 1, 0){ 55}}
\put(255,410){\line( 0, 1){130}}
\put(255,540){\vector( 1, 0){ 25}}
\put(140,615){\makebox(0,0)[b]{\raisebox{0pt}[0pt][0pt]{ML file}}}
\put(240,775){\makebox(0,0)[b]{\raisebox{0pt}[0pt][0pt]{MASTER}}}
\put(140,695){\makebox(0,0)[b]{\raisebox{0pt}[0pt][0pt]{\tangle}}}
\put(140,535){\makebox(0,0)[b]{\raisebox{0pt}[0pt][0pt]{HOL}}}
\put( 80,455){\makebox(0,0)[b]{\raisebox{0pt}[0pt][0pt]{THEORY}}}
\put(200,455){\makebox(0,0)[b]{\raisebox{0pt}[0pt][0pt]{TAG FILE}}}
\put(320,455){\makebox(0,0)[b]{\raisebox{0pt}[0pt][0pt]{\LaTeX\ file}}}
\put(320,535){\makebox(0,0)[b]{\raisebox{0pt}[0pt][0pt]{\weave}}}
\put(320,375){\makebox(0,0)[b]{\raisebox{0pt}[0pt][0pt]{\LaTeX}}}
\put(320,295){\makebox(0,0)[b]{\raisebox{0pt}[0pt][0pt]{DOCUMENT}}}
\end{picture}

\end{center}
\caption{Operation of \mweb\label{fig-web-proc}}
\end{figure}
Both utilities share a common core, thus parameters, command line
options, etc. described in Section~\ref{sec-cmd-opt}--\ref{sec-para}
apply to both. 

\subsection{How to use {\tt mweb}?}

In the following tutorial example, we assume that the text formatter
is \LaTeX\ and the proof script is in \HOL88, i.e., in classic ML. The
procedures can easily be generalised to use other formatter and
programming languages.
The basic procedures of developing and managing proof script using
\mweb\ are:
\begin{enumerate}
\item preparing a master file, say {\tt foo.m};
\item running \tangle\ to generate an ML file {\tt foo.ml};
\item loading {\tt foo.ml} into \HOL\ to perform the proof;
\item preparing a tag file using the result from \HOL\ if there is any
tag command in the master file;
\item running \weave\ to generate a \LaTeX\ file {\tt foo.tex};
\item running \LaTeX\ to produce a document.
\end{enumerate}
These procedures can be managed automatically using a {\tt Makefile}
with the utility {\tt make}. The remainder of this section describes a
simple example step by step. Suppose we have a master file called {\tt
lmem.m} which contains the following lines.

\begingroup
 \makeatletter
%redefine font used in verbatim environment
 \catcode`\`=\active
 \def\verbatim@font{\footnotesize\tt \catcode96\active
   \def`{\leavevmode\kern\z@\char96 }}
 \makeatother

\verbatimnumbered{../Examples/lmem.m}

This file creates a theory named {\tt lmem}. A
constant definition defines a \HOL\ constant \CONST*{LMEM}. A theorem
about this constant is proved and stored in the theory. In the master
file listed above, there are a number of \mweb\ commands which tell
the utility programs how to process the file. Commands start with a
\verb|@| character at the beginning of a line. This is followed by a
command character. A master file is divided into sections. For
example, the command \verb|@*| in line~9 begins 
a major section. In addition to the \verb|@*|
command, there are three other commands which begin a section. Two of
these has been used in this file. They are \verb*|@ | which begins an
ordinary section as in lines~18 and~28, and \verb|@M| which begins an
ML-only section as in lines~12, 38 and~47. The command \verb|@-|
terminates a section. There are six sections in this master file.

Each section consists of three parts: the document part, the
other-language part and the native-language part. All parts are
optional. The document part
contains text lines which describe the programs that appear in the other
two parts. These lines are simply copied to the \LaTeX\ file. Any
\LaTeX\ macros can appear in this part. For example, lines 19--21 are
the documentation for the constant definition in that section. No
other-language part has been used in this file, so its usage will be
explained in a later section. The native-language part contains \HOL\
text or ML programs. It starts with the command \verb|@N| as in
lines~13, 22, 32, 39 and~48. Lines in this part will be copied to the
ML file, and they may also appear in the \LaTeX\ file. Whether lines
in this part appear in the \LaTeX\ file and how they appear in the
final document can be controlled by the user using
various means. For simplicity, this master file has been processed in
a way that no program lines appear in the final document. Any
\HOL\ and ML commands can appear in the native-language part. 
The ML file, {\tt lmem.ml}, generated by \tangle\ from this master file
is listed below.
\newpage
\verbatimnumbered{../Examples/lmem.ml}

As can be seen in the above listing, the ML program  from each section in
the master file is preceded by a comment line. It contains the line
number, the section number which is automatically generated by
\tangle\ and the name of the section if it is present. This helps to
trace the source of problems if failure occurs when the file is loaded
into \HOL. However, one usually works directly with the master file
while developing a proof in an interactive \HOL\ session.

There are still a couple of commands in the master file that have not been
explained. First is the command \verb|@D| which appeared in lines~2
to~7. This is the parameter definition command. \mweb\ has been
developed in a way in which the actual input command and output
format can be programmed by the user. This is done by redefining the
values of a set of parameters using the \verb|@D| command. The precise
meanings of the parameters will be described in
Section~\ref{sec-para}. Another command appearing in lines~21 and 31 in
the master file is \verb|@t|, known as the tag command. The string
enclosed in a pair of braces following the tag command is the name of
the tag. This command instructs \weave\ to find a text block in a tag
file which has the same name, and to output this text block in place
of the tag command to the \LaTeX\ file. These text blocks usually
contain theorems generated by \HOL. In fact, any text and macros
understood by \LaTeX\ can appear in these blocks. In the second last
section of the master file, the function
\mlname{latex_all_theorems_to} in the {\tt latex-hol} library is
called to generate the tag file which is listed below.

\verbatimnumbered{../Examples/lmem.tag}

This file contains two tagged blocks. Each of them is a theorem in
\LaTeX\ format. They will be output to the \LaTeX\ file in place of
the tag commands in Line~21 and 31 when the following system command
is executed:
\begin{verbatim}
   mweave -Ilmem.tag lmem.m lmem.tex
\end{verbatim}
This command produces a \LaTeX\ file named {\tt lmem.tex}. In a simple
application which consists of only a single master, all \LaTeX\
preamble and post-amble materials can be put into the master file.
More often, applications consists of more than a single master file.
In this case, the preamble and post-amble materials can be put into a main
\LaTeX\ file which, in turn, includes all the files produced by \weave.
The result of formatting the file {\tt lmem.tex} is shown in the box below.

\noindent
\begin{center}
\begin{boxedminipage}{\textwidth}
\def\sect{\subsection}\def\subsect{\subsubsection}
\def\mlname{\begingroup\makeulother\dotoken{\tt}}
\def\CONST{\begingroup\makeulother\dotoken{\constfont}}
\def\KEYWD{\begingroup\makeulother\dotoken{\keyfont}}
\vspace*{12pt}
\sect{The theory {\tt lmem}}

This theory contains definition of the constant \CONST{LMEM} and some theorems.

\subsect{Definition of \CONST{LMEM}}

The constant \CONST{LMEM} is a predicate asserting the membership of an
element in a list.
\begin{equation}
\THM (\FORALL x\DOT \CONST{LMEM} \,x \,\NIL  = \CONST{F}) \AND 
     (\FORALL x \,h \,t\DOT \CONST{LMEM} \,x \,(\CONST{CONS} \,h \,t) =
                               (x = h) \OR  \CONST{LMEM} \,x \,t)
\end{equation}
\begin{comment} 
\pline{}{let\ LMEM\char"5F DEF\ =\ new\char"5F list\char"5F rec\char"5F definition(\char"60 LMEM\char"5F DEF\char"60 ,}
\pline{}{\ "(LMEM\ x\ []\ =\ F)\ /\char"5C }
\pline{}{\ \ (LMEM\ x\ (CONS\ (h:*)\ t)\ =\ (x\ =\ h)\ \char"5C /\ (LMEM\ x\ t))");;}
\end{comment}

\subsect{A theorem}

The theorem \mlname{NULL_NOT_LMEM} states that nothing is an element
of an empty list. 
\begin{equation}
\THM \FORALL x \,l\DOT \CONST{NULL} \,l \IMP  \NOT \CONST{LMEM} \,x \,l
\end{equation}
\begin{comment} 
\pline{}{let\ NULL\char"5F NOT\char"5F LMEM\ =\ prove\char"5F thm(\char"60 NULL\char"5F NOT\char"5F LMEM\char"60 ,}
\pline{}{\ \ \ \ "!(x:*)\ l.\ NULL\ l\ ==>\ \char"7E (LMEM\ x\ l)",}
\pline{}{\ \ \ \ GEN\char"5F TAC\ THEN\ LIST\char"5F INDUCT\char"5F TAC\ THEN\ REWRITE\char"5F TAC[NULL;LMEM\char"5F DEF]);;}
\end{comment}


\vspace*{12pt}
\end{boxedminipage}
\end{center}
Note that \weave\ assumes a set of \LaTeX\ macros for formatting the
document. A set of default definitions of macros is in the file {\tt
  mwecmac.tex} which should be included in the \LaTeX\ file explicitly.
\endgroup

\section{The command line}\label{sec-cmd-opt}

The command  given to the operating system to invoke the \mweb\
programs takes the following form:
\begin{quote}
{\it progname} [ {\it options} ] [ {\it infile\/} ] [ {\it outfile\/} ]
\end{quote}
where {\it progname\/} is the program name, e.g., {\tt mtangle} or
{\tt mweave}. {\it infile\/} is the input file name. If this name has
no suffix, i.e., the dot character \mbox{{\tt.}} does not
occur in the name, the current value of the parameter \para{insuffix}
is appended to {\it infile\/}. Similarly, {\it outfile\/} is the name
of the output file. If it has no suffix, the current value 
of the {\tt outsuffix} parameter of the program is appended. For
example, the {\tt outsuffix} parameter of \weave\ is
\para{outsuffixmweave}. The default input file suffix is {\tt .m}. The
default \tangle\ output file suffix is {\tt .ml}, and the default
\weave\ output file suffix is {\tt .tex}.

Both the input and output file names can be omitted.  In such a case,
the program reads from the standard input and writes to the standard
output. If only one file name is given, it is taken as the name of the
input file. The name of the output file is derived from this by
replacing (if it has suffix) or appending the suffix. If two file
names are given, no suffix checking is carried out. They are taken
literally as the input and output file names.

\index{command line options}
Options which may change the behaviour of the program can be specified
when invoking the \mweb\ programs. For example,  the command
\begin{verbatim}
   mweave -f filename
\end{verbatim}
will cause \weave\ to run in filter mode. All options are single
character and begin with a hyphen(-). Some of the options can take an
argument. All options are listed below.
\begin{description}
\item[{\tt -h -H}] --- help: print this message
\item[{\tt -v}] --- verbose: print processing information (default)
\item[{\tt -s}] --- silent: do not print processing information
\item[{\tt -L\ }{\it lang\/}] --- specify a language for \tangle
\item[{\tt -I\ }{\it file\/}] --- specify an include file
\item[{\tt -D\ }{\it para\/}{\tt=}{\it value\/}] --- set the parameter {\it
para\/} to {\it value\/}
\item[{\tt -m}] --- activate ML only mode
\item[{\tt -f}] --- activate filter mode
\item[{\tt -d}] --- activate document only mode
\item[{\tt -r}] --- activate raw text mode
\item[{\tt -p}] --- deactivate parallel mode
\item[{\tt -l}] --- activate line mode
\end{description}
\index{command line options!-h@{\tt -h}}
\index{command line options!-H@{\tt -H}}
\index{command line options!-v@{\tt -v}}
\index{command line options!-s@{\tt -s}}
\index{command line options!-I@{\tt -L}}
\index{command line options!-I@{\tt -I}}
\index{command line options!-D@{\tt -D}}
\index{command line options!-m@{\tt -m}}
\index{command line options!-f@{\tt -f}}
\index{command line options!-d@{\tt -d}}
\index{command line options!-r@{\tt -r}}
\index{command line options!-p@{\tt -p}}
\index{command line options!-p@{\tt -l}}
The exact behaviour of the different operation modes will be described
in Section~\ref{sec-modes}

\section{Operation of \mweb}

Since \mweb\ is highly programmable, the actual input commands and
output strings can be changed very easily. In the example described in
the previous section, the concrete strings of the default settings
have been shown. In the following
description, generic command names and parameter names will be used
instead of the concrete strings. For instance, the default command for
starting a section is the string \verb*|@ |, instead the command
\cmd{begin_section} will be used in the following description.
Command names will be typeset in a {\sf sans serif} font, and
parameter names will typeset in a {\tt typewriter} font.

\subsection{Master file format}

\begin{figure}
\begin{center}
\begin{boxedminipage}{5in}
\it
\begin{quote}
 \vspace*{12pt}
\hspace*{1.5em} ... \\
\cmd{begin_section} optional section name \\
\hspace*{1.5em} ... \\
\hspace*{1.5em} document part \\
\hspace*{1.5em} ... \\
\cmd{other_part} \\
\hspace*{1.5em} ... \\
\hspace*{1.5em} other-language part \\
\hspace*{1.5em} ... \\
\cmd{native_part} \\
\hspace*{1.5em} ... \\
\hspace*{1.5em} native-language part \\
\hspace*{1.5em} ... \\
\cmd{end_section}
 \\
\hspace*{1.5em} ... more sections\\
 \\
\end{quote}
\end{boxedminipage}
\end{center}
\caption{Master file format\label{fig-input-format}}
\end{figure}
The master file is divided into a number of {\it sections\/}. Each
section is divided into three parts: the {\it document} part, the {\it
other-language} part and the {\it native-language} part. They should
appear in the order as illustrated in Figure~\ref{fig-input-format}.
However, all the parts  
are optional. For instance, a section may contain only the document
part or the document part and the native-language part.

The document part contains text describing the \HOL\ proof script or
specification in the other
two parts. It will be formatted by the text formatter, i.e., \LaTeX. All text
formatter commands can appear in this part providing the first
character of any line is not a command escape character (by default \verb|@|).
The native-language part contains program text which will be output to
a file which will eventually be executed. \HOL\ or ML is the default
native language. The purpose of the
other-language part is to allow specifications written in other
languages to be included in the same document so that it can be
compared to the native language program. For instance, a circuit
specification in a 
hardware description language can be included in the
other-language part. The translation of this specification into \HOL\ is
written in the native-language part. The document part describes the
circuit and any differences in syntax or semantics in the two.
These two specifications will appear in the final
document side-by-side. A human reader will be able to compare them
very easily.

A section starts with one of the four section begin commands:
\cmd{begin_section}, \cmd{begin_star_section},
\cmd{begin_ml_section} and \cmd{begin_anon_section}. The other-language
part starts with the \cmd{other_part} command, and similarly the
native-language part starts with the \cmd{native_part} command.
The command \cmd{end_section} terminates a section. 

The intention of providing four different kinds of sections is to
allow them to be processed in different ways. The sections started by
the commands \cmd{begin_section} and \cmd{begin_star_section} are
ordinary sections. The difference between them is that they are output
into the document as two levels of sectioning commands. The purpose of
the ML sections, i.e., started by the command \cmd{begin_ml_section},
is to hide the contents of the whole section.
For example, some libraries have to be loaded to set up the operation
environment in \HOL\ before the real work begins. The ML code for this
can be put in an ML section, and it will not appear in the final
document. The anonymous section stops the section heading from being output
to the document file. Thus, one can divided a logical section that
will appear in the final document into several smaller sections. All
these sections except the first one become anonymous sections. No
extra section headings will be printed. 


\subsection{Command format}\label{sec-command-format}

All commands are on separate lines called {\it command lines\/}.
A command can be divided into three parts. The first is the command
escape character which must be the first character 
of a command line. It marks the beginning of a command. The command
escape character is usually a single character, but it can be
programmed to be a multi-character string.

The second part is the command string. It must follow the command
escape character immediately without leaving any intervening spaces.
The actual string representing each command is the current value of
the parameter associated with that command. For instance, the
parameter associated with the \cmd{begin_section} command is
\para{char_begin_sec} whose default value is a single space character
\verb*| |. In fact, the default strings of all commands are single
characters.

The last part of a command is the argument. Some commands have
mandatory arguments, such as the \cmd{incl_file} command which requires
an argument to specify the name of the include file. Other commands
have optional arguments, such as the \cmd{begin_section} command whose
argument is the name of the section. If the argument is absent, the
section will have no name. Only the section number will appear in the
document. Still, other commands require no argument at all, such as
the \cmd{native_part} command.

The command argument, if it is present must follow immediately the
command string without leaving any intervening spaces. It must be
enclosed by a pair of braces(\verb|{}|). In fact, the braces are just
the default values of the parameters \para{char_begin_para} and
\para{char_end_para}. They can be changed to other characters, but they
must be a single character. Using the default settings, a concrete
example of a command starting a 
section whose name is `First section' is
\begin{verbatim}
@ {First section}
\end{verbatim}

\subsection{Operation modes}\label{sec-modes}

In ordinary operation, text before the first section, after the last
section and in between sections is ignored by the programs. Only the
command lines and text lines enclosed by  the \cmd{begin_section} and
\cmd{end_section} commands are significant.

The \tangle\ program  copies to the output file all the text lines
in the native-language part. For the purpose of debugging, a comment
line in the native language precedes the text of each part in the
output file. It contains the line number and the section number. If a name
is given to a section, it also appears in this comment line. The
comment line starts with the current value of the parameter
\para{ml_comm_begin} and ends with the current value of the parameter
\para{ml_comm_end}. The default strings of these parameters are
\verb|%-| and \verb|-%|, respectively. Figure~\ref{fig-tangleio} shows
the general form of input and the corresponding output written by \tangle.
\begin{figure}
\begin{center}\it
\begin{tabular}{|l|c|l|}
\multicolumn{1}{c}{\rm input } & \multicolumn{1}{c}{} &
 \multicolumn{1}{c}{\rm \tangle\ output}\\
 \cline{1-1}\cline{3-3}
 & & \\
\cmd{begin_section} section name & \hspace*{1em} &
                       \para{ml_comm_begin} line No. section No.\\
  document lines   & & \hspace*{2em} section name \para{ml_comm_end}\\
 \cmd{other_part}  & & \\
 otherline 1       & & \\
 otherline 2       & & \\
  \ldots           & & \\
 \cmd{native_part} & & \\
 nativeline 1      & & nativeline 1 \\
 nativeline 2      & & nativeline 2 \\
  \ldots           & &  \ldots \\
                   & & \\
 \cmd{end_section} & & \\
 & & \\
\cline{1-1}\cline{3-3}
\end{tabular}
\end{center}
\caption{Input format and corresponding output from \tangle\label{fig-tangleio}}
\end{figure}

The \weave\ program copies to the output file all the text lines
in the document part. If a \cmd{begin_tag} command occurs in this
part, instead of copying the command, \weave\ will output the text
block whose name is the same as the name specified in the command.
This block of text is preceded by the string which is the current
value of the parameter \para{mac_before_tag} and terminated by the
string which is the current value of the parameter \para{mac_after_tag}.
Tagged text blocks are read by \weave\ from the include files, and
saved in an internal buffer. If such a text block cannot be found,
\weave\ issues a warning message and carries on.

By default, \weave\ outputs to the output file all the 
lines in the language parts in a side-by-side formatting environment.
The default environment is called {\tt paralines}. When entering an
other-language part, \weave\ saves all lines in an internal buffer
until it finds the native-language part. It then saves all lines in
the native-language part in
another internal buffer until the end of the section. When it sees the
\cmd{end_section} command, it processes the two lists of saved lines
and outputs the `begin environment' string which is the current value of
the parameter \para{mac_begin_env}. It then outputs the corresponding
lines from the two parts, i.e. Line~1 of both parts and then Line~2 of
both parts, etc., as the arguments to a line macro which is the
current value of the parameter \para{mac_line}. This is terminated by
the `end environment' string. For instance, the
default line macro is the string \verb|\pline|. The output of a
section looks like
\begin{verbatim}
\begin{paralines}
\pline{otherline 1}{nativeline 1}
\pline{otherline 2}{nativeline 2}
...
\pline{otherline m}{nativeline m}
\pline{otherline m+1}{}
...
\pline{otherline n}{}
\end{paralines}
\end{verbatim}
If any part is missing or has fewer lines, the corresponding
argument to the line macro is simply empty. The braces enclosing the
lines are also programmable. They are the current value of the parameters
\para{mac_begin_arg} and \para{mac_end_arg}.
Figure~\ref{fig-weaveio} shows the general form of 
input and the corresponding output from \weave.
\begin{figure}
\begin{center}\it
\begin{tabular}{|l|c|l|}
\multicolumn{1}{c}{\rm input } & \multicolumn{1}{c}{} &
 \multicolumn{1}{c}{\rm \weave\ output}\\
 \cline{1-1}\cline{3-3}
 & & \\
\cmd{begin_section} section name & \hspace*{1em} &
 \para{mac_begin_sec} section name\hspace*{2em} \\
                   & & \para{mac_begin_doc} \\
   document lines   & & document lines  \\
 \cmd{begin_tag}   & & \para{mac_before_tag}\\
                   & & tagged text block \\
                   & & \para{mac_after_tag} \\
  more lines       & & more lines \\
                   & & \para{mac_end_doc} \\
 \cmd{other_part}  & & \para{mac_begin_env}\ \para{mac_changed}  \\
 otherline 1       & & \para{mac_line} \\
 otherline 2       & & \para{mac_begin_arg} otherline 1 \para{mac_end_arg}\\
  \ldots           & & \para{mac_begin_arg} nativeline 1 \para{mac_end_arg}\\
 \cmd{native_part} & & \\
 nativeline 1      & & \para{mac_line} \\
 nativeline 2      & & \para{mac_begin_arg} otherline 2 \para{mac_end_arg}\\
  \ldots           & & \para{mac_begin_arg} nativeline 2 \para{mac_end_arg}\\
  \ldots           & &  \ldots \\
                   & & \para{mac_end_env} \\
 \cmd{end_section} & & \para{mac_end_sec} \\
 & & \\
\cline{1-1}\cline{3-3}
\end{tabular}
\end{center}
\caption{Input format and corresponding output from \weave\label{fig-weaveio}}
\end{figure}

Text in the default {\tt paralines} environment will be printed in two
columns. Each column contains text from a language part. Different
fonts can be used to typeset them to provide visual distinction between
the two languages. 
A \LaTeX\ macro file, named {\tt mwebmac.tex}, containing the definition of 
this parallel environment is distributed with the package. This should
be explicitly input into the \LaTeX\ file.\index{LATEX file@\LaTeX\ file}

\index{operation modes}
There are a number of special operation modes which can be activated
by the user to perform special tasks. By default, all modes are in
their inactive states except parallel mode which is active. The
operation modes are: 
\begin{description}
\item[{\it document only mode}]--- \index{operation modes!document only}
If this mode is activated, \weave\ will only output the text in the
document part to the \LaTeX\ file. Thus, one can omit the program text
from the document. 
\item[{\it filter mode}]--- \index{operation modes!filter}
If this mode is activated, \weave\ will copy all lines outside any
sections to the output file.
\item[{\it ML only mode}]--- \index{operation modes!ML only}
If this mode is activated, \weave\ will ignore all sections which
start with the command \cmd{begin_ml_section}. They will not appear in
the final document at all. However, \tangle\ is
not affected by this mode, so lines in the native language part of
these sections will still be output to the ML file.
\item[{\it raw text mode}]--- \index{operation modes!raw text}
If this mode is activated, the program lines in both the other-language
part and the native-language part will be copied to the output file as
they are. Otherwise, the program lines are processed. Special characters
are converted to their ASCII code in hexadecimal notation and prefixed with the
string which is the current value of the parameter
\para{mac_spec_char}. Special characters are defined as one of the
characters in the string which is the current value of the parameter
\para{spec_chars}. For example, with the default setting, the
character \verb|#| is a special character which is converted to
\verb|\char"23| since its ASCII code is 23 in hexadecimal.
\item[{\it parallel mode}]--- \index{operation modes!parallel}
This is the default operation mode of \weave. If it is deactivated,
the programs in the other-language part and in the native-language
part will output as two separate blocks of text. The lines from the
other-language part will be output first, and they are followed by the
lines from the native-language part.
The lines from the other-language part are enclosed by the
strings which are the current values of the parameters
\para{mac_begin_other} and \para{mac_end_other}. Similarly, the lines
from the native-language part are enclosed by the
strings which are the current values of the parameters
\para{mac_begin_native} and \para{mac_end_native}.
\item[{\it line mode}]---\index{operation modes!line}
This operation mode affects \tangle\ only. When this mode is active,
the program lines in the output file will have the same line
number as in the master file. The lines of document part and other
language part will become blank lines in the output file. This is
useful in debugging since many compilers will report errors with a line
number. By keeping the same line number in the program file, one can
easily find and fix the errors in the master file.
\end{description}
All these modes can be activated either by using command line options
or by changing the value of their associated parameter. If
using command line options, the modes are active from the beginning of
the file. If changing the parameter values, filter mode will become
active from the next input line. The other modes will become active
from the next section. All modes can be deactivated by changing the
parameter value. The new value takes effect in the same way as
described above.

\subsection{Multiple language parts}

Although two different languages are used in many applications in
formal verification, sometimes more than two languages or more than
two specifications in the same language are needed. To handle such
cases, the language parts can be generalised to allow any number of
different language parts in a section. A new command
\cmd{language_part} can be used instead of the \cmd{other_part} and
\cmd{native_part} commands. This command takes the name of the
language as an argument. In addition to the standard command argument
syntax described in Section~\ref{sec-command-format}, a shorthand
format is also allowed in this command. This takes the form of a
string immediately following the command character with no braces. It
is terminated by the first blank character. For example, given the
default command character of \cmd{language_part} is {\tt L}, the following
two command lines have the same meaning which begins the language part
whose name is {\tt top-level}:
\begin{verbatim}
@L {top-level}
@Ltop-level
\end{verbatim}

A command line option for \tangle\ allows the user to specify which
language to output to the program file.  The default action of
\tangle\ is to output the lines in the language part whose name is the
same as that specified at the command line option.  If no language
option is specified, the \cmd{native-part} is output.  Note that no
checking is performed on the language names so be sure to spell them
precisely in every section and in the command line option, otherwise,
lines in the misspelled part will be missing from the output file.

The lines in the language parts are output to the \LaTeX\ file by
\weave\ if the {\it document only\/} mode is not active. Each language
part is enclosed in an environment whose name is the name of the
language as illustrated below:
\begin{center}\it
\begin{boxedminipage}{5in}
\begin{quote}
\para{mac_begin_lang} \para{mac_begin_arg} $\langle name\rangle$ \para{mac_end_arg}\\
lines in the language part whose name is $\langle name\rangle$\\
\para{mac_end_lang} \para{mac_begin_arg} $\langle name\rangle$ \para{mac_end_arg}
\end{quote}
\end{boxedminipage}
\end{center}

Within a section, both the generalized language parts and the other-
and native-parts can be occur. The generalised language parts may be
place either before or after the other- and native parts. However, the
use of only generalised language parts are recommended if side-by-side
printing of two language parts is not required. Note also that
\tangle\ is unable to output lines in the other-language part.

\section{Commands}

The utility programs have been implemented in a way that the actual command
strings and the exact output strings can be customised by changing the
values of a set of parameters by the user. The exact meanings of the
commands and their associate parameters are described in this section.
\vspace{20pt}

\noindent\cmdrule

\begin{describecmd}{begin_anon_section}{char_anon_sec}{\opt}
{\verb|@A|}
\describe
This command begins an anonymous section. 
The actions for this command is the same as the \cmd{begin_section}
command except that \weave\ does not output the section macro and the
section name to the \LaTeX\ file.
\end{describecmd}

\begin{describecmd}{begin_ml_section}{char_ml_sec}{\opt}
{\verb|@M|}
\describe
This command begins an `ML only' section. 
\tangleaction: save the section name if it is present.
\weaveaction: same as \cmd{begin_section} command if the value of
the parameter \para{ml_only} is false, otherwise, nothing is generated
in the output file for the whole section.
\end{describecmd}

\begin{describecmd}{begin_section}{char_begin_sec}{\opt}
{\verb*|@ |}
\describe
This command begins an ordinary section.
\tangleaction: save the section name if it is present.
\weaveaction: output  section macro \para{mac_sec} and the section
name as the argument to the macro if a name is present, otherwise,
output a default section macro \para{mac_sec_default}. 
Then, copy all input lines to the output until one of the following
commands is found: \cmd{end_sec}, \cmd{other_part}, \cmd{native_part}.
\end{describecmd}

\begin{describecmd}{begin_star_section}{char_begin_star_sec}{\man}
{\verb|@*|}
\describe
This command begins a major section.
\tangleaction: save the section name.
\weaveaction: output major section macro \para{mac_star_sec} and the section
name as the argument to the macro. Then, copy all input lines to the
output until one of the following 
commands is found: \cmd{end_sec}, \cmd{other_part}, \cmd{native_part}.
\end{describecmd}

\begin{describecmd}{begin_tag}{char_begin_tag}{\man}
{\verb|@t|}
\describe
In a master file, this command marks the position a tagged text
block should be placed. The argument to this command is the name of
the block. There must be a block having the same name appear in a tag
file. This file is included by \weave\ when processing the master file.
In a tag file, this command begins a text block. All input lines are
saved until an \cmd{end_tag} command is found. 
\tangleaction: do nothing.
\weaveaction:  output the text block whose name is the
same as the argument. If no block of the same name can be found, a
warning message is printed. The block is enclosed by the parameters
\para{mac_before_tag} and \para{mac_after_tag}.
\end{describecmd}

\begin{describecmd}{end_section}{char_end_sec}{\none}
{\verb|@-|}
\describe
This command terminates a section.
\tangleaction: do nothing.
\weaveaction: output the section. The text line saved in the
internal buffers (if any) is processed and output in an environment
enclosed by the parameters \para{mac_begin_env} and \para{mac_end_env}.
\end{describecmd}

\begin{describecmd}{end_tag}{char_end_tag}{\opt}
{\verb|@e|}
\describe
This command marks the end of a tagged text block in a tag file.
If the optional argument is present it should be the same as the
argument to the associated \cmd{begin_block} command.
\end{describecmd}

\begin{describecmd}{incl_file}{char_incl_file}{\man}
{\verb|@I|}
\describe
This command inputs an include file. The argument specifies the name
of a file to be read. It can only appear outside any section. The file
is expected to contain only parameter definitions and tagged text
blocks. 
\end{describecmd}

\begin{describecmd}{language_part}{char_lang}{\man}
{\verb|@L|}
\describe
This command begins a programming language part. It can only occur
after one of the `begin section' commands. It takes the name of the
language as its argument. Two formats are allowed for this argument.
The first is a string enclosed in a pair of braces which are the
values of the parameters \para{char_begin_para} and
\para{char_end_para} or a pair of quotes which are the value of the
parameter \para{char_str_quote}. Blank space between the command and
its argument in this form is optional. The second format is a string
immediately after the command. It is ended by the first blank
character.
\tangleaction: if the language name is equal to the language specified
in the command line option, output a comment line, then output
subsequent input lines until an \cmd{end_sec}, \cmd{other_part},
\cmd{native_part} or another \cmd{language_part} command is found.
\weaveaction: output the subsequent input lines in an environment
whose name is the language name.
\end{describecmd}

\begin{describecmd}{native_part}{char_new}{\none}
{\verb|@N|}
\describe
This command begins the native-language part. It can only occur after
one of the `begin section' commands. It must follow an
\cmd{other_part} command if one is present in the same section.
\tangleaction: output a comment line, then output subsequent input
lines until an \cmd{end_sec} command is found.
\weaveaction: save the subsequent input lines in an internal buffer
 until an \cmd{end_sec} is encountered.
\end{describecmd}

\begin{describecmd}{other_part}{char_old}{\none}
{\verb|@O|}
\describe
This command begins the other-language part. It can only occur after one
of the `begin section' commands.
\tangleaction: do nothing.
\weaveaction: save the subsequent input lines in an internal buffer
 until a command line is encountered.
\end{describecmd}

\begin{describecmd}{para_def}{char_para_def}{\man}
{\verb|@D|}
\describe
The command defines the value of a parameter.
It can only appear outside any section. The parameter value defines by
this command will take effect from the next section.
The argument to this command should be an equation. The left hand side
is a string representing the name of a parameter. The right hand side
is also a string representing the value of the parameter. If the
parameter takes only a boolean value, a null string, i.e., \verb|""| is
interpreted as false.
\end{describecmd}


\section{Parameters}\label{sec-para}

This section describes all parameters understood by \mweb. Their
values are all strings. A few of them must be single character string
which are indicated in the description. Some of the strings are
interpreted as boolean value. A null ({\tt""}) string means false, and
any other string means true. 

The value of a parameter can be redefined by the user using the
\cmd{para_def} command or the command line option {\tt -D}. The syntax
of parameter definition is
\begin{quote}\it
name {\tt =} value
\end{quote}
Both $name$ and $value$ are strings. A string is a sequence of
non-blank characters optionally enclosed by a pair of characters which
is the value of the parameter \para{char_str_quote} or by a left
bracket which is the value of \para{char_begin_para} and a right
bracket which is the value of \para{char_end_para}. The default
characters are  double quote({\tt "}), left brace({\tt\{}) and
right brace({\tt\}}), respectively. For example, \verb|{String}| and
\verb|"String"| are legal strings. A null string is a string of zero
character. This can be written as either \verb|""| or \verb|{}|.

Spaces around the equal sign are optional. The
string on the left hand side of the equation should be the name of one
of the parameters listed below. The string on the right hand side
denotes the new value of this parameter.

Within a string, all characters are literal except the escape
character. This is the current value of the parameter
\para{char_str_esc}. The default string escape character is
backslash(\verb|\|). The interpretation of the string escape will be
explained in the entry of the parameter \para{char_sstr_esc}. The
default string of the parameters shown in the entries is in the form
one has to type on the right hand side of the parameter definition
equation. For example, the equation below defines the parameter
\para{mac_sec} to the \LaTeX\ macro \verb|\subsec|:
\begin{verbatim}
  "mac_sec" = "\\subsec"
\end{verbatim}
where the double backslash is interpreted as a single backslash by the
\mweb\ programs.

\noindent\cmdrule

\begin{describepara}{char_anon_sec}{\verb*|A|}\describe
This parameter associates with the \cmd{begin_anon_section} command
which begins an anonymous section. 
\end{describepara}
\begin{describepara}{char_begin_para}{\verb*|{|}\describe
This parameter begins the argument to a command. It must be a single
character string.
\end{describepara}
\begin{describepara}{char_begin_sec}{\verb*| |}\describe
This parameter associates with the \cmd{begin_section} command
which begins an ordinary section. 
\end{describepara}
\begin{describepara}{char_begin_star_sec}{\verb*|*|}\describe
This parameter associates with the \cmd{begin_star_section} command
which begins a major section. 
\end{describepara}
\begin{describepara}{char_begin_tag}{\verb*|t|}\describe
This parameter associates with the \cmd{begin_tag} command.
which begins a tagged text block in a tag file or marks the position
for replacing by a tagged text block in a master file.
\end{describepara}
\begin{describepara}{char_cmd}{\verb*|@|}\describe
This parameter is the command escape character.
\end{describepara}
\begin{describepara}{char_end_para}{\verb*|}|}\describe
This parameter terminates the argument to a command. It must be a single
character string.
\end{describepara}
\begin{describepara}{char_end_sec}{\verb*|-|}\describe
This parameter associates with the \cmd{end_section} command
which terminates a section. 
\end{describepara}
\begin{describepara}{char_end_tag}{\verb*|e|}\describe
This parameter associates with the \cmd{end_tag} command
which terminates a tagged text block in a tag file.
\end{describepara}
\begin{describepara}{char_incl_file}{\verb*|I|}\describe
This parameter associates with the \cmd{incl_file} command
which specifies the name of a tag file to be included.
\end{describepara}
\begin{describepara}{char_lang}{\verb*|L|}\describe
This parameter associates with the \cmd{language_part} command
which begins a language part within a section. 
\end{describepara}
\begin{describepara}{char_ml_sec}{\verb*|M|}\describe
This parameter associates with the \cmd{begin_ml_section} command
which begins an `ML only' section. Sections started with this command
will not appear in the document if the ML only mode is active.
\end{describepara}
\begin{describepara}{char_new}{\verb*|N|}\describe
This parameter associates with the \cmd{native_part} command
which begins the native-language part within a section. 
\end{describepara}
\begin{describepara}{char_old}{\verb*|O|}\describe
This parameter associates with the \cmd{other_part} command
which begins the other-language part within a section. 
\end{describepara}
\begin{describepara}{char_para_def}{\verb*|D|}\describe
This parameter associates with the \cmd{para_def} command
which defines the value of a parameter.
\end{describepara}
\begin{describepara}{char_str_esc}{\verb*|\\|}\describe
This parameter specifies the escape character in a string. Its value must
be a single character string. When defining the value of a parameter,
the string at the right hand side of the equation may contain this
escape character. All characters in the string are literal except this
escape character. It is interpreted together with the next character.
If the next character is also an escape character, it becomes a single
escape character. If the next character is an `{\tt n}', it is
interpreted as a `newline' character. If the next character is a `{\tt
t}', it becomes a TAB character (ASCII code 9). If the next character
is the current value of the parameter \para{char_str_quote}, it
becomes that character. If the next character is a digit
from 0 to 7, then the next two characters must be digits as well. This
three digit number is interpreted as an octal number specifying the
ASCII code of a character.
\end{describepara}
\begin{describepara}{char_str_quote}{\verb*|\"|}\describe
This parameter specifies the quote character for strings. Its value
must be a character string. This character marks the beginning and the
end of a string in parameter value definition.
\end{describepara}
\begin{describepara}{doc_only_mode}{null string}\describe
The value of this parameter is boolean. When it is true, the document
only mode is active.
\end{describepara}
\begin{describepara}{filter_mode}{null string}\describe
The value of this parameter is boolean. When it is true, the filter
mode is active.
\end{describepara}
\begin{describepara}{insuffix}{\verb*|.m|}\describe
The value of this parameter is the default suffix of the input file.
If the name of the input file given on the command line has no suffix,
this string is appended to it. The \mweb\ programs attempt to open a
file of this resulting name for inputting.
\end{describepara}
\begin{describepara}{line_mode}{null string}\describe
The value of this parameter is boolean with default value false.
It affects \tangle\ only. When this mode is active,
the program lines in the output file will have the same line
number as in the master file. The lines of the document part and other
language part will become blank lines in the output file. This is
useful in debugging since many compilers will report errors with line
munber. By keeping the same line number in the program file, one can
easily find and fix the errors in the master file.
\end{describepara}
\begin{describepara}{mac_after_tag}{\verb*|\\end{verbatim}\n|}\describe
The value of this parameter is output to the document file after the
tagged text block.
\end{describepara}
\begin{describepara}{mac_before_tag}{\verb*|\\begin{verbatim}\n|}\describe
The value of this parameter is output to the document file before the
tagged text block.
\end{describepara}
\begin{describepara}{mac_begin_arg}{\verb*|{|}\describe
The value of this parameter is output to the document file before a line
from other-language part or native-language part.
\end{describepara}
\begin{describepara}{mac_begin_doc}{null string}\describe
The value of this parameter is output to the document file to mark
the beginning of the document part of every section.
\end{describepara}
\begin{describepara}{mac_begin_env}{\verb*|\\begin{paralines}|}\describe
The value of this parameter is output to the document file to
mark the beginning of the environment for program lines.
\end{describepara}
\begin{describepara}{mac_begin_lang}{\verb*|\\begin|}\describe
The value of this parameter is output to the document file before the
subsequent lines in a language part. It is followed immediately by the
name of the language enclosed in the delimiters which are the values of the
parameters \para{mac_begin_arg} and \para{mac_end_arg}.
\end{describepara}
\begin{describepara}{mac_begin_native}{\verb*|\\begin{native}\n|}\describe
The value of this parameter is output to the document file before the lines
from the native-language part if the operation mode is not parallel.
\end{describepara}
\begin{describepara}{mac_begin_other}{\verb*|\\begin{other}\n|}\describe
The value of this parameter is output to the document file before the lines
from the other-language part if the operation mode is not parallel.
\end{describepara}
\begin{describepara}{mac_changed}{\verb*|[c]\n|}\describe
The value of this parameter is output to the document file following
the string of the parameter \para{mac_begin_env} if the contents of the two
language parts are different.
\end{describepara}
\begin{describepara}{mac_end_arg}{\verb*|}|}\describe
The value of this parameter is output to the document file after a line
from other-language part or native-language part.
\end{describepara}
\begin{describepara}{mac_end_doc}{null string}\describe
The value of this parameter is output to the document file to mark
end of document part of every section.
\end{describepara}
\begin{describepara}{mac_end_env}{\verb*|\\end{paralines}\n|}\describe
The value of this parameter is output to the document file to
mark the end of the environment for program lines.
\end{describepara}
\begin{describepara}{mac_end_lang}{\verb*|\\end|}\describe
The value of this parameter is output to the document file after the
lines in a language part. It is followed immediately by the
name of the language enclosed in the delimiters which are the values of the
parameters \para{mac_begin_arg} and \para{mac_end_arg}.
\end{describepara}
\begin{describepara}{mac_end_native}{\verb*|\\end{native}\n|}\describe
The value of this parameter is output to the document file after the lines
from the native-language part if the operation mode is not parallel.
\end{describepara}
\begin{describepara}{mac_end_other}{\verb*|\\end{other}\n|}\describe
The value of this parameter is output to the document file after the lines
from the other-language part if the operation mode is not parallel.
\end{describepara}
\begin{describepara}{mac_end_sec}{\verb*|\n|}\describe
The value of this parameter is output to the document file to
mark the end of a section.
\end{describepara}
\begin{describepara}{mac_filler}{\verb|\n|}\describe
The value of this parameter is output to the document file following
the string of the parameter \para{mac_begin_env} if the contents of the two
language parts are the same.
\end{describepara}
\begin{describepara}{mac_line}{\verb*|\\pline|}\describe
The value of this parameter is output to the document file at the
beginning of every pair of program lines.
\end{describepara}
\begin{describepara}{mac_native_line}{\verb*|\n|}\describe
The value of this parameter is output to the document file at the
beginning of every line of the native-language part if parallel mode
is inactive.
\end{describepara}
\begin{describepara}{mac_other_line}{\verb*|\n|}\describe
The value of this parameter is output to the document file at the
beginning of every line of the other-language part if parallel mode
is inactive.
\end{describepara}\begin{describepara}{mac_sec}{\verb*|\\subsect|}\describe
The value of this parameter is output to the document file to start an
ordinary section, or ML only section if the ML only mode is inactive.
\end{describepara}
\begin{describepara}{mac_sec_default}{\verb*|\\subsect{}|}\describe
The value of this parameter is output to the document file to start an
ordinary section if there is no section name given.
\end{describepara}
\begin{describepara}{mac_spec_char}{\verb*|\\char\"|}\describe
The value of this parameter is prefixed to the ASCII code in
hexadecimal notation of each special character which occurs in the program
text in either other-language part or native-language part when raw
tex mode is not active.
\end{describepara}
\begin{describepara}{mac_star_sec}{\verb*|\\sect|}\describe
The value of this parameter is output to the document file to start a
major section.
\end{describepara}
\begin{describepara}{ml_comm_begin}{\verb*|%-|}\describe
The value of this parameter is output to the program file to mark the
beginning of a comment.
\end{describepara}
\begin{describepara}{ml_comm_end}{\verb*|-%|}\describe
The value of this parameter is output to the program file to mark the
end of a comment.
\end{describepara}
\begin{describepara}{ml_line_count}{\verb*|200|}\describe
The value of this parameter specifies the maximum number of program
lines in a section. It must be a string consisting of only decimal digits.
\end{describepara}
\begin{describepara}{ml_line_length}{\verb*|80|}\describe
The value of this parameter specifies the maximum number of characters
on a program line. It must be a string consisting of only decimal digits.
\end{describepara}
\begin{describepara}{ml_only}{null string}\describe
The value of this parameter is boolean. When it is true, the ML
only mode is active.
\end{describepara}
\begin{describepara}{outsuffixmtangle}{\verb*|.ml|}\describe
The value of this parameter is the default suffix of the output file
of \tangle. If only one file name is given on the command line,
this string is appended to it after any suffix has been stripped to
obtain a name for the output file.
\end{describepara}
\begin{describepara}{outsuffixmweave}{\verb*|.tex|}\describe
The value of this parameter is the default suffix of the output file
of \weave. If only one file name is given on the command line,
this string is appended to it after any suffix has been stripped to
obtain a name for the output file.\end{describepara}
\begin{describepara}{outsuffixwinnow}{\verb*|.m|}\describe
The value of this parameter is the default suffix of the output file
of {\tt winnow}. If only one file name is given on the command line,
this string is appended to it after any suffix has been stripped to
obtain a name for the output file.
\end{describepara}
\begin{describepara}{parallel_mode}{\verb*|true|}\describe
The value of this parameter is boolean. When it is false, the lines
from the other-language part and from the native-language part are
output to the \LaTeX\ file as two blocks of text with the other-language
block appearing first. The blocks are enclosed by the strings which
are the current values of the parameters \para{mac_begin_other},
\para{mac_end_other} and \para{mac_begin_native},
\para{mac_end_native}, respectively.
\end{describepara}
\begin{describepara}{spec_chars}{\verb*|`#$%&~_^{}\\|}\describe
Characters in this string are special characters. If the raw text mode
is inactive, every special character in either other-language part or
native-language part will be converted to their ASCII codes and
prefixed with the string value of the parameter \para{mac_spec_char}
when output to the document file.
\end{describepara}
\begin{describepara}{str_blank}{\verb*|\n\t |}\describe
This parameter string contains all characters which are interpreted as
blank character in a parameter definition.
\end{describepara}
\begin{describepara}{raw_text_mode}{null string}\describe
The value of this parameter is boolean. When it is false, the lines
from the other-language part and from the native-language part are
copied to the \LaTeX\ file without being processed. When it is true,
special characters appearing in these lines are converted into their
ASCII codes and prefixed with the string value of the parameter
\para{mac_spec_char}. The parameter string \para{spec_chars} defines
which character is special.
\end{describepara}
\begin{describepara}{tab_spaces}{\verb*|8|}\describe
A TAB character in  any language parts will be converted into a number
of space characters when the raw text mode is not active.
The value of this parameter specifies the exact number of space characters
to be output into the \LaTeX\ file.
It must be a string consisting of only decimal digits.
\end{describepara}
\begin{describepara}{tangle_lang}{null string}\describe
The value of this parameter is the name of the language \tangle\
is to output to the program file. It is set either by the command line
option {\tt -L} or by assigning a new value directly to this parameter.
\end{describepara}

\vspace*{14pt}

Note: a new parameter can be defined by simply using a name on the left
hand side of a parameter definition equation, but basic \mweb\
programs do not understand it. There is no way to tell the programs
how to use new parameters yet.


% Local Variables: 
% mode: latex
% TeX-master: t
% End: 
