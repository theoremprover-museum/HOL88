\documentstyle[11pt,a4,bezier]{article}
\parskip=8pt plus2pt minus2pt
\begin{document}

%%%%%%%%%%%%%%%%%%%%%%%%%%%%%%%%%%%%%%%%%%%%%%%%%%%%%%%%%%%%%%%%%%%%%%%%%%
%%                                                                      %%
%%   Set of LaTeX macros for typesetting scripts framed in a box.       %%
%%                                                                      %%
%%   Some of this code is taken from the TeX book p421.                 %%
%%                                                                      %%
%%----------------------------------------------------------------------%%
%%   DATE:    1.Mar.89                                                  %%
%%   AUTHOR:  ISD                                                       %%
%%----------------------------------------------------------------------%%
%%                                                                      %%
%%   The following macros provide two commands, namely:                 %%
%%                                                                      %%
%%       \begintt                                                       %%
%%                                                                      %%
%%       \endtt                                                         %%
%%                                                                      %%
%%   Anything between these two commands is typeset using the almost    %%
%%   verbatim environment.  There are three major diffeences;           %%
%%                                                                      %%
%%   1.  The back-slash "\" is retained as the escape character.        %%
%%       So now other funny characters can easily be included.  For     %%
%%       example the forall symbol can be included using \(\forall\)    %%
%%                                                                      %%
%%   2.  A box is drawn around the entire block of text with width      %%
%%       equal to the present \textwidth and height variable to         %%
%%       accomodate the typed text.                                     %%
%%                                                                      %%
%%   3.  There is no check to ensure that the text isn't too wide,      %%
%%       it merely runs off the right side of the page without any      %%
%%       errors or warnings.                                            %%
%%                                                                      %%
%%   Finally there are twoplaces which allow the macro to be            %%
%%   customised depending on which pointsize of documentstyle one is    %%
%%   using.  These places are clearly indicated with comments.          %%
%%   Comment out the appropriate lines depending on which               %%
%%   documentstyle you are using.                                       %%
%%%%%%%%%%%%%%%%%%%%%%%%%%%%%%%%%%%%%%%%%%%%%%%%%%%%%%%%%%%%%%%%%%%%%%%%%%

\chardef\other=12

\def\ttverbatim{\begingroup
                \catcode`\{=\other
                \catcode`\}=\other
                \catcode`\$=\other	% Matching $
                \catcode`\&=\other
                \catcode`\#=\other
                \catcode`\%=\other
                \catcode`\~=\other
                \catcode`\_=\other
                \catcode`\^=\other
                \catcode`\<=\other
                \obeyspaces
                \def\par{\leavevmode\endgraf}
                \obeylines
                \tt}

{\obeyspaces\gdef {\ }}

\newlength{\ttboxwidth}
\setlength{\ttboxwidth}{\textwidth}
\addtolength{\ttboxwidth}{-0.4pt}    % Width of two lines on either side
\addtolength{\ttboxwidth}{-0.8em}    % hspace inserted after the left line.

\outer\def\begintt{\begin{flushleft}
                   \begin{tabular}{@{}|@{\hspace{0.8em}}l@{}|@{}}
                   \hline
                   \begin{minipage}{\ttboxwidth}
%		   \begin{normalsize}	% for default documentstyle (10pt)
		   \begin{small}  	% for documentstyle[11pt, ... ]
%		   \begin{footnotesize}	% for documentstyle[12pt, ... ]
                   $$
                   %	Matching $$
                   \let\par=\endgraf
                   \ttverbatim
                   \parskip=0pt
                   \rightskip=-5pc
                   \ttfinish}

{\obeylines\gdef\ttfinish#1^^M#2\endtt{#1\vbox{#2}\endgroup$$
                   %	Matching $$
                   \vspace{0.5ex}
%		   \end{normalsize}	% for default documentstyle (10pt)
		   \end{small}		% for documentstyle[11pt, ... ]
%		   \end{footnotesize}	% for documentstyle[12pt, ... ]
                   \end{minipage}\\ \hline\end{tabular}
                   \end{flushleft}}}

%%%%%%%%%%%%%%%%%%%%%%%%%%%%%%%%%%%%%%%%%%%%%%%%%%%%%%%%%%%%%%%%%%%%%%%%%%
%%                           End of Macro...                            %%
%%%%%%%%%%%%%%%%%%%%%%%%%%%%%%%%%%%%%%%%%%%%%%%%%%%%%%%%%%%%%%%%%%%%%%%%%%


% newcommand definitions
% ======================
\newcommand{\ie}	{\mbox{\it i.e.}}
\newcommand{\inbox}[1]	{\begin{center}
			 \framebox{\parbox{0.984\textwidth}{#1}}
			 \end{center}}
\newcommand{\AND}	{\mbox{$\wedge$}}

% title, etc.
% ======================
{\LARGE
\begin{center} Dealing with the Choice Operator in 
HOL88\footnote{supported in part by Contract No.W2213-8-6362/01-SS
with the Department of National Defence}
\end{center}
}
\begin{center}
by\\
\ \ \\
Brian Graham\\
\ \ \\
Computer Science Department,
University of Calgary \\
Calgary, Alberta, Canada T2N 1N4 \\
Tel: (403) 220 7691 \\
Fax: (403) 284 4707 \\
Net: grahamb@cpsc.ucalgary.ca \\
\ \ \\
April 3, 1990 \\
\end{center}
\vfill
{\abstract This paper discusses the choice operator \verb+@+ in HOL88.
It gives an intuitive interpretation, as well as the definition, and
presents proof strategies for goals that contain this operator, broken
down into cases based on the number of values which satisfy the
predicate argument.  A working knowlege of the HOL system is assumed.}
\vfill
\vfill
\vfill
\vfill
\begin{center}
\copyright Brian Graham
\end{center}
\clearpage

% the paper ...
% ======================
My recent work on the proof of the SECD microprocessor
(\cite{Graham89c}, \cite{Simpson89c}, \cite{Birtwistle90a}) has
suffered from close encounters with the choice operator
(``\verb+@+'').  It is a powerful device for making definitions, but
we have quite limited capabilities of manipulating this constant.
This is unfortunate, as it shows up in many fundamental definitions,
including the definitions of {\it REP\/} functions created when new
types are defined.  It is also used in the definition of the temporal
abstraction function {\it TimeOf\/}\footnote{This function was defined
by Tom Melham in~\cite{Melham88a} and Inder Dhingra
in~\cite{Dhingra88b}.}, which is used when relating two granularities
of time in hardware representation.

I thought it useful to draw together all the axioms, rules,
conversions, and tactics concerned with it, and suggest some
techniques to handle it in (backward) proofs.

\section{What the HOL88 System Supplies}

The constant ``\verb+$@+'' is a binder, having the type
\verb+$@:(*->bool)->*+.  
In~\cite{Hol89b}, it is described as follows:
\begin{quotation}
\ldots if $t$ is a term having type $\sigma${\small\verb%->bool%}, 
then {\small\verb%@x.%}$t${\small\verb% x%} (or, equivalently,
{\small\verb%$@%}$t$) denotes {\it some\/} member of the set whose
characteristic
function is $t$. If the set is empty, then
{\small\verb%@x.%}$t${\small\verb% x%} denotes an arbitrary member of the
set denoted by $\sigma$. The constant {\small\verb%@%} is a higher order
version of Hilbert's
$\varepsilon$-operator; it is related to the constant
$\iota$ in Church's formulation of higher order logic. For more details,
see Church's original paper \cite{Church}, Leisenring's book 
on Hilbert's $\varepsilon$-symbol \cite{Leisenring}, or
Andrews' textbook on type theory \cite{Andrews}.
\end{quotation}

When applied to a predicate, it returns an
arbitrary value of the correct type that satisfies the predicate, and
in the case where no such value exists, it returns an arbitrary but
undetermined value of the correct type.  It can be used to make a
total function from a partial function.

The following listing includes all built-in tools in the HOL88 system
that are specific to the choice operator.

\subsection{Axioms}
\begintt
    SELECT_AX : thm                   (in theory "bool")
    
    |- !(P:* -> bool) (x:*). P x ==> P($@ P)
\endtt
\subsection{Forward Inference Rules}
\begintt
    SELECT_INTRO : (thm -> thm)

        A |- P t
      -----------------
       A |- P($@ P)


    SELECT_ELIM : (thm -> (term # thm -> thm))		(cases)

       A1 |- P($@ P)    ,    A2, "P v" |- t
      ------------------------------------------ (v occurs nowhere
                  A1 u A2 |- t			   except in "P v")

    SELECT_RULE : (thm -> thm)

        A |- ?x. t[x]
      -----------------
       A |- t[@x.t[x]]


    SELECT_EQ : (term -> thm -> thm)		(@ abstraction)

              A |- t1 = t2
         -----------------------
          A |- (@x.t1) = (@x.t2)
\endtt
\subsection{Conversions}
\begintt
    SELECT_CONV : conv
    
     "P [@x.P [x]]" ---> |- P [@x.P [x]] = ?x. P[x]
\endtt
\subsection{Tactics}
\begintt
    SELECT_TAC : (term -> tactic)           ( term = @x.P(x) )
	(found in start_groups.ml in the "group" library)    

        [A] P[@x.P(x)]
    ======================
         [A] ?x.P(x)
\endtt
\section{Manipulating the Choice Operator}

We distinguish three cases for the values that the operator may return:
\begin{enumerate}
\item the predicate holds for no values, so the operator returns an
   arbitrary but unspecified value of the correct type.
\item the predicate holds for a single unique value.
\item the predicate holds for more than one value.
\end{enumerate}
We deal with these cases individually in Sections~\ref{sec:none}~to~\ref{sec:many}.

\subsection{No values satisfy}\label{sec:none}

It is useful to observe the equivalence given by SELECT\_CONV.  The
SELECT'ed value satisfies the predicate if and only if some value
exists that satisfies the predicate.  From this we can prove the
following is false:
\begin{center}
\verb+((@x. Q x) = y)  ==> Q y+
\end{center}
(\ie{} if the value at which Q holds equals y then Q holds at y.)
By using AP\_TERM {\it Q\/} on the lhs of the implication, we get:
\begin{center}
\verb+(Q(@x. Q x) = Q y) ==> Q y+
\end{center}
Using SELECT\_CONV on the lhs gives:
\begin{center}
\verb+((?x. Q x) = Q y) ==> Q y+
\end{center}
If no value exists, then the equation simplifies to:
\begin{center}
\verb+(F = F) ==> F+
\end{center}
which is clearly false.  

In short this means that it is not possible to prove
anything about the specific value returned by the choice operator
{\it unless\/} some value exists at which the predicate holds.  This 
is relevent to the attempt to define an `arbitrary' value of the
form \verb+"@x.F"+. 
It is not possible to prove inequality of this term to any value of
the appropriate type, but all such arbitrary terms are equal.

\subsection{One value satisfies}\label{sec:one}

In the case where a predicate is satisfied by a unique value, the
choice operator applied to the predicate can be shown to be equivalent
to that unique value.  We can prove:
\begin{center}
\verb+|- (?!x. Q x) ==> Q c ==> ((@x. Q x) = c)+
\end{center}
The definition of ``\verb+$?!+'' is:
\begin{center}
\verb+|- $?! = (\P. $? P /\ (!x y. P x /\ P y ==> (x = y)))+
\end{center}
From ``\verb+Q c+'' one can derive ``\verb+?x. Q x+'' (using EXISTS),
so we can reduce the earlier theorem to :
\begin{center}
\verb+|- (!x y. Q x /\ Q y ==> (x = y)) ==> Q c ==> ((@x. Q x) = c)+
\end{center}
This has been implemented as a forward inference rule and tactic
(see Appendix~\ref{AppendixA} for definitions).  SELECT\_UNIQUE\_RULE
takes a pair of terms, a theorem that the predicate holds at the
second term, and a theorem that the predicate has a unique value that
satisfies it, and generates a theorem that the value returned by the
choice operator applied to the predicate is the unique value.  
\begintt
    SELECT_UNIQUE_RULE : (term # term) -> thm -> thm -> thm

    ("x","y")   A1 |- Q[y]  A2 |- !x y.(Q[x] \AND Q[y]) ==> (x=y)
    ===========================================================
    	A1 U A2 |- (@x.Q[x]) = y


    SELECT_UNIQUE_TAC : tactic

            [ A ] "(@x. Q[x]) = y"
    =====================================================
    [ A ] "Q[y]"   [ A ] "!x y.(Q[x] \AND Q[y]) ==> (x=y)"
\endtt

An example problem using this tactic arose in the proof of the
correctness of the initial state of the SECD machine.  The essentials
of the situation in that complex system are captured in the following
simple example.

We define a circuit composed of a mux, 2 inverters, and 2 D-type flip
flops.  We sample all lines once per clock cycle, and assume that the
D-types start up with the value {\it F\/} initially (a forced reset at a
lower level of description).  Definitions for all components are in
Appendix~\ref{AppendixB}.  Here we define only the circuit
implementation:
\begintt
let circuit_imp = new_definition
(`circuit_imp`,
 "circuit_imp s0 s1 =
  ? q c in0 in1:num->bool.
    (D_type q s0)  \AND
    (D_type s0 s1) \AND
    (inv s1 in1)   \AND
    (inv s0 c)     \AND
    (mux2 c in0 in1 q) \AND
    (gnd in0)"
 );;
\endtt
\inbox{
% origin:      ll
% size:        13 x 13
% inputs left: (0,2) (0,11)
% input right: (13,6.5)
% input top:   (5,13)
% input bot:   (5,0)
\newsavebox{\MUX}
\savebox{\MUX}
{\setlength{\unitlength}{1mm}\begin{picture}(13,13)
\put(0,0) {\line (0,1) {4}}
\put(0,4) {\line (4,1) {4}}
\put(4,5) {\line (0,1) {3}}
%\bezier{50}(3,6)(1.5,6.5)(0,7)
\put(0,9) {\line (4,-1) {4}}
\put(0,9) {\line (0,1) {4}}
\put(0,13) {\line (1,0) {5}}
\put(5,13) {\line (2,-1) {8}}
\put(13,4) {\line (0,1) {5}}
\put(5,0) {\line (2,1) {8}}
\put(0,0) {\line (1,0) {5}}
%\put(5.5,4.5) {\makebox(0,0)[b]{\tiny 2:1}}
%\put(5.5,2.5) {\makebox(0,0)[b]{\tiny mux}}
\put(5,5) {\shortstack{\tiny 2:1 \\ \tiny mux}}
\end{picture}}

\newsavebox{\BUFGATE}
\savebox{\BUFGATE}
{\setlength{\unitlength}{1mm}\begin{picture}(6,8)
\put(0,0) {\line (0,1) {8}}
\put(0,0) {\line (3,2) {6}}
\put(0,8) {\line (3,-2) {6}}
\end{picture}}

% origin       ll
% size:        8 x 8
% inputs left: (0,4)
% input right: (8,4)
\newsavebox{\INVGATE}
\savebox{\INVGATE}
{\setlength{\unitlength}{1mm}\begin{picture}(8,8)
\put(0,0){\usebox{\BUFGATE}}
\put(7,4) {\circle {2}}
\end{picture}}

% origin       ll
% size:        8 x 8
% inputs left: (0,4)
% input right: (8,4)
\newsavebox{\revINVGATE}
\savebox{\revINVGATE}
{\setlength{\unitlength}{1mm}\begin{picture}(8,8)
\put(8,0) {\line (0,1) {8}}
\put(8,0) {\line (-3,2) {6}}
\put(8,8) {\line (-3,-2) {6}}
\put(1,4) {\circle {2}}
\end{picture}}

\newsavebox{\GND}
\savebox{\GND}
{\setlength{\unitlength}{1mm}\begin{picture}(4,4)
\put(2,2) {\line (0,1) {2}}
\put(0,2) {\line (1,0) {4}}
\put(0.75,1) {\line (1,0) {2.5}}
\put(1.5,0) {\line (1,0) {1}}
\end{picture}}



\begin{center}
\setlength{\unitlength}{1mm}
\begin{picture}(104,31)(0,6)
    \put(10,15){\usebox{\MUX}}
    \put(40,15){\framebox(13,13){D\_type}}
    \put(70,15){\framebox(13,13){D\_type}}
    \put(0,22){\usebox{\GND}}
    \put(25,6){\usebox{\revINVGATE}}
    \put(25,29){\usebox{\revINVGATE}}

    \put(23,21.5){\line(1,0){17}}
    \put(53,21.5){\line(1,0){17}}
    \put(61.5,21.5){\line(0,1){11.5}}
    \put(61.5,21.5){\circle*{1}}
    \put(61.5,33){\circle*{1}}
    \put(83,21.5){\vector(1,0){17}}
    \put(91.5,21.5){\line(0,-1){11.5}}
    \put(91.5,21.5){\circle*{1}}

    \put(33,10){\line(1,0){58.5}}
    \put(25,10){\line(-1,0){20}}
    \put(5,10){\line(0,1){7}}
    \put(5,17){\line(1,0){5}}
    \put(33,33){\vector(1,0){67}}
    \put(25,33){\line(-1,0){10}}
    \put(15,33){\line(0,-1){5}}
    \put(10,26){\line(-1,0){8}}

\tiny
% Internal labels
    \put(30,23){\makebox(0,0)[b]{q}}
    \put(16,30){\makebox(0,0)[l]{c}}
    \put(7.5,27){\makebox(0,0)[b]{in$_0$}}
    \put(7.5,16){\makebox(0,0)[t]{in$_1$}}
% Output labels
    \put(102,33){\makebox(0,0)[l]{s$_0$}}
    \put(102,21.5){\makebox(0,0)[l]{s$_1$}}


\end{picture}
\end{center}
}

This circuit implements a modulo 3 counter (convince yourself).  In
order to show that the first time that the \verb+s1+ line is
asserted is at time t = 2, we establish the goal:
\begintt
"TimeOf s1 0 = 2"
    [ "circuit_imp s0 s1" ]
\endtt
Note that this use of the TimeOf function defines a mapping from the
clock cycle time granularity to one that is sampled only when the
{\it s1\/} line is asserted.   

The definition of TimeOf is:
\begin{center}
\verb+|- !f n. TimeOf f n = (@t. IsTimeOf n f t)+.
\end{center}
After rewriting the goal with this definition, the choice operator is
introduced. 
\begintt
"(@t. IsTimeOf 0 s1 t) = 2"
    [ "circuit_imp s0 s1" ]
\endtt
Applying SELECT\_UNIQUE\_TAC reduces the goal to two subgoals, neither
of which contains the choice operator.
\begintt
"!t t'. IsTimeOf 0 s1 t \AND IsTimeOf 0 s1 t' ==> (t = t')"
    [ "circuit_imp s0 s1" ]

"IsTimeOf 0 s1 2"
    [ "circuit_imp s0 s1" ]
\endtt
The proof is simple. The first goal is proven by an existing
theorem IsTimeOf\_IDENTITY\footnote{See Appendix~\ref{AppendixC}.}:
\begin{center}
\verb+|- !n f t1 t2. IsTimeOf n f t1 /\ IsTimeOf n f t2 ==> (t1 = t2)+
\end{center}
The second is solved by rewriting with the definition of IsTimeOf, and
the various component definitions.
(A complete proof is given in Appendix~\ref{AppendixB}.)

The use of SELECT\_UNIQUE\_TAC has permitted us to eliminate the
choice operator entirely from the subgoals, and prove that the value
it represents is equal to a specific value.  Thus time {\it 0\/} of the
coarser granularity of time corresponds to time {\it 2\/} at the finer
granularity.  

\subsection{More than one value satisfies}\label{sec:many}

When the choice operator is applied to a predicate that is satisfied
at multiple different values, there is no way to determine which
specific value the expression represents.  One can only make use of
the fact that the predicate is satisfied by the value.  An approach to
proof in this situation is supplied by Elsa Gunter, and contained
within the {\it start\_groups.ml\/} file in the {\it group\/}
library\footnote{See the Modular Arithmetic Case Study
in~\cite{Hol89c}.}.  The following tactic is provided:
\begintt
SUPPOSE_TAC : term -> tactic  (term = t)

            [A] t1
    =======================
       [A;t] t1    [A] t
\endtt
(There is also a REV\_SUPPOSE\_TAC which merely reverses the order of
the subgoals.)  The idea is to add an assumption that will be useful
in solving the goal, and also requiring that the assumtion itself be
proved.  This approach is useful in splitting a goal into major
subgoals, and proceeding along one branch, while deferring some of the
proof steps to a more convenient time.  This tactic can be useful on a
goal of the form:
\begin{center}
\verb+[ A ] "Q (@x. P x)"+
\end{center}
\verb+SUPPOSE_TAC "?x. P x"+ gives the following subgoals:
\vspace{-\parskip}
\begin{center}
\makebox[60mm][l]{\verb+[ A ; "?x. P x" ] "Q (@x. P x)"+}\\
\makebox[60mm][l]{\verb+[ A ] "?x. P x"+}
\end{center}
In the first subgoal, using SELECT\_RULE on the new assumption gives:
\begin{center}
\verb+[ A ; "P (@x. P x)" ] "Q (@x. P x)"+
\end{center}
The properties given by the assumption \verb+"P (@x. P x)"+ may be
helpful in solving the goal.

We revert to the previously given circuit for a concrete example.  We
start with the goal:
\begintt
"~s0(SUC(TimeOf s1 n))"
    [ "circuit_imp s0 s1" ]
\endtt
We want to show that the line \verb+s0+ is always F following the
point corresponding to the coarser granularity of time.  Using the
definition of TimeOf to rewrite the goal,
\begintt
"~s0(SUC(@t. IsTimeOf n s1 t))"
    [ "circuit_imp s0 s1" ]
\endtt
followed by the tactic:
\verb+SUPPOSE_TAC "!n:num. ?t. IsTimeOf n s1 t"+, we get 2 subgoals:
\begintt
"!n. ?t. IsTimeOf n s1 t"
    [ "circuit_imp s0 s1" ]

"~s0(SUC(@t. IsTimeOf n s1 t))"
    [ "circuit_imp s0 s1" ]
    [ "!n. ?t. IsTimeOf n s1 t" ]
\endtt
The upper subgoal corresponds to proving a {\it liveness\/} property for
the circuit: there is a finer grain time {\it t\/} that corresponds
to every point in the coarser grain time.  In effect, the predicate
{\it s1 t\/} is satisfied infinitely often.  The lower subgoal
will permit us to use this property and the definition of the circuit
in its proof.  Thus we have effectively separated
the goal into 2 distinct parts: the {\it liveness\/} of the circuit, and
the essential property of the circuit that we are trying to prove.
Tackling the bottom subgoal, we can apply SELECT\_RULE to the bottom
assumption to obtain
\begin{center}
\verb+"IsTimeOf n s1(@t. IsTimeOf n s1 t)"+,
\end{center}
and resolve this with the theorem IsTimeOf\_TRUE:
\begin{center}
\verb+IsTimeOf_TRUE = |- !n f t.  IsTimeOf n f t ==> f t+
\end{center}
to get the goal:
\begintt
"~s0(SUC(@t. IsTimeOf n s1 t))"
    [ "circuit_imp s0 s1" ]
    [ "IsTimeOf n s1(@t. IsTimeOf n s1 t)" ]
    [ "s1(@t. IsTimeOf n s1 t)" ]
\endtt

Resolving the first assumption with an unwound circuit definition:
\begin{center}
\makebox[60mm][l]{\tt circuit\char'137unwound = }\\
\makebox[60mm][l]{\tt |- circuit\char'137{}imp s0 s1 ==>}\\
\makebox[60mm][l]{\tt \ \ \ (!t. s0 0 = F) /\char'134 x}\\
\makebox[60mm][l]{\tt \ \ \ (!t. s0(SUC t) = ((~s0 t) => ~s1 t | F)) /\char'134 }\\
\makebox[60mm][l]{\tt \ \ \ (!t. s1 0 = F) /\char'134 }\\
\makebox[60mm][l]{\tt \ \ \ (!t. s1(SUC t) = s0 t) }
\end{center}
then rewriting with the theorem same\_branches:
\begin{center}
\verb+same_branches = |- !x (y:*). x => y | y = y+,
\end{center}
solves this subgoal.

The previous subgoal can be solved using another application of
SUPPOSE\_TAC, with the assumption \verb+"Inf s1"+, giving the
subgoals: 
\begintt
"Inf s1"
    [ "circuit_imp s0 s1" ]

"!n. ?t. IsTimeOf n s1 t"
    [ "circuit_imp s0 s1" ]
    [ "Inf s1" ]
\endtt
The bottom subgoal is solved by resolving with the theorem
IsTimeOf\_EXISTS: 
\begin{center}
\verb+|- !f. Inf f ==> !n. ?t. IsTimeOf n f t+.
\end{center}
The last subgoal requires proving that \verb+s1+ is asserted
infinitely often.  This simple subgoal can be proven by a surprisingly
lengthy proof using induction.  It is provided in
Appendix~\ref{AppendixB} for the interested reader.  

It should be noted that SUPPOSE\_TAC is more generally useful than
just the cases given here, and it can be applied in the case where the
predicate holds for only one value as well.

\section{Conclusions}
We have attempted to provide some useful examples of the approach to
solving problems that involve the choice operator in HOL.  An informal
analysis based on the number of values satisfying the predicate to
which the choice operator was applied, examined what is provable about
goals containing the operator.  A new tactic and inference rule for
the case where a unique value satisfies the predicate were supplied.
The techniques described were developed during the verification of the
SECD microprocessor design.  

\section*{Acknowledgements}
This work was supported by Strategic, Operating, and Equipment Grants
from the Natural Sciences and Engineering Research Council of Canada
and The Canadian Microelectronics Corporation.  The Strategic Grant
was also supported by The Alberta Microelectronic Centre and LSI
Canada Inc.  The SECD verification effort was also supported by The
Communication Research Establishment, Ottawa.  
We are indebted to Mike Gordon, Tom Melham, and Inder Dhingra for
assistance in learning and using the HOL system, and to Elsa Gunter,
whose initial input on this topic led to this report.


\nocite{Hol89a}
{\small
\bibliography{./select}
\bibliographystyle{alpha}
\appendix

\section{SELECT\_UNIQUE.ml}\label{AppendixA}
\begin{verbatim}
% Filename: SELECT_UNIQUE.ml

	SELECT_UNIQUE_RULE:
	===================

	("x","y")   A1 |- Q[y]  A2 |- !x y.(Q[x]/\Q[y]) ==> (x=y)
	=========================================================
		A1 U A2 |- (@x.Q[x]) = y

Permits substitution for values specified by the Hilbert Choice
operator with a specific value, if and only if unique existance
of the specific value is proven.
%

let SELECT_UNIQUE_RULE (x,y) th1 th2 =
  let qvs,tm = (strip_forall o concl)th2
  in			% use th2 to get a template %
  let Q = (hd o conjuncts o fst o dest_imp) tm
  in			% to get quantified variable for 1st conjuct %
  let x' = (hd o (filter (C mem (frees Q)))) qvs
  in			% in case quantified variables have wrong order %
  let th2' = (GENL (x' . (filter (\x. not(x = x')) qvs))(SPEC_ALL th2))
  in
  let Q' = mk_abs (x,subst[x,x']Q)
  in
  let th1' = SUBST [SYM (BETA_CONV "^Q' ^y"), "b:bool"] "b:bool" th1
  in
  (MP (SPECL ["$@ ^Q'"; y] th2')
      (CONJ (CONV_RULE BETA_CONV (SELECT_INTRO th1')) th1));;

%
	SELECT_UNIQUE_TAC:
	==================

	        [ A ] "(@x. Q[x]) = y"
	===================================================
        [ A ] "Q[y]"   [ A ] "!x y.(Q[x]/\Q[y]) ==> (x=y)"

Given a goal that requires proof of the value specified by the 
Hilbert choice operator, it returns 2 subgoals:
  1. "y" satisfies the predicate, and
  2. unique existance of the value that satisfies the predicate.
%
	
let SELECT_UNIQUE_TAC:tactic (gl,g) =
  let Q,y = dest_eq g
  in
  let x,Qx = dest_select Q
  in
  let x' = variant (x.freesl(g.gl))x
  in
  let Qx' = subst [x', x] Qx
  in
  ([gl,subst [y,x]Qx;
    gl, "!^x ^x'. (^Qx /\ ^Qx') ==> (^x = ^x')"],
   (\thl. SELECT_UNIQUE_RULE (x,y) (hd thl) (hd (tl thl))));;
\end{verbatim}

% \appendix
\newpage

\section{hilbert.ml}\label{AppendixB}
\begin{verbatim}
% Filename: hilbert.ml                                             %

new_theory `hilbert`;;

new_parent `when`;;

loadt `SELECT_UNIQUE`;;

map (load_definition `when`)
[ `TimeOf`
; `IsTimeOf`
; `Inf`
];; 
map (load_theorem `when`)
[ `IsTimeOf_TRUE`
; `IsTimeOf_IDENTITY`
; `IsTimeOf_EXISTS`
];; 

% The following tactic is lifted from the file:
    /Library/group/start_groups.ml
  written by Elsa Gunter:

 SUPPOSE_TAC : term -> tactic  (term = t)

            [A] t1
    =======================
       [A;t] t1    [A] t
%

let SUPPOSE_TAC thisterm thisgoal =
(if type_of thisterm = ":bool"
 then [((thisterm.(fst thisgoal)),(snd thisgoal));((fst thisgoal),thisterm)],
      (\[goalthm;termthm].MP (DISCH thisterm goalthm) termthm)
 else fail)?failwith `SUPPOSE_TAC`;;

% ***************************************************************** %
let mux2 = new_definition
(`mux2`,
 "!cnt in0 in1 out:num->bool.
   mux2 cnt in0 in1 out = !t:num. out t = cnt t => in1 t | in0 t"
 );;

let inv = new_definition
(`inv`, 
 "!in out:num->bool. inv in out = !t. out t = ~in t"
 );;

let D_type = new_definition
(`D_type`, 
 "!in out:num->bool.
   D_type in out =
     (out 0 = F) /\
     !t. out (SUC t) = in t"
 );;

let gnd = new_definition
(`gnd`,
 "!p:num->bool. gnd p = !t. p t = F"
 );;

let circuit_imp = new_definition
(`circuit_imp`,
 "circuit_imp s0 s1 =
  ? q c in0 in1:num->bool.
    (D_type q s0)  /\
    (D_type s0 s1) /\
    (inv s1 in1)   /\
    (inv s0 c)     /\
    (mux2 c in0 in1 q) /\
    (gnd in0)"
 );;
close_theory();;
% ***************************************************************** %
let lemma_0 = TAC_PROOF
(([], "circuit_imp s0 s1 ==> ~s1 0"),
 PURE_REWRITE_TAC[circuit_imp; D_type]
 THEN DISCH_THEN (REPEAT_TCL CHOOSE_THEN (\th.REWRITE_TAC[th]))
 );;

let lemma_1 = TAC_PROOF
(([], "circuit_imp s0 s1 ==> ~s1 1"),
 PURE_REWRITE_TAC[circuit_imp; D_type; num_CONV "1"]
 THEN DISCH_THEN (REPEAT_TCL CHOOSE_THEN (\th.REWRITE_TAC[th])));;

let lemma_2 = TAC_PROOF
(([], "circuit_imp s0 s1 ==> s1 2"),
 PURE_REWRITE_TAC[circuit_imp; D_type; inv; mux2; num_CONV "2"; num_CONV "1"]
 THEN DISCH_THEN (REPEAT_TCL CHOOSE_THEN (\th. REWRITE_TAC[th])));;

% ***************************************************************** %
set_goal(["circuit_imp s0 s1"], "TimeOf s1 0 = 2");;

expand (PURE_ONCE_REWRITE_TAC[TimeOf]);;
expand (SELECT_UNIQUE_TAC);;
rotate 1;;
expand (REWRITE_TAC[IsTimeOf_IDENTITY]);;

expand (PURE_ONCE_REWRITE_TAC[IsTimeOf]);;
expand (IMP_RES_THEN (\th.REWRITE_TAC[th]) lemma_2);;
expand ((CONV_TAC o REDEPTH_CONV o CHANGED_CONV) num_CONV);;
expand (GEN_TAC THEN PURE_REWRITE_TAC[LESS_THM] THEN STRIP_TAC);;

expand (ASM_REWRITE_TAC[]);;
expand (PURE_ONCE_REWRITE_TAC[(SYM o num_CONV) "1"]);;
expand (IMP_RES_THEN ACCEPT_TAC lemma_1);;

expand (ASM_REWRITE_TAC[]);;
expand (IMP_RES_THEN ACCEPT_TAC lemma_0);;

expand (IMP_RES_TAC NOT_LESS_0);;
let initial_thm = save_top_thm `initial_thm`;;

% ***************************************************************** %
let same_branches = TAC_PROOF (([],"!x (y:*). x => y | y = y"),
	GEN_TAC THEN GEN_TAC THEN COND_CASES_TAC THEN REWRITE_TAC[]);;

let not_both = TAC_PROOF
((["circuit_imp s0 s1"], "!t:num. s1 t ==> ~s0 t"),
 INDUCT_TAC
 THENL
 [ IMP_RES_TAC lemma_0
   THEN ASM_REWRITE_TAC[]
 ; RULE_ASSUM_TAC (PURE_REWRITE_RULE[circuit_imp; D_type; inv; mux2; gnd])
   THEN FIRST_ASSUM
     (\th.(CHOOSE_THEN (REPEAT_TCL CHOOSE_THEN ASSUME_TAC))th ?
      NO_TAC)
   THEN ASM_REWRITE_TAC[]
   THEN ASM_CASES_TAC "(s1:num->bool) t"
   THENL
   [ RES_TAC
     THEN ASM_REWRITE_TAC[]
   ; ASM_CASES_TAC "(s0:num->bool) t"
     THEN ASM_REWRITE_TAC[]
   ]
 ]);;

let circuit_unwound =
(fst o EQ_IMP_RULE o SPEC_ALL o (PURE_REWRITE_RULE[D_type; inv; mux2; gnd]))
  circuit_imp;;

% ***************************************************************** %
set_goal (["circuit_imp s0 s1"], "~s0 (SUC(TimeOf s1 n))");;

expand (PURE_ONCE_REWRITE_TAC[TimeOf]);;
expand (SUPPOSE_TAC "!n:num. ?t. IsTimeOf n s1 t");;

 expand (POP_ASSUM(\th. ASSUME_TAC ((SELECT_RULE o (SPEC "n:num")) th)));;
 expand (IMP_RES_TAC IsTimeOf_TRUE);;
 expand (IMP_RES_THEN STRIP_ASSUME_TAC circuit_unwound);;
 expand (ASM_REWRITE_TAC[same_branches]);;

 expand (SUPPOSE_TAC "Inf s1");;

  expand (IMP_RES_THEN (\th.REWRITE_TAC[th]) IsTimeOf_EXISTS);;

   expand (PURE_ONCE_REWRITE_TAC[Inf]);;
   expand (INDUCT_TAC);;
   
    expand (EXISTS_TAC "2"
            THEN CONJ_TAC
	    THENL
	    [ PURE_REWRITE_TAC[num_CONV "2"; LESS_0]
	    ; IMP_RES_THEN ACCEPT_TAC lemma_2
	    ]);;

    expand (POP_ASSUM
            (\th. ASSUME_TAC
                  (PURE_ONCE_REWRITE_RULE
                   [(CONV_RULE o ONCE_DEPTH_CONV) SYM_CONV LESS_MONO_EQ]th)));;
    expand (POP_ASSUM(\th. ASSUME_TAC (PURE_ONCE_REWRITE_RULE[LESS_THM] th)));;
    expand (POP_ASSUM STRIP_ASSUME_TAC);;    

     expand (IMP_RES_THEN STRIP_ASSUME_TAC circuit_unwound);;
     expand (EXISTS_TAC "SUC (SUC (SUC t'))");;
     expand (ASM_REWRITE_TAC[LESS_THM; same_branches]);;
     expand (IMP_RES_TAC not_both);;

     expand (EXISTS_TAC "t':num" THEN ASM_REWRITE_TAC[]);;
let s0_thm = save_top_thm `s0_thm`;;

quit();;
% ***************************************************************** %
\end{verbatim}


% \appendix
\newpage

\section{when.ml}\label{AppendixC}
\begin{verbatim}
%------------------------------------------------------------------------
| FILE		: when.ml
| 
| DESCRIPTION	: Defines the predicates `Next`, `Inf`, `IsTimeOf`
|		  and `TimeOf` and derives several major theorems
|		  which provide a basis for temporal abstraction.
|
|		  These predicates and theorems are taken from
|		  T.Melham's paper, "Abstraction Mechanisms for
|		  Hardware Verification", Hardware Verification
|		  Workshop, University of Calgary, January 1987.
|
|                 This file was written by I.S.Dhingra.
|
| Modified:
| 06.08.91 - BtG - updated to HOL2                                
------------------------------------------------------------------------%

new_theory `when`;;

let Next = new_definition
  (`Next`, 
   "Next t1 t2 f  =  (t1<t2)  /\
                     ( f t2)  /\
                 !t. (t1<t )  /\  (t<t2)  ==>  ~f t"
  );;

let IsTimeOf = new_prim_rec_definition
  (`IsTimeOf`,
  "(IsTimeOf      0  f t  =  f t  /\  !t'.  (t'<t) ==> ~f t') /\
   (IsTimeOf (SUC n) f t  =  ?t'.  IsTimeOf n f t' /\
                                   Next t' t f              )"
  );;

let TimeOf = new_definition
  (`TimeOf`,
   "TimeOf f n  =  @t. IsTimeOf n f t"
  );;

let when = new_infix_definition
  (`when`,
   "when (s:num->*) (p:num->bool)  =  \n. s (TimeOf p n)"
  );;

let Inf = new_definition
  (`Inf`,
   "Inf f =  !t. ?t'.  (t<t') /\ (f t')"
  );;

%------------------------------------------------------------------------
| Define "LEAST P" to represent that P has a smallest element.
------------------------------------------------------------------------%
let LEAST = new_definition
  (`LEAST`,
   "LEAST P  =  ?x. P x  /\  (!y. y<x ==> ~P y)"
 );;

close_theory();;
%------------------------------------------------------------------------
|  The first assumption which matches the term  tm  is undischarged
|  (c) ISD--Aug.1989.
------------------------------------------------------------------------%
let MATCH_UNDISCH_TAC tm =
  (FIRST_ASSUM \th. UNDISCH_TAC (if  (can (match tm) (concl th))
                                then (concl th)
                                else fail
                               ) ? NO_TAC
  )
  ORELSE FAIL_TAC `MATCH_UNDISCH_TAC`;;

%------------------------------------------------------------------------
|   wop = |- !P.  (?n. P n)  ==>  LEAST P
------------------------------------------------------------------------%
let wop = prove_thm
  (`wop`,
   "!P. (?n. P n)  ==>  LEAST P",
   REWRITE_TAC [WOP; LEAST]
  );;

%------------------------------------------------------------------------
|   Inf_EXISTS = |- !f.  Inf f  ==>  ?n.  f n
------------------------------------------------------------------------%
let Inf_EXISTS = prove_thm
  (`Inf_EXISTS`,
   "!f.  Inf f  ==>  ?n.  f n",
   PURE_REWRITE_TAC [Inf]
   THEN REPEAT STRIP_TAC
   THEN FIRST_ASSUM (STRIP_ASSUME_TAC  o (SPEC "t:num"))
   THEN EXISTS_TAC "t':num"
   THEN FIRST_ASSUM ACCEPT_TAC
  );;

%------------------------------------------------------------------------
|   Inf_LEAST = |- !f.  Inf f  ==>  LEAST f
------------------------------------------------------------------------%
let Inf_LEAST = prove_thm
  (`Inf_LEAST`,
   "!f.  Inf f  ==>  LEAST f",
   REPEAT STRIP_TAC
   THEN IMP_RES_TAC Inf_EXISTS
   THEN IMP_RES_TAC wop
  );;

%------------------------------------------------------------------------
|   Inf_Next = |- !f. Inf f ==> !t. f t ==> ?t'. Next t t' f
------------------------------------------------------------------------%
let Inf_Next = prove_thm
  (`Inf_Next`,
   "!f. Inf f ==> !t. f t ==> ?t'. Next t t' f",
   PURE_REWRITE_TAC [Inf; Next]
   THEN GEN_TAC
   THEN DISCH_THEN (\th0. X_GEN_TAC "v:num"
                          THEN STRIP_TAC
                          THEN X_CHOOSE_THEN "n:num" STRIP_ASSUME_TAC
                                   (MATCH_MP wop' (SPEC "v:num" th0)))
   THEN EXISTS_TAC "n:num"
   THEN ASM_REWRITE_TAC []
   THEN REPEAT STRIP_TAC
   THEN RES_THEN (STRIP_ASSUME_TAC o (REWRITE_RULE[DE_MORGAN_THM]))
   THEN RES_TAC
  )
where wop' =
 CONV_RULE (DEPTH_CONV BETA_CONV) (SPEC (( mk_abs
                                         o dest_exists
                                         o snd
                                         o dest_forall
                                         o rhs
                                         o concl
                                         o SPEC_ALL
                                         ) Inf)
                                        WOP
                                  );;

%------------------------------------------------------------------------
|   Next_ADD1 = |- !f t.  f (t+1)  ==>  Next t (t+1) f
------------------------------------------------------------------------%
let Next_ADD1 = prove_thm
  (`Next_ADD1`,
   "!f t.  f (t+1)  ==>  Next t (t+1) f",
   REWRITE_TAC [ Next
               ; SYM (SPEC_ALL ADD1)
               ; LESS_SUC_REFL
               ]
   THEN REPEAT STRIP_TAC
   THENL [ FIRST_ASSUM ACCEPT_TAC
         ; IMP_RES_THEN
             (STRIP_ASSUME_TAC o (CONV_RULE (ONCE_DEPTH_CONV SYM_CONV)))
             LESS_NOT_EQ
          THEN IMP_RES_TAC LESS_SUC_IMP
          THEN IMP_RES_TAC LESS_ANTISYM
         ]
  );;

%------------------------------------------------------------------------
|   Next_INCREAST = |- !f t1 t2.   ~f(t1+1)       ==>
|                               Next (t1+1) t2 f  ==>  Next t1 t2 f
------------------------------------------------------------------------%
let Next_INCREASE = prove_thm
  (`Next_INCREASE`,
   "!f t1 t2.   ~f(t1+1)       ==>
             Next (t1+1) t2 f  ==>  Next t1 t2 f",
   PURE_REWRITE_TAC [Next; SYM (SPEC_ALL ADD1)]
   THEN REPEAT STRIP_TAC
   THENL [ IMP_RES_TAC SUC_LESS
         ; FIRST_ASSUM ACCEPT_TAC
         ; MATCH_UNDISCH_TAC "~^(genvar":bool")"
           THEN IMP_RES_TAC (PURE_REWRITE_RULE [LESS_OR_EQ] LESS_OR)
           THEN RES_TAC
           THEN ASM_REWRITE_TAC []
        ]
  );;

%------------------------------------------------------------------------
|   Next_IDENTITY = |- !t1 t2 f.   Next t1 t2 f  ==>
|                           !t3.   Next t1 t3 f  ==>  (t2 = t3)
------------------------------------------------------------------------%
let Next_IDENTITY = prove_thm
  (`Next_IDENTITY`,
   "!t1 t2 f.   Next t1 t2 f  ==>
          !t3.  Next t1 t3 f  ==>  (t2 = t3)",
   PURE_REWRITE_TAC [Next]
   THEN REPEAT STRIP_TAC
   THEN PURE_ONCE_REWRITE_TAC
          [(SYM o SPEC_ALL o hd o CONJUNCTS) NOT_CLAUSES]
   THEN DISCH_TAC
   THEN STRIP_ASSUME_TAC
          (SPECL ["t2:num"; "t3:num"]
                 (REWRITE_RULE [DE_MORGAN_THM] LESS_ANTISYM))
   THENL [ ALL_TAC
         ; RULE_ASSUM_TAC (CONV_RULE (ONCE_DEPTH_CONV SYM_CONV))
         ]
   THEN IMP_RES_TAC LESS_CASES_IMP
   THEN RES_TAC
  );;

%------------------------------------------------------------------------
|   IsTimeOf_TRUE = |- !n f t.  IsTimeOf n f t ==> f t
------------------------------------------------------------------------%
let IsTimeOf_TRUE = prove_thm
  (`IsTimeOf_TRUE`,
   "!n f t.  IsTimeOf n f t ==> f t",
   INDUCT_TAC
   THEN REWRITE_TAC [IsTimeOf; Next]
   THEN REPEAT STRIP_TAC
  );;

%------------------------------------------------------------------------
|   IsTimeOf_EXISTS = |- !f. Inf f ==> !n. ?t. IsTimeOf n f t
------------------------------------------------------------------------%
let IsTimeOf_EXISTS = prove_thm
  (`IsTimeOf_EXISTS`,
   "!f. Inf f ==> !n. ?t. IsTimeOf n f t",
   GEN_TAC
   THEN DISCH_TAC
   THEN INDUCT_TAC
   THENL [ IMP_RES_TAC Inf_EXISTS
           THEN IMP_RES_THEN
                  (ASSUME_TAC o (PURE_REWRITE_RULE [LEAST]))
                  Inf_LEAST
           THEN ASM_REWRITE_TAC [IsTimeOf]
         ; FIRST_ASSUM STRIP_ASSUME_TAC
           THEN IMP_RES_TAC IsTimeOf_TRUE
           THEN IMP_RES_TAC Inf_Next
           THEN FIRST_ASSUM STRIP_ASSUME_TAC
           THEN REWRITE_TAC [IsTimeOf]
           THEN EXISTS_TAC "t':num"
           THEN EXISTS_TAC "t:num"
           THEN ASM_REWRITE_TAC []
         ]
  );;

%------------------------------------------------------------------------
|   TimeOf_DEFINED = |- !f. Inf f ==> (!n. IsTimeOf n f (TimeOf f n))
------------------------------------------------------------------------%
let TimeOf_DEFINED = save_thm
  (`TimeOf_DEFINED`, ( GEN_ALL
                     o DISCH_ALL
                     o GEN_ALL
                     o (REWRITE_RULE [SYM(SPEC_ALL TimeOf)])
                     o CONV_RULE (DEPTH_CONV BETA_CONV)
                     o (REWRITE_RULE [EXISTS_DEF])
                     o SPEC_ALL
                     o UNDISCH_ALL
                     o SPEC_ALL
                     ) IsTimeOf_EXISTS
  );;

%------------------------------------------------------------------------
|   TimeOf_TRUE = |- !f. Inf f ==> (!n. f (TimeOf f n))
------------------------------------------------------------------------%
let TimeOf_TRUE = save_thm
  (`TimeOf_TRUE`, ( GEN_ALL
                  o DISCH_ALL
                  o GEN_ALL
                  o (MATCH_MP IsTimeOf_TRUE)
                  o SPEC_ALL
                  o UNDISCH_ALL
                  o SPEC_ALL
                  ) TimeOf_DEFINED
  );;

%------------------------------------------------------------------------
|   IsTimeOf_IDENTITY =
|   |- !n f t1 t2. IsTimeOf n f t1 /\ IsTimeOf n f t2 ==> (t1 = t2)
------------------------------------------------------------------------%
let IsTimeOf_IDENTITY = prove_thm
  (`IsTimeOf_IDENTITY`,
   "!n f t1 t2. IsTimeOf n f t1 /\ IsTimeOf n f t2 ==> (t1 = t2)",
   INDUCT_TAC
   THEN PURE_REWRITE_TAC [IsTimeOf; Next]
   THEN X_GEN_TAC "f:num->bool"
   THEN X_GEN_TAC "t1:num"
   THEN X_GEN_TAC "t2:num"
   THEN REPEAT STRIP_TAC
   THENL [ ALL_TAC
         ; RES_THEN (TRY o IMP_RES_THEN
             (\th. EVERY_ASSUM
                     (STRIP_ASSUME_TAC o (\thm. SUBS [th] thm? thm))))
         ]
   THEN STRIP_ASSUME_TAC
          (SPECL ["t2:num"; "t1:num"]
                 (REWRITE_RULE [LESS_OR_EQ] LESS_CASES))
   THEN RES_TAC
  );;

%-----------------------------------------------------------------------
|   TimeOf_INCREASING =
|   |- !f. Inf f  ==>  !n. (TimeOf f n) < (TimeOf f (n+1))
-----------------------------------------------------------------------%
let TimeOf_INCREASING = prove_thm
  (`TimeOf_INCREASING`,
   "!f. Inf f ==> (!n. (TimeOf f n) < (TimeOf f(n+1)))",
   GEN_TAC
   THEN DISCH_TAC
   THEN X_GEN_TAC "n:num"
   THEN IMP_RES_TAC Inf_Next
   THEN IMP_RES_THEN (STRIP_ASSUME_TAC o (SPEC "n:num")) TimeOf_DEFINED
   THEN IMP_RES_THEN
         ( CHOOSE_THEN ( STRIP_ASSUME_TAC
                       o (\thl. CONJ (el 1 thl) (el 2 thl))
                       o CONJUNCTS
                       )
         o (REWRITE_RULE [IsTimeOf; Next])
         o (SPEC "SUC n")
         ) TimeOf_DEFINED
   THEN MATCH_UNDISCH_TAC "x<y"
   THEN IMP_RES_TAC IsTimeOf_TRUE
   THEN IMP_RES_THEN (IMP_RES_THEN
         (\th. ONCE_REWRITE_TAC[ADD1; th])) IsTimeOf_IDENTITY
  );;

%-----------------------------------------------------------------------
|   TimeOf_INTERVAL =
|   |- !f. Inf f ==> 
|      !n t. (TimeOf f n)<t  /\  t<(TimeOf f (n+1)) ==> ~f t
-----------------------------------------------------------------------%
let TimeOf_INTERVAL = prove_thm
  (`TimeOf_INTERVAL`,
   "!f. Inf f ==> 
    !n t. (TimeOf f n)<t  /\  t<(TimeOf f (n+1)) ==> ~f t",
   GEN_TAC
   THEN DISCH_TAC
   THEN X_GEN_TAC "n:num"
   THEN X_GEN_TAC "t:num"
   THEN IMP_RES_TAC Inf_Next
   THEN IMP_RES_THEN (STRIP_ASSUME_TAC o (SPEC "n:num")) TimeOf_DEFINED
   THEN IMP_RES_THEN
         ( CHOOSE_THEN ( STRIP_ASSUME_TAC
                       o (\thl.  CONJ (el 1 thl) (SPEC "t:num" (el 4 thl)))
                       o CONJUNCTS
                       )
         o (REWRITE_RULE [IsTimeOf; Next])
         o (SPEC "SUC n")
         ) TimeOf_DEFINED
   THEN FIRST_ASSUM (UNDISCH_TAC o concl)
   THEN IMP_RES_TAC IsTimeOf_TRUE
   THEN IMP_RES_THEN (IMP_RES_THEN (\th. ONCE_REWRITE_TAC[ADD1; th]))
                     IsTimeOf_IDENTITY
  );;

%-----------------------------------------------------------------------
|   TimeOf_Next = |- !f. Inf f ==> !n. Next (TimeOf f n) (TimeOf f (n+1)) f
-----------------------------------------------------------------------%
let TimeOf_Next = prove_thm
  (`TimeOf_Next`,
   "!f. Inf f ==> !n. Next (TimeOf f n) (TimeOf f (n+1)) f",
   PURE_REWRITE_TAC [Next]
   THEN REPEAT STRIP_TAC
   THENL [ IMP_RES_THEN (\th. REWRITE_TAC [th]) TimeOf_INCREASING
         ; IMP_RES_THEN (\th. REWRITE_TAC [th]) TimeOf_TRUE
         ; IMP_RES_TAC TimeOf_INTERVAL THEN RES_TAC
         ]
  );;

print_theory `when`;;
%<----------------------------------------------------------------------
The Theory when
Parents --  HOL     
Constants --
  when ":(num -> *) -> ((num -> bool) -> (num -> *))"
  Next ":num -> (num -> ((num -> bool) -> bool))"
  IsTimeOf ":num -> ((num -> bool) -> (num -> bool))"
  TimeOf ":(num -> bool) -> (num -> num)"
  Inf ":(num -> bool) -> bool"     LEAST ":(num -> bool) -> bool"     
Infixes --  when ":(num -> *) -> ((num -> bool) -> (num -> *))"     
Definitions --
  Next
    |- !t1 t2 f.
        Next t1 t2 f =
        t1 < t2 /\ f t2 /\ (!t. t1 < t /\ t < t2 ==> ~f t)
  IsTimeOf
    |- (!f t. IsTimeOf 0 f t = f t /\ (!t'. t' < t ==> ~f t')) /\
       (!n f t.
         IsTimeOf(SUC n)f t = (?t'. IsTimeOf n f t' /\ Next t' t f))
  TimeOf  |- !f n. TimeOf f n = (@t. IsTimeOf n f t)
  when  |- !s p. s when p = (\n. s(TimeOf p n))
  Inf  |- !f. Inf f = (!t. ?t'. t < t' /\ f t')
  LEAST  |- !P. LEAST P = (?x. P x /\ (!y. y < x ==> ~P y))
  
Theorems --
  wop  |- !P. (?n. P n) ==> LEAST P
  Inf_EXISTS  |- !f. Inf f ==> (?n. f n)
  Inf_LEAST  |- !f. Inf f ==> LEAST f
  Inf_Next  |- !f. Inf f ==> (!t. f t ==> (?t'. Next t t' f))
  Next_ADD1  |- !f t. f(t + 1) ==> Next t(t + 1)f
  Next_INCREASE
    |- !f t1 t2. ~f(t1 + 1) ==> Next(t1 + 1)t2 f ==> Next t1 t2 f
  Next_IDENTITY
    |- !t1 t2 f. Next t1 t2 f ==> (!t3. Next t1 t3 f ==> (t2 = t3))
  IsTimeOf_TRUE  |- !n f t. IsTimeOf n f t ==> f t
  IsTimeOf_EXISTS  |- !f. Inf f ==> (!n. ?t. IsTimeOf n f t)
  TimeOf_DEFINED  |- !f. Inf f ==> (!n. IsTimeOf n f(TimeOf f n))
  TimeOf_TRUE  |- !f. Inf f ==> (!n. f(TimeOf f n))
  IsTimeOf_IDENTITY
    |- !n f t1 t2. IsTimeOf n f t1 /\ IsTimeOf n f t2 ==> (t1 = t2)
  TimeOf_INCREASING
    |- !f. Inf f ==> (!n. (TimeOf f n) < (TimeOf f(n + 1)))
  TimeOf_INTERVAL
    |- !f.
        Inf f ==>
        (!n t. (TimeOf f n) < t /\ t < (TimeOf f(n + 1)) ==> ~f t)
  TimeOf_Next  |- !f. Inf f ==> (!n. Next(TimeOf f n)(TimeOf f(n + 1))f)
  
******************** when ********************--------------------------->%
\end{verbatim}

}
\end{document}
