\chapter{ML Functions in the wordn Library}
\label{entries}
This chapter provides documentation on some of the \ML\ functions and values
that are made available in \HOL\ when the \ml{trs} library is loaded. Those
functions and values not documented are internal and not intended for use.
This documentation is also available online via the \ml{help} facility.



\DOC{define\_wordn\_bit\_ops}

\TYPE {\small\verb%define_wordn_bit_ops : thm -> int -> (thm) list%}\egroup

\SYNOPSIS
Defines a set of bitwise logical operators for the given word length.

\DESCRIBE
{\small\verb%define_wordn_bit_ops wordn n%}, where {\small\verb%wordn%} is the theorem returned
by {\small\verb%define_wordn_type%} and {\small\verb%n%} is the word length, defines four
commonly used bitwise logical operators: {\small\verb%NOTn%}, {\small\verb%ANDn%}, {\small\verb%ORn%}, and
{\small\verb%XORn%}.  The definitions are stored in the current theory under the
names: {\small\verb%NOTn_DEF%}, {\small\verb%ANDn_DEF%}, {\small\verb%ORn_DEF%} and {\small\verb%XORn_DEF%}  and are
returned in a list as  the value of this function.

\FAILURE
Fails if either it is not in draft mode, or the input theorem is not
in the expected form or the type {\small\verb%:wordn%} has not been defined.

\EXAMPLE
{\small\verb%define_wordn_bit_ops word3 3%} returns the following list of
definitons:
{\par\samepage\setseps\small\begin{verbatim}
   [ |- !b0 b1 b2. NOT3(WORD3[b0;b1;b2]) = WORD3(B_NOT[b0;b1;b2]);
     |- !b0 b1 b2 w.
        (WORD3[b0;b1;b2]) AND3 w = WORD3(B_AND[b0;b1;b2](BITS3 w));
     |- !b0 b1 b2 w.
        (WORD3[b0;b1;b2]) OR3 w = WORD3(B_OR[b0;b1;b2](BITS3 w));
     |- !b0 b1 b2 w.
        (WORD3[b0;b1;b2]) XOR3 w = WORD3(B_XOR[b0;b1;b2](BITS3 w)) ]
\end{verbatim}}
\noindent and they are stored in the current theory under the names
{\small\verb%NOT3_DEF%}, {\small\verb%AND3_DEF%}, {\small\verb%OR3_DEF%} and {\small\verb%XOR3_DEF%}, repectively.

\SEEALSO
{\small\verb%prove_bit_opthms%}, {\small\verb%wordn_bit_ops%}.

\ENDDOC
\DOC{define\_wordn\_type}

\TYPE {\small\verb%define_wordn_type : string -> int -> thm%}\egroup

\SYNOPSIS


\DESCRIBE


\FAILURE


\EXAMPLE


\USES


\COMMENTS


\SEEALSO


\ENDDOC
\DOC{define\_word\_val}

\TYPE {\small\verb%define_word_val : int -> thm list%}\egroup

\SYNOPSIS
Defines two constants for converting between the type {\small\verb%:wordn%} and {\small\verb%:num%}.

\DESCRIBE
{\small\verb%define_word_val%} takes a single argument, a positive integer {\small\verb%n%}. It
defines two logical  constants for converting between the types
{\small\verb%:wordn%} and {\small\verb%:num%}. The constants are named {\small\verb%NVALn%} and {\small\verb%NWORDn%}, and
they have the types {\small\verb%:wordn -> num%} and {\small\verb%:num -> wordn%}, respectively.
Natural numbers are represented by an n-bit word in a big-endian
format, for example, {\small\verb%#001%} denotes 1 as a 3-bit word, and the
following expressions are true:
{\par\samepage\setseps\small\begin{verbatim}
   NVAL3 #001 = 1
   NWORD3 1 = #001
\end{verbatim}}

These constants are defined in terms of {\small\verb%VAL%} and {\small\verb%WORDN%}, which are
the generic conversion 
functions between the represention type of the n-bit words, namely
{\small\verb%:(bool)list%}, and the type {\small\verb%:num%}. The definitions of these generic
conversion functions are as follows:
{\par\samepage\setseps\small\begin{verbatim}
   |- (VAL[] = 0) /\
      (!b bs. VAL(CONS b bs) = ((BV b) * (2 EXP (LENGTH bs))) + (VAL bs))
   |- !n m. (LENGTH(WORDN n m) = n) /\ (VAL(WORDN n m) = m MOD (2 EXP n))
\end{verbatim}}

For a particular word length {\small\verb%n%}, the conversion from  {\small\verb%:wordn%} to
{\small\verb%:num%} is defined to be
{\par\samepage\setseps\small\begin{verbatim}
   !w. NVALn w = VAL(BITSn w)
\end{verbatim}}
and the conversion from {\small\verb%:num%} to {\small\verb%:wordn%} is defined to be
{\par\samepage\setseps\small\begin{verbatim}
   !m. NWORDn m = WORDn(WORDN n m)
\end{verbatim}}

The functions {\small\verb%NVALn%} converting words to natural numbers are one-one
and onto, but the functions {\small\verb%NWORDn%} are not one-one due to the finite
length of a word. From the definition of {\small\verb%WORDN%}, the theorem
{\par\samepage\setseps\small\begin{verbatim}
   |- !m. NWORDn m = NWORDn (m MOD (2 EXP n))
\end{verbatim}}
can be deduced. The conversions can be better characterized by the
theorem
{\par\samepage\setseps\small\begin{verbatim}
   |- (!w. NWORDn(NVALn w) = w) /\
      (!m. m < (2 EXP m) ==> (NVALn(NWORDn m) = m))
\end{verbatim}}
which can be obtained by calling the function {\small\verb%prove_word_val%}.

\FAILURE
Fails if the type {\small\verb%:wordn%} of the specified word length {\small\verb%n%} has not
been defined.

\EXAMPLE
The following HOL session illustrates the use of this function:
{\par\samepage\setseps\small\begin{verbatim}
   # let [NVAL3_DEF; NWORD3_DEF] = define_word_val 3;;

   NVAL3_DEF = |- !w. NVAL3 w = VAL(BITS3 w)
   NWORD3_DEF = |- !n. NWORD3 n = WORD3(WORDN 3 n)
\end{verbatim}}

\SEEALSO
{\small\verb%define_wordn_type%}, {\small\verb%prove_word_val%}.

\ENDDOC
\DOC{define\_word\_partition}

\TYPE {\small\verb%define_word_partition : int -> int -> thm%}\egroup

\SYNOPSIS
Defines operators for splitting and joining words.

\DESCRIBE
{\small\verb%define_word_partition m n%} defines the following two constants:
{\small\verb%JOIN_m_n%} which has type {\small\verb%:(wordm # wordn) -> wordl%} and
{\small\verb%SPLIT_m_n%} which has type {\small\verb%:wordl -> (wordm # wordn)%} where {\small\verb%l = m + n%}.
They are specified by the following theorem:
{\par\samepage\setseps\small\begin{verbatim}
   |- (!wm wn. SPLIT_m_n(JOIN_m_n(wm,wn)) = wm,wn) /\
      (!w. JOIN_m_n(SPLIT_m_n w) = w)
\end{verbatim}}
\noindent which is stored in the current theory as a definition under the name
{\small\verb%PARTITION_m_n%} and is returned as the value of this function.

\FAILURE
Fails if either it is not in draft mode or any of the word types
involved has not been defined.

\EXAMPLE
{\small\verb%define_word_partition 3 5%} returns the following definition:
{\par\samepage\setseps\small\begin{verbatim}
   |- (!wm wn. SPLIT_3_5(JOIN_3_5 (wm,wn)) = wm,wn) /\
      (!w. JOIN_3_5(SPLIT_3_5 w) = w)
\end{verbatim}}
where {\small\verb%wm%} is of type {\small\verb%:word3%}, {\small\verb%wn%} is of type {\small\verb%:word5%} and {\small\verb%w%} is
{\small\verb%:word8%}.

\ENDDOC
\DOC{new\_wordn\_definition}

\TYPE {\small\verb%new_wordn_definition : (bool -> thm -> string -> conv)%}\egroup

\SYNOPSIS
Defines a function over the types of n-bit words.

\DESCRIBE
The function {\small\verb%new_wordn_definition%} provides a facility for defining
functions on the types of n-bit words. It takes four arguments. The
first is a boolean flag which indicates if the function being
defined will be an infix or not. The second is a theorem stating the
existence of functions on the type of n-bit words; this must be a
theorem returned by {\small\verb%prove_function_defn_thm%}. The third argument is a
name under which the resulting definition will be saved in the current
theory segment. And the fourth argument is a term giving the desired
function definition. The value returned by {\small\verb%new_wordn_definiton%} is a
theorem stating the definition requested by the user. This theorem is
derived by formal proof from an instance of the theorem given as the
second argument.

\FAILURE
Failure occurs if HOL cannot prove there is a function satisfying the
specification supplied as the last argument, or if any other condition
for making constant specification (ie, the failure conditions of
{\small\verb%new_specification%}) is violated.

\EXAMPLE
Suppose that the type of 3-bit word has been defined by calling
{\small\verb%define_wordn_type 3%}, and the theorem returned by this type
definition is bound to the name {\small\verb%word3%}. Then, a call to the function
{\small\verb%prove_function_defn_thm%} with the theorem {\small\verb%word3%}
will result in the following theorem:
{\par\samepage\setseps\small\begin{verbatim}
   word3_FNDEF |- !f. ?! fn. !b0 b1 b2. fn(WORD3[b0;b1;b2]) = f b0 b1 b2
\end{verbatim}}
\noindent which is in the form expected by the function {\small\verb%new_wordn_definition%}.
The following definition will define a new constant on the type
{\small\verb%:word3%}:
{\par\samepage\setseps\small\begin{verbatim}
   #let BIT1 = new_wordn_definition false word3_FNDEF `BIT1`
   #    "BIT1(WORD3[b0;b1;b2]) = b1";;
   BIT1 =
   |- !b0 b1 b2. BIT1(WORD3[b0;b1;b2]) = b1
\end{verbatim}}


\SEEALSO
{\small\verb%define_wordn_type%}, {\small\verb%new_specification%}, {\small\verb%new_recursive_definition%},
{\small\verb%prove_function_defn_thm%}, {\small\verb%use_wordn%}.

\ENDDOC
\DOC{prove\_BITS\_one\_one}

\TYPE {\small\verb%prove_BITS_one_one : thm -> thm%}\egroup

\SYNOPSIS


\DESCRIBE


\FAILURE


\EXAMPLE


\USES


\COMMENTS


\SEEALSO


\ENDDOC
\DOC{prove\_BITS\_onto}

\TYPE {\small\verb%prove_BITS_onto : thm -> thm%}\egroup

\SYNOPSIS


\DESCRIBE


\FAILURE


\EXAMPLE


\USES


\COMMENTS


\SEEALSO


\ENDDOC
\DOC{prove\_BITS\_WORD}

\TYPE {\small\verb%prove_BITS_WORD : thm -> thm%}\egroup

\SYNOPSIS
Proves a theorem for reducing {\small\verb%BITSn(WORDn [...])%} to {\small\verb%[...]%}.

\DESCRIBE
{\small\verb%prove_BITS_WORDS%} takes a theorem returned by {\small\verb%define_wordn_type%} and
deduces a theorem which is an instance of the second conjunct of the
type definition theorem. This theorem is usually used in rewriting to
simplify terms containing the expression {\small\verb%BITSn(WORDn [...]%}. 

\FAILURE
Fails if the input theorem is not in the expected form.

\EXAMPLE
{\small\verb%prove_BITS_WORD word3%} where {\small\verb%word3%} is the theorem 
{\par\samepage\setseps\small\begin{verbatim}
   |- (!w. WORD3(BITS3 w) = w) /\
      (!l. (LENGTH l = 3) = (BITS3(WORD3 l) = l))
\end{verbatim}}
\noindent returns the following theorem:
{\par\samepage\setseps\small\begin{verbatim}
   |- !b0 b1 b2. BITS3(WORD3[b0; b1; b2]) = [b0; b1; b2]
\end{verbatim}}

\ENDDOC
\DOC{prove\_bit\_op\_thms}

\TYPE {\small\verb%prove_bit_op_thms : (int -> thm list)%}\egroup

\SYNOPSIS
Proves some basic theorems about the bitwise logical operators.

\DESCRIBE
{\small\verb%prove_bit_op_thms%} takes a single integer as its argument and returns
a list of theorems about the basic properties of the bitwise logical
operators. These theorems are the self-inversion of the {\small\verb%NOTn%}
operator, the symmetry and associativity of the {\small\verb%ANDn%}, {\small\verb%ORn%} and
{\small\verb%XORn%} operators. They are also stored in the current theory under the
names {\small\verb%NOTn%}, {\small\verb%ANDn_SYM%}, {\small\verb%ORn_SYM%}, {\small\verb%XORn_SYM%}, {\small\verb%ANDn_ASSOC%},
{\small\verb%ORn_ASSOC%} and {\small\verb%XORn_ASSOC%}.

\FAILURE
Fails if either it is not in draft mode, or any of the operators has
not been defined.

\EXAMPLE
{\small\verb%prove_bit_op_thms 3%} returns the following list of theorems and
stores them in the current theory under the names {\small\verb%NOT3%},
{\small\verb%AND3_SYM%}, {\small\verb%OR3_SYM%}, {\small\verb%XOR3_SYM%}, {\small\verb%AND3_ASSOC%}, {\small\verb%OR3_ASSOC%} and
{\small\verb%XOR3_ASSOC%}, respectively.
{\par\samepage\setseps\small\begin{verbatim}
   [ |- !w. NOT3(NOT3 w) = w;
     |- !w1 w2. w1 AND3 w2 = w2 AND3 w1;
     |- !w1 w2. w1 OR3 w2 = w2 OR3 w1;
     |- !w1 w2. w1 XOR3 w2 = w2 XOR3 w1;
     |- !w1 w2 w3. w1 AND3 (w2 AND3 w3) = (w1 AND3 w2) AND3 w3;
     |- !w1 w2 w3. w1 OR3 (w2 OR3 w3) = (w1 OR3 w2) OR3 w3;
     |- !w1 w2 w3. w1 XOR3 (w2 XOR3 w3) = (w1 XOR3 w2) XOR3 w3]
\end{verbatim}}

\SEEALSO
{\small\verb%define_wordn_bit_ops%}, {\small\verb%wordn_bit_ops%}.

\ENDDOC
\DOC{prove\_function\_defn\_thm}

\TYPE {\small\verb%prove_function_defn_thm : thm -> thm%}\egroup

\SYNOPSIS


\DESCRIBE


\FAILURE


\EXAMPLE


\USES


\COMMENTS


\SEEALSO


\ENDDOC
\DOC{prove\_LENGTH\_BITS\_thm}

\TYPE {\small\verb%prove_LENGTH_BITS_thm : thm -> thm%}\egroup

\SYNOPSIS


\DESCRIBE


\FAILURE


\EXAMPLE


\USES


\COMMENTS


\SEEALSO


\ENDDOC
\DOC{prove\_NWORD\_MOD}

\TYPE {\small\verb%prove_NWORD_MOD : int -> thm%}\egroup

\SYNOPSIS
Proves a theorem stating the periodicity of the function {\small\verb%NWORDn%}.

\DESCRIBE
{\small\verb%prove_NWORD_MOD n%} proves a theorem about the num-to-word conversion
{\small\verb%NWORDn%} which takes the following form:
{\par\samepage\setseps\small\begin{verbatim}
   |- !m. NWORDn m = NWORDn (m MOD e)
\end{verbatim}}
\noindent where {\small\verb%e%} is a constant number whose value equals to {\small\verb%2 EXP n%}.

\FAILURE
Fails if the type {\small\verb%:wordn%} or the conversion function {\small\verb%NWORDn%}
has not been defined.

\EXAMPLE
The following HOL session illustrates the use of this function:
{\par\samepage\setseps\small\begin{verbatim}
   # let word3_NWORD_MOD = prove_NWORD_MOD 3;;
   word3_NWORD_MOD = |- !m. NWORD3 m = NWORD3(m MOD 8)
\end{verbatim}}

\SEEALSO
{\small\verb%define_wordn_type%}, {\small\verb%define_word_val%}, {\small\verb%prove_word_val%}.

\ENDDOC
\DOC{prove\_wordn\_cases\_thm}

\TYPE {\small\verb%prove_wordn_cases_thm : thm -> thm%}\egroup

\SYNOPSIS


\DESCRIBE


\FAILURE


\EXAMPLE


\USES


\COMMENTS


\SEEALSO


\ENDDOC
\DOC{prove\_wordn\_const\_cases}

\TYPE {\small\verb%prove_wordn_const_cases : thm -> thm%}\egroup

\SYNOPSIS
Exhaustive cases theorem for wordn constants.

\DESCRIBE
{\small\verb%prove_wordn_const_cases%} derives an exhaustive cases theorem for
constants of type {\small\verb%:wordn%}.  Use only for small n.
The input is a theorem of the form returned by {\small\verb%prove_word_cases_thm%}:
{\par\samepage\setseps\small\begin{verbatim}
    |- !w. ?b0...bn-1. w = WORDn [b0;...;bn-1]
\end{verbatim}}
\noindent The corresponding output has the form:
{\par\samepage\setseps\small\begin{verbatim}
   |- !w. ((w = #0...0) \/ ...)  \/ (... \/ w = #1...1)
\end{verbatim}}

\FAILURE


\EXAMPLE
{\small\verb%prove_wordn_const_cases word3_CASES%} returns the following theorem:
{\par\samepage\setseps\small\begin{verbatim}
   |- !w.
    (((w = #000) \/ (w = #001)) \/ (w = #010) \/ (w = #011)) \/
    ((w = #100) \/ (w = #101)) \/
    (w = #110) \/
    (w = #111)
\end{verbatim}}

\SEEALSO
{\small\verb%prove_wordn_cases_thm%}

\ENDDOC
\DOC{prove\_wordn\_induction\_thm}

\TYPE {\small\verb%prove_wordn_induction_thm : thm -> thm%}\egroup

\SYNOPSIS


\DESCRIBE


\FAILURE


\EXAMPLE


\USES


\COMMENTS


\SEEALSO


\ENDDOC
\DOC{prove\_WORD\_one\_one}

\TYPE {\small\verb%prove_WORD_one_one : thm -> thm%}\egroup

\SYNOPSIS


\DESCRIBE


\FAILURE


\EXAMPLE


\USES


\COMMENTS


\SEEALSO


\ENDDOC
\DOC{prove\_WORD\_onto}

\TYPE {\small\verb%prove_WORD_onto : thm -> thm%}\egroup

\SYNOPSIS


\DESCRIBE


\FAILURE


\EXAMPLE


\USES


\COMMENTS


\SEEALSO


\ENDDOC
\DOC{prove\_word\_val}

\TYPE {\small\verb%prove_word_val : int -> thm%}\egroup

\SYNOPSIS
Proves the theorem characterizing the conversion functions between the
types {\small\verb%:wordn%} and {\small\verb%:num%}.

\DESCRIBE
{\small\verb%prove_word_val%} takes a natural number as its single argument. It
proves a theorem which characterizes the conversion functions {\small\verb%NVALn%}
and {\small\verb%NWORDn%}. The theorem is returned as the function value, and it
takes the following form:
{\par\samepage\setseps\small\begin{verbatim}
   |- (!w. NWORDn(NVALn w) = w) /\ 
      (!m. m < (2 EXP n) ==> (NVALn(NWORDn m) = m))
\end{verbatim}}

\FAILURE
Fails if the type {\small\verb%:wordn%} or the conversion functions {\small\verb%:NVALn%} and
{\small\verb%:NWORDn%} have not been defined.

\EXAMPLE
The following HOL session illustrates the use of this function:
{\par\samepage\setseps\small\begin{verbatim}
   # let WORD_VAL_3 = save_thm(`WORD_VAL_3`, prove_word_val 3);;
   WORD_VAL_3 = 
   |- (!w. NWORD3(NVAL3 w) = w) /\
      (!n. n < (2 EXP 3) ==> (NVAL3(NWORD3 n) = n))
\end{verbatim}}

\SEEALSO
{\small\verb%define_wordn_type%}, {\small\verb%define_word_val%}.

\ENDDOC
\DOC{use\_wordn}

\TYPE {\small\verb%use_wordn : int -> (thm)list%}\egroup

\SYNOPSIS
Defines the type of n-bit words and proves some basic theorems.

\DESCRIBE
The function {\small\verb%use_wordn%} is one of the top level user functions of the
wordn library. It provides a facility to define the type of n-bit
words and to prove some basic theorems on this type. When the user
needs to use n-bit words in an application, he/she can simply make
a call to {\small\verb%use_wordn%} with a single argument {\small\verb%n%}. This will define the
type {\small\verb%:wordn%}, and a list of basic theorems about the type are returned.
{\small\verb%use_wordn%} combines the function {\small\verb%define_wordn_type%} and the following
functions for proving theorems about the type of n-bit words:
{\small\verb%prove_function_defn_thm%},\ {\small\verb%prove_wordn_induction_thm%},\
{\small\verb%prove_wordn_cases_thm%},\ {\small\verb%prove_BITS_one_one%},\
{\small\verb%prove_BITS_onto%},\ {\small\verb%prove_WORD_one_one%},\
{\small\verb%prove_WORD_onto%},\ {\small\verb%prove_LENGTH_BITS_thm%}.

\FAILURE
Failure occurs if any of the functions invoked fails.


\EXAMPLE
The HOL session below illustrates the use of the function {\small\verb%use_wordn%}:
{\par\samepage\setseps\small\begin{verbatim}
 # let [word3; word3_BITS_11; word3_BITS_onto;
 # word3_WORD_11; word3_WORD_onto; word3_LENGTH_BITS;
 # word3_FNDEF; word3_INDUCT; word3_CASES] =
 # use_wordn 3;;
   word3 = 
   |- (!w. WORD3(BITS3 w) = w) /\
      (!l. (LENGTH l = 3) = (BITS3(WORD3 l) = l))
   word3_BITS_11 = |- !w' w. (BITS3 w = BITS3 w') ==> (w = w')
   word3_BITS_onto = |- !l. (LENGTH l = 3) ==> (?w. BITS3 w = l)
   word3_WORD_11 = 
   |- !l l'.
       (LENGTH l = 3) /\ (LENGTH l' = 3) ==>
       (WORD3 l = WORD3 l') ==>
       (l = l')
   word3_WORD_onto = |- !w. ?l. (LENGTH l = 3) /\ (WORD3 l = w)
   word3_LENGTH_BITS = |- !w. LENGTH(BITS3 w) = 3
   word3_FNDEF = |- !f. ?! fn. !b0 b1 b2. fn(WORD3[b0;b1;b2]) = f b0 b1 b2
   word3_INDUCT = |- !P. (!b0 b1 b2. P(WORD3[b0;b1;b2])) ==> (!w. P w)
   word3_CASES = |- !w. ?b0 b1 b2. w = WORD3[b0;b1;b2]
\end{verbatim}}

\SEEALSO
{\small\verb%define_wordn_type%}, {\small\verb%prove_function_defn_thm%}, {\small\verb%prove_wordn_induction_thm%},
{\small\verb%prove_wordn_cases_thm%}, {\small\verb%prove_BITS_one_one%},
{\small\verb%prove_BITS_onto%}, {\small\verb%prove_WORD_one_one%},
{\small\verb%prove_WORD_onto%}, {\small\verb%prove_LENGTH_BITS_thm%}.


\ENDDOC
\DOC{wordn\_bit\_ops}

\TYPE {\small\verb%wordn_bit_ops : thm -> int -> (thm)list%}\egroup

\SYNOPSIS
Defines bitwise operators and proves some basic theorems about them.

\DESCRIBE
{\small\verb%wordn_bit_ops wordn n%}, where {\small\verb%wordn%} is the theorem returned by
{\small\verb%define_wordn_type%} and {\small\verb%n%} is the word length, defines a set of
commonly used bitwise logical operators and proves some basic theorems
about these operators. It combines the functions of
{\small\verb%define_wordn_bit_ops%}, and {\small\verb%prove_bit_op_thms%}. 

\FAILURE
Fails if either it is not in draft mode, or the input theorem is not
in the expected form or the type {\small\verb%:wordn%} has not been defined.

\EXAMPLE
{\small\verb%wordn_bit_ops word3 3%} returns the following list of definitions and
theorems:
{\par\samepage\setseps\small\begin{verbatim}
   [ |- !b0 b1 b2. NOT3(WORD3[b0;b1;b2]) = WORD3(B_NOT[b0;b1;b2]);
     |- !b0 b1 b2 w.
        (WORD3[b0;b1;b2]) AND3 w = WORD3(B_AND[b0;b1;b2](BITS3 w));
     |- !b0 b1 b2 w.
        (WORD3[b0;b1;b2]) OR3 w = WORD3(B_OR[b0;b1;b2](BITS3 w));
     |- !b0 b1 b2 w.
        (WORD3[b0;b1;b2]) XOR3 w = WORD3(B_XOR[b0;b1;b2](BITS3 w));
     |- !w. NOT3(NOT3 w) = w;
     |- !w1 w2. w1 AND3 w2 = w2 AND3 w1;
     |- !w1 w2. w1 OR3 w2 = w2 OR3 w1;
     |- !w1 w2. w1 XOR3 w2 = w2 XOR3 w1;
     |- !w1 w2 w3. w1 AND3 (w2 AND3 w3) = (w1 AND3 w2) AND3 w3;
     |- !w1 w2 w3. w1 OR3 (w2 OR3 w3) = (w1 OR3 w2) OR3 w3;
     |- !w1 w2 w3. w1 XOR3 (w2 XOR3 w3) = (w1 XOR3 w2) XOR3 w3 ]
\end{verbatim}}

\SEEALSO
{\small\verb%define_wordn_bit_ops%}, {\small\verb%prove_bit_op_thms%}.

\ENDDOC
\DOC{wordn\_CASES\_TAC}

\TYPE {\small\verb%wordn_CASES_TAC : term -> thm_tactic%}\egroup

\SYNOPSIS
Instantiates variables of type {\small\verb%:wordn%} with a list of bits.

\DESCRIBE
The tactic {\small\verb%wordn_CASES_TAC%} instantiates occurrences of a variable of
type {\small\verb%:wordn%} in the goal with a list of bits. It takes, as its
arguments, a term which indicates which variable is to be instantiated
and a theorem which should in the form  returned by
{\small\verb%prove_wordn_cases%}. The specification of {\small\verb%wordn_CASES_TAC%} is:
{\par\samepage\setseps\small\begin{verbatim}
              A ?- t[w]
   =============================== wordn_CASES_TAC "w" wordn_CASES
      A ?- t[WORDn[b0';...;bn-1']/w]
\end{verbatim}}
where {\small\verb%wordn_CASES%} is the following theorem:
{\par\samepage\setseps\small\begin{verbatim}
   |- !w. ?b0...bn-1. w = WORDn[b0;...bn-1]
\end{verbatim}}

This function specializes the universally quantified variable in the
theorem {\small\verb%wordn_CASES%} with the given term which should match the
variable required to be instantiated in the goal. It then eliminaates
the existentially quantified vairables {\small\verb%b0%},...,{\small\verb%bn-1%} by choosing some
variants which are not free in the assumptions; normally these are
just {\small\verb%b0%},...,{\small\verb%bn-1%}. It then  substitutes the right-hand side of the
equation for the occurrences of the variable {\small\verb%w%} in the goal.

\FAILURE
Fails if the substitution cannot be performed due to  no matching
variable of the correct type is found or other reasons as given in the
failure conditions of {\small\verb%CHOOSE_TAC%} and {\small\verb%SUBST1_TAC%}.

\EXAMPLE
Suppose that one wants to solve the following  goal:
{\par\samepage\setseps\small\begin{verbatim}
   "!w. LENGTH(BITS3 w) = 3"
\end{verbatim}}
The tactic {\small\verb%GEN_TAC THEN wordn_CASES_TAC "w" word3_CASES%}
 may be used to transform it to
{\par\samepage\setseps\small\begin{verbatim}
   "LENGTH(BITS3(WORD3[b0;b1;b2])) = 3"
\end{verbatim}}

\SEEALSO
{\small\verb%wordn_X_CASES_TAC%}, {\small\verb%prove_wordn_cases_thm%},
{\small\verb%CHOOSE_TAC%}, {\small\verb%SUBST1_TAC%}.

\ENDDOC
\DOC{wordn\_CONV}

\TYPE {\small\verb%wordn_CONV : conv%}\egroup

\SYNOPSIS
Definition scheme for wordn constants.

\DESCRIBE
{\small\verb%wordn_CONV "#b1...bn"%} returns
{\par\samepage\setseps\small\begin{verbatim}
   |- #b1...bn = WORDn [b1;...;bn]
\end{verbatim}}

\FAILURE


\EXAMPLE
{\small\verb%wordn_CONV "#110"%} returns the following theorem:
{\par\samepage\setseps\small\begin{verbatim}
   |- #110 = WORD3 [T; T; F]
\end{verbatim}}

\SEEALSO
{\small\verb%define_wordn_type%}, {\small\verb%wordn_NVAL_CONV%}, {\small\verb%wordn_NWORD_CONV%}.

\ENDDOC
\DOC{wordn\_EQ\_CONV}

\TYPE {\small\verb%wordn_EQ_CONV : thm -> conv%}\egroup

\SYNOPSIS
Derives a theorem stating the equality of two words.

\DESCRIBE
{\small\verb%wordn_EQ_CONV wordn "w1 = w2"%} returns a theorem
{\par\samepage\setseps\small\begin{verbatim}
   |- (w1 = w2) = T
\end{verbatim}}
if and only if {\small\verb%w1%} is equal to {\small\verb%w2%}, otherwise, it returns
{\par\samepage\setseps\small\begin{verbatim}
   |-  (w1 = w2) = F
\end{verbatim}}
{\small\verb%wordn%} is the type definition theorem returned by {\small\verb%define_wordn_type%}.

\FAILURE


\EXAMPLE
{\small\verb%wordn_EQ_CONV word3 "#010 = #110";;%} return the following theorem:
{\par\samepage\setseps\small\begin{verbatim}
   |- (#010 = #110) = F
\end{verbatim}}

\SEEALSO
{\small\verb%define_wordn_type%}, {\small\verb%wordn_CONV%}.

\ENDDOC
\DOC{wordn\_NVAL\_CONV}

\TYPE {\small\verb%wordn_NVAL_CONV : conv%}\egroup

\SYNOPSIS
Derives a theorem stating the value of a word.

\DESCRIBE
{\small\verb%wordn_NVAL_CONV%} takes a term in the form {\small\verb%"NVALn #..."%} and derives
a theorem stating the value of the given word when interpreted as
natural number in a big-endian format. The length of the string
following the hash {\small\verb%#%} should be {\small\verb%n%}, and it should consist of only
the digits {\small\verb%0%} and {\small\verb%1%}.

\FAILURE
Fails if the type {\small\verb%:wordn%} or the constant {\small\verb%NVALn%} has not been
defined, or the string is not in the form specified above.

\EXAMPLE
The HOL session below illustrates the use of this conversion:
{\par\samepage\setseps\small\begin{verbatim}
   # let word3_val_6 = wordn_NVAL_CONV "NVAL3 #110";;
   word3_val_6 = |- NVAL3 #110 = 6
\end{verbatim}}

\SEEALSO
{\small\verb%define_wordn_type%}, {\small\verb%define_word_val%}, {\small\verb%wordn_NWORD_CONV%}.

\ENDDOC
\DOC{wordn\_NWORD\_CONV}

\TYPE {\small\verb%wordn_NWORD_CONV : conv%}\egroup

\SYNOPSIS
Derives a theorem stating the  representation of a natural number in
an n-bit word.

\DESCRIBE
{\small\verb%wordn_NWORD_CONV%} takes a term in the form {\small\verb%"NWORDn m"%} and derives a
theorem stating the n-bit word representing the natural number {\small\verb%m%} as
below:
{\par\samepage\setseps\small\begin{verbatim}
	|- NWORDn m = #...
\end{verbatim}}
If {\small\verb%m%} is greater than the maximium that an n-bit word can represent,
{\small\verb%m MOD (2 EXP n)%} will be used to work out the n-bit word instead. By
virtue of the theorem returned by {\small\verb%prove_NWORD_MOD%}, the above theorem
is still valid.

\FAILURE
Fails if the type {\small\verb%:wordn%} or the function {\small\verb%NWORDn%} has not been defined.

\EXAMPLE
The HOL session below illustrates the use of this conversion:
{\par\samepage\setseps\small\begin{verbatim}
   # wordn_NWORD_CONV "NWORD3 6";;
   |- NWORD3 6 = #110

   # wordn_NWORD_CONV "NWORD3 16";;
   |- NWORD3 16 = #000
\end{verbatim}}

\SEEALSO
{\small\verb%define_wordn_type%}, {\small\verb%define_word_val%}, {\small\verb%prove_NWORD_MOD%}, {\small\verb%wordn_NVAL_CONV%}.

\ENDDOC
\DOC{wordn\_X\_CASES\_TAC}

\TYPE {\small\verb%wordn_X_CASES_TAC : term -> thm -> (term list) -> tactic%}\egroup

\SYNOPSIS
Instantiates variables of type {\small\verb%:wordn%} with a list of bits.

\DESCRIBE
The tactic {\small\verb%wordn_X_CASES_TAC%} instantiates occurrences of a variable of
type {\small\verb%:wordn%} in the goal with a list of bits. It takes, as its
arguments, a term which indicates which variable is to be
instantiated, a theorem which should in the form  returned by
{\small\verb%prove_wordn_cases%} and a list of terms which is used as the bits in
the instantiation. The specification of {\small\verb%wordn_X_CASES_TAC%} is:
{\par\samepage\setseps\small\begin{verbatim}
              A ?- t[w]
   =============================== wordn_X_CASES_TAC "w" wordn_CASES
      A ?- t[WORDn[c0;...;cn-1]/w]       ["c0";..."cn-1"]
\end{verbatim}}
\noindent where {\small\verb%wordn_CASES%} has the following theorem:
{\par\samepage\setseps\small\begin{verbatim}
   |- !w. ?b0...bn. w = WORDn[b0;...bn]
\end{verbatim}}

This function specializes the universally quantified variable in the
theorem {\small\verb%wordn_CASES%} with the given term which should match the
variable required to be instantiated in the goal. It then eliminates
the existentially quantified vairables {\small\verb%b0%},...,{\small\verb%bn-1%} using the terms
given in the last argument as specific witness.
It then  substitutes the right-hand side of the
equation for the occurrences of the variable {\small\verb%w%} in the goal.

\FAILURE
Fails if the substitution cannot be performed due to  no matching
variable of the correct type is found, or the supplied
terms involve variables which are free in the assumption or other reasons
as given in the failure conditions of {\small\verb%X_CHOOSE_TAC%} and {\small\verb%SUBST1_TAC%}.

\EXAMPLE
Suppose that one wants to solve the following  goal:
{\par\samepage\setseps\small\begin{verbatim}
   "!w. LENGTH(BITS3 w) = 3"
\end{verbatim}}
\noindent The tactic {\small\verb%GEN_TAC THEN wordn_X_CASES_TAC "w" word3_CASES ["c1";"c2";"c3"]%}
 may be used to transform it to
{\par\samepage\setseps\small\begin{verbatim}
   "LENGTH(BITS3(WORD3[c1;c2;c3])) = 3"
\end{verbatim}}

\SEEALSO
{\small\verb%wordn_CASES_TAC%}, {\small\verb%prove_wordn_cases_thm%},
{\small\verb%X_CHOOSE_TAC%}, {\small\verb%SUBST1_TAC%}.

\ENDDOC
