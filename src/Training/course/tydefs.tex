% =====================================================================
% HOL Course Slides: the type definition package      (c) T Melham 1990
% =====================================================================

\documentstyle[12pt,layout]{article}

% ---------------------------------------------------------------------
% Preliminary settings.
% ---------------------------------------------------------------------

\renewcommand{\textfraction}{0.01}	  % 0.01 of the page must contain text
\setcounter{totalnumber}{10}	 	  % max of 10 figures per page
\flushbottom				  % text extends right to the bottom
\pagestyle{slides}			  % slides page style
\setlength{\unitlength}{1mm}		  % unit = 1 mm

% ---------------------------------------------------------------------
% load macros
% ---------------------------------------------------------------------


\input{holmacs}
\input{tokmac}

%\newtokmac{ter}{\tt}
%\newtokmac{nter}{\tt}
\newtokmac{mlname}{\tt}
\newtokmac{CONST}{\constfont}


\newenvironment{hproof}{\begin{center}
 \begin{tabular*}{5.25in}{r>{$}l<{$}@{\extracolsep{0pt plus 1fill}}>{[}r<{]}}}
 {\end{tabular*}\end{center}}

\makeatletter

%% \verbatimnumbered{<file>}
%% read a file and format the contains in a verbatim environment with
%% line number.
%\newcounter{VerbatimLineNo}
%\def\verbatimnumbered#1{\begingroup
%  \setcounter{VerbatimLineNo}{0}%
%  \def\verbatim@processline{%
%    \addtocounter{VerbatimLineNo}{1}%
%    \leavevmode
%    \llap{\theVerbatimLineNo
%          \ \hskip\@totalleftmargin}%
%    \the\verbatim@line\par}%
%  \verbatiminput{#1}\endgroup}

%% Redefine the theindex environment
%%
%\renewenvironment{theindex}{\begin{multicols}{2}[\section*{\indexname}]%
% \columnseprule \z@ \columnsep 35\p@
% \parindent\z@ \parskip\z@ plus.3\p@\relax\let\item\@idxitem}{\end{multicols}}
%
%\def\@idxitem{\par\hangindent 40\p@}
%
%\def\subitem{\par\hangindent 40\p@ \hspace*{20\p@}}
%
%\def\subsubitem{\par\hangindent 40\p@ \hspace*{30\p@}}
%
%\def\indexspace{\par \vskip 10\p@ plus5\p@ minus3\p@\relax}

\makeatother

%% macros for special words
%%
%\def\HOL{{\sc  hol}}
%\def\CHOL{{\sc hol88}}

%% Redfine constfont. The font cmssc12 is a vertual font based on
%% cmss12. The only difference is that the character at '137 becomes
%% an underline in cmssc12.
\font\sfc=cmssc12 \def\constfont{\sfc}
\def\ul#1{$\underline{#1}$}



% ---------------------------------------------------------------------
% set caption at the foot of pages for this series of slides
% ---------------------------------------------------------------------
\ftext{Type definition package}{10}

% ---------------------------------------------------------------------
% Slides
% ---------------------------------------------------------------------
\begin{document}

% ---------------------------------------------------------------------
% Title page for this series of slides
% ---------------------------------------------------------------------

\bsectitle
The\\
Type Definition\\
Package
\esectitle


% =====================================================================
\slide{Type Definitions in HOL}

\point{The type definition mechanism:}

\vskip10mm

{\setlength{\unitlength}{1.2mm}
\begin{center}
\begin{picture}(115,25)
\Large
{\thicklines
\put(88,12.5){\oval(40,25)}
\put(20.5,12.5){\oval(15,15)}
\put(80.5,12.5){\oval(15,15)}}
\put(20.5,20){\line(1,0){60}}
\put(20.5,5){\line(1,0){60}}
\put(20.5,15){\makebox(0,0){\large\bf new}}
\put(20.5,11.5){\makebox(0,0){\large\bf type}}
\put(98,20){\makebox(0,0){\large\bf existing}}
\put(98,16.5){\makebox(0,0){\large\bf type}}
{\thicklines \put(33,10.5){\vector(1,0){30}}}
\put(48,14.4){\makebox(0,0){{\large \bf bijection}}}
\put(3,12.5){\makebox(0,0){{\tt newty}}}
\put(80.5,12.5){\makebox(0,0){{\LARGE\tt P}}}
\put(113,12.5){\makebox(0,0){{\tt ty}}}
\end{picture}
\end{center}
}

\vskip7mm

\point{The axiom-scheme for type definitions:}

\vskip7mm

\bspindent
\begin{picture}(100,11)
\def\_{\char'137}
\put(0,8){\makebox(50,0)[l]{\Large\tt
new\_type\_definition:(string \# term \# thm) -> thm}}
\put(77,0){\makebox(0,0)[b]{\large\bf 1}}
\put(102,0){\makebox(0,0)[b]{\large\bf 2}}
\put(123,0){\makebox(0,0)[b]{\large\bf 3}}
\end{picture}
\espindent


%\vskip7mm

\bpindent
\LARGE\bf where:
\epindent

\vskip7mm

\bspindent
\Large\bf
1: is the name of the new type: {\tt `newty`}
\bigskip

2: is the `subset predicate': {\tt "P:ty->bool"}
\bigskip

3: is a required existence theorem: \verb!|- ?x. P x!
\espindent 

% =====================================================================
\slide{Example: A Type with One Value}

\vskip-9mm

\point{Type definition:%
{\setlength{\unitlength}{0.9mm}\begin{picture}(115,25)(-8,11)
\Large{\thicklines
\put(88,12.5){\oval(40,25)}
\put(20.5,12.5){\oval(15,15)}
\put(80.5,12.5){\oval(15,15)}}
\put(20.5,20){\line(1,0){60}}
\put(20.5,5){\line(1,0){60}}
\put(20.5,13.25){\makebox(0,0){\tt one}}
\put(98,20){\makebox(0,0){\tt bool}}
{\thicklines \put(33,9.5){\vector(1,0){30}}}
\put(48,14.4){\makebox(0,0){\tt $\exists\;\:$rep}}
\put(80.5,13.25){\makebox(0,0){{{\large $\bullet\;$}\tt T}}}
\put(98,5){\makebox(0,0){{{\large $\bullet\;$}\tt F}}}
\end{picture}
}}

\vskip17mm

\point{Prove existence theorem:}

\begin{session}\begin{verbatim}
#let eth = TAC_PROOF(([], "?x:bool.(\b.b)x"),
                     BETA_TAC THEN EXISTS_TAC "T" 
                     THEN ACCEPT_TAC TRUTH);;
eth = |- ?x. (\b. b)x
\end{verbatim}\end{session}

\point{Make type definition:}

\begin{session}\begin{verbatim}
#let def = 
     new_type_definition(`one`, "\b:bool.b", eth);;
def = |- ?rep. TYPE_DEFINITION(\b. b)rep

#REWRITE_RULE [TYPE_DEFINITION] def;;
|- ?rep.
    (!x' x''. (rep x' = rep x'') ==> (x' = x'')) /\
    (!x. (\b. b)x = (?x'. x = rep x'))
\end{verbatim}\end{session}

% =====================================================================
\slide{Defining Bijections}

\point{The function:}

\vskip7mm

\bspindent
{\Large\verb!define_new_type_bijections!}
\espindent

\vskip7mm

\bpindent
{\LARGE\bf defines bijections {\tt ABS} and {\tt REP} such that:}
\epindent

\vskip7mm

{\setlength{\unitlength}{1.2mm}
\begin{picture}(115,25)
\Large
{\thicklines
\put(88,12.5){\oval(40,25)}
\put(20.5,12.5){\oval(15,15)}
\put(80.5,12.5){\oval(15,15)}}
\put(20.5,20){\line(1,0){60}}
\put(20.5,5){\line(1,0){60}}
\put(20.5,12.5){\makebox(0,0){\tt newty}}
\put(98,20){\makebox(0,0){\tt ty}}
{\thicklines \put(54,16){\vector(1,0){10}}}
{\thicklines \put(54,8.6){\line(1,0){10}}}
{\thicklines \put(42,16){\line(-1,0){10}}}
{\thicklines \put(42,8.6){\vector(-1,0){10}}}
\put(48,16){\makebox(0,0){\Large\tt REP}}
\put(48,8.6){\makebox(0,0){\Large\tt ABS}}
\put(80.5,12.5){\makebox(0,0){{\tt P}}}
\end{picture}}

\vskip10mm

\point{For example:}

\begin{session}\begin{verbatim}
#let def = 
     new_type_definition(`one`, "\b:bool.b", eth);;
def = |- ?rep. TYPE_DEFINITION(\b. b)rep

#let iso = define_new_type_bijections
           `one_ISO_DEF` `ABS` `REP` def;;
iso = |- (!a. ABS(REP a) = a) /\ 
         (!r. (\b. b)r = (REP(ABS r) = r))
\end{verbatim}\end{session}

% =====================================================================
\slide{Proving Facts about the Bijections}

\point{Functions for proving (trivial) facts:}

\vskip5mm

\bspindent\Large
\verb!prove_rep_fn_one_one : thm -> thm!
\medskip

\verb!prove_rep_fn_onto    : thm -> thm!
\medskip

\verb!prove_abs_fn_one_one : thm -> thm!
\medskip

\verb!prove_abs_fn_onto    : thm -> thm!
\espindent

\point{For example:}

\begin{session}\begin{verbatim}
#let iso = define_new_type_isomorphisms 
           `one_ISO_DEF` `ABS` `REP` def;;
iso = |- (!a. ABS(REP a) = a) /\ 
         (!r. (\b. b)r = (REP(ABS r) = r))

#prove_rep_fn_one_one iso;;
|- !a a'. (REP a = REP a') = (a = a')

#prove_abs_fn_onto iso;;
|- !a. ?r. (a = ABS r) /\ (\b. b)r
\end{verbatim}\end{session}

% =====================================================================
\slide{Method for Defining a New Type}

\point{The usual three steps are:}

\begin{center}
\setlength{\unitlength}{1.5mm}
\begin{picture}(80,100)(12,0)
\thicklines

% box 1
{\thicklines \put(40,90){\oval(80,14){}}}
\put(4,89.5){\parbox{108mm}{\Large\bf Find an appropriate representation}} 

% vector down
\put(40,83){\vector(0,-1){13}}
\put(42,76.5){\line(1,0){19}}
\put(40,76.5){\thinlines\oval(4,2)}
\put(64,76.5){\makebox(0,0)[l]{\Large\bf Subset predicate: {\tt P}}}

% box 2
{\thicklines \put(40,62){\oval(80,16){}}}
\put(4,61.5){\parbox{108mm}{\def\_{\char'137}%
			     \Large {\tt new\_type\_definition}\\
			     {\tt define\_new\_type\_bijections}}}

% vector down
\put(40,54){\vector(0,-1){13}}
\put(42,47.5){\line(1,0){19}}
\put(40,47.5){\thinlines\oval(4,2)}
\put(64,47.5){\makebox(0,0)[l]{\Large\bf Bijections: {\tt ABS} and {\tt REP}}}

% box 3
{\thicklines \put(40,33){\oval(80,16){}}}
\put(4,32.5){\parbox{109mm}{\Large \bf 
			   Derive an abstract characterization of the 
			   new type by formal proof.}}

% output
\put(40,25){\vector(0,-1){7}}

\put(40,14){\makebox(0,0){\Large\bf $\vdash$ abstract `axiomatization'}}

\end{picture}
\end{center}

\vskip-10mm

\point{This process has been completely automated for arbitrary concrete
recursive types.}

% =====================================================================
\slide{Automatic Type Definitions}

\point{The derived rule is:}

\vskip7mm

\bspindent
\begin{picture}(100,11)
\def\_{\char'137}
\put(0,8){\makebox(50,0)[l]{\Large \tt
define\_type:string -> string -> thm}}
\put(49,0){\makebox(0,0)[b]{\large\bf 1}}
\put(80.5,0){\makebox(0,0)[b]{\large\bf 2}}
\end{picture}
\espindent

%\vskip7mm

\bpindent
\Large\bf where: 
\epindent

\vskip7mm

\bspindent
\Large\bf
1: is the name under which the results will be stored
\vskip12pt

2: is a `grammar' describing the required type
\espindent 

\vskip3mm

\point{The resulting theorem is a complete and \newline
      \mbox{abstract} characterization of the desired type.}

% =====================================================================
\slide{Example: an Enumerated Type}

\point{A three-valued type is specified by:}

\vskip7mm

\bspindent
{\LARGE \verb!tri  =  A  |  B  |  C!}
\espindent

\vskip4mm

\point{In the system:}

\begin{session}\begin{verbatim}
#let tri = define_type `tri` `tri = A | B | C`;;
tri = |- !e0 e1 e2. ?! fn:tri->*. 
         (fn A = e0) /\ (fn B = e1) /\ (fn C = e2)
\end{verbatim}\end{session}

\vskip4mm

\point{The resulting theorem says:}

{\setlength{\unitlength}{2.4mm}
\begin{center}
\begin{picture}(40,25)
\thicklines
\put(5,23){\makebox(0,0){\LARGE\tt :tri}}
\put(35,23){\makebox(0,0){\LARGE\tt :*}}
\put(20,18){\makebox(0,0){\LARGE\tt fn}}

\put(5,10){\oval(10,20)}

\put(3,5){\makebox(0,0){\Large\tt C}}
\put(3,10){\makebox(0,0){\Large\tt B}}
\put(3,15){\makebox(0,0){\Large\tt A}}

\put(6,5){\circle*{1}}
\put(6,10){\circle*{1}}
\put(6,15){\circle*{1}}

\put(7.5,5){\vector(1,0){25}}
\put(7.5,10){\vector(1,0){25}}
\put(7.5,15){\vector(1,0){25}}

\put(35,10){\oval(10,20)}


\put(37,5){\makebox(0,0){\Large\tt e2}}
\put(37,10){\makebox(0,0){\Large\tt e1}}
\put(37,15){\makebox(0,0){\Large\tt e0}}

\put(34,5){\circle*{1}}
\put(34,10){\circle*{1}}
\put(34,15){\circle*{1}}

\end{picture}
\end{center}
}

% =====================================================================
\slide{Example: the Natural Numbers}

\point{The recursive type {\tt :nat} is specified by:}

\vskip7mm

\bspindent
\LARGE \verb!nat  =  Zero  |  Suc nat!
\espindent

\point{In the system:}

\begin{session}\begin{verbatim}
#let nat = 
     define_type `nat` `nat = Zero | Suc nat`;;
nat = |- !e f. ?! fn. 
         (fn Zero = e) /\ 
         (!n. fn(Suc n) = f (fn n) n)
\end{verbatim}\end{session}

\point{The result is the Primitive Recursion Theorem for the natural numbers.}

% =====================================================================
\slide{Example: Binary Trees}

\point{The recursive type {\tt :(*)tree} is specified by:}

\vskip7mm

\bspindent
\LARGE \verb!tree  =  Leaf *  |  Node tree tree!
\espindent

\point{In the system:}

\begin{session}\begin{verbatim}
#let tree = 
     define_type `tree` 
     `tree = Leaf * | Node tree tree`;; 
tree = 
|- !f0 f1. ?! fn.
    (!x. fn(Leaf x) = f0 x) /\
    (!t t'. fn(Node t t') = f1 (fn t) (fn t') t t')
\end{verbatim}\end{session}

\point{The result is a `primitive recursion theorem' 
       for binary trees.}

% =====================================================================
\slide{General Form of Input}

\point{The type {\tt :ty} specified by:}

\vskip7mm

\bspindent
\LARGE
$
\verb!ty! \;\: {\tt{=}} \;\:
{\tt C}_1\;\:ty{}_1^1\;\:\ldots\;\:ty{}_1^{k_1}\quad\verb!|! \;\:{\cdots}\;\: 
\verb!|! \quad {\tt C}_n\;\:ty{}_n^1\;\:\ldots\;\:ty{}_n^{k_n}$
\espindent

\vskip7mm

\bpindent
\LARGE\bf denotes the smallest set of values:
\espindent

\vskip7mm

\bspindent
\LARGE
${\tt C}_i\;\;x{}_i^1\;\;\ldots\;\;x{}_i^{k_i}$
\espindent

\vskip7mm

\bpindent
\LARGE\bf where $x_i^j$ is a term of type $ty_i^j$ for
		$1 \leq j \leq k_i$.
\epindent

\vskip10mm

\point{The resulting theorem states the uniqueness of functions
       defined by `primitive recursion' on the new type {\tt :ty}.}


% =====================================================================
\slide{Primitive Recursive Functions}

\point{The derived rule is:}

\vskip7mm

\bspindent
\Large\verb!new_recursive_definition!
\espindent

\vskip7mm

\bpindent
\LARGE\bf with ML type:
\epindent

\vskip7mm

%let new_recursive_definition infix_flag th name tm

\bspindent
\begin{picture}(100,11)
\def\_{\char'137}
\put(0,8){\makebox(50,0)[l]{\Large \tt
:bool -> thm -> string -> term -> thm}}
\put(10,0){\makebox(0,0)[b]{\large\bf 1}}
\put(33.5,0){\makebox(0,0)[b]{\large\bf 2}}
\put(61,0){\makebox(0,0)[b]{\large\bf 3}}
\put(89,0){\makebox(0,0)[b]{\large\bf 4}}
\end{picture}
\espindent

\bpindent 
\Large\bf where
\epindent

\vskip7mm

\bspindent
\Large\bf
1: is an `infix flag'
\bigskip

2: is the abstract axiom for the recursive type
\bigskip

3: is the name under which the result is stored
\bigskip

4: is a term giving the desired definition
\espindent

\vskip3mm

\point{The output is a theorem that states the
		     \mbox{desired} function definition.}


% =====================================================================
\slide{An Example}

\point{Define the type:}

\begin{session}\begin{verbatim}
#let tree = 
     define_type `tree` 
     `tree = Leaf * | Node tree tree`;; 
tree = 
|- !f0 f1. ?! fn.
    (!x. fn(Leaf x) = f0 x) /\
    (!t t'. fn(Node t t') = f1 (fn t) (fn t') t t')
\end{verbatim}\end{session}

\point{Define a recursive function:}

\begin{session}\begin{verbatim}
#let Sum = 
     new_recursive_definition false tree `Sum`
      "(Sum(Leaf n) = n) /\
       (Sum(Node t1 t2) = (Sum t1) + (Sum t2))";;
Sum = 
 |- (!n. Sum(Leaf n) = n) /\
    (!t1 t2. Sum(Node t1 t2) = (Sum t1) + (Sum t2))
\end{verbatim}\end{session}

% =====================================================================
\slide{Special Features}

\point{Partially specified functions:}

\begin{session}\begin{verbatim}
#let VAL = 
     new_recursive_definition false tree `VAL`
     "VAL (Leaf x) = (x:*)";;

VAL = |- !x. VAL(Leaf x) = x
\end{verbatim}\end{session}


\point{Proving existence only:}

\begin{session}\begin{verbatim}
#let ethm = 
     prove_rec_fn_exists tree 
       "(Sum(Leaf n) = n) /\
        (Sum(Node t1 t2) = (Sum t1) + (Sum t2))";;
ethm = 
|- ?Sum.
    (!n. Sum(Leaf n) = n) /\
    (!t1 t2. Sum(Node t1 t2) = (Sum t1) + (Sum t2))
\end{verbatim}\end{session}

% =====================================================================
\slide{Deriving Induction}

\point{An Example:}

\begin{session}\begin{verbatim}
#let tree = 
     define_type `tree` 
     `tree = Leaf * | Node tree tree`;; 
tree = 
|- !f0 f1. ?! fn.
    (!x. fn(Leaf x) = f0 x) /\
    (!t t'. fn(Node t t') = f1 (fn t) (fn t') t t')

#let ind = prove_induction_thm tree;;
ind = 
|- !P. (!x. P(Leaf x)) /\ 
       (!t t'. P t /\ P t' ==> P(Node t t')) 
         ==> 
       (!t. P t)
\end{verbatim}\end{session}

\point{For enumerated types:}

\begin{session}\begin{verbatim}
#prove_induction_thm tri;;
|- !P. P A /\ P B /\ P C ==> (!t. P t)
\end{verbatim}\end{session}


% =====================================================================
\slide{Induction Tactics}

\point{The general tactic for induction is:}

\vskip7mm

\bspindent
\def\_{\char'137}
\begin{picture}(100,11)
\put(0,8){\makebox(50,0)[l]{\Large \tt
INDUCT\_THEN:thm -> (thm -> tactic) -> thm}}
\put(44,0){\makebox(0,0)[b]{\large\bf 1}}
\put(80.5,0){\makebox(0,0)[b]{\large\bf 2}}
\end{picture}
\espindent

\bpindent
\Large\bf where:
\epindent

\vskip7mm

\bspindent
\Large\bf

1: is the induction theorem for the defined type
\bigskip

2: determines what to do with induction hypotheses
\espindent



% =====================================================================
\slide{Induction Example}

\point{The goal:}

\begin{session}\begin{verbatim}
#set_goal ([], "!t:(*)tree. ?n. P t n");;
"!t. ?n. P t n"
\end{verbatim}\end{session}

\point{Hypotheses are just assumed:}

\begin{session}\begin{verbatim}
#expand (INDUCT_THEN ind ASSUME_TAC);;
OK..
2 subgoals
"?n. P(Node t t')n"
    [ "?n. P t n" ]
    [ "?n. P t' n" ]

"!x. ?n. P(Leaf x)n"
\end{verbatim}\end{session}


\point{Hypotheses are stripped first:}

\begin{session}\begin{verbatim}
#expand (INDUCT_THEN ind STRIP_ASSUME_TAC);;
OK..
2 subgoals
"?n. P(Node t t')n"
    [ "P t n" ]
    [ "P t' n'" ]

"!x. ?n. P(Leaf x)n"
\end{verbatim}\end{session}

% =====================================================================
\slide{Other tools}

\point{Proving constructors one-to-one:}

\begin{session}\begin{verbatim}
#prove_constructors_one_one tree;;
|- (!x x'. (Leaf x = Leaf x') = (x = x')) /\
   (!t0 t1 t2 t3. (Node t0 t1 = Node t2 t3)
                    = 
                  (t0=t2) /\ (t1=t3))
\end{verbatim}\end{session}

\point{Proving constructors yield distinct values:}


\begin{session}\begin{verbatim}
#prove_constructors_distinct tree;;
|- !x t t'. ~(Leaf x = Node t t')
\end{verbatim}\end{session}


\point{Proving an exhastive cases theorem:}

\begin{session}\begin{verbatim}
#prove_cases_thm ind;;
|- !t. (?x. t = Leaf x) \/ (?t1 t2. t = Node t1 t2)
\end{verbatim}\end{session}

% =====================================================================
\slide{Summary}

\def\_{\char'137}

\point{Type definition primitive}

  \subpoint{{\tt new\_type\_definition}}

\point{Defining bijections}

  \subpoint{{\tt define\_new\_type\_bijections}}

\point{Automatic type definitions}

  \subpoint{{\tt define\_type}}

\point{Primitive recursive functions}

  \subpoint{{\tt new\_recursive\_definition}}

\point{Induction}

  \subpoint{{\tt prove\_induction\_thm}}
  \subpoint{{\tt INDUCT\_THEN}}

\point{Other tools}

  \subpoint{{\tt prove\_constructors\_one\_one}}
  \subpoint{{\tt prove\_constructors\_distinct}}
  \subpoint{{\tt prove\_cases\_thm}}


\end{document}
