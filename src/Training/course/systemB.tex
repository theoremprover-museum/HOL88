% =====================================================================
% HOL Course Slides: overview of higher order logic   (c) T melham 1990
% =====================================================================

% latest work : rco 8/ii/91

\documentstyle[12pt,layout]{article}

% ---------------------------------------------------------------------
% Preliminary settings.
% ---------------------------------------------------------------------

\renewcommand{\textfraction}{0.01}	  % 0.01 of the page must contain text
\setcounter{totalnumber}{10}	 	  % max of 10 figures per page
\flushbottom				  % text extends right to the bottom
\pagestyle{slides}			  % slides page style
\setlength{\unitlength}{1mm}		  % unit = 1 mm

% ---------------------------------------------------------------------
% load macros
% ---------------------------------------------------------------------


\input{holmacs}
\input{tokmac}

%\newtokmac{ter}{\tt}
%\newtokmac{nter}{\tt}
\newtokmac{mlname}{\tt}
\newtokmac{CONST}{\constfont}


\newenvironment{hproof}{\begin{center}
 \begin{tabular*}{5.25in}{r>{$}l<{$}@{\extracolsep{0pt plus 1fill}}>{[}r<{]}}}
 {\end{tabular*}\end{center}}

\makeatletter

%% \verbatimnumbered{<file>}
%% read a file and format the contains in a verbatim environment with
%% line number.
%\newcounter{VerbatimLineNo}
%\def\verbatimnumbered#1{\begingroup
%  \setcounter{VerbatimLineNo}{0}%
%  \def\verbatim@processline{%
%    \addtocounter{VerbatimLineNo}{1}%
%    \leavevmode
%    \llap{\theVerbatimLineNo
%          \ \hskip\@totalleftmargin}%
%    \the\verbatim@line\par}%
%  \verbatiminput{#1}\endgroup}

%% Redefine the theindex environment
%%
%\renewenvironment{theindex}{\begin{multicols}{2}[\section*{\indexname}]%
% \columnseprule \z@ \columnsep 35\p@
% \parindent\z@ \parskip\z@ plus.3\p@\relax\let\item\@idxitem}{\end{multicols}}
%
%\def\@idxitem{\par\hangindent 40\p@}
%
%\def\subitem{\par\hangindent 40\p@ \hspace*{20\p@}}
%
%\def\subsubitem{\par\hangindent 40\p@ \hspace*{30\p@}}
%
%\def\indexspace{\par \vskip 10\p@ plus5\p@ minus3\p@\relax}

\makeatother

%% macros for special words
%%
%\def\HOL{{\sc  hol}}
%\def\CHOL{{\sc hol88}}

%% Redfine constfont. The font cmssc12 is a vertual font based on
%% cmss12. The only difference is that the character at '137 becomes
%% an underline in cmssc12.
\font\sfc=cmssc12 \def\constfont{\sfc}
\def\ul#1{$\underline{#1}$}


\def\meta#1{{$\langle\hbox{\it #1\/}\rangle$}}
\def\Rule#1#2{\mbox{${\displaystyle\raise 6pt\hbox{$\;\;\;#1\;\;\;$}} \over 
                     {\displaystyle\lower7pt\hbox{$\;\;\;#2\;\;\;$}}$}}

% ---------------------------------------------------------------------
% set caption at the foot of pages for this series of slides
% ---------------------------------------------------------------------
\ftext{The Core System}{6}

% ---------------------------------------------------------------------
% Slides
% ---------------------------------------------------------------------
\begin{document}

% ---------------------------------------------------------------------
% Title page for this series of slides
% ---------------------------------------------------------------------

\bsectitle
The Core HOL System
\esectitle

% =====================================================================
\slide{Outline}

\point{Theories:}
\subpoint{What a theory consists of.}
\subpoint{Functions for creating and building theories.}
\subpoint{Functions for accessing theory components.}

\point{Core HOL theories:}
\subpoint{The theory {\tt bool}.}
\subpoint{The theory {\tt ind}.}

\point{Primitive inference rules in HOL.}

\point{The built-in derived theories.}


% =====================================================================
\slide{Theories}


\point{A theory is a collection of}

\subpoint{type constants}
\subpoint{type operators}
\subpoint{constants}
\subpoint{constant definitions}
\subpoint{type definitions}
\subpoint{axioms}
\subpoint{theorems}

\point{Theories also have {\it parents}.}

\point{The user always works within some current theory:}

\subpoint{in {\it draft mode\/} when creating a theory.}

\subpoint{in {\it working mode} otherwise.}

\point{Theories are stored in files on disk.}


% =====================================================================
\slide{Core HOL theories}

\point{The core theory heirarchy:}

\vskip10mm
\begin{center}
\setlength{\unitlength}{1mm}
\begin{picture}(100,80)
\put(50,75){\makebox(0,0)[c]{\LARGE {\bf {\tt PPLAMB} }}}
\put(50,55){\makebox(0,0)[c]{\LARGE {\bf {\tt bool } }}}
\put(50,35){\makebox(0,0)[c]{\LARGE {\bf {\tt ind } }}}
\put(50,15){\makebox(0,0)[c]{\LARGE {\bf {\tt BASIC-HOL } }}}

{
\thicklines
\put(46,70){\line(0,-1){10}}
\put(46,50){\line(0,-1){10}}
\put(46,30){\line(0,-1){10}}
}

\end{picture}
\end{center}



\point{{\tt bool} and {\tt ind} contain the axioms and basic \\definitions of
higher order logic.}

\point{In principle, only these two theories need to contain axioms that are
not definitional!}

% =====================================================================
\slide{Hierarchy of Built-in Theories}

\point{The built-in HOL theories are:}

\vskip10mm

\begin{center}
\setlength{\unitlength}{0.0150in}
\begin{picture}(245,292)(90,520)
\thicklines
\put(120,674){\line( 0,-1){ 23}}
\put(120,715){\line( 0,-1){ 23}}
\put(120,757){\line( 0,-1){ 23}}
\put(222,635){\line( 1,-2){ 30}}
\put(335,555){\line(-4,-1){ 80}}
\put(255,535){\line(-4, 1){ 80}}
\put(255,555){\line( 0,-1){ 20}}
\put(175,590){\line( 0,-1){ 18}}
\put(222,635){\line(-2,-1){ 50}}
\put(172,609){\line(-2, 1){ 50}}
\put(120,795){\line( 0,-1){ 25}}
\put(235,520){\makebox(0,0)[lb]{\raisebox{0pt}[0pt][0pt]{\elvrm 
 {\Large \tt HOL}}}}
\put(325,560){\makebox(0,0)[lb]{\raisebox{0pt}[0pt][0pt]{\elvrm 
 {\Large \tt one}}}}
\put(240,560){\makebox(0,0)[lb]{\raisebox{0pt}[0pt][0pt]{\elvrm 
 {\Large \tt sum}}}}
\put(150,560){\makebox(0,0)[lb]{\raisebox{0pt}[0pt][0pt]{\elvrm 
 {\Large \tt tydefs}}}}
\put(150,595){\makebox(0,0)[lb]{\raisebox{0pt}[0pt][0pt]{\elvrm 
 {\Large \tt TREE}}}}
\put(195,640){\makebox(0,0)[lb]{\raisebox{0pt}[0pt][0pt]{\elvrm 
 {\Large \tt combin}}}}
\put(105,640){\makebox(0,0)[lb]{\raisebox{0pt}[0pt][0pt]{\elvrm 
 {\Large \tt tree}}}}
\put(105,680){\makebox(0,0)[lb]{\raisebox{0pt}[0pt][0pt]{\elvrm 
 {\Large \tt list}}}}
\put( 90,720){\makebox(0,0)[lb]{\raisebox{0pt}[0pt][0pt]{\elvrm 
 {\Large \tt arithmetic}}}}
\put( 95,760){\makebox(0,0)[lb]{\raisebox{0pt}[0pt][0pt]{\elvrm 
 {\Large \tt prim\_rec}}}}
\put(100,800){\makebox(0,0)[lb]{\raisebox{0pt}[0pt][0pt]{\elvrm 
 {\Large \tt num}}}}
\end{picture}
\end{center}

\vskip 7mm

\point{These are all definitional and are built on top of the core theories.}

\point{They will be described later.}

% =====================================================================
\slide{Creating Theories in HOL}

\point{There are two modes of working in HOL:}

\subpoint{{\it draft mode}: used when building a new theory by making 
definitions.}

\subpoint{{\it working mode}: used when proving theorems in the current theory
but not adding new definitions}

\vskip5mm
\point{There is always a {\it current\/} theory.}

\vskip5mm

\point{Current theory and testing for draft mode:}

\begin{session}\begin{verbatim}
#current_theory();;
`HOL` : string

#draft_mode();;
false : bool

#new_theory `foo`;;
() : void

#draft_mode();;
true : bool
\end{verbatim}\end{session}


% =====================================================================
\slide{Drafting a New Theory}

\point{To create a new theory:}
\vskip5mm
\bspindent\LARGE\bf
     \verb!new_theory : string -> void!
\espindent
\vskip5mm

\point{Evaluating {\tt new\_theory `\meta{name}`}:}

\subpoint{enters draft mode and creates the theory `\meta{name}'}
\subpoint{creates a theory file {\tt \meta{name}.th} on disk}
\subpoint{makes \meta{name} the current theory}

\vskip4mm
\point{The previous current theory becomes a parent of the new theory.}
\vskip5mm

\point{Example:}

\begin{session}\begin{verbatim}
#current_theory();;
`HOL` : string

#new_theory `foo`;;
() : void

#parents `foo`;;
[`HOL`] : string list
\end{verbatim}\end{session}


% =====================================================================
\slide{Adding Parents}

\point{The function:}
\vskip5mm
\bspindent\LARGE\bf
\verb!new_parent : string -> void!
\espindent
\vskip5mm
\bpindent\LARGE\bf
adds a new parent to the current theory. 
\epindent

\vskip7mm

\point{Example:}
\begin{session}\begin{verbatim}
#current_theory();;
`foo` : string

#new_parent `bar`;;
Theory bar loaded
() : void

#new_parent `baz`;;
Error: File "baz.th" does not exist.
evaluation failed     new_parent

#parents `foo`;;
[`bar`; `HOL`] : string list
\end{verbatim}\end{session}



% =====================================================================
\slide{Adding Axioms}

\point{The function:}
\vskip5mm

\bpindent\LARGE\bf
\begin{verbatim}
      new_axiom : (string # term) -> thm
\end{verbatim}
\epindent
\vskip5mm
\bpindent\LARGE\bf
adds a new axiom to the current theory. 
\epindent

\point{Evaluating:}

\bspindent\LARGE\bf {\tt new\_axiom (`\meta{name}`,$t$)}
\espindent

\subpoint{makes $t$ into an axiom $\vdash t$}
\subpoint{stores the axiom under the name `\meta{name}' in the current theory}

\vskip7mm
\point{Warning:}

\newbox\warn
\setbox\warn=\hbox\bgroup\hskip5mm\begin{minipage}{140mm}\Huge\bf\vskip8mm
\obeylines
this function can be used to add 
inconsistent axioms to a theory!
\mbox{}

\end{minipage}\hskip5mm\egroup

\begin{center}\Frame{\box\warn}\end{center}

% =====================================================================
\slide{Defining Constants}

\point{The function for making constant definitions:}
\vskip5mm
\bspindent\LARGE\bf
\verb!new_definition: (string # term) -> thm!
\espindent
\vskip7mm

\point{Evaluating:}
\vskip5mm

\bspindent\LARGE\bf
{\tt new\_definition (`\meta{name}`,"$c$ = $t$")}
\espindent

\subpoint{makes $c$ a constant defined by {\tt |- $c$ = $t$}}
\subpoint{stores {\tt |- $c$ = $t$} in the current theory under the name 
`\meta{name}'}

\vskip7mm
\point{Variants:}
\subpoint{{\tt new\_infix\_definition} makes the constant an infix}
\subpoint{{\tt new\_binder\_definition} makes it a binder} 


% =====================================================================
\slide{Definition Examples}

\point{A constant definition:}
\vskip4mm

\begin{session}\begin{verbatim}
#new_definition(`foo`, "foo x = x+1");;
|- !x. foo x = x + 1
\end{verbatim}\end{session}


\point{Defining an infix:}
\vskip4mm
\begin{session}\begin{verbatim}
#new_infix_definition
   (`element`, "$e x P = (P:*->bool) x");;
|- !x P. x e P = P x

#"$e (y:num) P";;
"y e P" : term

#"y e (\y. y = T)";;
"y e (\y. y = T)" : term
\end{verbatim}\end{session}

\point{Defining a binder:}
\vskip4mm
\begin{session}\begin{verbatim}
#new_binder_definition
   (`Forall`, "$A = (\P:*->bool. P = (\x. T))");;
|- $A = (\P. P = (\x. T))

#"$A (\x. x < (x+1))";;
"A x. x < (x + 1)" : term

#"A y. y < 10";;
"A y. y < 10" : term
\end{verbatim}\end{session}


% =====================================================================
\slide{Common Errors}

\point{Constant already defined:}
\vskip4mm
\begin{session}\begin{verbatim}
#new_definition(`foo`, "foo x = x+2");;
evaluation failed
   attempt to redefine the constant foo
\end{verbatim}\end{session}

\point{Recursive definition:}
\vskip4mm
\begin{session}\begin{verbatim}
#new_definition
   (`fact`, 
    "fact n = ((n=0) => 1 | n * fact(n-1))");;
evaluation failed  
   recursive definitions not allowed
\end{verbatim}\end{session}

\point{Free variables on the right-hand side:}
\vskip4mm
\begin{session}\begin{verbatim}
#new_definition(`bar`, "bar x y = x + y + z");;
evaluation failed     unbound var in rhs
\end{verbatim}\end{session}

% =====================================================================
\slide{Loose Specification of Constants}

\point{Constant specification is a rule of definition.}

\point{{\tt new\_specification} has ML type:}
\vskip 5mm
\bspindent\Large\bf
   \verb!string -> ((string # string) list) -> thm -> thm!
\espindent
\vskip 7mm

\point{Example:}
\begin{session}\begin{verbatim}
#th1;;
|- ?x y. x < y

#new_specification `LOHI`
   [`constant`,`LO`; `constant`,`HI`] th1;;
|- LO < HI
\end{verbatim}\end{session}

\vskip7mm
\point{The first string in the list of pairs is one of:}
\vskip5mm
\bspindent\LARGE\bf
{\tt `constant`},~~~{\tt `infix`},~~or~~{\tt `binder`}
\espindent

\point{See the manual for details.}


% =====================================================================
\slide{Type Definitions}

\point{To define a new type use:}
\vskip5mm
\bspindent\LARGE\bf
\verb!new_type_definition!
\espindent
\vskip5mm
\bpindent\LARGE\bf
which has ML type
\epindent
\vskip5mm
\bspindent\LARGE\bf
\verb!(string # term # thm) -> thm!
\espindent
\vskip5mm
\bpindent\LARGE\bf
where
\epindent
\vskip5mm
\subpoint{the string is the name of the new type}
\subpoint{the term is the subset predicate}
\subpoint{the theorem is the required existence theorem}

\vskip 7mm
\point{Example:}
\vskip4mm
\begin{session}\begin{verbatim}
#eth;;
|- ?x. (\b. b)x

#new_type_definition
   (`unit`, "\b:bool. b", eth);;
|- ?rep:unit->bool. TYPE_DEFINITION(\b. b)rep

#"x:unit";;
"x" : term
\end{verbatim}\end{session}



% =====================================================================
\slide{Other Theory Functions}

\point{Leave draft mode:}
\vskip4mm
\begin{session}\begin{verbatim}
#draft_mode();;
true : bool

#close_theory();;
() : void

#draft_mode();;
false : bool
\end{verbatim}\end{session}

\point{Return to draft mode:}
\vskip4mm
\begin{session}\begin{verbatim}
#extend_theory `foo`;;
Theory foo loaded
() : void

#draft_mode();;
true : bool
\end{verbatim}\end{session}

\point{Make a theory the current theory:}
\vskip4mm
\begin{session}\begin{verbatim}
#load_theory `foo`;;
Theory foo loaded
() : void
\end{verbatim}\end{session}


% =====================================================================
\slide{User Programmable Search Path}

\point{The {\it search path\/} is used when:}
\subpoint{the system searches for files (e.g.\ via {\tt load})}
\subpoint{the system loads or reads theories from
theory files (e.g.\ using {\tt load\_theory}}

\point{Getting the present search path:}
\vskip4mm
\begin{session}\begin{verbatim}
#search_path();;
[``; `~/`; `/usr/lib/hol/theories/`] : string list
\end{verbatim}\end{session}


\point{Setting the search path:}
\vskip4mm
\begin{session}\begin{verbatim}
#set_search_path [``; `~/mylib/`];;
() : void

#search_path();;
[``; `~/mylib/`] : string list
\end{verbatim}\end{session}


\point{Seeing the result of a search:}
\vskip4mm
\begin{session}\begin{verbatim}
#find_file `foo.th`;;
`foo.th` : string

#find_file `baz.ml`;;
`~/mylib/baz.ml` : string
\end{verbatim}\end{session}

% =====================================================================
\slide{Adding and Deleting Theorems}

\point{Function for saving theorems:}
\vskip5mm
\bspindent\LARGE\bf
\verb!save_thm: (string # thm) -> thm!
\espindent
\vskip 7mm

\point{Function for deleting theorems:}
\vskip5mm
\bspindent\LARGE\bf
\verb!delete_thm: string -> string -> thm!
\espindent
\vskip 7mm

\point{Functions used to fetch a named theorem, \\
axiom or definition from a theory:}

\subpoint{\tt theorem: string -> string -> thm}

\subpoint{\tt axiom: string -> string -> thm}

\subpoint{\tt definition: string -> string -> thm}

% =====================================================================
\slide{Accessing Theory Components}

\point{The following functions extract components from a theory:}

\vskip5mm
\subpoint{\tt parents: string -> string list}
\subpoint{\tt types: string -> (int \# string) list}
\subpoint{\tt constants: string -> term list}
\subpoint{\tt infixes: string -> term list}
\subpoint{\tt binders: string -> term list}
\subpoint{\tt axioms: string -> (string \# thm) list}
\subpoint{\tt definitions: string -> (string \# thm) list}
\subpoint{\tt theorems: string -> (string \# thm) list}

\point{The first argument is the theory name.}


% =====================================================================
\slide{Inspecting a Theory}


\point{Function for inspecting a theory}
\vskip5mm
\bspindent\LARGE\verb!print_theory: string -> void!\espindent

\vskip7mm
\point{Example:}
\vskip4mm
\begin{session}\begin{verbatim}
#print_thory `foo`;;

The Theory foo
Parents --  HOL     bar     
Types --  ":unit"     
Constants --
  foo ":num -> num"
  e ":* -> ((* -> bool) -> bool)"     
  LO ":num"     HI ":num"     
Infixes --  e ":* -> ((* -> bool) -> bool)"     
Definitions --
  foo  |- !x. foo x = x + 2
  element  |- !x P. x e P = P x
  LOHI  |- LO < HI
  unit_TY_DEF  |- ?rep. TYPE_DEFINITION(\b. b)rep
  
******************** foo ********************

() : void
\end{verbatim}\end{session}


% =====================================================================
\slide{Summary}

\point{Creating/extending/loading a theory:}

  \subpoint{{\tt new\_theory}}
  \subpoint{{\tt extend\_theory}}
  \subpoint{{\tt load\_theory}}

\point{Adding axioms and definitions:}

  \subpoint{{\tt new\_axiom}}
  \subpoint{{\tt new\_definition}}
  \subpoint{{\tt new\_infix\_definition}}
  \subpoint{{\tt new\_binder\_definition}}
  \subpoint{{\tt new\_specification}}
  \subpoint{{\tt new\_type\_definition}}

\point{The search path:}

  \subpoint{\tt search\_path}
  \subpoint{\tt set\_search\_path}


% =====================================================================
\slide{Summary (cont'd)}

\point{Saving/deleting theorems:}

  \subpoint{{\tt save\_thm}}
  \subpoint{{\tt delete\_thm}}

\point{Other theory-accessing functions:}

\subpoint{\tt parents}
\subpoint{\tt types}
\subpoint{\tt constants}
\subpoint{\tt infixes}
\subpoint{\tt binders}
\subpoint{\tt axioms}
\subpoint{\tt definitions}
\subpoint{\tt theorems}

\point{Inspecting a theory:}

\subpoint{\tt print\_theory}

% =====================================================================
\slide{The Primitive HOL Theory}

\point{{\tt bool} is the most basic HOL theory.}

\point{It contains:}

\subpoint{four of the five axioms}

\subpoint{definitions of basic logical constants}

\subpoint{definitions of constants for syntatic abbreviations}

\subpoint{axioms for pairs (explained later)}

% =====================================================================
\slide{Definitions}

\point{Definitions of the basic logical constants:}
\vskip5mm
\bspindent
\Large\obeylines
  \verb+|- T = ((\x. x) = (\x. x))+
  \verb+|- $! = (\P. P = (\x. T))+
  \verb+|- $? = (\P. P($@ P))+
  \verb+|- $/\ = (\t1 t2. !t. (t1 ==> t2 ==> t) ==> t)+
  \verb+|- $\/ = (\t1 t2. !t. (t1 ==> t) ==>+
\hskip70mm\verb!(t2 ==> t) ==> t)!
  \verb+|- F = (!t. t)+
  \verb+|- $~ = (\t. t ==> F)+
  \verb+|- $?! = (\P. $? P /\ +
\hskip44mm\verb+(!x y. P x /\ P y ==> (x = y)))+
\espindent

\point{These are bound to the ML variables:}
\vskip5mm
\bspindent\Large\obeylines
\verb!T_DEF!
\verb!FORALL_DEF!
\verb!EXISTS_DEF!
\verb!AND_DEF!
\verb!OR_DEF!
\verb!F_DEF!
\verb!NOT_DEF!
\verb!EXISTS_UNIQUE_DEF!
\espindent
% =====================================================================
\slide{Support for Syntactic Sugar}

\point{{\tt bool} also contains the definitions of constants used to support
syntactic abbreviations.}

\vskip7mm
\point{For example:}
\vskip 5mm
\bspindent
\Large\obeylines
  \verb!|- LET = (\f x. f x)!
  \verb+|- !f x y. UNCURRY f(x,y) = f x y+
\espindent
\vskip7mm
\point{We have already seen how these are used.}

% =====================================================================
\slide{Axioms}

\point{The axioms in {\tt bool} are:}

\vskip 7mm
\begin{session}\begin{verbatim}
#BOOL_CASES_AX;;
|- !t:bool. (t = T) \/ (t = F)

#IMP_ANTISYM_AX;;
|- !t1 t2. 
   (t1 ==> t2) ==> (t2 ==> t1) ==> (t1 = t2)

#ETA_AX;;
|- !t:*->**. (\x. t x) = t

#SELECT_AX;;
|- !P:*->bool. !x. P x ==> P($@ P)
\end{verbatim}\end{session} 


% =====================================================================
\slide{Pairs}

\point{For historical and implementation reasons, pairs are axiomatized in the
theory {\tt bool}}

\vskip7mm
\point{The following theorems are in {\tt bool}:}
\vskip5mm
\bspindent\Large\bf\obeylines
\verb+|- !x. FST x,SND x = x+
\verb+|- !x y. FST(x,y) = x+
\verb+|- !x y. SND(x,y) = y+
\verb+|- !x y a b. (x,y = a,b) = (x = a) /\ (y = b)+
\espindent
\vskip5mm
\bpindent\LARGE\bf
bound to {\tt PAIR}, {\tt FST}, {\tt SND}, and {\tt PAIR\_EQ}.
\epindent
 
\vskip7mm
\point{In principle pairs can be {\it defined}.}


% =====================================================================
\slide{The Theory of Individuals}

\point{The theory {\tt ind} introduces the infinite type {\tt ind}, which is 
needed to define numbers.}
\vskip7mm

\point{It has one axiom:}
\vskip5mm
\bspindent\LARGE\bf
\verb!INFINITY_AX  |- ?f. ONE_ONE f /\ ~ONTO f!
\espindent
\vskip5mm

\bpindent\LARGE\bf
where
\epindent
\vskip5mm
\bspindent\Large\bf\obeylines
\verb+|- !f. ONTO f = !y. ?x. y = f x+
\verb+|- !f. ONE_ONE f = +
\hskip30mm\verb+!x1 x2. (f x1 = f x2) ==> (x1 = x2)+
\espindent

\vskip7mm

\point{This axiom states that the type {\tt ind} is infinite.}


% =====================================================================
\slide{Primitive Inference Rules}

\point{There are 8 identifiers in the core system which are bound to
functions that implement primitive inference rules:}

\vskip5mm

\subpoint{\tt ASSUME}
\subpoint{\tt REFL}
\subpoint{\tt BETA\_CONV}
\subpoint{\tt SUBST}
\subpoint{\tt ABS}
\subpoint{\tt INST\_TYPE}
\subpoint{\tt DISCH}
\subpoint{\tt MP}

\vskip7mm
\point{These rules are primitive, not defined.}


% =====================================================================
\slide{Introducing an Assumption}

\point{In logic, the rule is:}
\vskip10mm
\bspindent\LARGE\bf
{\Large\tt ASSUME:\quad}
$\Rule{\mbox{-}}{\{ t \} \vdash t}$
\espindent

\vskip7mm

\point{This rule allows us to deduce $\vdash t=t$ for any boolean term $t$.}
\vskip7mm
\point{The HOL inference rule:}
\vskip5mm
\bspindent\LARGE
\verb!ASSUME : term -> thm!
\espindent
\vskip5mm
\bpindent\LARGE\bf
therefore takes this term as an argument. \epindent

% =====================================================================
\slide{An Example}

\point{An example using {\tt ASSUME}:}
\vskip 4mm
\begin{session}\begin{verbatim}
#ASSUME;;
- : (term -> thm)

#ASSUME "P /\ Q";;
. |- T

#top_print print_all_thm
- : (thm -> void)

#ASSUME "P /\ Q";;
P /\ Q |- P /\ Q
\end{verbatim}\end{session}


\point{The assumed term must be boolean:}
\vskip 4mm
\begin{session}\begin{verbatim}
#ASSUME "n:num";;
evaluation failed     ASSUME
\end{verbatim}\end{session}


% =====================================================================
\slide{Reflexivity of Equality}

\point{The rule is:}
\vskip10mm
\bspindent\LARGE\bf
{\Large\tt REFL:\quad}
$\Rule{\mbox{-}}{\vdash t=t}$
\espindent

\vskip 7mm
\point{The HOL implementation again takes a term as an argument:}
\vskip 4mm
\begin{session}\begin{verbatim}
#REFL "T";;
|- T = T

#REFL "x:bool";;
|- x = x
\end{verbatim}\end{session}


% =====================================================================
\slide{Modus Ponens}

\point{The rule is:}
\vskip10mm
\bspindent\LARGE\bf
{\Large\tt MP:\quad} 
$\Rule{\Gamma_1 \vdash t_1 \supset t_2 \qquad \Gamma_2 \vdash t_1}
{\Gamma_1 \cup \Gamma_2 \vdash t_2}$
\espindent

\vskip 7mm
\point{In HOL, the arguments are:}
\vskip5mm
\subpoint{first: the implication $\Gamma_1 \vdash t_1 \supset t_2$}
\subpoint{second: the antedecent $\Gamma_2 \vdash t_1$}

\vskip 7mm
\point{Example:}
\vskip 4mm
\begin{session}\begin{verbatim}
#MP;;
- : (thm -> thm -> thm)

#let th1 = DISCH "T=T" (REFL "F");;
th1 = |- (T = T) ==> (F = F)

#MP th1 (REFL "T");;
|- F = F

#MP th1 (ASSUME "T=T");;
. |- F = F
\end{verbatim}\end{session}


% =====================================================================
\slide{Modus Ponens}


\point{Modus ponens works up to $\alpha$-equivalence:}
\vskip 4mm
\begin{session}\begin{verbatim}
#let th1 = ASSUME "(!x:num. P x) ==> ?x. P x";;
th1 = . |- (!x. P x) ==> (?x. P x)

#let th2 = ASSUME "!y:num. P y";;
th2 = . |- !y. P y

#MP th1 th2;;
.. |- ?x. P x
\end{verbatim}\end{session}

\vskip7mm

\point{Of course, the antedent must match:}
\vskip 4mm
\begin{session}\begin{verbatim}
#let th1 = ASSUME "(!x:num. P x) ==> ?x. P x";;
th1 = . |- (!x. P x) ==> (?x. P x)

#MP th1 (ASSUME "P(x+y):bool");;
evaluation failed     MP
\end{verbatim}\end{session}




% ==================================================================
\slide{Discharging an Assumption}

\point{The rule is:}
\vskip10mm
\bspindent\LARGE\bf
{\Large\tt DISCH:\quad}
$\Rule{\Gamma \vdash t_2}
{ \Gamma - \{ t_1 \} \vdash t_1 \supset t_2 }$
\espindent

\point{In HOL, the arguments to {\tt DISCH} are:}
\vskip5mm
\subpoint{first: the term $t_1$}
\subpoint{second: the theorem $\Gamma \vdash t_2$}

\vskip 7mm

\point{Example:}
\vskip 4mm
\begin{session}\begin{verbatim}
#let th = ASSUME "F";;
th = . |- F

#DISCH;;
- : (term -> thm -> thm)

#DISCH "F" th;;
|- F ==> F

#DISCH "T" th;;
. |- T ==> F

#DISCH "x=3" (REFL "x=9");;
|- (x = 3) ==> ((x = 9) = (x = 9))
\end{verbatim}\end{session}


% =====================================================================
\slide{The Abstraction Rule}

\point{The rule is:}
\vskip10mm
\bspindent\LARGE\bf
{\Large\bf ABS:\quad}
$\Rule{\Gamma \vdash t_1 = t_2}
{\Gamma \vdash (\lambda x. \; t_1) = (\lambda x.\; t_2) }$
\vskip4mm
{\Large\bf $x$ not free in $\Gamma$}
\espindent
\vskip7mm
\point{You must tell {\tt ABS} what the variable $x$ is.}


\vskip 7mm
\point{An example:}
\vskip 4mm
\begin{session}\begin{verbatim}
#ABS;;
- : (term -> thm -> thm)

#let eq = REFL "x:num";;
eq = |- x = x

#ABS "x:*" eq;;
|- (\x. x) = (\x. x)

#ABS "y:*" it;;
|- (\y x. x) = (\y x. x)
\end{verbatim}\end{session}


% =====================================================================
\slide{Abstraction: Failure Conditions}

\point{The term argument must be a variable:}
\vskip 4mm
\begin{session}\begin{verbatim}
#ABS "x+y" (ASSUME "x:* = x");;
evaluation failed     ABS
\end{verbatim}\end{session}

\vskip 7mm
\point{The theorem must be an equation:}
\vskip 4mm
\begin{session}\begin{verbatim}
#ABS "x:*" (ASSUME "T");;
evaluation failed     ABS
\end{verbatim}\end{session}


\vskip 7mm
\point{The variable must not be free in the\\
assumptions of the theorem:}
\begin{session}\begin{verbatim}
#let th = ASSUME "x:*=y";;
th = . |- x = y

#ABS "x:*" th;;
evaluation failed     ABS

#ABS "b:bool" th;;
. |- (\b. x) = (\b. y)

#DISCH "x:*=y" it;;
|- (x = y) ==> ((\b. x) = (\b. y))
\end{verbatim}\end{session}


% =====================================================================
\slide{Beta Conversion}

\point{The $\beta$-conversion rule is:}
\vskip10mm
\bspindent\LARGE\bf
{\Large\tt BETA\_CONV:\quad}
$\Rule{\mbox{-}}
{\vdash (\lambda x. \; t_1) t_2 = t_1 [ t_2 / x ] }$
\espindent

\point{In HOL, the function {\tt BETA\_CONV} takes the term $(\lambda x. t_1)
t_2$ as an argument.}

\vskip 7mm
\point{Examples:}
\vskip 4mm
\begin{session}\begin{verbatim}
#BETA_CONV;;
- : conv

#BETA_CONV "(\x. x=3) 4";;
|- (\x. x = 3)4 = (4 = 3)

#BETA_CONV "(\x:*. \y. x=y) y";;
|- (\x y. x = y)y = (\y'. y = y')

#BETA_CONV "x=0";;
evaluation failed     BETA_CONV
\end{verbatim}\end{session}

\vskip5mm
\point{The ML type {\tt conv} abbreviates {\tt term -> thm}.}



% =====================================================================
\slide{Substitution}

\point{The rule for substitution is:}
\vskip10mm
\bspindent\LARGE\bf
{\Large\tt SUBST:\quad}
$\Rule{\Gamma_1 \vdash t_1 = t_2 \qquad \Gamma_2 \vdash t [ t_1 ] }
{\Gamma_1 \cup \Gamma_2 \vdash t [ t_2 ]}$
\espindent

\vskip5mm

\point{We must tell {\tt SUBST} which free occurences of $t_1$ in 
the theorem $\Gamma_2 \vdash t[t_1]$ to replace by $t_2$.}

\vskip5mm
\point{This turns out to be quite complicated:}
\begin{session}\begin{verbatim}
#SUBST;;
- : ((thm # term) list -> term -> thm -> thm)
\end{verbatim}\end{session}
\vskip5mm

\point{{\tt SUBST} takes three arguments:}
\vskip5mm
\subpoint{first: a list of substitution equations $\vdash t_1 = t_2$ and
{\it marker variables}}
\subpoint{second: a {\it template} with marked subterms.}
\subpoint{third: the theorem $\Gamma_2 \vdash t[t_1]$}


% =====================================================================
\slide{Examples of Substitution}

\point{A simple example:}
\vskip 4mm
\begin{session}\begin{verbatim}
#let th1 = ASSUME "x=7";;
th1 = . |- x = 7

#let th2 = REFL "x+x";;
th2 = |- x + x = x + x

#SUBST [(th1,"m:num")] "m+x = x+m" th2;;
. |- 7 + x = x + 7

#SUBST [(th1,"m:num")] "x+x = m+x" th2;;
. |- x + x = 7 + x
\end{verbatim}\end{session}
\vskip 7mm

\point{Another example:}
\vskip 4mm
\begin{session}\begin{verbatim}
#let th1 = BETA_CONV "(\x. x) 1";;
th1 = |- (\x. x)1 = 1

#let th2 = REFL "(\x.x) 1";;
th2 = |- (\x. x)1 = (\x. x)1

#let th3 = SUBST [(th1,"m:num")] "m=(\x.x) 1" th2;;
th3 = |- 1 = (\x. x)1

#SUBST [(th3,"m:num")] "1=(\x.x)m" th3;;
|- 1 = (\x. x)((\x. x)1)
\end{verbatim}\end{session}


% =====================================================================
\slide{Simultaneous Substitution}

\point{{\tt SUBST} allows simultaneous substitutions:}
\vskip 4mm
\begin{session}\begin{verbatim}
#let th1 = ASSUME "x=7"
 and th2 = ASSUME "y=x+10";;
th1 = . |- x = 7
th2 = . |- y = x + 10

#let th3 = REFL "x+y";;
th3 = |- x + y = x + y

#SUBST [(th1,"m1:num");(th2,"m2:num")] 
       "m1+y = x+m2" th3;;
.. |- 7 + y = x + (x + 10)
\end{verbatim}\end{session}


% =====================================================================
\slide{Type Instantiation}


\point{The type instantiation rule is:}
\vskip10mm
\bspindent\LARGE\bf
{\Large\tt INST\_TYPE:\quad}
$\Rule{\Gamma \vdash t}
{ \Gamma \vdash t 
[ \sigma_1 \ldots \sigma_n / \alpha_1 \ldots \alpha_n]}$
\espindent

\vskip 7mm
\point{First argument to {\tt INST\_TYPE} a list of the type\\
instantiations to be made:}
\vskip 4mm
\begin{session}\begin{verbatim}
#show_types true;;
false : bool

#let th = REFL "\x:*. f(x) = (y:**)";;
th = |- (\(x:*). (f:* -> **) x = y) 
         = 
        (\(x:*). (f:* -> **) x = y)

#INST_TYPE;;
- : ((type # type) list -> thm -> thm)

#INST_TYPE [(":num",":*");(":bool",":**")] th;;
|- (\(x:num). (f:num -> bool) x = y)
    = 
   (\(x:num). (f:num -> bool) x = y)
\end{verbatim}\end{session}


% =====================================================================
\slide{Hierarchy of Standard Theories}

\point{The hierarchy of standard theories:}

\vskip10mm

\begin{center}
\setlength{\unitlength}{0.0150in}
\begin{picture}(245,292)(90,520)
\thicklines
\put(120,674){\line( 0,-1){ 23}}
\put(120,715){\line( 0,-1){ 23}}
\put(120,757){\line( 0,-1){ 23}}
\put(222,635){\line( 1,-2){ 30}}
\put(335,555){\line(-4,-1){ 80}}
\put(255,535){\line(-4, 1){ 80}}
\put(255,555){\line( 0,-1){ 20}}
\put(175,590){\line( 0,-1){ 18}}
\put(222,635){\line(-2,-1){ 50}}
\put(172,609){\line(-2, 1){ 50}}
\put(120,795){\line( 0,-1){ 25}}
\put(235,520){\makebox(0,0)[lb]{\raisebox{0pt}[0pt][0pt]{\elvrm 
 {\Large \tt HOL}}}}
\put(325,560){\makebox(0,0)[lb]{\raisebox{0pt}[0pt][0pt]{\elvrm 
 {\Large \tt one}}}}
\put(240,560){\makebox(0,0)[lb]{\raisebox{0pt}[0pt][0pt]{\elvrm 
 {\Large \tt sum}}}}
\put(150,560){\makebox(0,0)[lb]{\raisebox{0pt}[0pt][0pt]{\elvrm 
 {\Large \tt tydefs}}}}
\put(150,595){\makebox(0,0)[lb]{\raisebox{0pt}[0pt][0pt]{\elvrm 
 {\Large \tt TREE}}}}
\put(195,640){\makebox(0,0)[lb]{\raisebox{0pt}[0pt][0pt]{\elvrm 
 {\Large \tt combin}}}}
\put(105,640){\makebox(0,0)[lb]{\raisebox{0pt}[0pt][0pt]{\elvrm 
 {\Large \tt tree}}}}
\put(105,680){\makebox(0,0)[lb]{\raisebox{0pt}[0pt][0pt]{\elvrm 
 {\Large \tt list}}}}
\put( 90,720){\makebox(0,0)[lb]{\raisebox{0pt}[0pt][0pt]{\elvrm 
 {\Large \tt arithmetic}}}}
\put( 95,760){\makebox(0,0)[lb]{\raisebox{0pt}[0pt][0pt]{\elvrm 
 {\Large \tt prim\_rec}}}}
\put(100,800){\makebox(0,0)[lb]{\raisebox{0pt}[0pt][0pt]{\elvrm 
 {\Large \tt num}}}}
\end{picture}
\end{center}

\vskip7mm
\point{These are all {\it derived\/} in HOL}


% =====================================================================
\slide{The Standard Built-in Theories}

\point{The theories {\tt bool} and {\tt ind} and the primitive inference rules
form the core of the HOL.}

\vskip7mm
\point{Standard theories built on top of {\tt bool} and {\tt ind}:}
\vskip 7mm
\bspindent\Large\bf
\begin{tabular}{l l}
{\tt num} & defines non-negative integers \\
{\tt prim\_rec} & theory of primitive recursion\\
{\tt arithmetic} & theory of arithmetic\\
{\tt list} & defines the type of lists\\
{\tt tree} & defines trees\\
{\tt combin} & defines the combinators {\tt S}, {\tt K}, {\tt I}, {\tt o}\\
{\tt TREE} & defines labelled trees\\
{\tt tydefs} & for the type definition package\\
{\tt sum} & defines disjoint sum\\
{\tt one} & defines a type with only one value\\
\end{tabular}
\espindent
\vskip7mm
\point{The contents of these theories will be\\ introduced as needed.}

% =====================================================================
\slide{Summary}

\point{Implementation of theories:}
\subpoint{Functions for creating and building theories.}
\subpoint{Functions for accessing theory components.}

\point{The core HOL theories:}
\subpoint{The theory {\tt bool}.}
\subpoint{The theory {\tt ind}.}

\point{Primitive inference rules in HOL.}

\point{The built-in derived theories.}
\end{document}
