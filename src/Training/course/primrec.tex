% =====================================================================
% HOL Course Slides: Pimitive recursion and induction (c) T melham 1990
% =====================================================================

\documentstyle[12pt,layout]{article}

% ---------------------------------------------------------------------
% Preliminary settings.
% ---------------------------------------------------------------------

\renewcommand{\textfraction}{0.01}	  % 0.01 of the page must contain text
\setcounter{totalnumber}{10}	 	  % max of 10 figures per page
\flushbottom				  % text extends right to the bottom
\pagestyle{slides}			  % slides page style
\setlength{\unitlength}{1mm}		  % unit = 1 mm

% ---------------------------------------------------------------------
% load macros
% ---------------------------------------------------------------------


\input{holmacs}
\input{tokmac}

%\newtokmac{ter}{\tt}
%\newtokmac{nter}{\tt}
\newtokmac{mlname}{\tt}
\newtokmac{CONST}{\constfont}


\newenvironment{hproof}{\begin{center}
 \begin{tabular*}{5.25in}{r>{$}l<{$}@{\extracolsep{0pt plus 1fill}}>{[}r<{]}}}
 {\end{tabular*}\end{center}}

\makeatletter

%% \verbatimnumbered{<file>}
%% read a file and format the contains in a verbatim environment with
%% line number.
%\newcounter{VerbatimLineNo}
%\def\verbatimnumbered#1{\begingroup
%  \setcounter{VerbatimLineNo}{0}%
%  \def\verbatim@processline{%
%    \addtocounter{VerbatimLineNo}{1}%
%    \leavevmode
%    \llap{\theVerbatimLineNo
%          \ \hskip\@totalleftmargin}%
%    \the\verbatim@line\par}%
%  \verbatiminput{#1}\endgroup}

%% Redefine the theindex environment
%%
%\renewenvironment{theindex}{\begin{multicols}{2}[\section*{\indexname}]%
% \columnseprule \z@ \columnsep 35\p@
% \parindent\z@ \parskip\z@ plus.3\p@\relax\let\item\@idxitem}{\end{multicols}}
%
%\def\@idxitem{\par\hangindent 40\p@}
%
%\def\subitem{\par\hangindent 40\p@ \hspace*{20\p@}}
%
%\def\subsubitem{\par\hangindent 40\p@ \hspace*{30\p@}}
%
%\def\indexspace{\par \vskip 10\p@ plus5\p@ minus3\p@\relax}

\makeatother

%% macros for special words
%%
%\def\HOL{{\sc  hol}}
%\def\CHOL{{\sc hol88}}

%% Redfine constfont. The font cmssc12 is a vertual font based on
%% cmss12. The only difference is that the character at '137 becomes
%% an underline in cmssc12.
\font\sfc=cmssc12 \def\constfont{\sfc}
\def\ul#1{$\underline{#1}$}


\def\_{\char'137}


% ---------------------------------------------------------------------
% set caption at the foot of pages for this series of slides
% ---------------------------------------------------------------------
\ftext{Primitive Recursion and Induction}{9}

% ---------------------------------------------------------------------
% Slides
% ---------------------------------------------------------------------
\begin{document}

% ---------------------------------------------------------------------
% Title page for this series of slides
% ---------------------------------------------------------------------

\bsectitle
Primitive Recursion\\
and\\
Induction
\esectitle

% =====================================================================
\slide{Proof by Induction}

\point{For example, to prove:}

\vskip 5mm
\bspindent\LARGE\bf
$\forall n.\: \forall m.\: n + m = m + n$
\espindent
\vskip5mm
\bpindent\LARGE\bf
it is sufficient to prove that
\epindent
\vskip5mm
\bspindent\LARGE\bf
$\forall m.\: 0 + m = m + 0$
\vskip5mm
\bpindent\LARGE\bf
and
\epindent
\vskip5mm
$\begin{array}[t]{@{}l}
\forall m.\: n + m = m+n \;\\
\;\;\;\;\;\;\;\supset\\
 \forall m.\; (SUC\; n) + m = m + (SUC\; n)\end{array}$
\espindent

\vskip7mm

\point{HOL has built-in tactics and inference rules for this kind of inductive
proof.}

% =====================================================================
\slide{Proof by Mathematical Induction}

\point{In general, to prove that $\forall n.\: P[n]$ it is sufficient to prove
that:}

\vskip 5mm
\subpoint{$P[0]$}
\vskip 5mm
\bpindent\LARGE\bf
and \epindent
\vskip 5mm
\subpoint{$\forall n.\: P[n] \supset P[SUC\;n]$}

\vskip7mm

\point{The HOL inference rule {\tt INDUCT} implements \\
this induction principle.}

% =====================================================================
\slide{Inference Rule for Induction}

\point{The rule is {\tt INDUCT : (thm \# thm) -> thm}}


\vskip7mm
\point{Given the theorems:}
\vskip6mm
\bspindent\LARGE\bf
\verb+|- P[0]+ \quad and\quad \verb+|- !n. P[n] ==> P[SUC n]+
\espindent
\vskip6mm
\bpindent\LARGE\bf
\verb!INDUCT! returns the theorem \verb+|- !n.P[n]+
\epindent

\vskip7mm
\point{Example:}
\vskip4mm
\begin{session}\begin{verbatim}
#base;;
 |- !m. (0 + m) = (m + 0)

#step;;
 |- !n. (!m. n + m = m + n) ==>
        (!m. SUC n + m = m + SUC n)

#INDUCT(base, step);;
 |- !n. !m. n + m = m + n
\end{verbatim}\end{session}

% =====================================================================
\slide{The Induction Tactic}

\point{There is also a tactic  for induction: {\tt INDUCT\_TAC}}
\vskip7mm
\point{Given a goal of the form:}
\vskip6mm
\bspindent\LARGE\bf
\verb+!n:num. P n+
\espindent
\vskip6mm
\bpindent\LARGE\bf
\verb!INDUCT_TAC! breaks it into two subgoals:
\epindent
\vskip6mm
\subpoint{the base case: {\tt [], "P 0"}}
\vskip 4mm
\subpoint{the step case: {\tt [P n], "P (SUC n)"}}
\vskip7mm
\point{For example:}
\begin{session}\begin{verbatim}
#let [g1;g2],p = INDUCT_TAC([],"!n. n >= 0");;
g1 = ([], "0 >= 0") : goal
g2 = (["n >= 0"], "(SUC n) >= 0") : goal
p = - : proof
\end{verbatim}\end{session}


% =====================================================================
\slide{Induction for Lists}

\point{HOL also has a rule and a tactic for structural induction on lists.}

\vskip7mm
\point{To prove $\forall l.\: P[l]$ it is sufficient to show:}
\vskip 5mm
\subpoint{$P[Nil]$ \quad and \quad
$\forall l.\: P[l] \supset \forall h.\: P[CONS\;h\;l]$}

\vskip7mm
\point{The inference rule is}
\vskip 6mm
\bspindent\LARGE\bf
\verb!LIST_INDUCT : (thm # thm) -> thm!
\espindent
\vskip 6mm
\bpindent\LARGE\bf
and the tactic is:
\epindent
\vskip 6mm
\bspindent\LARGE\bf
\verb!LIST_INDUCT_TAC : tactic!
\espindent

\vskip7mm
\point{These are similar to {\tt INDUCT} and {\tt INDUCT\_TAC}}

% =====================================================================
\slide{Definition by Primitive Recursion}

\point{Recursive definitions are not allowed by the HOL primitive rules of
definition.}
\vskip7mm
\point{The problem with recursion is that one could 
give a `defining' equation, for example:}
\vskip6mm 
\bspindent \LARGE $f(x) \; = \; (f\;x) + 1$\espindent
\vskip6mm
\bpindent\LARGE\bf
that is not satisfied by any total function.
\epindent
\vskip7mm
\point{We must instead {\it derive\/} recursive definitions from equivalent
non-recursive ones.}
\vskip7mm
\point{For {\it primitive recursive} functions, the system does this
automatically.}


% =====================================================================
\slide{Primitive Recursion}

\point{A primitive recursive definition has the form:}
\vskip6mm
\bspindent\LARGE\bf
$f(0) \;\; = \;\; E_0$
\vskip4mm
$f(SUC \; n) \;\; = \;\; E_1 [f \; n, \; n]$
\espindent
\vskip6mm
\bpindent\LARGE\bf
where\epindent
\vskip6mm
\subpoint{the first equation is the {\it base case\/}, and}
\subpoint{the second is the {\it recursive case}}

\vskip7mm
\point{Example:}
\vskip 6mm
\bspindent\LARGE\bf
$plus\; 0 = \lambda m.\: m$
\vskip 4mm
$plus\; (SUC\;n) = \lambda m.\: SUC((plus\; n)\;m)$
\espindent

\vskip7mm
\point{In more familiar notation:}
\vskip 6mm
\bspindent\LARGE\bf
$0 + m = m$
\vskip 4mm
$(SUC\;n) + m = SUC(n + m)$
\espindent

% =====================================================================
\slide{Primitive Recursion in HOL}

\point{The derived rule is:}

\vskip 5mm
\bspindent\Large\bf
\verb!new_prim_rec_definition: (string # term) -> thm!
\espindent
\vskip 5mm
\bpindent\LARGE\bf
where:
\epindent
\vskip 5mm
\subpoint{the string is the name under which the definition will be stored in
the current theory,}
\subpoint{the term gives the desired definition, and}
\subpoint{the result is the definition as a theorem.}

\vskip7mm

\point{Example:}
\vskip4mm
\begin{session}\begin{verbatim}
#new_prim_rec_definition
  (`plus_def`,
   "(plus 0 m = m) /\
    (plus (SUC n) m = SUC(plus n m))");;
|- (!m. plus 0 m = m) /\ 
   (!n m. plus(SUC n)m = SUC(plus n m))
\end{verbatim}\end{session}

\vskip5mm

\point{Definitions allowed only in draft mode.}


% =====================================================================
\slide{Primitive Recursion}

\point{The ML function }

\vskip 6mm
\bspindent\LARGE\bf
\verb!new_prim_rec_definition!
\espindent
\vskip 6mm

\subpoint{checks that the given equations are 
primitive\\ recursive, and}

\subpoint{proves that there is a function that satisfies them.}

\vskip7mm

\point{There is also a function for defining infixes:}

\vskip 6mm
\bspindent\LARGE\bf
\verb!new_infix_prim_rec_definition!
\espindent
\vskip 7mm
\point{For example:}
\vskip4mm
\begin{session}\begin{verbatim}
#new_infix_prim_rec_definition
   (`iplus`,
    "(++ 0 n = n) /\ 
     (++ (SUC n) m = SUC(++ n m))");;
|- (!n. 0 ++ n = n) /\ 
   (!n m. (SUC n) ++ m = SUC(n ++ m))
\end{verbatim}\end{session}

% =====================================================================
\slide{Primitive Recursion on Lists}

\point{There are corresponding derived rules of\\
definition for primitive recursion on lists:}

\vskip 7mm
\bspindent\LARGE\bf
\verb!new_list_rec_definition!
\vskip 4mm
\verb!new_infix_list_rec_definition!
\espindent
\vskip7mm
\point{For example, the {\tt map} function is defined by:}
\vskip4mm
\begin{session}\begin{verbatim}
#new_list_rec_definition
  (`map_def`,
   "(map (f:*->**) ([]:* list) = ([]:** list)) /\
    (map f (CONS h t) = CONS (f h) (map f t))");;
|- (!f. map f[] = []) /\
   (!f h t. map f(CONS h t) = CONS(f h)(map f t))
\end{verbatim}\end{session}

% =====================================================================
\slide{Summary}

\point{Mathematical induction:}

  \subpoint{{\tt INDUCT}}
  \subpoint{{\tt INDUCT\_TAC}}

\point{Structural induction on lists:}

  \subpoint{{\tt LIST\_INDUCT}}
  \subpoint{{\tt LIST\_INDUCT\_TAC}}

\point{Primitive recursive definitions:}

  \subpoint{{\tt new\_prim\_rec\_definition}}
  \subpoint{{\tt new\_infix\_prim\_rec\_definition}}


\point{Primitive recursion on lists:}

  \subpoint{{\tt new\_list\_rec\_definition}}
  \subpoint{{\tt new\_infix\_list\_rec\_definition}}


\end{document}
