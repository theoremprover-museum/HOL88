% =====================================================================
% HOL Course Slides: introductory slides              (c) T Melham 1990
% =====================================================================

\documentstyle[12pt,layout]{article}

% ---------------------------------------------------------------------
% Preliminary settings.
% ---------------------------------------------------------------------

\renewcommand{\textfraction}{0.01}	  % 0.01 of the page must contain text
\setcounter{totalnumber}{10}	 	  % max of 10 figures per page
\flushbottom				  % text extends right to the bottom
\pagestyle{slides}			  % slides page style
\setlength{\unitlength}{1mm}		  % unit = 1 mm

% ---------------------------------------------------------------------
% load macros
% ---------------------------------------------------------------------


\input{holmacs}
\input{tokmac}

%\newtokmac{ter}{\tt}
%\newtokmac{nter}{\tt}
\newtokmac{mlname}{\tt}
\newtokmac{CONST}{\constfont}


\newenvironment{hproof}{\begin{center}
 \begin{tabular*}{5.25in}{r>{$}l<{$}@{\extracolsep{0pt plus 1fill}}>{[}r<{]}}}
 {\end{tabular*}\end{center}}

\makeatletter

%% \verbatimnumbered{<file>}
%% read a file and format the contains in a verbatim environment with
%% line number.
%\newcounter{VerbatimLineNo}
%\def\verbatimnumbered#1{\begingroup
%  \setcounter{VerbatimLineNo}{0}%
%  \def\verbatim@processline{%
%    \addtocounter{VerbatimLineNo}{1}%
%    \leavevmode
%    \llap{\theVerbatimLineNo
%          \ \hskip\@totalleftmargin}%
%    \the\verbatim@line\par}%
%  \verbatiminput{#1}\endgroup}

%% Redefine the theindex environment
%%
%\renewenvironment{theindex}{\begin{multicols}{2}[\section*{\indexname}]%
% \columnseprule \z@ \columnsep 35\p@
% \parindent\z@ \parskip\z@ plus.3\p@\relax\let\item\@idxitem}{\end{multicols}}
%
%\def\@idxitem{\par\hangindent 40\p@}
%
%\def\subitem{\par\hangindent 40\p@ \hspace*{20\p@}}
%
%\def\subsubitem{\par\hangindent 40\p@ \hspace*{30\p@}}
%
%\def\indexspace{\par \vskip 10\p@ plus5\p@ minus3\p@\relax}

\makeatother

%% macros for special words
%%
%\def\HOL{{\sc  hol}}
%\def\CHOL{{\sc hol88}}

%% Redfine constfont. The font cmssc12 is a vertual font based on
%% cmss12. The only difference is that the character at '137 becomes
%% an underline in cmssc12.
\font\sfc=cmssc12 \def\constfont{\sfc}
\def\ul#1{$\underline{#1}$}



% ---------------------------------------------------------------------
% set caption at the foot of pages for this series of slides
% ---------------------------------------------------------------------
\ftext{Introduction}{1}

% ---------------------------------------------------------------------
% Slides
% ---------------------------------------------------------------------
\begin{document}

% ---------------------------------------------------------------------
% Title page for this series of slides
% ---------------------------------------------------------------------

\bsectitle
Introduction\\
to the\\
HOL Theorem Prover\\
\esectitle

\vskip20mm

\begin{center}
\large\bf
T F Melham
\end{center}
\vskip10mm
\begin{center}
\bf
University of Cambridge\\
Computer Laboratory\\
New Museums Site\\
Pembroke Street\\
Cambridge, CB2 3QG\\
England
\end{center}

% =====================================================================
\slide{What is HOL?}

\point{A system for proving theorems in 
Higher\\ Order Logic.}

\point{Features of HOL:}
\subpoint{has a rigorous formal basis}
\subpoint{supports both `forward' and goal-directed proof}
\subpoint{secure: can't prove false theorems}
\subpoint{embedded in a general purpose functional\\ programming language (ML)}
\subpoint{user-extendable, without compromising security}
\subpoint{NOT an `automatic' theorem prover}

% =====================================================================
\slide{History of HOL}

\vskip10mm

\begin{center}
\def\_{\leavevmode \kern-0.5mm \vbox{\hrule height0.2mm width0.3em}}
\setlength{\unitlength}{1mm}
\begin{picture}(100,120)
{
\thicklines
\put(45,55){\line(-3,-1){30}}
\put(55,55){\line( 3,-1){30}}
\put(85,19){\line( 0,-1){10}}
\put(40,95){\line(-3,-1){30}}
\put(50,95){\line( 0,-1){30}}
\put(50,115){\line( 0,-1){10}}

\put(1,16){\bf {\Large{\bf 
  {\shortstack{LCF{\kern-.3mm}\_{\kern.7mm}LSM\\[2mm]
 (logic for\\ sequential\\ machines)}}}}}
\put(85,5){\makebox(0,0)[c]{\Large{\bf HOL88}}}
\put(65,21){\Large{\bf {\shortstack{ HOL\\[2mm] (Higher Order\\Logic) }}}}
\put(50,60){\makebox(0,0)[c]{\Large{\bf Cambridge LCF (PP\kern-1mm$\lambda$)}}}
\put(10,80){\makebox(0,0)[c]{\Large{\bf Standard ML}}}
\put(50,100){\makebox(0,0)[c]{\Large{\bf Edinburgh LCF (ML, PP\kern-1mm$\lambda$)}}}
\put(50,120){\makebox(0,0)[c]{\Large{\bf Stanford LCF}}} 
}
\end{picture}
\end{center}
\vskip 7mm

% =====================================================================
\slide{HOL and ML}

\point{HOL is built on top of ML.}


\point{Roughly speaking, HOL = ML plus:}

\vskip7mm
\bspindent\LARGE
{\bf some predefined ML programs (functions),} 
\vskip7mm
{\bf and some data type declarations.}
\espindent

\point{There are also a few enhancements to the ML parser and pretty-printer.}

% =====================================================================
\slide{Tools and System Support}

\point{HOL tools include:}

\subpoint{support for goal-directed proof}
\subpoint{many built-in inference rules}
\subpoint{automatic recursive type definitions}
\subpoint{structural induction tools}
\subpoint{rewriting tools (from LCF)}
\subpoint{automatic primitive recursive definitions}
\subpoint{built-in theories of arithmetic, lists, sets,\dots}
\subpoint{tautology checker}
\subpoint{automatic inductive definitions}
\subpoint{parser and pretty-printer generator}
\subpoint{online help facility}
\subpoint{full documentation}
\subpoint{user-loadable libraries}


% =====================================================================
\slide{Applications}

\point{HOL applications include:}

\subpoint{hardware design and verification}
\subpoint{reasoning about security}
\subpoint{verification of fault-tolerant computers}
\subpoint{reasoning about real-time system}
\subpoint{semantics of HDLs (e.g.\ VHDL, ELLA)}
\subpoint{compiler verification}
\subpoint{program refinement calculus}
\subpoint{software verification (e.g.\ Hoare logic)}
\subpoint{modelling concurrency (e.g.\ CCS, CSP)}
\subpoint{automata theory}
\subpoint{\dots\ et cetera}


% =====================================================================
\slide{Course Outline}

\point{Overview of higher order logic}

\subpoint{Syntax of the logic}

\subpoint{Primitive basis of the logic}

\point{Brief introduction to ML}

\point{Introduction to the HOL system}

\subpoint{Embedding the logic in ML}

\subpoint{Inference rules, theorems and proof}

\subpoint{The core system}

\point{Forward proof in HOL}

\point{Goal-directed proof in HOL}

\point{Primitive recursion and induction}

\point{The recursive types package}

\end{document}
