% =====================================================================
% HOL Course Slides: overview of higher order logic   (c) T melham 1990
% =====================================================================

\documentstyle[12pt,layout]{article}

% ---------------------------------------------------------------------
% Preliminary settings.
% ---------------------------------------------------------------------

\renewcommand{\textfraction}{0.01}	  % 0.01 of the page must contain text
\setcounter{totalnumber}{10}	 	  % max of 10 figures per page
\flushbottom				  % text extends right to the bottom
\pagestyle{slides}			  % slides page style
\setlength{\unitlength}{1mm}		  % unit = 1 mm

% ---------------------------------------------------------------------
% load macros
% ---------------------------------------------------------------------


\input{holmacs}
\input{tokmac}

%\newtokmac{ter}{\tt}
%\newtokmac{nter}{\tt}
\newtokmac{mlname}{\tt}
\newtokmac{CONST}{\constfont}


\newenvironment{hproof}{\begin{center}
 \begin{tabular*}{5.25in}{r>{$}l<{$}@{\extracolsep{0pt plus 1fill}}>{[}r<{]}}}
 {\end{tabular*}\end{center}}

\makeatletter

%% \verbatimnumbered{<file>}
%% read a file and format the contains in a verbatim environment with
%% line number.
%\newcounter{VerbatimLineNo}
%\def\verbatimnumbered#1{\begingroup
%  \setcounter{VerbatimLineNo}{0}%
%  \def\verbatim@processline{%
%    \addtocounter{VerbatimLineNo}{1}%
%    \leavevmode
%    \llap{\theVerbatimLineNo
%          \ \hskip\@totalleftmargin}%
%    \the\verbatim@line\par}%
%  \verbatiminput{#1}\endgroup}

%% Redefine the theindex environment
%%
%\renewenvironment{theindex}{\begin{multicols}{2}[\section*{\indexname}]%
% \columnseprule \z@ \columnsep 35\p@
% \parindent\z@ \parskip\z@ plus.3\p@\relax\let\item\@idxitem}{\end{multicols}}
%
%\def\@idxitem{\par\hangindent 40\p@}
%
%\def\subitem{\par\hangindent 40\p@ \hspace*{20\p@}}
%
%\def\subsubitem{\par\hangindent 40\p@ \hspace*{30\p@}}
%
%\def\indexspace{\par \vskip 10\p@ plus5\p@ minus3\p@\relax}

\makeatother

%% macros for special words
%%
%\def\HOL{{\sc  hol}}
%\def\CHOL{{\sc hol88}}

%% Redfine constfont. The font cmssc12 is a vertual font based on
%% cmss12. The only difference is that the character at '137 becomes
%% an underline in cmssc12.
\font\sfc=cmssc12 \def\constfont{\sfc}
\def\ul#1{$\underline{#1}$}


\def\_{\char'137}


% ---------------------------------------------------------------------
% Macro for a tricky underbrace
% ---------------------------------------------------------------------
\def\rul#1{\vrule width0.25mm height#1mm depth-1mm}
\newbox\ubox
\def\und#1#2{%
    \setbox\ubox=\hbox{${\stackrel{\rul{#2}}{\scriptstyle\hbox{\rm #1}}}$}%
    {\wd\ubox=0mm\box\ubox}}


% ---------------------------------------------------------------------
% set caption at the foot of pages for this series of slides
% ---------------------------------------------------------------------
\ftext{The HOL System}{5}

% ---------------------------------------------------------------------
% Slides
% ---------------------------------------------------------------------
\begin{document}

% ---------------------------------------------------------------------
% Title page for this series of slides
% ---------------------------------------------------------------------

\bsectitle
Introduction \\
to the\\
HOL System
\esectitle

% =====================================================================
\slide{Outline}

\point{HOL = Higher Order Logic embedded in ML.}

\point{How the logic is embedded in ML:}

\subpoint{Terms represented by ML type {\tt term}}

\subpoint{Types represented by ML type {\tt type}}

\subpoint{Theorems represented by ML type {\tt thm}}

\point{Inference rules in HOL:}

\subpoint{ML functions that yield values of type {\tt thm}}

% =====================================================================
\slide{The ML Representation of Terms}

\point{Higher order logic terms are represented by values of ML type {\tt
term}.}

\point{Term literals are written in double quotes:}
\vskip4mm
\begin{session}\begin{verbatim}
#"T /\ F ==> T";;
"T /\ F ==> T" : term

#["~T"; "x=7"];;
["~T"; "x = 7"] : term list

#("T","3");;
("T", "3") : (term # term)

#"T,3";;
"T,3" : term
\end{verbatim}\end{session}


% =====================================================================
\slide{Ascii Notation for Terms}

\point{The following transliteration is used:}
\vskip7mm

\bpindent\setlength{\arrayrulewidth}{\lwid}
\renewcommand{\arraystretch}{1.2}
{\LARGE{\bf
\vskip 7mm
\begin{center}
\begin{tabular}{| l | l|}
\hline 
Logic    & HOL \\ \hline
$\lambda x. t$   & \verb%"\x. t"% \\
$\forall x. \; t$ & \verb%"!x. t"% \\
$\exists x. \; t$ & \verb%"?x. t"% \\
$\exists{\em !} x. \; t$ & \verb%"?!x. t"% \\
$\epsilon x.\; t$& \verb%"@x. t"% \\
$\neg x$         & \verb%"~x"% \\
$t_1 \wedge t_2$ & \verb%"t1 /\ t2"% \\
$t_1 \vee t_2$   & \verb%"t1 \/ t2"% \\
$t_1 \supset t_2$& \verb%"t1 ==> t2"% \\
$(t_1 \rightarrow t_2 \mid t_3)$ & \verb%"(t1 => t2 | t3)"% \\ \hline
\end{tabular}
\end{center}
}}
\epindent

% =====================================================================
\slide{Term Literals}

\point{Examples:}
\vskip4mm
\begin{session}\begin{verbatim}
#"!b. (b=T) \/ (b=F)";;
"!b. (b = T) \/ (b = F)" : term

#let tm = "?x. x = (\b. b)F";;
tm = "?x. x = (\b. b)F" : term
\end{verbatim}\end{session}
\vskip5mm

\point{Terms must be well-typed:}
\vskip4mm
\begin{session}\begin{verbatim}
#"7 + T";;
Badly typed application of:  "$+ 7"
   which has type:           ":num -> num"
to the argument term:        "T"
   which has type:           ":bool"

evaluation failed     mk_comb in quotation
\end{verbatim}\end{session}

% =====================================================================
\slide{The Representation of Types in ML}

\point{The ML type {\tt type} represents logical types:}
\vskip4mm
\begin{session}\begin{verbatim}
#":bool";;
":bool" : type

#":num -> num";;
":num -> num" : type
\end{verbatim}\end{session}

\point{The function {\tt type\_of}:}
\vskip4mm
\begin{session}\begin{verbatim}
#type_of "T \/ b";;
":bool" : type
\end{verbatim}\end{session}

\point{The parser sometimes fails to infer types:}
\vskip4mm
\begin{session}\begin{verbatim}
#"f a = b";;
Indeterminate types:  "$=:?1 -> (?2 -> bool)"

evaluation failed  types indeterminate in quotation
\end{verbatim}\end{session}


\point{You must explictly enter polymorphic types:}
\vskip4mm
\begin{session}\begin{verbatim}
#let tm = "(f:*->**) a = b";;
tm = "f a = b" : term
\end{verbatim}\end{session}



% =====================================================================
\slide{Printing Types}

\point{You can ask the system to show types when printing terms:}
\vskip4mm
\begin{session}\begin{verbatim}
#show_types true;;
false : bool

#let tm = "(f:*->**) a = b";;
tm = "(f:* -> **) a = b" : term

#"!b. b=T";;
"!(b:bool). b = T" : term

#"\x. x+1";;
"\(x:num). x + 1" : term
\end{verbatim}\end{session}

% =====================================================================
\slide{Type-checking Errors}


\point{Badly-typed applications:}
\vskip4mm
\begin{session}\begin{verbatim}
#"SUC T";;
Badly typed application of:  "SUC"
   which has type:           ":num -> num"
to the argument term:        "T"
   which has type:           ":bool"

evaluation failed     mk_comb in quotation

#"x /\ (x=1)";;
Badly typed application of:  "$= x"
   which has type:           ":bool -> bool"
to the argument term:        "1"
   which has type:           ":num"

evaluation failed     mk_comb in quotation
\end{verbatim}\end{session}

\point{Illegal types:}
\vskip4mm
\begin{session}\begin{verbatim}
#"!x:foo->bar. x = x";;
evaluation failed     mk_type in quotation

#"!P:bool Q. P /\ Q";;
evaluation failed     mk_type in quotation
\end{verbatim}\end{session}

% =====================================================================
%\slide{Type Variables}
%
%\point{The ?1, ?2 etc. on the last slide indicate indeterminate types.}
%
%\point{Type variables are not considered to be indeterminate.}
%
%\vskip7mm
%\vskip4mm
%\begin{session}\begin{verbatim}
%#"INL(x:*) INL";;
%Badly typed application of:  "INL x"
%   which has type:           ":* + ?"
%to the argument term:        "INL"
%   which has type:           ":?1 -> ?1 + ?2"
%
%evaluation failed     mk_comb in quotation
%\end{verbatim}\end{session}
%
%\point{These messages can be switched off by setting the flag
%`type\_error` to false.}

% =====================================================================
\slide{Warning}

\point{Do not confuse terms with ML programs:}
\vskip4mm
\begin{session}\begin{verbatim}
#"(\x. x+7) 3";;
"(\x. x + 7)3" : term

#(\x. x+7) 3;;
10 : int

#"(\x. x+7)" 3;;

ill-typed phrase: " ... "
has an instance of type  term
which should match type  (* -> **)
1 error in typing
typecheck failed     
\end{verbatim}\end{session}

\point{Do not confuse ML types and logical types:}
\vskip4mm
\begin{session}\begin{verbatim}
#"3 : int";;
evaluation failed     mk_type in quotation

#"3" : int;;

ill-typed phrase: " ... "
has an instance of type  term
which should match type  int
1 error in typing
typecheck failed     
\end{verbatim}\end{session}



% =====================================================================
\slide{Antiquotation}

\point{This allows an ML expression which evaluates to a term to occur inside
the quotation marks.}

\vskip7mm

\point{Items tagged with `{\tt\char'136}' are evaluated:}
\vskip4mm
\begin{session}\begin{verbatim}
#let t = "f:num->num";;
t = "f" : term

#"(t 0):num";;
"t 0" : term

#"^t 0";;
"f 0" : term

#"^t 0 + t 1";;
"(f 0) + (t 1)" : term
\end{verbatim}\end{session}

\point{In {\tt "\dots \char'136$\langle{\it item\/}\rangle$ \dots"}, the
$\langle{\it item\/}\rangle$ must be an ML\\ expression that can be evaulated:}
\vskip4mm
\begin{session}\begin{verbatim}
#"^x /\ y";;

unbound or non-assignable variable x
1 error in typing
typecheck failed     
\end{verbatim}\end{session}


% =====================================================================
\slide{Let-terms}

\point{The term parser accepts {\tt let}-terms like those of local declarations
in ML:}

\vskip4mm
\begin{session}\begin{verbatim}
#"let x=1 and y=2 in x+y";;
"let x = 1 and y = 2 in x + y" : term
\end{verbatim}\end{session}

\point{{\tt let}-terms are just abbreviations for ordinary terms
of the logic.}

\point{They are encoded using the constant {\tt LET}:}

\vskip 5mm
\bspindent\LARGE
\verb!|- LET = (\f x. f x)!
\espindent
\vskip 5mm

\bpindent\LARGE\bf For example:\epindent


\vskip 5mm
\bspindent\LARGE\bf
\verb!"let x = 1 in x+2"!
\espindent
\vskip5mm

\bpindent\LARGE\bf parses to:
\epindent
\vskip5mm

\vskip 5mm
\bspindent\LARGE\bf
\verb!"LET(\x. x + 2)1"!
\espindent

% =====================================================================
\slide{Parser Support for Let-terms}


\point{The parser applies the transformations:}

\vskip7mm
\vskip4mm
\bspindent
\obeylines\Large\bf
    \verb!"let f v1...vn = t1 in t2"!
    \begin{picture}(2,11)(-8,0)\thicklines
      \put(1,8){\vector(0,-1){8}}
    \end{picture}
    \verb!"LET (\f. t2) (\v1...vn. t1)"!
\vskip10mm
    \verb!"let (v1,...,vn) = t1 in t2"!
    \begin{picture}(2,11)(-8,0)\thicklines
      \put(1,8){\vector(0,-1){8}}
    \end{picture}
    \verb!"LET (\(v1,...,vn).t2) t1"!
\vskip10mm
    \verb!"let v1=t1 and...and vn=tn in t"!
    \begin{picture}(2,11)(-8,0)\thicklines
      \put(1,8){\vector(0,-1){8}}
    \end{picture}
    \verb!"LET(...(LET(LET (\v1...vn.t) t1)t2)...)tn"!
\espindent

\vskip11mm
\bpindent\LARGE\bf
and the pretty-printer inverts them.
\epindent


% =====================================================================
\slide{Printing Let-terms}

\point{The {\tt print\_let} flag controls the pretty-printing of {\tt
let}-terms:}
\vskip4mm
\begin{session}\begin{verbatim}
#"let x=1 and y=2 in x+y";;
"let x = 1 and y = 2 in x + y" : term

#set_flag(`print_let`,false);;
true : bool

#"let x=1 and y=2 in x+y";;
"LET(LET(\x y. x + y)1)2" : term

#set_flag(`print_let`,true);;
false : bool

#"let x=1 and y=2 in x+y";;
"let x = 1 and y = 2 in x + y" : term
\end{verbatim}\end{session}


% =====================================================================
\slide{Paired Abstractions}

\point{The following terms are allowed by the parser:}

\vskip7mm
\vskip4mm
\begin{center}
\bspindent\LARGE \bf
\begin{verbatim}
      "\(x,y). x + y"

      "!(x,y),(p,q). (x = p) /\ (y = q)"

      "?((x,y),(p,q,r)). (x+y) < ((p+q)*r)"
\end{verbatim} 
\espindent
\end{center}

\vskip5mm

\point{These are just surface syntax: they are parsed into standard terms}

\vskip7mm

\point{The transformation uses the constant {\tt UNCURRY}:}

\vskip 7mm
\vskip 4mm
\bspindent\LARGE\bf
\verb+|- !f x y. UNCURRY f(x,y) = f x y+
\espindent


% =====================================================================
\slide{Paired Abstractions}

\point{The parser removes all variable structure by applying recursively the
transformation:}

\vskip 10mm
\bspindent
\Large
      \verb!"\(v1,v2).t"!
    \begin{picture}(13,2)(-2,-1)\thicklines
      \put(1,1){\vector(1,0){8}}
    \end{picture}
      \verb!"UNCURRY(\v1. (\v2.t))"!
\espindent
\vskip 10mm

\bpindent{\LARGE{\bf
where \verb%"v1"% and \verb%"v2"% are variables, or tuples of variables
(called {\it variable structures}).
}}
\epindent

\vskip 5mm
\point{The transformation algorithm is:}
\vskip 7mm
\bspindent
\obeylines\Large\bf
    \verb!"\(x,y).t"!
    \begin{picture}(2,11)(-8,0)\thicklines
      \put(1,8){\vector(0,-1){8}}
    \end{picture}
    \verb!"UNCURRY(\x y. t)"!
\vskip8mm
    \verb!"\(x1,x2,...,xn).t"!
    \begin{picture}(2,11)(-8,0)\thicklines
      \put(1,8){\vector(0,-1){8}}
    \end{picture}
    \verb!"UNCURRY(\x.\(x2,...,xn).t)"!
\vskip8mm
    \verb!"\((x1,...,xn),y1,...,ym).t" !
    \begin{picture}(2,11)(-8,0)\thicklines
      \put(1,8){\vector(0,-1){8}}
    \end{picture}
    \verb!"UNCURRY(\(x1,...,xn). \y1,...,ym).t"!
\espindent


% =====================================================================
\slide{Printing Paired Abstractions}

\point{The {\tt print\_uncurry} flag controls the pretty-printing of 
paired abstractions:}

\vskip4mm
\begin{session}\begin{verbatim}
#"\(P,Q). P /\ Q";;
"\(P,Q). P /\ Q" : term

#set_flag(`print_uncurry`,false);;
true : bool

#"\(P,Q). P /\ Q";;
"UNCURRY(\P Q. P /\ Q)" : term
\end{verbatim}\end{session}



% =====================================================================
\slide{Summary}

\point{HOL has the two ML data types:}

\subpoint{{\tt term}: to represent logical terms}

\subpoint{{\tt type}: to represent logical types}

\point{Term and type literals in quotations:}

\subpoint{e.g.\ {\tt ":bool->bool"}}
\subpoint{e.g.\ {\tt "P ==> Q"}}

\point{Parser and pretty-printer support:}

\subpoint{for {\tt let}-terms (the constant {\tt LET})}
\subpoint{for paired abstractions (the constant {\tt UNCURRY})}

\point{Built-in ML functions:}

\subpoint{{\tt type\_of : term -> type}}
\subpoint{{\tt show\_types : bool -> bool}}
\subpoint{{\tt set\_flag : (string \# bool) -> bool}}

% =====================================================================
\slide{Syntax Functions}

\point{HOL provides functions for }

\subpoint{constructing terms and types}

\subpoint{taking terms and types apart}

\subpoint{testing the syntactic classes of terms and types}

\point{For example:}
\vskip4mm
\begin{session}\begin{verbatim}
#is_var "x:bool";;
true : bool

#is_var "\x. x+y";;
false : bool

#mk_comb ("f:bool -> bool", "x:bool");;
"f x" : term

#dest_forall "!x. x ==> F";;
("x", "x ==> F") : (term # term)

#dest_forall "?x. x";;
evaluation failed     dest_forall
\end{verbatim}\end{session}

% =====================================================================
\slide{Primitive Syntax Functions}

\def\m#1{\(#1\)}

\point{Syntax functions for types:}
\vskip5mm
\bspindent\Large
\obeylines
\begin{tabular}{@{\hskip3mm}l@{\hskip3mm$=$\hskip3mm}l@{}}
{\tt mk\_vartype `*\m{\dots}`}     & {\tt ":*\m{\dots}"}\\
{\tt mk\_type(`\m{op}`,[\m{ty_1};\dots;\m{ty_n}])} &
    {\tt ":(\m{ty_1},\dots,\m{ty_n})\m{op}"}
\end{tabular}
\vskip5mm

\begin{tabular}{@{\hskip3mm}l@{\hskip3mm$=$\hskip3mm}l@{}}
{\tt dest\_vartype ":*\m{\dots}"} & {\tt `*\m{\dots}`}\\
{\tt dest\_type ":(\m{ty_1},\dots,\m{ty_n})\m{op}"} &
    {\tt (`\m{op}`,[\m{ty_1};\dots;\m{ty_n}])}
\end{tabular}
\vskip5mm
\begin{tabular}{@{\hskip3mm}l@{\hskip3mm{iff}\hskip3mm}l}
{\tt is\_vartype ":\m{ty}"} &  {\tt :}\m{ty} is a type variable
\end{tabular}
\espindent

\point{Examples:}
\vskip4mm

\begin{session}\begin{verbatim}
#dest_type ":bool";;
(`bool`, []) : (string # type list)

#mk_type (`prod`,[":bool";":ind"]);;
":bool # ind" : type

#is_vartype ":*";;
true : bool

#is_vartype ":bool";;
false : bool
\end{verbatim}\end{session}

% =====================================================================
\slide{Primitive Syntax Functions}

\point{Term constructors:}
\vskip3mm
\bspindent\Large
\def\bk{{\tt\char'134}}
\def\eps{{\tt\char'100}}
\def\neg{{\tt\char'176}}

\begin{tabular}{@{\hskip3mm}l@{\hskip3mm$=$\hskip3mm}l@{}}
  {\tt mk\_abs("\m{v}","\m{t}")}     & {\tt "\bk\m{v}.\m{t}"}\\
  {\tt mk\_comb("\m{t_1}","\m{t_2}")}& {\tt "\m{t_1} \m{t_2}"}\\
  {\tt mk\_const(`\m{c}`,":\m{ty}")} & {\tt"\m{c}:\m{ty}"} \\
  {\tt mk\_var(`\m{v}`,":\m{ty}")}   & {\tt "\m{v}:\m{ty}"}
\end{tabular}
\espindent

\point{Term destructors:}
\vskip3mm
\bspindent\Large
\def\bk{{\tt\char'134}}
\def\eps{{\tt\char'100}}
\def\neg{{\tt\char'176}}

\begin{tabular}{@{\hskip3mm}l@{\hskip3mm$=$\hskip3mm}l}
{\tt dest\_abs "\bk\m{v}.\m{t}"}  &    {\tt ("\m{v}","\m{t}")} \\
{\tt dest\_comb "\m{t_1} \m{t_2}"}&    {\tt ("\m{t_1}","\m{t_2}")}\\
{\tt dest\_const "\m{c}:\m{ty}"}  &    {\tt (`\m{c}`,":\m{ty}")} \\
{\tt dest\_var "\m{v}:\m{ty}"}    &    {\tt (`\m{v}`,":\m{ty}")}
\end{tabular}
\espindent

\point{Term discriminators:}
\vskip3mm
\bspindent\Large\bf
\def\bk{{\tt\char'134}}
\def\eps{{\tt\char'100}}
\def\neg{{\tt\char'176}}

\begin{tabular}{@{\hskip3mm}l@{\hskip3mm{iff}\hskip3mm}l@{}}
{\tt is\_abs "\m{tm}"}   & 
   \m{tm} = {\tt "\bk\m{v}.\m{t}"} for some \m{v} and \m{t}\\
{\tt is\_const "\m{tm}" } & 
   \m{tm} = {\tt "\m{c}:\m{ty}"} for some \m{c} and \m{ty}\\
{\tt is\_comb "\m{tm}" }  & 
   \m{tm} = {\tt "\m{t_1} \m{t_2}"} for some \m{t_1} and \m{t_2}\\
{\tt is\_var  "\m{tm}"}  &   
   \m{tm} = {\tt "\m{v}:\m{ty}"} for some \m{v} and \m{ty} 
\end{tabular}
\espindent

\point{Examples:}
\begin{session}\begin{verbatim}
#dest_abs "\P. P ==> Q";;
("P", "P ==> Q") : (term # term)

#mk_var (`x`,":bool");;
"x" : term
\end{verbatim}\end{session}

% =====================================================================
\slide{Derived Syntax Functions}

\point{There are derived syntax functions for some non-primitive terms:}
\vskip4mm

\begin{session}\begin{verbatim}
#dest_forall "!P. P /\ Q";;
("P", "P /\ Q") : (term # term)

#mk_conj ("P:bool", "Q:bool");;
"P /\ Q" : term
\end{verbatim}\end{session}

\point{Naming convention:}

\subpoint{Prefixes: {\tt mk}, {\tt dest}, {\tt is}}

\subpoint{Roots:}
\bspindent\Large
\obeylines
\vskip4mm
\qquad {\tt pair}, {\tt list}, {\tt cond}
\qquad {\tt eq}, {\tt neq}, {\tt conj}, {\tt disj}, {\tt imp}
\qquad{\tt forall}, {\tt exists}, {\tt select}
\espindent

\point{Hence, for example:}

\bspindent\LARGE\bf
\begin{verbatim}
      mk_conj : term # term -> term
      is_conj : term -> bool
      dest_conj : term -> (term # term)
\end{verbatim}
\espindent

% =====================================================================
\slide{Iterated Syntax Functions}

\point{Various built-in iterated syntax functions\\ exist, for example:}
\vskip4mm
\begin{session}\begin{verbatim}
#strip_forall "!P Q R. P /\ Q /\ R";;
(["P"; "Q"; "R"], "P /\ Q /\ R") : goal

#list_mk_conj ["P:bool";"Q:bool";"R:bool"];;
"P /\ Q /\ R" : term
\end{verbatim}\end{session}

\vskip7mm

\point{Their names generally begin with `{\tt strip\_}' or with `{\tt
list\_}'.}

\vskip7mm

\point{See the manual for details \dots}

% =====================================================================
\slide{Other Syntax Functions}

\point{Generating a variant variable name:}
\vskip4mm
\begin{session}\begin{verbatim}
#variant ["x:bool";"y:bool";"z:bool"] "y:bool" ;;
"y'" : term
\end{verbatim}\end{session}


\point{Testing if a variable is free in a term:}
\vskip4mm
\begin{session}\begin{verbatim}
#free_in "Q:bool" "P /\ Q";;
true : bool

#free_in "Q:bool" "!P Q. P /\ Q";;
false : bool
\end{verbatim}\end{session}


\point{Get the free variables of a term:}
\vskip4mm
\begin{session}\begin{verbatim}
#frees "!Q. P ==> (Q /\ R)";;
["P"; "R"] : term list
\end{verbatim}\end{session}

\point{Test for alpha-equivalence:}
\vskip4mm
\begin{session}\begin{verbatim}
#aconv "!P Q. P /\ Q" "!x y. x /\ y";;
true : bool
\end{verbatim}\end{session}


% =====================================================================
\slide{Matching and Instantation}

\point{Example:}

\vskip4mm
\begin{session}\begin{verbatim}
#let tm1 = "\x:*. ((x = y) => (a:**) | b)"
tm1 = "\x. ((x = y) => a | b)" : term

#let s,i = match tm1 "\b. ((b = T) => P | F)";;
s = [("F", "b"); ("P", "a"); ("T", "y")] 
      : (term # term) list
i = [(":bool", ":**"); (":bool", ":*")] 
      : (type # type) list

#show_types true;;
false : bool

#let tm2 = inst [] i tm1;;
tm2 = "\(x:bool). ((x = y) => a | b:bool)" : term

#subst s tm2;;
"\(x:bool). ((x = T) => P | F)" : term
\end{verbatim}\end{session}

% =====================================================================
\slide{Sequents}

\point{A sequent is just a pair $(\Gamma,t)$, where}

\subpoint{$\Gamma$ is set of formulas and}
\subpoint{$t$ is a formula (term of type bool)}

\vskip7mm

\point{In HOL, sequents are just pairs of ML type:}

\vskip 7mm
\bspindent\LARGE 
$\underbrace{\mathstrut \verb!term!\;\;\verb!list!}_{\und{assumptions}{10}}\;\;
\verb!#!\;\;
\underbrace{\mathstrut \verb!term!}_{\und{conclusion}{15}}$
\espindent
\vskip 7mm

\point{Note, in HOL sequents have a {\it list\/} of \\
assumptions, not a {\it
set} of assumptions.}


% =====================================================================
\slide{Sequents in HOL}

\point{The ML type {\tt goal} is just an abbreviation for  
{\tt term list \# term}.}

\point{Examples:}
\vskip4mm
\begin{session}\begin{verbatim}
#[ "x=T" ; "y=F" ], "x=~y" ;;
(["x = T"; "y = F"], "x = ~y") : goal

#fst it;;
["x = T"; "y = F"] : term list

#[],"T";;
([], "T") : (* list # term)

#[]:term list, "T";;
([], "T") : goal
\end{verbatim}\end{session}


% =====================================================================
\slide{Theorems}

\point{In HOL, theorems are values of ML type {\tt thm}.}

\point{A value of type {\tt thm} is printed as follows:}
\vskip4mm
\begin{session}\begin{verbatim}
#BOOL_CASES_AX;;
|- !t. (t = T) \/ (t = F)

#ASSUME "P ==> Q";;
. |- P ==> Q
\end{verbatim}\end{session}

\vskip4mm
\bpindent\LARGE\bf
The dot represents an assumption which has not been printed.
\epindent


\point{To get assumptions to print:}
\vskip4mm
\begin{session}\begin{verbatim}
#top_print print_all_thm;;
- : (thm -> void)

#ASSUME "P ==> Q";;
P ==> Q |- P ==> Q
\end{verbatim}\end{session}


% =====================================================================
\slide{Syntax Functions for Theorems}

\point{Syntax functions for destructing theorems:}

\vskip 7mm
\bspindent\LARGE
\begin{verbatim}
      dest_thm : thm -> goal
         concl : thm -> term
           hyp : thm -> term list
\end{verbatim}
\espindent
\vskip 7mm



\point{Example:}
\vskip4mm
\begin{session}\begin{verbatim}
#let t1 = MP (ASSUME "P ==> Q") (ASSUME "P:bool");;
t1 = P ==> Q, P |- Q

#dest_thm t1;;
(["P ==> Q"; "P"], "Q") : goal
\end{verbatim}\end{session}

\point{Note: there are no literals of type {\tt thm}.}

\point{That is, you can't just type arbitrary theorems in, you have to prove
them!}


% =====================================================================
\slide{Inference Rules}

\point{In HOL, an inference rule is represented by an ML function that:}

\subpoint{takes 0 or more theorems (values of type {\tt thm}) as arguments,
and possibly other auxiliary\\ arguments as well}

\subpoint{returns a theorem as a result}


\vskip5mm
\point{For example, {\it modus ponens\/} in HOL is:}

\vskip 7mm
\bspindent\LARGE
\begin{verbatim}
      MP : thm -> thm -> thm
\end{verbatim}
\espindent

\vskip5mm

\point{These functions return only objects of type {\tt thm} that are logically
deducable by the inference rules which they represent.}


% =====================================================================
\slide{Theorems and Proof in HOL}

\point{In the core of HOL we have:}

\subpoint{Axioms: five built-in values of type {\tt thm}}

\subpoint{Primitive inference rules: eight built-in functions that return
values of type {\tt thm}}

\subpoint{Rules of definition: three built-in functions that return
values of type {\tt thm}}

\point{Thus, one can obtain a value of type {\tt thm}:}

\subpoint{directly - as an axiom.}

\subpoint{by computation - using the built-in functions\\ that
represent the inference rules and rules of\\ definition.}

\point{These are the {\it only\/} ways obtain a value of type {\tt thm} in the
HOL system.}

\point{Hence, in HOL, {\it all\/} theorems must be proved!}


% =====================================================================
\slide{Summary}

\point{Logical terms and types;}
\subpoint{the ML data types {\tt term} and {\tt type}.}
\subpoint{Syntax functions for manipluating terms.}

\point{Sequents:}
\subpoint{the ML data type {\tt goal}.}

\point{Theorems:}
\subpoint{the ML data type {\tt thm}.}

\point{Inference Rules:}
\subpoint{ML functions that return values of type {\tt thm}.}

\point{Proof in HOL:}
\subpoint{must be done by ML programs that execute the\\
appropriate sequence of inference rules.}

\end{document}

