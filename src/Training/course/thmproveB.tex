% =====================================================================
% HOL Course Slides: overview of higher order logic   (c) T melham 1990
% =====================================================================

\documentstyle[alltt,12pt,layout]{article}

% ---------------------------------------------------------------------
% Preliminary settings.
% ---------------------------------------------------------------------

\renewcommand{\textfraction}{0.01}	  % 0.01 of the page must contain text
\setcounter{totalnumber}{10}	 	  % max of 10 figures per page
\flushbottom				  % text extends right to the bottom
\pagestyle{slides}			  % slides page style
\setlength{\unitlength}{1mm}		  % unit = 1 mm

% ---------------------------------------------------------------------
% load macros
% ---------------------------------------------------------------------


\input{holmacs}
\input{tokmac}

%\newtokmac{ter}{\tt}
%\newtokmac{nter}{\tt}
\newtokmac{mlname}{\tt}
\newtokmac{CONST}{\constfont}


\newenvironment{hproof}{\begin{center}
 \begin{tabular*}{5.25in}{r>{$}l<{$}@{\extracolsep{0pt plus 1fill}}>{[}r<{]}}}
 {\end{tabular*}\end{center}}

\makeatletter

%% \verbatimnumbered{<file>}
%% read a file and format the contains in a verbatim environment with
%% line number.
%\newcounter{VerbatimLineNo}
%\def\verbatimnumbered#1{\begingroup
%  \setcounter{VerbatimLineNo}{0}%
%  \def\verbatim@processline{%
%    \addtocounter{VerbatimLineNo}{1}%
%    \leavevmode
%    \llap{\theVerbatimLineNo
%          \ \hskip\@totalleftmargin}%
%    \the\verbatim@line\par}%
%  \verbatiminput{#1}\endgroup}

%% Redefine the theindex environment
%%
%\renewenvironment{theindex}{\begin{multicols}{2}[\section*{\indexname}]%
% \columnseprule \z@ \columnsep 35\p@
% \parindent\z@ \parskip\z@ plus.3\p@\relax\let\item\@idxitem}{\end{multicols}}
%
%\def\@idxitem{\par\hangindent 40\p@}
%
%\def\subitem{\par\hangindent 40\p@ \hspace*{20\p@}}
%
%\def\subsubitem{\par\hangindent 40\p@ \hspace*{30\p@}}
%
%\def\indexspace{\par \vskip 10\p@ plus5\p@ minus3\p@\relax}

\makeatother

%% macros for special words
%%
%\def\HOL{{\sc  hol}}
%\def\CHOL{{\sc hol88}}

%% Redfine constfont. The font cmssc12 is a vertual font based on
%% cmss12. The only difference is that the character at '137 becomes
%% an underline in cmssc12.
\font\sfc=cmssc12 \def\constfont{\sfc}
\def\ul#1{$\underline{#1}$}


\def\_{\char'137}
\def\meta#1{{$\langle\hbox{\it #1\/}\rangle$}}

% ---------------------------------------------------------------------
% Macro for a tricky underbrace
% ---------------------------------------------------------------------
\def\rul#1{\vrule width0.25mm height#1mm depth-1mm}
\newbox\ubox
\def\und#1#2{%
    \setbox\ubox=\hbox{${\stackrel{\rul{#2}}{\scriptstyle\hbox{\rm #1}}}$}%
    {\wd\ubox=0mm\box\ubox}}


% ---------------------------------------------------------------------
% set caption at the foot of pages for this series of slides
% ---------------------------------------------------------------------
\ftext{Backward Proof}{8}

% ---------------------------------------------------------------------
% Slides
% ---------------------------------------------------------------------
\begin{document}

% ---------------------------------------------------------------------
% Title page for this series of slides
% ---------------------------------------------------------------------

\bsectitle
Basic Theorem Proving\\
{\vrule width20mm height2.5mm depth-2mm}\\
Backward Proof
\esectitle

% =====================================================================
\slide{Backward Proof}

\point{Forward proof style:}

\subpoint{use the rules to derive new theorems from old}

\vskip5mm
\point{Backward proof style:}

\subpoint{begin with a term that states what we want to prove. This is called
the {\it goal\/}.}

\subpoint{reduce the goal to simpler and simpler subgoals.}

\subpoint{stop when we reach subgoals that we can prove immediately.}

\subpoint{then let the system construct the proof for us.}

% =====================================================================
\slide{Example}

\point{The basic idea is:}

\begin{center}
\setlength{\unitlength}{1mm}
\begin{picture}(100,125)\thicklines
\put(47,110){\makebox(0,0){\Large\bf (P $\wedge$ Q $\supset$ R) 
$ \wedge$
(A $\supset$ $\neg$ B)}}
\put(50,105){\line(2,-1){32}}
\put(50,105){\line(-2,-1){32}}

\put(18,80){\makebox(0,0){\Large\bf (P $\wedge$ Q $\supset$ R)}}
\put(18,75){\line(0,-1){8}}
\put(18,60){\makebox(0,0){\Large\bf ([P;Q], R)}}
\multiput(30,53)(0,-2){3}{\circle*{1}}


\put(82,80){\makebox(0,0){\Large\bf (A $\supset$ $\neg$ B)}}
\put(82,75){\line(0,-1){8}}
\put(82,60){\makebox(0,0){\Large\bf [A], $\neg$ B}}
\put(82,55){\line(0,-1){8}}
\put(82,40){\makebox(0,0){\Large\bf [A], B $\supset$ F}}
\put(82,35){\line(0,-1){8}}
\put(82,20){\makebox(0,0){\Large\bf [A;B], F}}
\multiput(82,13)(0,-2){3}{\circle*{1}}
\end{picture}
\end{center}

\point{Each reduction of goals to subgoals is logically justified by
a proof in the `opposite direction'.}


% =====================================================================
\slide{Theorems and Goals}

\point{A theorem in HOL has the form:}
\vskip7mm
\bpindent\LARGE\bf
\qquad\qquad$\underbrace{\mathstrut t_1,\ldots,t_n}_
   {\und{assumptions}{7.5}}  \vdash 
   \underbrace{\mathstrut t}_{\und{conclusion}{15}}$
\epindent
\vskip8mm
\bpindent\LARGE\bf
This asserts `if $t_1,\ldots,t_n$ are true then so is $t$'.
\epindent
\vskip7mm
\point{A goal is likewise a pair:}

\vskip7mm
\bpindent\LARGE\bf
\qquad\qquad([$t_1$;\ldots;$t_n$], $t$)
\epindent
\vskip7mm
\bpindent\LARGE\bf
but goals and theorems are separate types!
\epindent

\vskip7mm
\point{The aim of backward proof is to turn goals into theorems.}


% =====================================================================
\slide{Overview of Tactics}

\point{Goals are represented by values of type:}
\vskip 7mm
\bspindent\Large\bf
\verb!goal!\quad  $\stackrel{def}{\equiv}$ \quad \verb!term list # term!
\espindent
\vskip7mm
\bpindent\LARGE\bf
That is, a goal is represented by a list of\\
assumptions and a conclusion.
\epindent

\vskip7mm

\point{A {\it tactic} is an ML function of type:}

\vskip 7mm
\bspindent\Large\bf
\verb!tactic!\quad  $\stackrel{def}{\equiv}$ \quad 
\verb!goal -> (goal list # proof)!
\espindent
\vskip7mm
\bpindent\LARGE\bf
which breaks a goal up into a
list of {\it subgoals\/} and a {\it proof\/}.
\epindent

\vskip7mm
\point{The list of subgoals generated by a tactic must imply the goal.}


% =====================================================================
\slide{Goals and Justifications}

\point{A tactic {\tt T} is a function:}
\vskip 5mm
\bspindent\Large\bf
\verb!T : goal -> !$
   \underbrace{\hbox{\verb!goal list!}}_{\und{subgoals}{5}}\;\:
\hbox{\tt \#}\;\:
   \underbrace{\hbox{\verb!(thm list -> thm)!}}_{\und{justification}{5}}$
\espindent
\vskip7mm
\point{Suppose: {\tt T} $g$ = ([$g_1$;\ldots;$g_n$], $f$)}

\point{Then if the theorems
$\vdash th_1$, \dots, $\vdash th_n$ solve the goals
$g_1$, \dots, $g_n$ then:}
\vskip 7mm
\bspindent\LARGE\bf
$f$ [$\vdash th_1$;\ldots;$\vdash th_n$] 
\espindent
\vskip7mm
\bpindent\LARGE\bf
should solve the original goal $g$.
\epindent

\vskip5mm
\point{That is, a tactic takes a goal and returns:}
\vskip3mm
\subpoint{a list of subgoals}
\subpoint{a justification}
\vskip5mm
\bpindent\LARGE\bf
When given a list of theorems that assert the subgoals, the justification
returns a theorem that asserts the goal.
\epindent


% =====================================================================
\slide{Notation}

\point{Notation used in the manual for goals:}
\vskip7mm
\bpindent\LARGE\bf
\qquad\qquad$A\:\;\hbox{\verb+?-+}\;\:t$
\epindent
\vskip7mm
\point{A tactic is written:}
\vskip7mm
{\Large\begin{alltt}
                \meta{goal}
     ===========================  \meta{name} \meta{args}
       \meta{subgoal}   \dots   \meta{subgoal}
\end{alltt}}

\vskip7mm
\point{Example:}
\vskip7mm
{\Large\begin{verbatim}
          A ?- t1 /\ t2
      ======================  CONJ_TAC
       A ?- t1      A ?- t2
\end{verbatim}}
\vskip7mm
\bpindent\LARGE\bf
This splits a conjunction into two conjuncts.
\epindent

% =====================================================================
\slide{A Tactic for Conjunctive Goals}

\point{The tactic is:}
\vskip7mm
{\Large\begin{verbatim}
          A ?- t1 /\ t2
      ======================  CONJ_TAC
       A ?- t1      A ?- t2
\end{verbatim}}
\vskip7mm

\point{Example:}
\vskip4mm
\begin{session}\begin{verbatim}
#let sgs,p = CONJ_TAC ([], "T /\ ~F");;
sgs = [([], "T"); ([], "~F")] : goal list
p = - : proof

#let th1 = TRUTH;;
th1 = |- T

#let th2 = NOT_INTRO(SPEC "F" FALSITY);;
th2 = |- ~F

#p [th1;th2];;
|- T /\ ~F
\end{verbatim}\end{session}

% =====================================================================
\slide{Implementation}

\point{The code for {\tt CONJ\_TAC} is:}
\vskip 4mm
\begin{session}\begin{verbatim}
#let CONJ_TAC : tactic(asl,w) =
     (let l,r = dest_conj w in
      [(asl,l); (asl,r)],
     \[th1;th2]. CONJ th1 th2
    )? failwith `CONJ_TAC`;;
CONJ_TAC = - : tactic
\end{verbatim}\end{session}

\point{Example:}
\vskip 4mm
\begin{session}\begin{verbatim}
#let l,r = dest_conj "T /\ ~F";;
l = "T" : term
r = "~F" : term

#let asl = []:term list;;
asl = [] : term list

#[(asl,l);(asl,r)];;
[([], "T"); ([], "~F")] : goal list

#\[th1;th2]. CONJ th1 th2;;
- : proof
\end{verbatim}\end{session}

\point{Note failure:}
\vskip 4mm
\begin{session}\begin{verbatim}
#CONJ_TAC ([], "!t. t /\ t");;
evaluation failed     CONJ_TAC
\end{verbatim}\end{session}



% =====================================================================
\slide{Solving Goals with Theorems}

\point{To use a theorem to solve a goal:}
\vskip6mm
{\Large\begin{verbatim}
       A ?- g
     =========    ACCEPT_TAC (A |- g)
         -
\end{verbatim}}
\bpindent\LARGE\bf
The theorem must exactly match the goal.
\epindent

\point{Example:}
\vskip 3mm
\begin{session}\begin{verbatim}
#ACCEPT_TAC;;
- : thm_tactic

#ETA_AX;;
|- !t. (\x. t x) = t

#let tac = ACCEPT_TAC ETA_AX;;
tac = - : tactic

#let sg,p = tac ([], "!t:*->**. (\x. t x) = t");;
sg = [] : goal list
p = - : proof

#p [];;
|- !t. (\x. t x) = t
\end{verbatim}\end{session}

\point{Notice the type abbreviation:}
\vskip 3mm
\bspindent\Large\bf
\verb!thm_tactic!\quad  $\stackrel{def}{\equiv}$ \quad 
\verb!thm -> tactic!
\espindent


% =====================================================================
\slide{Summary}

\point{Goals:}
\vskip5mm
\bspindent\Large\bf
\verb!goal!\quad  $\stackrel{def}{\equiv}$ \quad \verb!term list # term!
\espindent
\vskip7mm
\point{Tactics:}
\vskip 5mm
\bspindent\Large\bf
\verb!tactic!\quad  $\stackrel{def}{\equiv}$ \quad 
\verb!goal -> (goal list # proof)!
\espindent
\vskip7mm
\point{Justifications:}
\vskip3mm
\subpoint{If: {\tt T} $g$ = ([$g_1$;\ldots;$g_n$], $f$)}
\vskip3mm
\subpoint{Then if the theorems
$\vdash th_1$, \dots, $\vdash th_n$ solve the goals
$g_1$, \dots, $g_n$ then:}
\vskip 5mm
\bspindent\Large\bf
\qquad$f$ [$\vdash th_1$;\ldots;$\vdash th_n$] 
\espindent
\vskip5mm
\bspindent\Large\bf
should solve the original goal $g$.
\epindent
\vskip7mm
\point{Notation and examples.}
\subpoint{{\tt CONJ\_TAC} and {\tt ACCEPT\_TAC}}

% =====================================================================
\slide{The Subgoal Package}

\point{HOL has a {\it subgoal package\/} for finding tactic proofs
interactively.}

\point{It maintains a stack of subgoals to be proved.}

\point{The following functions are provided:}
\vskip 5mm
\bspindent\Large\bf
\verb!set_goal : goal -> void!
\vskip 5mm
\verb!expand : tactic -> void!
\vskip 5mm
\verb!print_state : int -> void!
\vskip 5mm
\verb!rotate : int -> void!
\vskip 5mm
\verb!backup : void -> void!
\vskip 5mm
\verb!top_goal : void -> goal!
\vskip 5mm
\verb!save_top_thm : string -> thm!
\espindent

\point{The subgoal package is for {\it finding\/} tactic proofs only. These
functions should not appear in completed proofs!}


% =====================================================================
\slide{Functions in the Subgoal Package}

\point{Put an initial goal {\tt($A$,$g$)} on the stack:}
\vskip4mm
\bspindent\Large\bf
\verb!set_goal (!$A$\verb!,!$g$\verb!);;!
\espindent

\point{Expand the current goal using a tactic $T$:}
\vskip4mm
\bspindent\Large\bf
\verb!expand !$T$\verb!;;!
\espindent

\point{print the top $n$ layers of the stack:}
\vskip4mm
\bspindent\Large\bf
\verb!print_state !$n$\verb!;;!
\espindent

\point{Rotate the subgoal stack by $n$ steps:}
\vskip4mm
\bspindent\Large\bf
\verb!rotate !$n$\verb!;;!
\espindent


\point{Undo the last change to the goal stack:}
\vskip4mm
\bspindent\Large\bf
\verb!backup ();;!
\espindent


\point{Get the top goal on the stack:}
\vskip4mm
\bspindent\Large\bf
\verb!top_goal ();;!
\espindent

\point{Save most recently proved theorem:}
\vskip4mm
\bspindent\Large\bf
\verb!save_top_thm !`\meta{name}`\verb!;;!
\espindent

% =====================================================================
\slide{Example}

\point{Set the goal:}
\vskip 4mm
\begin{session}\begin{verbatim}
#set_goal ([], "T /\ ~F");;
"T /\ ~F"

() : void
\end{verbatim}\end{session}

\point{Expand with {\tt CONJ\_TAC}:}
\begin{session}\begin{verbatim}
#expand CONJ_TAC;;
OK..
2 subgoals
"~F"

"T"

() : void
\end{verbatim}\end{session}


\point{Solve the first subgoal:}
\begin{session}\begin{verbatim}
#expand (ACCEPT_TAC TRUTH);;
OK..
goal proved
|- T

Previous subproof:
"~F"

() : void
\end{verbatim}\end{session}

% =====================================================================
\slide{Example Continued}

\point{Solve the second subgoal:}
\begin{session}\begin{verbatim}
#let thm = NOT_INTRO(SPEC "F" FALSITY);;
thm = |- ~F

#expand (ACCEPT_TAC thm);;
OK..
goal proved
|- ~F
|- T /\ ~F

Previous subproof:
goal proved
() : void
\end{verbatim}\end{session}

\point{Save the theorem:}
\begin{session}\begin{verbatim}
#let foo = save_top_thm `foo`;;
foo = |- T /\ ~F
\end{verbatim}\end{session}


% =====================================================================
\slide{Proof by Contradiction}

\point{The contradiction tactic:}
\vskip7mm
{\Large\begin{verbatim}
          A ?- g
      =============  CONTR_TAC  (B |- F)
            -
\end{verbatim}}
\bpindent\LARGE\bf
We should have ${\tt B} \subseteq {\tt A}$ and, obviously, ${\tt B} \neq \{\}$.
\epindent

\point{Example:}
\vskip4mm
\begin{session}\begin{verbatim}
#set_goal (["n < n"], "1 = 0");;
"1 = 0"
    [ "n < n" ]

() : void

#top_print print_all_thm;;
- : (thm -> void)

#let thm = UNDISCH(SPEC "n:num" LESS_REFL);;
thm = n < n |- F

#expand (CONTR_TAC thm);;
OK..
goal proved
. |- 1 = 0

Previous subproof:
goal proved
() : void
\end{verbatim}\end{session}


% =====================================================================
\slide{Adding Assumptions}

\point{To put a theorem onto the assumption list:}
\vskip7mm
{\Large\begin{verbatim}
          A ?- g
      ============== ASSUME_TAC  (B |- t)
       A u {t} ?- g
\end{verbatim}}
\bpindent\LARGE\bf
We require that ${\tt B} \subseteq {\tt A}$.
\epindent

\point{Example:}
\vskip4mm
\begin{session}\begin{verbatim}
#set_goal (["n < m"], "n < SUC(SUC m)");;
"n < (SUC(SUC m))"
    [ "n < m" ]

() : void

#let thm1 = SPECL ["n:num";"m:num"] LESS_SUC;;
thm1 = |- n < m ==> n < (SUC m)

#let thm2 = UNDISCH thm1;;
thm2 = n < m |- n < (SUC m)

#expand (ASSUME_TAC thm2);;
OK..
"n < (SUC(SUC m))"
    [ "n < m" ]
    [ "n < (SUC m)" ]

() : void
\end{verbatim}\end{session}


% =====================================================================
\slide{Validity}


\point{A tactic whose proof part that doesn't actually prove the goal, but some
{\it other\/} proposition, is called {\it invalid}.}

\point{To be valid, all assumptions of the theorem arguments to tactics must
also appear among the assumptions of the goal.}

\point{The subgoal package checks validity:}
\begin{session}\begin{verbatim}
#set_goal ([], "1 = 0");;
"1 = 0"

() : void

#let thm = UNDISCH (SPEC "1=0" FALSITY);;
thm = F |- 1 = 0

#expand (ACCEPT_TAC thm);;
OK..
evaluation failed     Invalid tactic
\end{verbatim}\end{session}


% =====================================================================
\slide{Proving Disjunctions}

\point{To prove a disjunctive goal:}
\vskip7mm
{\Large\begin{verbatim}
       A \/ B                     A \/ B
     ========== DISJ1_TAC       ========== DISJ2_TAC
         A                          B
\end{verbatim}}

\point{Example:}
\vskip 4mm
\begin{session}\begin{verbatim}
#set_goal ([], "F \/ T");;
"F \/ T"

() : void

#expand DISJ2_TAC;;
OK..
"T"

() : void

#expand (ACCEPT_TAC TRUTH);;
OK..
goal proved
|- T
|- F \/ T

Previous subproof:
goal proved
() : void
\end{verbatim}\end{session}


% =====================================================================
\slide{Boolean Equality}

\point{The tactic is:}
\vskip7mm
{\Large\begin{verbatim}
                 A ?- t1 = t2
     ==================================== EQ_TAC
      A ?- t1 ==> t2      A ?- t2 ==> t1

\end{verbatim}}

\point{Example:}
\vskip4mm
\begin{session}\begin{verbatim}
#set_goal ([], "(?n. n < m) = ~(m = 0)");;
"(?n. n < m) = ~(m = 0)"

() : void

#expand EQ_TAC;;
OK..
2 subgoals
"~(m = 0) ==> (?n. n < m)"

"(?n. n < m) ==> ~(m = 0)"

() : void
\end{verbatim}\end{session}

% =====================================================================
\slide{Universal Quantification}

\point{The tactic is:}
\vskip7mm
{\Large\begin{verbatim}
         A ?- !x.P
     ================ GEN_TAC
       A ?- P[x'/x]
\end{verbatim}}\vskip 5mm
\bpindent\LARGE\bf
The system chooses the variable {\tt x'} so that it is not free in \verb+!x.P+
or {\tt A}.  
\epindent

\vskip7mm
\point{For example, set the goal:}
\vskip4mm
\begin{session}\begin{verbatim}
#set_goal (["x > 0"], "!x:num. x = x");;
"!x. x = x"
    [ "x > 0" ]

() : void
\end{verbatim}\end{session}

\point{Expand with {\tt GEN\_TAC}:}
\vskip4mm
\begin{session}\begin{verbatim}
#expand GEN_TAC;;
OK..
"x' = x'"
    [ "x > 0" ]

() : void
\end{verbatim}\end{session}

% =====================================================================
\slide{Example Continued}

\point{Current goal:}
\vskip4mm
\begin{session}\begin{verbatim}
#print_state 1;;
"x' = x'"
    [ "x > 0" ]

() : void
\end{verbatim}\end{session}
\vskip4mm
\point{We can solve this with:}
\vskip7mm
{\Large\begin{verbatim}
         A ?- t = t
     ================ REFL_TAC
             -
\end{verbatim}}
\vskip2mm
\point{Continuing the proof:}
\vskip4mm
\begin{session}\begin{verbatim}
#expand REFL_TAC;;
OK..
goal proved
. |- x' = x'
. |- !x. x = x

Previous subproof:
goal proved
() : void
\end{verbatim}\end{session}


% =====================================================================
\slide{Generalizing the Goal}

\point{To prove a more general goal:}
\vskip7mm
{\Large\begin{verbatim}
          A ?- P[t]
      =================  SPEC_TAC ("t","x")
        A ?- !x. P[x]
\end{verbatim}}
\bpindent\LARGE\bf
This is useful for preparing to do induction.
\epindent


\point{Example:}
\vskip4mm
\begin{session}\begin{verbatim}
#set_goal ([], "F ==> (1 = 0)");;
"F ==> (1 = 0)"

() : void

#expand (SPEC_TAC ("1=0","t:bool"));;
OK..
"!t. F ==> t"

() : void
\end{verbatim}\end{session}

% =====================================================================
\slide{Existential Goals}

\point{Providing a witness:}
\vskip7mm
{\Large\begin{verbatim}
       A ?- ?x.P
     ============= EXISTS_TAC "t"
      A ?- P[t/x]
\end{verbatim}}
\bpindent\LARGE\bf
You must know an appropriate term \verb!"t"!.
\epindent

\point{Example:}
\begin{session}\begin{verbatim}
#set_goal ([], "?n. !m. ~(SUC m = n)");;
"?n. !m. ~(SUC m = n)"

() : void

#expand (EXISTS_TAC "0");;
OK..
"!m. ~(SUC m = 0)"

() : void
\end{verbatim}\end{session}


% =====================================================================
\slide{Substitution}

\point{Parallel substitution in a goal:}
\vskip7mm
{\Large\begin{verbatim}
         A ?- P
     ===============   SUBST_TAC [|- ti == ui]
      A ?- P[ui/ti]
\end{verbatim}}

\point{Example:}
\vskip 5mm
\begin{session}\begin{verbatim}
#set_goal ([], "~F");;
"~F"

() : void

#expand (SUBST_TAC [NOT_DEF;F_DEF]);;
OK..
"(\t. t ==> F)(!t. t)"

() : void
\end{verbatim}\end{session}



% =====================================================================
\slide{Implications}

\point{To assume the antecdent:}
\vskip7mm
{\Large\begin{verbatim}
       A ?- P ==> Q
      ==============  DISCH_TAC
       A u {P} ?- Q
\end{verbatim}}

\point{Example:}
\vskip 7mm
\begin{session}\begin{verbatim}
#set_goal ([], "(?x:*. P x) ==> ~!x. ~P x");;
"(?x. P x) ==> ~(!x. ~P x)"

() : void

#expand DISCH_TAC;;
OK..
"~(!x. ~P x)"
    [ "?x. P x" ]

() : void
\end{verbatim}\end{session}


% =====================================================================
\slide{Stripping a Goal}

\point{{\tt STRIP\_TAC} strips the outermost connective or quantifier using
one of the tactics:}
\vskip6mm
\bspindent\LARGE\bf
{\tt CONJ\_TAC}, {\tt GEN\_TAC} or {\tt DISCH\_TAC}.
\espindent

\vskip7mm 
\point{It is often (but {\it not always\/}) the right thing to do
first in typical tactic proofs.}

\vskip7mm
\point{For example, set the goal:}
\vskip 4mm
\begin{session}\begin{verbatim}
#set_goal ([], "!x. (x \/ ~x) ==> (t1 /\ t2)");;
"!x. x \/ ~x ==> t1 /\ t2"

() : void
\end{verbatim}\end{session}

\point{First, strip the universal quantifier:}
\vskip 4mm
\begin{session}\begin{verbatim}
#expand STRIP_TAC;;
OK..
"x \/ ~x ==> t1 /\ t2"
\end{verbatim}\end{session}

% =====================================================================
\slide{The Example Continued}

\point{The current goal:}
\vskip4mm
\begin{session}\begin{verbatim}
#print_state 1;;
"x \/ ~x ==> t1 /\ t2"

() : void
\end{verbatim}\end{session}


\point{Now, assume the antecedent:}
\vskip 4mm
\begin{session}\begin{verbatim}
#expand STRIP_TAC;;
OK..
2 subgoals
"t1 /\ t2"
    [ "~x" ]

"t1 /\ t2"
    [ "x" ]

() : void
\end{verbatim}\end{session}

\point{The tactic {\tt STRIP\_TAC} (unlike {\tt DISCH\_TAC}) breaks the
disjunctive antecedent into two subgoals.}

\point{In fact, {\tt STRIP\_TAC} `breaks up' assumed \\antecedents as much as
possible.}


% =====================================================================
\slide{The Example Concluded}

\point{The current goal:}
\vskip4mm
\begin{session}\begin{verbatim}
#print_state 1;;
2 subgoals
"t1 /\ t2"
    [ "~x" ]

"t1 /\ t2"
    [ "x" ]

() : void
\end{verbatim}\end{session}

\point{Continue to break up the goal:}
\vskip 4mm
\begin{session}\begin{verbatim}
#expand STRIP_TAC;;
OK..
2 subgoals
"t2"
    [ "x" ]

"t1"
    [ "x" ]

() : void
\end{verbatim}\end{session}


\point{... et cetera.}

% =====================================================================
\slide{Assuming a Theorem}

\point{This tactic puts the supplied theorem on the assumption list, but breaks
it up first.}

\point{Conjunctions:}
\vskip3mm
{\Large\begin{verbatim}
      STRIP_ASSUME_TAC (B |- P /\ Q) :
   
                 A ?- t
         ======================  
            A u {P,Q} ?- t
\end{verbatim}}

\point{Disjunctions:}
\vskip3mm
{\Large\begin{verbatim}
      STRIP_ASSUME_TAC (B |- P \/ Q) :

                      A ?- t
         ==============================  
          A u {P} ?- t    A u {Q} ?- t
\end{verbatim}}

\point{Existential quantifier:}
\vskip3mm
{\Large\begin{verbatim}
      STRIP_ASSUME_TAC (B |- ?x.P) :

                A ?- t
         ====================  
          A u {P[x'/x]} ?- t
\end{verbatim}}


% =====================================================================
\slide{Assuming a Theorem}

\point{Example:}
\vskip4mm
\begin{session}\begin{verbatim}
#set_goal ([], "0 <= n");;
"0 <= n"

() : void

#let th = SPEC "n:num" num_CASES;;
th = |- (n = 0) \/ (?n'. n = SUC n')

#expand (STRIP_ASSUME_TAC th);;
OK..
2 subgoals
"0 <= n"
    [ "n = SUC n'" ]

"0 <= n"
    [ "n = 0" ]

() : void
\end{verbatim}\end{session}

% =====================================================================
\slide{Rewriting Tactics}

\point{Rewrite the goal using supplied theorems and basic simplifications:}

\vskip 4mm
\bspindent\Large\bf
\verb!REWRITE_TAC: thm list -> tactic!
\espindent


\point{Rewrite the goal using only the supplied\\ theorems:}
\vskip 4mm
\bspindent\Large\bf
\verb!PURE_REWRITE_TAC: thm list -> tactic!
\espindent


\point{Rewrite the using the supplied theorems and the assumptions of the
goal:}
\vskip 4mm
\bspindent\Large\bf\obeylines
\verb!PURE_ASM_REWRITE_TAC: thm list -> tactic!
\verb!ASM_REWRITE_TAC: thm list -> tactic!
\espindent

\point{There are also `{\tt ONCE\_}' variants of these.}

% =====================================================================
\slide{Summary}

\point{The subgoal package: {\tt set\_goal} and {\tt expand}.}

\point{Various basic tactics.}

\point{Valid and invalid tactics.}

\point{Rewriting tactics.}


% =====================================================================
\slide{Tacticals}

\point{Tacticals are:}

\subpoint{ML functions used to combine tactics.}
\subpoint{used for writing complete tactic proofs.}
\vskip5mm

\point{In general, a {\it tactical\/} is a specific operation on tactics that
takes one or more tactics as input and return a tactic as output.}

% =====================================================================
\slide{Choice of Alternatives}

\point{The tactical {\tt ORELSE} is an infix ML function:}
\vskip6mm
\bspindent\LARGE\bf
\verb!ORELSE : tactic -> tactic -> tactic!
\espindent
\vskip 7mm
\point{If $T_1$ and $T_2$ are tactics, then:}
\vskip5mm
\bspindent\LARGE\bf
$T_1\;\;$\verb!ORELSE!$\;\;T_2$
\espindent
\vskip5mm
\bpindent\LARGE\bf
is a tactic which:
\epindent
\vskip 5mm
\subpoint{first tries the tactic $T_1$, and}
\subpoint{applies $T_1$ if it succeeds, or else}
\subpoint{tries $T_2$ if $T_1$ fails}

\vskip 7mm
\point{The ML definition of {\tt ORELSE} is simple:}
\vskip 4mm
\begin{session}\begin{verbatim}
#let (t1 ORELSE t2:tactic) g = t1 g ? t2 g;;
ORELSE = - : (tactic -> tactic -> tactic)
\end{verbatim}\end{session}

% =====================================================================
\slide{Example}

\point{Example, try {\tt CONJ\_TAC} or else {\tt GEN\_TAC}}
\vskip 4mm
\begin{session}\begin{verbatim}
#let tac = CONJ_TAC ORELSE GEN_TAC;;
tac = - : tactic

#set_goal ([], "!x:*. P x");;
"!x. P x"

() : void

#expand CONJ_TAC;;
OK..
evaluation failed     CONJ_TAC

#expand tac;;
OK..
"P x"

() : void
\end{verbatim}\end{session}


% =====================================================================
\slide{Sequencing}

\point{The tactical {\tt THEN} is an ML infix:}
\vskip6mm
\bspindent\LARGE\bf
\verb!THEN : tactic -> tactic -> tactic!
\espindent

\vskip7mm
\point{If $T_1$ and $T_2$ are tactics, then:}
\vskip5mm
\bspindent\LARGE\bf
$T_1\;\;$\verb!THEN!$\;\;T_2$
\espindent
\vskip5mm
\bpindent\LARGE\bf
is a tactic which:
\epindent
\vskip 5mm
\subpoint{first applies the tactic $T_1$, and}
\subpoint{then applies $T_2$ to {\it all\/} the resulting subgoals}


% =====================================================================
\slide{Example of Sequencing}

\point{Example:}
\vskip4mm
\begin{session}\begin{verbatim}
#set_goal ([], "!x:bool. T \/ P x");;
"!x. T \/ P x"
() : void

#expand
  (GEN_TAC THEN DISJ1_TAC THEN ACCEPT_TAC TRUTH);;
OK..
goal proved
|- !x. T \/ P x

Previous subproof:
goal proved
() : void
\end{verbatim}\end{session}


% =====================================================================
\slide{Alternative Sequencing}


\point{$T_1$ {\tt THEN} $T_2$ applies its second tactic to {\it all\/} the
subgoals generated by the first tactic.}
\vskip5mm

\point{If you want to do something different to each resulting subgoals, use
{\tt THENL}:}
\vskip6mm
\bspindent\LARGE\bf
\verb!THENL : tactic -> tactic list -> tactic!
\espindent

\vskip7mm
\point{If $T$, $T_1$, \dots, $T_n$ are tactics, then:}
\vskip5mm
\bspindent\LARGE\bf
{\tt $T\;\;$\verb!THENL!$\;\;$[$T_1$;$\dots$;$T_n$]}
\espindent
\vskip5mm
\bpindent\LARGE\bf
is a tactic which:
\epindent
\vskip 5mm
\subpoint{first applies the tactic $T$, and}
\subpoint{then applies $T_i$ to the $i$th resulting subgoal}

\point{Note: if $T$ generates $n$ subgoals, there must be $n$ tactics in the
supplied list.}


% =====================================================================
\slide{Example of Alternative Sequencing}

\point{An Example:}
\vskip4mm
\begin{session}\begin{verbatim}
#set_goal ([], "T /\ ~F");;
"T /\ ~F"

() : void

#let tac = 
  CONJ_TAC THENL
    [ACCEPT_TAC TRUTH ;
     STRIP_TAC THEN 
     CONTR_TAC (ASSUME "F")];;
tac = - : tactic


#expand tac;;
OK..
goal proved
|- T /\ ~F

Previous subproof:
goal proved
() : void
\end{verbatim}\end{session}


% =====================================================================
\slide{Repetition}

\point{Tactical that repeats a tactic until failure:}
\vskip6mm
\bspindent\LARGE\bf
\verb!REPEAT : tactic -> tactic!
\espindent
\vskip7mm

\point{If $T$ is a tactic, then the tactic {\tt REPEAT $T$}\\
applies $T$ repeatedly until it fails.}
\vskip7mm
\point{The ML definition of {\tt REPEAT}:}
\vskip4mm
\begin{session}\begin{verbatim}
#let ALL_TAC : tactic = \g. [g],hd;;
ALL_TAC = - : tactic

#letrec REPEAT tac g = 
    ((tac THEN REPEAT tac) ORELSE ALL_TAC) g ;;
REPEAT = - : (tactic -> tactic)
\end{verbatim}\end{session}

% =====================================================================
\slide{Example of Repetition}

\point{A very common first step:}
\vskip 4mm
\begin{session}\begin{verbatim}
#set_goal ([],"!P Q. ~ P \/ Q ==> (P ==> Q)");;
"!P Q. ~P \/ Q ==> P ==> Q"

() : void

#expand (REPEAT STRIP_TAC);;
OK..
2 subgoals
"Q"
    [ "Q" ]
    [ "P" ]

"Q"
    [ "~P" ]
    [ "P" ]

() : void
\end{verbatim}\end{session}

% =====================================================================
\slide{Auxiliary Functions for Tactics}

\point{To prove a theorem with a tactic:}
\vskip5mm
\bspindent\Large\bf
\verb!TAC_PROOF : (goal # tactic) -> thm!
\espindent
\vskip5mm
\bpindent\LARGE\bf
This function:
\epindent
\vskip5mm
\subpoint{takes as arguments a goal $(A,g)$ and a tactic $T$,}
\subpoint{tries to prove the goal using the tactic, and}
\subpoint{if the tactic succeeds in
proving the goal, returns the resulting theorem.}

\point{To prove a theorem and save it:}
\vskip5mm
\bspindent\Large\bf
\verb!prove_thm : (string # term # tactic) -> thm!
\espindent
\vskip5mm
\bpindent\LARGE\bf
This function:
\epindent
\vskip5mm
\subpoint{takes as arguments name {\tt `\meta{name}`}, a goal $(A,g)$ and a
tactic $T$,}
\subpoint{tries to prove the goal using the tactic,}
\subpoint{if the tactic succeeds in
proving the goal, returns the resulting theorem, and}

\subpoint{saves the theorem under the name {\tt `\meta{name}`} in the current
theory.}



% =====================================================================
\slide{Summary of Tacticals}

\point{Choice of alternatives:}
\vskip5mm
\bspindent\Large\bf
\verb!ORELSE : tactic -> tactic -> tactic!
\espindent
\point{Sequencing:}
\vskip5mm
\bspindent\Large\bf
\verb!THEN : tactic -> tactic -> tactic!
\espindent
\point{Alternative sequencing:}
\vskip5mm
\bspindent\Large\bf
\verb!THENL : tactic -> tactic list -> tactic!
\espindent
\point{Repetition:}
\vskip5mm
\bspindent\Large\bf
\verb!REPEAT : tactic -> tactic!
\espindent
\point{Auxiliary functions:}
\vskip5mm
\bspindent\Large\bf\obeylines
\verb!TAC_PROOF : (goal # tactic) -> thm!
\verb!prove_thm : (string # goal # tactic) -> thm!
\espindent



% =====================================================================
\slide{List Tacticals}

\point{We now consider the HOL {\it list tacticals}, which operate on lists of
theorems, tactics, or\\
 assumptions.}

\point{The simplest of these are:}
\vskip5mm
\bspindent\LARGE\bf\obeylines
\verb!EVERY : tactic list -> tactic!
\mbox{}
\verb!FIRST : tactic list -> tactic!
\espindent
\vskip5mm
\bpindent\LARGE\bf
which behave as follows:
\vskip5mm
\epindent

\subpoint{{\tt EVERY [$T_1$;\dots;$T_n$]} applies the tactics $T_1$, \dots
$T_n$ in sequence to the goal.}


\subpoint{{\tt FIRST [$T_1$;\dots;$T_n$]} applies the first of the tactics
$T_1$, \dots $T_n$ that succeeds when applied to the goal.}

\point{Thus:}

\vskip5mm
\bspindent\Large\bf\obeylines
{\tt EVERY [$T_1$;\dots;$T_n$]} %
$=$ {\tt $T_1$ THEN \dots THEN $T_n$}
\mbox{}
{\tt FIRST [$T_1$;\dots;$T_n$]} %
$=$ {\tt $T_1$ ORELSE \dots ORELSE $T_n$}
\espindent


% =====================================================================
\slide{Mapping Tacticals}

\point{The tacticals:}
\vskip5mm
\bspindent\Large\bf\obeylines
\verb!MAP_EVERY : (* -> tactic) -> * list -> tactic!
\mbox{}
\verb!MAP_FIRST : (* -> tactic) -> * list -> tactic!
\espindent
\vskip5mm
\bpindent\LARGE\bf
take an ML function $f$ of type \verb!* -> tactic! and 
a list $l$ of type \verb!* list! and:
\vskip5mm
\epindent
\subpoint{map $f$ it down $l$ to get a list of tactics, and}
\subpoint{apply every resulting tactic (the first resulting tactic that
succeeds) to the goal.}

\point{For example}
\vskip5mm
\bspindent\Large\bf\obeylines
{\tt MAP\_EVERY ASSUME\_TAC [|- $t_1$; \dots; |-$t_n$]}
\espindent
\vskip5mm
\bpindent\LARGE\bf
is equivalent to
\epindent
\vskip5mm
\bspindent\Large\bf\obeylines
{\tt EVERY [ASSUME\_TAC |- $t_1$; \dots; ASSUME\_TAC |- $t_n$]}
\espindent

% =====================================================================
\slide{Using the Assumptions}

\point{There are three list tacticals for using the \\
assumption of a goal:}
\vskip7mm
\bspindent\Large\bf\obeylines
\verb!EVERY_ASSUM : (thm -> tactic) -> tactic!
\mbox{}
\verb!FIRST_ASSUM : (thm -> tactic) -> tactic!
\mbox{}
\verb!ASSUM_LIST : (thm list -> tactic) -> tactic!
\espindent
\vskip5mm

\point{These are useful when you want to use the assumptions of a goal as
theorems.}

\point{For example, you may want to use the \\
assumptions of a goal as inputs to
tactics\\ like {\tt ACCEPT\_TAC} and {\tt CONTR\_TAC}.}


% =====================================================================
\slide{Using Every Assumption}

\point{The tactical:}
\vskip5mm
\bspindent\Large\bf\obeylines
\verb!EVERY_ASSUM : (thm -> tactic) -> tactic!
\espindent
\vskip5mm
\bpindent\LARGE\bf
takes an ML function $f$ of type \verb!thm -> tactic! and 
a goal {\tt ($A$,$g$)} and:
\vskip5mm
\epindent
\subpoint{maps {\tt ASSUME} down the list of assumptions $A$,}
\subpoint{maps $f$  down the resulting list of theorems,}
\subpoint{applies every tactic in the resulting list of tactics to the goal.}

\vskip7mm
\point{For example, condider the goal:}
\vskip4mm
\begin{session}\begin{verbatim}
#set_goal(["n = 1";"m = 2"], "n < m");;
"n < m"
    [ "m = 2" ]
    [ "n = 1" ]

() : void
\end{verbatim}\end{session}

% =====================================================================
\slide{Example Continued}

\point{The tactic {\tt SUBST1\_TAC}:}
\vskip4mm
\begin{session}\begin{verbatim}
#SUBST1_TAC;;
- : thm_tactic
\end{verbatim}\end{session}


\point{Substitute into the conclusion:}
\vskip4mm
\begin{session}\begin{verbatim}
#print_state 1;;
"n < m"
    [ "m = 2" ]
    [ "n = 1" ]

() : void

#expand (EVERY_ASSUM SUBST1_TAC);;
OK..
"1 < 2"
    [ "m = 2" ]
    [ "n = 1" ]

() : void
\end{verbatim}\end{session}


% =====================================================================
\slide{Using an Assumption}

\point{The tactical:}
\vskip5mm
\bspindent\Large\bf\obeylines
\verb!FIRST_ASSUM : (thm -> tactic) -> tactic!
\espindent
\vskip5mm
\bpindent\LARGE\bf
takes an ML function $f$ of type \verb!thm -> tactic! and 
a goal {\tt ($A$,$g$)}, and:
\vskip5mm
\epindent
\subpoint{maps {\tt ASSUME} down the list of assumptions $A$,}
\subpoint{maps $f$  down the resulting list of theorems,}
\subpoint{applies the first tactic in the resulting list of tactics that
succeeds when applied to the goal.}
\vskip7mm
\point{For example, consider the goal:}
\vskip 4mm
\begin{session}\begin{verbatim}
#set_goal ([],"!P Q.  (P /\ Q) ==> (Q /\ P)");;
"!P Q. P /\ Q ==> Q /\ P"

() : void
\end{verbatim}\end{session}

% =====================================================================
\slide{Example Continued}

\point{After stripping:}
\vskip4mm
\begin{session}\begin{verbatim}
#expand (REPEAT STRIP_TAC);;
OK..
2 subgoals
"P"
    [ "P" ]
    [ "Q" ]

"Q"
    [ "P" ]
    [ "Q" ]

() : void
\end{verbatim}\end{session}

\point{The conclusion is an assumption, so:}
\vskip4mm
\begin{session}\begin{verbatim}
#expand (FIRST_ASSUM ACCEPT_TAC);;
OK..
goal proved
. |- Q

Previous subproof:
"P"
    [ "P" ]
    [ "Q" ]

() : void
\end{verbatim}\end{session}

% =====================================================================
\slide{Using the Entire Assumption List}

\point{The tacticals {\tt FIRST\_ASSUM} and {\tt EVERY\_ASSUM} are special
cases of:}
\vskip5mm
\bspindent\Large\bf\obeylines
\verb!ASSUM_LIST : (thm list -> tactic) -> tactic!
\espindent
\vskip5mm
\bpindent\LARGE\bf
which takes an ML function $f$ of type
\vskip5mm
\epindent
\bspindent\Large\bf\obeylines
\verb!thm list -> tactic! 
\espindent
\vskip5mm
\bpindent\LARGE\bf
and a goal {\tt ($A$,$g$)}, and:
\vskip5mm
\epindent
\subpoint{maps {\tt ASSUME} down the list of assumptions $A$,}
\subpoint{applies $f$ to the resulting list of theorems,}
\subpoint{applies the resulting tactic to the goal.}

\vskip7mm
\point{For example, {\tt EVERY\_ASSUM} is defined:}
\vskip4mm
\begin{session}\begin{verbatim}
#let EVERY_ASSUM  = ASSUM_LIST o MAP_EVERY;;
EVERY_ASSUM = - : (thm_tactic -> tactic)
\end{verbatim}\end{session}


% =====================================================================
\slide{Summary: List Tacticals}

\point{For lists of tactics:}
\subpoint{{\tt EVERY : tactic list -> tactic}}
\subpoint{{\tt FIRST : tactic list -> tactic}}

\point{Mapping tacticals:}
\subpoint{{\tt MAP\_EVERY : (* -> tactic) -> * list -> tactic}}
\subpoint{{\tt MAP\_FIRST : (* -> tactic) -> * list -> tactic}}

\point{Using assumptions:}
\subpoint{{\tt EVERY\_ASSUM : thm\_tactic -> tactic}}
\subpoint{{\tt FIRST\_ASSUM : thm\_tactic -> tactic}}
\subpoint{{\tt ASSUM\_LIST : (thm list -> tactic) -> tactic}}

% =====================================================================
\slide{`Resolution' Tactics}

\point{In HOL, `resolution' involves of deducing from a list
of facts}
\vskip 5mm
\bspindent\Large\bf
{\tt [|- $t_1$; |- $t_2$; \dots; |- $t_n$]}
\espindent
\vskip5mm
\bpindent\LARGE\bf
and an implication
\epindent
\vskip5mm
\bspindent\Large\bf
{\tt |- $A_1$ ==> ($A_2$ ==> \dots ==> ($A_n$ ==> $C$))}
\espindent
\vskip5mm
\bpindent\LARGE\bf
some instance of {\tt |- $C$}.
\epindent

\point{This is done by matching antecedants $A_j$ with facts $t_i$ and
using modus ponens.}

\point{HOL provides one-way matching for this task.}


% =====================================================================
\slide{Matching Modus Ponens}

\point{Suppose we have two theorems:}
\vskip 5mm
\bspindent\Large\bf\obeylines
\verb+thm1 = |- !x y z. t1[x,y,z] ==> t2[x,y,z]+
\mbox{}
\verb!thm2 = |- t1[a,b,c]!
\espindent
\vskip5mm
\bpindent\LARGE\bf
and wish to deduce:
\epindent
\vskip5mm
\bspindent\Large\bf\obeylines
\verb!|- t2 [a,b,c]!
\espindent

\point{This can be done as follows:}
\vskip4mm
\begin{session}\begin{verbatim}
#let th = SPECL ["a";"b";"c"] thm1;;
th = |- t1[a,b,c] ==> t2[a,b,c]

#let thm3 = MP th thm2;;
|- t2[a,b,c]
\end{verbatim}\end{session}

\point{The rule {\tt MATCH\_MP} does this for us:}
\vskip4mm
\begin{session}\begin{verbatim}
#let th3 = MATCH_MP thm1 thm2;;
|- t2[a,b,c]
\end{verbatim}\end{session}

% =====================================================================
\slide{Matching Modus Ponens}

\point{{\tt MATCH\_MP} also handles bound variables and type variables:}
\vskip4mm
\begin{session}\begin{verbatim}
#let thm1 = SKOLEM_CONV "!x:*. ?y:*. P x y";;
thm1 = |- (!x. ?y. P x y) = (?y. !x. P x(y x))

#let thm2 = fst(EQ_IMP_RULE thm1);;
thm2 = |- (!x. ?y. P x y) ==> (?y. !x. P x(y x))

#let thm3 = ASSUME "!n. ?m. n < m";;
thm3 = . |- !n. ?m. n < m

#MATCH_MP thm2 thm3;;
. |- ?y''. !x. x < (y'' x)
\end{verbatim}\end{session}

% =====================================================================
\slide{Resolution}

\point{The main resolution tactic is:}
\vskip 5mm
\bspindent\Large\bf\obeylines
\verb!IMP_RES_TAC : thm -> tactic!
\espindent
\vskip5mm
\bpindent\LARGE\bf
which takes an implication
\epindent
\vskip5mm
\bspindent\Large\bf
{\tt |- $A_1$ ==> ($A_2$ ==> \dots ==> ($A_n$ ==> $C$))}
\espindent
\vskip5mm
\bpindent\LARGE\bf
and does the following:
\epindent
\subpoint{match the antecedants $A_i$ to the assumptions of the goal}
\subpoint{derive conclusions from the resulting theorems by modus ponens and
add them to the\\ assumptions of the goal.}

% =====================================================================
\slide{Example of Resolution}

\point{Example:}
\vskip4mm
\begin{session}\begin{verbatim}
#set_goal (["a < b";"b < c"], "~(c < a)");;
"~c < a"
    [ "b < c" ]
    [ "a < b" ]

() : void

#let [_;thm] = RES_CANON LESS_TRANS;;
thm = |- !n p. n < p ==> (!m. m < n ==> m < p)

#expand (IMP_RES_TAC thm);;
OK..
"~c < a"
    [ "b < c" ]
    [ "a < b" ]
    [ "!m. m < a ==> m < b" ]
    [ "a < c" ]

() : void
\end{verbatim}\end{session}

\point{If the matching can be done in more than one way, all possible results
are put onto the\\
assumption list.}

% =====================================================================
\slide{Example Continued}

\point{{\tt IMP\_RES\_TAC} can solve goals:}
\vskip4mm
\begin{session}\begin{verbatim}
#set_goal (["a < b";"b < c"], "(a < c)");;
"a < c"
    [ "b < c" ]
    [ "a < b" ]

() : void

#LESS_TRANS;;
|- !m n p. m < n /\ n < p ==> m < p

#expand (IMP_RES_TAC LESS_TRANS);;
OK..
goal proved
.. |- a < c

Previous subproof:
goal proved
() : void
\end{verbatim}\end{session}

% =====================================================================
\slide{Canonical form}

\point{The implication given to {\tt IMP\_RES\_TAC} is put into canonical form
by the function:}
\vskip7mm
\bspindent\Large\bf
\verb!RES_CANON : thm -> thm list!
\espindent
\vskip7mm
\bpindent\LARGE\bf
which does the following transformations:
\espindent
\vskip7mm
{\Large
\noindent\begin{tabular}{@{\hskip\sind}l%
@{\quad$\Longrightarrow$\quad}l@{\qquad}}
  \verb!~t!    
  &  \verb!t ==> F!\\
  \verb!t1 /\ t2!	
  &  \verb!t1!, \verb!t2!\\
  \verb!(t1\/t2) ==> t !	
  &  \verb!t1 ==> t!, \verb!t2 ==> t!\\
  \verb+(t1/\t2) ==> t+ 
  &  \verb!t1 ==> (t2 ==> t)!, \\
  \verb+(t1/\t2) ==> t+  & \verb!t2 ==> (t1 ==> t)! \\
  \verb!t1 = t2!
  &  \verb!t1 ==> t2!, \verb!t2 ==> t1! \\
  \verb+!x. t1 ==> t2+ &
  \verb+t1 ==> !x.t2+ \\
  \verb+(?x. t1) ==> t2+ &
  \verb+!x'. t1+[\verb!x'!/\verb!x!]\verb+ ==> t2+ \\
  \verb!t1 ==> t2! & \verb!t1 ==> t2'! \\
\end{tabular}}


\point{When {\tt RES\_CANON} generates more than one\\
implication, they are all used for resolution.}


% =====================================================================
\slide{Resolution}

\point{The tactic:}
\vskip5mm
\bspindent\Large\bf
\verb!RES_TAC : tactic!
\espindent
\vskip5mm
\bpindent\LARGE\bf
does resolution {\tt among\/} the assumptions of a goal by 
calling {\tt IMP\_RES\_TAC} for all the implications in the assumption list.
\epindent

\point{We shall see some examples later.}

% =====================================================================
\slide{Solving Goals with Resolution}

\point{A goal is solved if resolution 
generates:}
\subpoint{the theorem {\tt $A$ |- F}, or }
\subpoint{a theorem $\alpha$-equivalent to the goal's conclusion}

\point{Example proof:}
\begin{session}\begin{verbatim}
#set_goal 
  ([], "((!x:*. P x ==> Q x) /\ P y) ==> Q y");;
"(!x. P x ==> Q x) /\ P y ==> Q y"

() : void

#expand (REPEAT STRIP_TAC);;
OK..
"Q y"
    [ "!x. P x ==> Q x" ]
    [ "P y" ]

() : void

#expand RES_TAC;;
OK..
goal proved
|- (!x. P x ==> Q x) /\ P y ==> Q y

Previous subproof:
goal proved
() : void
\end{verbatim}\end{session}


% =====================================================================
\slide{Summary: Resolution}

\point{The resolution tactics:}
\subpoint{{\tt IMP\_RES\_TAC : thm -> tactic}}
\subpoint{{\tt RES\_TAC : tactic}}

\point{Implications put in canonical form by:}
\subpoint{{\tt RES\_CANON : thm -> thm list}}


% =====================================================================
\slide{Example Tactic Proof}

\point{The goal to prove is:}
\vskip4mm
\begin{session}\begin{verbatim}
#set_goal([], "!m n. (n<=m) ==> ?p. m=n+p");;
"!m n. n <= m ==> (?p. m = n + p)"

() : void
\end{verbatim}\end{session}

\point{Expand with the definition of {\tt <=} and strip:}
\vskip4mm
\begin{session}\begin{verbatim}
#expand(PURE_ONCE_REWRITE_TAC [LESS_OR_EQ]);;
OK..
"!m n. n < m \/ (n = m) ==> (?p. m = n + p)"

() : void

#expand(REPEAT STRIP_TAC);;
OK..
2 subgoals
"?p. m = n + p"
    [ "n = m" ]

"?p. m = n + p"
    [ "n < m" ]

() : void
\end{verbatim}\end{session}

% =====================================================================
\slide{Example Continued}

\point{We can use the fact {\tt LESS\_ADD\_{1}}:}
\vskip4mm
\begin{session}\begin{verbatim}
#LESS_ADD_1;;
|- !m n. n < m ==> (?p. m = n + (p + 1))

#expand(IMP_RES_TAC LESS_ADD_1);;
OK..
"?p. m = n + p"
    [ "n < m" ]
    [ "m = n + (p + 1)" ]

() : void
\end{verbatim}\end{session}

\point{We now substitute for {\tt p}:}
\vskip4mm
\begin{session}\begin{verbatim}
#expand (FIRST_ASSUM SUBST1_TAC);;
OK..
"?p'. n + (p + 1) = n + p'"
    [ "n < m" ]
    [ "m = n + (p + 1)" ]

() : void
\end{verbatim}\end{session}


% =====================================================================
\slide{Example Continued}

\point{The proof of this subgoal is now easy:}
\vskip4mm
\begin{session}\begin{verbatim}
#expand (EXISTS_TAC "p + 1");;
OK..
"n + (p + 1) = n + (p + 1)"
    [ "n < m" ]
    [ "m = n + (p + 1)" ]

() : void

#expand REFL_TAC;;
OK..
goal proved
.. |- n + (p + 1) = n + (p + 1)
.. |- ?p'. n + (p + 1) = n + p'
.. |- ?p. m = n + p
. |- ?p. m = n + p

Previous subproof:
"?p. m = n + p"
    [ "n = m" ]

() : void
\end{verbatim}\end{session}



% =====================================================================
\slide{Example Continued}


\point{The correct value is {\tt 0}:}
\vskip4mm
\begin{session}\begin{verbatim}
#expand (EXISTS_TAC "0");;
OK..
"m = n + 0"
    [ "n = m" ]

() : void
\end{verbatim}\end{session}



\point{An important built-in theorem helps now:}
\vskip4mm
\begin{session}\begin{verbatim}
#ADD_CLAUSES;;
|- (0 + m = m) /\
   (m + 0 = m) /\
   ((SUC m) + n = SUC(m + n)) /\
   (m + (SUC n) = SUC(m + n))

#expand (ASM_REWRITE_TAC [ADD_CLAUSES]);;
OK..
goal proved
. |- m = n + 0
. |- ?p. m = n + p
|- !m n. n < m \/ (n = m) ==> (?p. m = n + p)
|- !m n. n <= m ==> (?p. m = n + p)

Previous subproof:
goal proved
() : void
\end{verbatim}\end{session}


% =====================================================================
\slide{Example Concluded}

\point{The entire proof is:}
\vskip4mm
\begin{session}\begin{verbatim}
let thm =
    TAC_PROOF
    (([], "!m n. (n<=m) ==> ?p. m=n+p"),
     PURE_ONCE_REWRITE_TAC [LESS_OR_EQ] THEN
     REPEAT STRIP_TAC THENL
     [IMP_RES_TAC LESS_ADD_1 THEN
      FIRST_ASSUM SUBST1_TAC THEN
      EXISTS_TAC "p + 1" THEN
      REFL_TAC;
      EXISTS_TAC "0" THEN
      ASM_REWRITE_TAC [ADD_CLAUSES]]);;
thm = |- !m n. n <= m ==> (?p. m = n + p)
\end{verbatim}\end{session}

\end{document}

