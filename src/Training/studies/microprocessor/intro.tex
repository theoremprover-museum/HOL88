\setlength{\unitlength}{1mm}
\mbox{}
\vskip .6in
\begin{center}\huge\bf
A Simple Microprocessor System
\end{center}
\vskip .4in

\section{Introduction}

This case study describes how the \HOL\ system has been
used to verify the design of a very simple microprocessor called
\Tamarack.
This microprocessor has just eight
programming level instructions and a single addressing mode.
The only kind of exception is
a single level, non-vectored hardware interrupt.
The microprocessor can be interfaced to external memory to
operate in one of three possible modes:
fully synchronous,
fully asynchronous,
and extended cycle mode.
All I/0 is memory-mapped.
Figure~\ref{fig-func}
shows a functional diagram for the externally available
signals of \Tamarack\
(excluding clock signals, reset signal and voltage sources).
These signals would be pins or groups of pins
in a microchip implementation of this design.

\begin{figure}
\begin{center}
% ===== Functional View ================================================
% ======================================================================
\setlength{\unitlength}{1mm}
\begin{picture}(100,80)(0,0)

  {\thicklines {\put(20,0){\framebox(60,80)[c]{\Large TAMARACK-3}}}}

%\put(0,66){
\put(-1.9,66){
  \begin{picture}(14,8)
  \put(0,1.5){\line(1,0){16}}
  \put(16,1.5){\line(0,-1){1.5}}
  \put(16,0){\line(1,1){4}}
  \put(20,4){\line(-1,1){4}}
  \put(16,8){\line(0,-1){1.5}}
  \put(16,6.5){\line(-1,0){16}}
  \put(0,6.5){\line(0,-1){5}}
  \end{picture}}

  { {\put(0,50){\vector(1,0){20}}}}
  { {\put(0,30){\vector(1,0){20}}}}
  { {\put(0,10){\vector(1,0){20}}}}

  \put(17,70){\makebox(0,0)[r]{\small{\verb"datain"}}}
  \put(17,53){\makebox(0,0)[r]{\small{\verb"dack"}}}
  \put(17,33){\makebox(0,0)[r]{\small{\verb"idle"}}}
  \put(17,13){\makebox(0,0)[r]{\small{\verb"ireq"}}}

  { {\put(80,72){\vector(1,0){20}}}}

%\put(80,52){
\put(78.7,52){
  \begin{picture}(14,8)
  \put(0,1.5){\line(1,0){16}}
  \put(16,1.5){\line(0,-1){1.5}}
  \put(16,0){\line(1,1){4}}
  \put(20,4){\line(-1,1){4}}
  \put(16,8){\line(0,-1){1.5}}
  \put(16,6.5){\line(-1,0){16}}
  \put(0,6.5){\line(0,-1){5}}
  \end{picture}}

  { {\put(80,40){\vector(1,0){20}}}}

\put(78.7,20){
  \begin{picture}(14,8)
  \put(0,1.5){\line(1,0){16}}
  \put(16,1.5){\line(0,-1){1.5}}
  \put(16,0){\line(1,1){4}}
  \put(20,4){\line(-1,1){4}}
  \put(16,8){\line(0,-1){1.5}}
  \put(16,6.5){\line(-1,0){16}}
  \put(0,6.5){\line(0,-1){5}}
  \end{picture}}

  { {\put(80,8){\vector(1,0){20}}}}

  \put(83,75){\makebox(0,0)[l]{\small{\verb"wmem"}}}
  \put(83,56){\makebox(0,0)[l]{\small{\verb"dataout"}}}
  \put(83,43){\makebox(0,0)[l]{\small{\verb"dreq"}}}
  \put(83,24){\makebox(0,0)[l]{\small{\verb"addr"}}}
  \put(83,11){\makebox(0,0)[l]{\small{\verb"iack"}}}

\end{picture}

\vspace*{.35in}

\begin{tabular}{ll@{\hspace{1cm}}ll}

\verb"datain"& - data from memory&
\verb"wmem"& - read/write select\\

\verb"dack"& - data acknowledge&
\verb"dataout"& - data to memory\\

\verb"idle"& - extended cycle mode&
\verb"dreq"& - data request\\

\verb"ireq"& - interrupt request&
\verb"addr"& - address to memory\\

 & &
\verb"iack"& - interrupt acknowledge

\end{tabular}

\vspace*{.15in}


\caption{Functional View of the TAMARACK-3 Microprocessor.}
\label{fig-func}
\end{center}
\end{figure}

The original version of this example was described by Gordon
\cite{Gordon81,Gordon:tech42} for a predecessor of the \HOL\ system called
\LCF\_LSM\ \cite{Gordon:tech41}.
This example was re-done in \HOL,
implemented as a CMOS microchip
(as an exercise in verification-driven design),
and given the name \mbox{\footnotesize TAMARACK-1}
\cite{Joyce:tech100,Joyce:calgary86,Joyce:integration}.
The design was then streamlined in \mbox{\footnotesize TAMARACK-2}
by eliminating some unessential control features
as a first step in the development of this case study.
The version described here,
\Tamarack,
is distinguished from earlier versions by the addition
of a hardware interrupt mechanism and the ability to interact
asynchronously with external devices using handshaking signals.
Gordon's original version of this example has also been used to illustrate
several other approaches to simulating and verifying
hardware.  This includes work by
Barrow \cite{Barrow}, Camilleri \cite{Camilleri:thesis,Camilleri:pisa},
Curzon \cite{Curzon:thesis}, Davie \cite{Davie:thesis},
Richards \cite{Richards:tech84},
Shankar et al. \cite{SRI}, Van Tassel \cite{VanTassel:thesis}, and
Weise \cite{Weise:thesis}.

Because
\Tamarack\ was designed as a verification example,
it is not seriously intended for practical applications.
\footnote{
But with a few more ALU functions,
more kinds of conditional branches, and
an alternative to absolute addressing for instruction operands,
it is possible to imagine the use
of a \mbox{\scriptsize TAMARACK}-like microprocessor in a very simple
real-time control application
(see suggested exercise in Section~\ref{sec-exer})}
We have deliberately avoided
some kinds of complexity found in `real designs' such
as multiple addressing modes
which do not necessarily contribute
interesting verification problems aside from the
very important problem of managing the sheer size of such proofs.
Multiple addressing modes are not provided because the problem of
establishing correctness results for a particular instruction
in different addressing modes is largely a matter of repeating
the same proof strategy with slight variations for each case.

The relatively simple design of \Tamarack\
is obviously an advantage when presenting this case study
as an introduction to verifying microprocessor systems.
But not every aspect of \Tamarack\ is completely trivial.
By avoiding as much uninteresting complexity as possible,
we have been left with a great deal of flexibility
to consider several problems at the forefront of
research into the use of formal methods
to verify microprocessor systems.
Some of the research problems considered in this work are:

\begin{itemize}
\item
Strategies
and techniques for structuring a formal verification into several layers.
\item
Trade-offs between using functions and relations for various parts
of the formal specification.
\item
Reasoning about
intrinsically complex aspects of system timing such as
the interaction of a microprocessor with external devices.
\end{itemize}

Most of the
basic proof techniques described in this case study
are standard in \HOL\ circles.
Readers curious about how these basic techniques scale upwards
to the complexity of a real design are referred to
work by Cohn \cite{Cohn:calgary86,Cohn:banff87,Cohn:jar}
on using the \HOL\ system to verify the commercially-available
\Viper\ microprocessor \cite{Cullyer:lncs331}.
In addition to these basic techniques,
some technical points of interest in this case study are:

\begin{itemize}
\item
The use of uninterpreted data types (represented as polymorphic variables)
in place of special-purpose data types for
reasoning about hardware.
\item
The `parameterization' of the formal
theory\footnote{
The term `theory' has a technical meaning in the \HOL\ system,
but we use this term more generally throughout this case study to
mean the formal specification of \mbox{\scriptsize TAMARACK-3} and
its logical consequences.}
by a representation variable which can be used
to relate this theory to other levels of proof.
\item
Embedding another calculi, namely, a form of temporal logic,
in the framework of higher-order logic.
\end{itemize}

The work described this case study owes much to
previous work by Hunt on the FM8501 microprocessor
\cite{Hunt:thesis,Bevier87,Bevier89} and by Cohn on the
\Viper\ microprocessor
\cite{Cohn:calgary86,Cohn:banff87,Cohn:jar}.
These two examples are widely seen as the current
state-of-the-art in microprocessor verification.
Hunt was the first (as far as we know) to consider
the problem of reasoning about the
implementation of a handshaking protocol
in a microprocessor system.
Cohn's work on \Viper\ has contributed a deep appreciation
of the scope and limits of using formal methods
to verify microprocessor systems.

The diversity of structure and behaviour in a typical microprocessor
provides a rich source of verification problems.
For this reason, it is likely
that microprocessor verification will be the main-spring of many future
developments in hardware verification.

\subsection{Overview}

This case study is less closely tied to
the intricacies of the \HOL\ system
than some of the other case studies in this tutorial.
This is partly due to the size of the formal specifications
and proofs of correctness.
We also interested in reaching a more general audience
than just \HOL\ users
as a basis for comparisons between the \HOL\ approach
(the system, the logic and conventional specification style)
and other approaches to verifying microprocessor systems.

Readers with access to the \HOL\ system who
want to experiment with the \HOL\ system by re-running
parts of this proof will find
essential documentation in
Sections~\ref{sec-abst} and \ref{sec-spec}.
A very straightforward version of the proof
(the so-called ``Bread and Butter'' method) is outlined
in Sections~\ref{sec-plan} and~\ref{sec-verf}.
Enhancements and extensions to this proof
(which may only interest some readers)
are described in Sections~\ref{sec-temp} and~\ref{sec-asyn}.

Because the proof of correctness is too large
to describe in terms of \HOL\ interactions,
most of our discussion consists of proof sketches
which outline the main steps in the proof.
However,
Sections~\ref{sec-verfphase} and~\ref{sec-verfmicro}
illustrate some details of how these proof steps are achieved by
showing a sequence of intermediate results extracted
from a \HOL\ session.
This may be helpful to readers
who are interested in the level of interaction with the \HOL\ system
required for this kind of proof.

Other readers with a more general interest
in how formal proof can be used to reason about a large hardware design
will find methodology discussed throughout this document.
Some of our discussion concerns the \HOL\ system,
but much of it is also relevant
to verifying hardware in other verification systems besides the
\HOL\ system.
One of the aims of the work described in this case study
is to avoid using special-purpose infra-structure for reasoning
about hardware which might make it difficult to re-produce
hardware proofs in other verification systems.

We begin by describing the design and operation of \Tamarack\
in a conventional style of informal microprocessor description.
This informal description is divided into three main parts:
programming level model, memory interface, and internal operation.
Some aspects of this informal description are
essential documentation.
Other, less-essential parts of this informal description
are mostly intended to provide background for proof strategies.

\subsection{Notation}

Machine-readable \HOL\ syntax
is used in this document except for the
following substitutions where conventional symbols from logic have been
used instead of standard {\footnotesize ASCII} characters.
Section~\ref{sec-asyn} describes a set of additional substitutions,
e.g., $\Box$, $\Diamond$,
used exclusively in our discussion of how
temporal logic can be embedded in the \HOL\ logic.

\begin{quote}
\begin{tabular}{lcl@{\hspace{1in}}lcl}
$\forall$ & replaces & \verb"!"  & $\neg$        & replaces & \verb"~"\\
$\exists$ & \verb+"+ & \verb"?"  & $\lambda$     & \verb+"+ & \verb"\"\\
$\wedge$  & \verb+"+ & \verb"/\" & $\varepsilon$ & \verb+"+ & \verb"@"\\
$\vee$    & \verb+"+ & \verb"\/"
\end{tabular}
\end{quote}

\subsection{Supporting files}

{\bf Formal Specification}
\vskip .075in

\noindent
Section~\ref{sec-abst}
$\left\{
\begin{minipage}[c]{125mm}
\begin{tabular}{ll}
\parbox{27mm}{header.ml} &	- representation variable type\\
\parbox{27mm}{rep.ml} &	- basic data types and operations
\end{tabular}
\end{minipage}
\right.$
\vskip .25in

\noindent
Section~\ref{sec-spec}
${\left\{
\begin{minipage}[c]{125mm}
\begin{tabular}{ll}
\parbox{27mm}{impspec.ml} &	- internal architecture\\
\parbox{27mm}{behspec.ml} &	- programming level model\\
\parbox{27mm}{synspec1.ml} &	- memory model and correctness conditions\\
\parbox{27mm}{example1.ml} &	- assembling microcode with proof rules
\end{tabular}
\end{minipage}
\right.}$
\vskip .25in

\noindent
{\bf Formal Verification: ``Bread and Butter'' Method}
\vskip .075in

\noindent
Section~\ref{sec-verf}
$\left\{
\begin{minipage}[c]{125mm}
\begin{tabular}{ll}
\parbox{27mm}{proof1.ml} &	- symbolic execution of phase level\\
\parbox{27mm}{example2.ml} &	- detailed view of phase level\\
\parbox{27mm}{proof2.ml} &	- symbolic execution of microprogramming level\\
\parbox{27mm}{example3.ml} &	- detailed view of microprogramming level\\
\parbox{27mm}{proof3.ml} &	- top level correctness result (version 1)
\end{tabular}
\end{minipage}
\right.$
\vskip .25in

\noindent
{\bf Synchronizing Multiple Levels of Timing}
\vskip .075in

\noindent
Section~\ref{sec-temp}
$\left\{
\begin{minipage}[c]{125mm}
\begin{tabular}{ll}
\parbox{27mm}{tempabs.ml} &	- temporal abstraction\\
\parbox{27mm}{synspec2.ml} &	- memory model and correctness conditions\\
\parbox{27mm}{proof4.ml} &	- symbolic execution of microprogramming level\\
\parbox{27mm}{proof5.ml} &	- top level correctness result (version 2)
\end{tabular}
\end{minipage}
\right.$
\vskip .25in

\noindent
{\bf Using Temporal Logic to Deal with Asynchrony}
\vskip .075in

\noindent
Section~\ref{sec-asyn}
$\left\{
\begin{minipage}[c]{125mm}
\begin{tabular}{ll}
\parbox{27mm}{tlops.ml} &	- temporal logic operators and derived rules\\
\parbox{27mm}{handspecs.ml} &	- sender and receiver specifications\\
\parbox{27mm}{handthms1.ml} &	- handshaking theorems (sender conditions)\\
\parbox{27mm}{handthms2.ml} &	- handshaking theorems (data transfers)\\
\parbox{27mm}{asynspec.ml} &	- memory model and correctness conditions\\
\parbox{27mm}{proof6.ml} &	- correct implementation of sender conditions\\
\parbox{27mm}{example4.ml} &	- simple implementation of sender conditions\\
\parbox{27mm}{proof7.ml} &	- collapse `waitidle' loops to single steps\\
\parbox{27mm}{proof8.ml} &	- collapse `waitdack' loops to single steps\\
\parbox{27mm}{proof9.ml} &	- symbolic execution of microprogramming level\\
\parbox{27mm}{proof10.ml} &	- top level correctness result (version 3)
\end{tabular}
\end{minipage}
\right.$
\vskip .25in

\noindent
{\bf Relating This Proof to Other Levels}
\vskip .075in

\noindent
Section~\ref{sec-relate}
$\left\{
\begin{minipage}[c]{125mm}
\begin{tabular}{ll}
\parbox{27mm}{example5.ml} &	- linking this proof into a verified stack
\end{tabular}
\end{minipage}
\right.$
\vskip .25in
