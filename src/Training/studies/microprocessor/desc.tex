\section{Programming level model}
\label{sec-progmodel}

The programming level model, or external architecture,
of \Tamarack\ is a description of its operation as seen by
a programmer.
This model hides
all aspects of the internal architecture
which the programmer does not need to know about when
writing programs for this microprocessor.

The programming level model can be viewed as an interpreter
for manipulating a set of variables which corresponds to the externally
visible state of the microprocessor.
It consists of four main parts:

\begin{itemize}
\item
Basic data types and primitive operations.
\item
Variables manipulated by the interpreter.
\item
Format of instructions.
\item
Instructions semantics.
\end{itemize}

Our presentation of the programming level model is organized
around these four main parts.
Although we describe hardware interrupts separatedly from the
semantics of ordinary program instructions,
hardware interrupts in \Tamarack\ can be regarded as just
another kind of instruction in the programming level model.
\footnote{
This is because the current instruction cycle executes to completion
before an interrupt request is allowed to interrupt the normal
flow of program execution.
Instructions are also indivisible (with respect to
external interrupt requests) in many commercially-available
microprocessors such as the M68000.}

\subsection{Basic data types and primitive operations}

We begin with the basic data types and primitive operations
used in the programming level model.
The data type \verb":bool" is used to represent voltage values
or logical conditions.
The data type \verb":num" is used when some lower level form
of data is interpreted as the representation of a natural number.
The remaining data types correspond to machine words,
groups of bits within a machine word,
and memory states.
A complete list of basic data types used in
the programming level model is shown below.

\begin{center}
\begin{tabular}{ll}
\verb":bool"& - Boolean values \{\verb"T",\verb"F"\}\\
\verb":num"& - natural numbers \{\verb"0",\verb"1",\verb"2",\ldots\}\\
\verb":*wordn"& - full-size machine words\\
\verb":*word3"& - instruction opcodes\\
\verb":*address"& - memory addresses\\
\verb":*memory"& - memory states
\end{tabular}
\end{center}

A conventional description would typically be very precise
about details such as the number of bits in a machine word
and the size of memory.
However, we avoid specifying these details by regarding
\verb":*wordn", \verb":*word3", \verb":*address" and \verb":*memory"
as {\it uninterpreted types}.
The actual representation of these basic data types
may be thought of as implementation dependent details.
We use the prefix \verb"*" to distinguish these as uninterpreted types.

Functional elements such as the ALU
({\it Arithmetic Logic Unit})
at the lowest level of architectural
description perform various operations on data.
It is also possible and desirable to
avoid specifying any details about these operations.
Instead, these operations are
regarded as {\it uninterpreted primitives}.
Primitive operations used to describe the design and operation of
\Tamarack\ are listed below
along with an informal description of their types.
Although we use the syntax of \HOL\ to describe these types,
a slightly different set of types is used in the formal theory
for reasons explained later in Section~\ref{sec-basis}.

\begin{center}
\begin{tabular}{lll}
\verb"iszero"&
\verb":*wordn"$\rightarrow$\verb"bool"&
test if zero\\
\verb"inc"&
\verb":*wordn"$\rightarrow$\verb"*wordn"&
increment\\
\verb"add"&
\verb":(*wordn"$\times$\verb"*wordn)"$\rightarrow$\verb"*wordn"&
addition\\
\verb"sub"&
\verb":(*wordn"$\times$\verb"*wordn)"$\rightarrow$\verb"*wordn"&
subtraction\\
\verb"wordn"&
\verb":num"$\rightarrow$\verb"*wordn"&
representation of a number\\
\verb"valn"&
\verb":*wordn"$\rightarrow$\verb"num"&
value of a full-size word\\
\verb"opcode"&
\verb":*wordn"$\rightarrow$\verb"*word3"&
extract opcode field\\
\verb"val3"&
\verb":*word3"$\rightarrow$\verb"num"&
value of an opcode\\
\verb"address"&
\verb":*wordn"$\rightarrow$\verb"*address"&
extract address field\\
\verb"fetch"&
\verb":(*memory"$\times$\verb"*address)"$\rightarrow$\verb"*wordn"&
read memory\\
\verb"store"&
\verb":(*memory"$\times$\verb"*address"$\times$\verb"*wordn)"$\rightarrow$\verb"*memory"&
write memory
\end{tabular}
\end{center}

The above list also gives a suggested interpretation for each of
the uninterpreted primitives.
Although we avoid specific details about
the operations denoted by these uninterpreted primitives,
we sometimes relax our presentation style
by referring to an uninterpreted primitive in terms of its
suggested interpretation.

Our use of uninterpreted types for data and uninterpreted primitives for
operations on these data types is one of the key concepts presented
in this case study.
It reflects our approach to formal specification
which aims to separate irrelevant detail from
those aspects of the design
have been formally considered in the formal proof of correctness.
We elaborate on this idea and
our approach to formal specification later in Section~\ref{sec-abst}.
Furthermore, by not specifying any details about
basic data types or primitive operations on data, we have an informal
description which is generalized
over a whole range of possible word and memory
sizes and possible interpretations
for the primitive operations.

\subsection{Externally visible state}

The set of variables manipulated by the programming level model
corresponds to the externally visible state of the microprocessor.
In \Tamarack, these variables are:

\begin{center}
\begin{tabular}{ll}
\verb"mem"& - memory\\
\verb"pc"& - program counter\\
\verb"acc"& - accumulator\\
\verb"rtn"& - return address register\\
\verb"iack"& - interrupt acknowledge flag
\end{tabular}
\end{center}

The memory stores memory states, represented by the
data type \verb":*memory".
Each of the registers stores full-size memory words, represented
by the data type \verb":*wordn".
The interrupt acknowledge flag is stored internally by a flipflop whose
value belongs to the data type \verb":bool".

\subsection{Instruction word format}

Instructions are exactly one full-size machine word.
Although specific details about word size and
instruction word format are not given in this description,
we can assume that the instruction word consists of a 3-bit opcode
(since there are eight different instructions)
with the remaining bits used as an operand address.
The operand address
is the absolute address of a memory word
which may be used as the address of either data or an instruction.

\begin{center}
% ===== Instruction Word Format ========================================
% ======================================================================
\setlength{\unitlength}{1mm}
\begin{picture}(125,16)(0,0)

  {\thicklines {\put(0,9){\framebox(125,7)[c]{}}}}
  {\thicklines {\put(35,9.5){\line(0,1){6}}}}

  \put(17.5,5){\makebox(0,0)[c]{$\underbrace{\rule{32mm}{0mm}}$}}
  \put(80,5){\makebox(0,0)[c]{$\underbrace{\rule{87mm}{0mm}}$}}
  \put(17.5,3){\makebox(0,0)[t]{3-bit opcode}}
  \put(80,3){\makebox(0,0)[t]{operand address}}

\end{picture}

\end{center}

Opcodes and operand addresses
are represented by the uninterpreted types \verb":*word3"
and \verb":*address".
They are extracted from an instruction word by the uninterpreted
primitives \verb"opcode" and \verb"address".

\subsection{Instruction set semantics}
\label{sec-semantics}

The eight \Tamarack\ programming level instructions
are in Table~\ref{tab-inst}.
Their opcode values and a brief explanation of each instruction are
also given in the table.
The opcode is extracted from the current instruction word by
\verb"opcode" and its numerical value is then obtained
by applying \verb"val3" to the extracted opcode.

\begin{table}
\begin{center}
\begin{tabular}{|l|c|l|}
\hline
Instruction& Opcode Value& Effect\\ \hline\hline
JZR& 0& jump if zero\\ \hline
JMP& 1& jump\\ \hline
ADD& 2& add accumulator\\ \hline
SUB& 3& subtract accumulator\\ \hline
LDA& 4& load accumulator\\ \hline
STA& 5& store accumulator\\ \hline
RFI& 6& return from interrupt\\ \hline
NOP& 7& no operation \\ \hline
\end{tabular}
\caption{TAMARACK-3 Instruction Set}
\label{tab-inst}
\end{center}
\end{table}

The four data processing instructions,
ADD, SUB, LDA and STA,
involve both the
accumulator \verb"acc" and memory \verb"mem".
The other four instructions,
JZR, JMP, RFI and NOP, are control instructions
with no effect on either the accumulator or memory.
The only conditional branch, JZR, tests
whether the accumulator \verb"acc"
contains the machine representation of zero.

The address of the current instruction is always given by
the program counter \verb"pc" at the beginning of
each instruction cycle.
Operationally, the program counter \verb"pc" is full-size register
but only the address field of this register
is used to access the current instruction word from memory.
When a jump is taken as a result of either a JMP or JZR instruction,
the entire instruction word is loaded into the program counter \verb"pc"
but only the address field has any significance.

The eight programming level instructions are described
in the below.
Only changes to the current state of the external architecture
are described.
Unaffected components of the externally visible state are not mentioned.
Some of these descriptions are simplified by using
the abbreviations,

\begin{quote}
\verb"inst = fetch (mem,(address pc))"

\verb"operand = fetch (mem,(address inst))"
\end{quote}

\noindent
for the current instruction word
and the operand addressed by this instruction.
The informal notation,

\begin{quote}
$<$destination$>$ $\leftarrow$ $<$expression$>$
\end{quote}

\noindent
is used to denote when a value computed from the current machine state
is loaded into a register, flipflop or memory to form a component of the
next machine state.

\subsubsection*{JZR - jump if zero}

\begin{quote}
\verb"pc" $\leftarrow$ if \verb"iszero acc" then \verb"inst" else \verb"inc pc"
\end{quote}

If the result of applying \verb"iszero" to the current contents of the
accumulator \verb"acc" is \verb"T",
then the current instruction word is loaded into the program
counter \verb"pc".
Otherwise, the \verb"inc" operation
is applied to the current contents of the program counter \verb"pc" to
obtain its new value.

\subsubsection*{JMP - jump}

\begin{quote}
\verb"pc" $\leftarrow$ \verb"inst"
\end{quote}

The current instruction word is unconditionally loaded into the program
counter \verb"pc".

\subsubsection*{ADD - add accumulator}

\begin{quote}
\verb"acc" $\leftarrow$ \verb"add (acc,operand)"\\
\verb"pc" $\leftarrow$ \verb"inc pc"
\end{quote}

The \verb"add" operation is applied to the current contents
of the accumulator \verb"acc" and the memory word addressed by the operand
address field of the current instruction. The result is loaded into
the accumulator \verb"acc".
The \verb"inc" operation
is applied to the current contents of the program counter \verb"pc" to
obtain its new value.

\subsubsection*{SUB - subtract accumulator}

\begin{quote}
\verb"acc" $\leftarrow$ \verb"sub (acc,operand)"\\
\verb"pc" $\leftarrow$ \verb"inc pc"
\end{quote}

The \verb"sub" operation is applied to the current contents
of the accumulator \verb"acc" and the memory word addressed by the operand
address field of the current instruction.  The result is loaded into
the accumulator \verb"acc".
The \verb"inc" operation
is applied to the current contents of the program counter \verb"pc" to
obtain its new value.

\subsubsection*{LDA - load accumulator}

\begin{quote}
\verb"acc" $\leftarrow$ \verb"operand"\\
\verb"pc" $\leftarrow$ \verb"inc pc"
\end{quote}

The memory word addressed by the operand
address field of the current instruction is loaded into
the accumulator \verb"acc".
The \verb"inc" operation
is applied to the current contents of the program counter \verb"pc" to
obtain its new value.

\subsubsection*{STA - store accumulator}

\begin{quote}
\verb"mem" $\leftarrow$ \verb"store (mem,address inst,acc)"\\
\verb"pc" $\leftarrow$ \verb"inc pc"
\end{quote}

The current contents of the accumulator \verb"acc" are stored in
external memory at the location specified by the operand
address field of the current instruction.
The \verb"inc" operation
is applied to the current contents of the program counter \verb"pc" to
obtain its new value.

\subsubsection*{RFI - return from interrupt}

\begin{quote}
\verb"pc" $\leftarrow$ \verb"rtn"\\
\verb"iack" $\leftarrow$ \verb"F"
\end{quote}

The current contents of the return address register \verb"rtn"
are loaded into the program counter \verb"pc" and the interrupt acknowledge
flag \verb"iack" is reset to \verb"F".
This instruction does not check whether the interrupt acknowledge
flag \verb"iack" is currently set.

\subsubsection*{NOP - no operation}

\begin{quote}
\verb"pc" $\leftarrow$ \verb"inc pc"
\end{quote}

The \verb"inc" operation
is applied to the current contents of the program counter \verb"pc" to
obtain its new value.

\subsection{Hardware interrupts}
\label{sec-ireq}

In the normal flow of program execution, instructions are
sequentially executed according their semantics described
in Section~\ref{sec-semantics}.
The only kind of exception is a 
single level, non-vectored, non-maskable hardware interrupt
which is generated by setting the interrupt request pin \verb"irq"
to \verb"T".

An interrupt request will be detected and processed as soon as execution
of the current instruction is completed.
Normally, the interrupt will be detected within a few clock cycles
but this may be delayed for an arbitrary number of
clock cycles when the microprocessor is
operating in either fully asynchronous mode or extended cycle mode.
Because only a single level of interrupt is supported,
the value of the interrupt request pin \verb"irq" will be ignored
if the interrupt acknowledge flag is already \verb"T" indicating
that a previous interrupt is still being serviced.

\begin{quote}
if \verb"iack = F" then

\hspace*{.25in}\verb"pc" $\leftarrow$ \verb"0"\\
\hspace*{.25in}\verb"rtn" $\leftarrow$ \verb"pc"\\
\hspace*{.25in}\verb"iack" $\leftarrow$ \verb"T"
\end{quote}

The interrupt request is processed by saving the current value of the program
counter \verb"pc" in
the return address register \verb"rtn",
setting the interrupt acknowledge flag \verb"iack" to \verb"T" and
loading the machine representation of zero into the program counter \verb"pc".
The interrupt service routine is assumed to begin at location zero
in memory.

At the end of the interrupt service routine,
a return-from-interrupt instruction RFI is executed causing
the saved return address
stored in \verb"rtn" to be loaded into the program counter
\verb"pc" and the interrupt acknowledge flag to be reset to \verb"F".

\section{Memory interface}
\label{sec-memory}

The microprocessor can be interfaced to external memory to
operate in one of three possible modes:
fully synchronous,
fully asynchronous,
or extended cycle mode.
The mode of operation is selected by the input pins
\verb"dack" and \verb"idle".
The \verb"dack" pin is used as a handshaking signal
in fully asynchronous mode and extended cycle mode.
In extended cycle mode,
the \verb"idle" pin is used
in place of a handshaking signal
to indicate when the external memory is idle and ready
to begin another interaction.

Although a bi-directional bus would typically be used to transfer data
between a microprocessor and external memory,
the design of \Tamarack\ uses
two separate uni-directional busses,
\verb"datain" and \verb"dataout".
Data is sent to external memory on the \verb"dataout" bus
and received from external memory on the \verb"datain" bus.
Memory addresses are sent to external memory on the \verb"addr" bus.

The operations performed by external memory are denoted by the
uninterpreted primitives \verb"fetch" and \verb"store".
Although the synchronization details depend on the memory mode,
the result of a read request is described by the equation,

\begin{quote}
\verb"datain = fetch (mem,addr)"
\end{quote}

\noindent
and the result of a write request
is described by the following update to the internal
state of memory.

\begin{quote}
\verb"mem" $\leftarrow$ \verb"store (mem,addr,dataout)"
\end{quote}

Synchronization details for each of the three memory modes
are described below.

\subsection{Fully synchronous mode}

In fully synchronous mode,
every memory interaction is completed in a single cycle.
The microprocessor is made to operate in this mode by
wiring both of the pins \verb"idle" and \verb"dack" to \verb"T".
A detailed timing analysis is needed to ensure that the
external memory can always satisfy memory requests within
a single clock cycle.

Figure~\ref{fig-syn} shows the interconnections between
external memory and \Tamarack\
when operating in fully synchronous mode.
The memory request pin \verb"dreq" is not needed in this mode
because every clock cycle is assumed to be either a read or write request.
The type of request is indicated by the \verb"wmem" pin which is
normally reset to \verb"F" except when writing to memory.

\begin{figure}
\begin{center}
% ===== Synchronous Mode ===============================================
% ======================================================================
\setlength{\unitlength}{1mm}
\begin{picture}(126,50)(0,0)

  {\thicklines {\put(20,0){\framebox(33,40)[c]{TAMARACK-3}}}}

  {{\put(3,35){\vector(1,0){17}}}}
  {{\put(3,25){\vector(1,0){17}}}}
  {{\put(3,15){\vector(1,0){17}}}}
  {{\put(3,5){\vector(1,0){17}}}}

  \put(18,38){\makebox(0,0)[r]{\small{\verb"datain"}}}
  \put(18,28){\makebox(0,0)[r]{\small{\verb"dack"}}}
  \put(18,18){\makebox(0,0)[r]{\small{\verb"idle"}}}
  \put(18,8){\makebox(0,0)[r]{\small{\verb"ireq"}}}

  \put(2,25){\makebox(0,0)[r]{\small{\verb"T"}}}
  \put(2,15){\makebox(0,0)[r]{\small{\verb"T"}}}

  {{\put(53,36){\vector(1,0){30}}}}
  {{\put(53,28){\vector(1,0){30}}}}
  {{\put(53,20){\vector(1,0){17}}}}
  {{\put(53,12){\vector(1,0){30}}}}
  {{\put(53,4){\vector(1,0){17}}}}

  \put(55,39){\makebox(0,0)[l]{\small{\verb"wmem"}}}
  \put(55,31){\makebox(0,0)[l]{\small{\verb"dataout"}}}
  \put(55,23){\makebox(0,0)[l]{\small{\verb"dreq"}}}
  \put(55,15){\makebox(0,0)[l]{\small{\verb"addr"}}}
  \put(55,7){\makebox(0,0)[l]{\small{\verb"iack"}}}

  {\thicklines {\put(83,0){\framebox(33,40)[c]{Memory}}}}

  {{\put(116,20){\line(1,0){10}}}}
  {{\put(126,20){\line(0,1){30}}}}
  {{\put(126,50){\line(-1,0){123}}}}
  {{\put(3,50){\line(0,-1){15}}}}

\end{picture}


\caption{Fully Synchronous Operation.}
\label{fig-syn}
\end{center}
\end{figure}

\subsection{Fully asynchronous mode}

Fully asynchronous interaction with external memory is achieved
when no assumptions are made about
the speed of the external memory relative to the
microprocessor clock speed.
This allows the microprocessor to be interfaced to a mixture
of fast and slow devices in the address space of external memory.
The transfer of data (and memory addresses) between the microprocessor
and external memory is synchronized by handshaking signals
following the four-phase bundled data convention
illustrated in Figure~\ref{fig-hand}.
It is only assumed that wire delays between the microprocessor
and external memory are approximately uniform.

\begin{figure}
\begin{center}
% ===== Handshaking Timing =============================================
% ======================================================================
\setlength{\unitlength}{1mm}
\begin{picture}(116,72)(3,3)

\put(34,3){\line(0,1){67}}
\put(40,3){\line(0,1){67}}
\put(54,3){\line(0,1){67}}
\put(60,3){\line(0,1){67}}
\put(74,3){\line(0,1){67}}
\put(80,3){\line(0,1){67}}
\put(94,3){\line(0,1){67}}
\put(100,3){\line(0,1){67}}

\put(18,14){\makebox(0,0)[r]{\verb"dack"}}
\put(20,10){\thicklines {\line(1,0){36}}}
\put(56,10){\thicklines {\line(1,4){2}}}
\put(58,18){\thicklines {\line(1,0){38}}}
\put(96,18){\thicklines {\line(1,-4){2}}}
\put(98,10){\thicklines {\line(1,0){22}}}

\put(18,29){\makebox(0,0)[r]{\verb"datain"}}
\put(20,29){\thicklines {\line(1,0){36}}}
\put(56,29){\thicklines {\line(1,2){2}}}
\put(56,29){\thicklines {\line(1,-2){2}}}
\put(58,33){\thicklines {\line(1,0){24}}}
\put(58,25){\thicklines {\line(1,0){24}}}
\put(82,33){\thicklines {\line(1,-2){2}}}
\put(82,25){\thicklines {\line(1,2){2}}}
\put(84,29){\thicklines {\line(1,0){36}}}

\put(18,44){\makebox(0,0)[r]{\verb"dreq"}}
\put(20,40){\thicklines {\line(1,0){16}}}
\put(36,40){\thicklines {\line(1,4){2}}}
\put(38,48){\thicklines {\line(1,0){38}}}
\put(76,48){\thicklines {\line(1,-4){2}}}
\put(78,40){\thicklines {\line(1,0){42}}}

\put(18,64){\makebox(0,0)[r]{\verb"wmem"}}
\put(18,59){\makebox(0,0)[r]{\verb"dataout"}}
\put(18,54){\makebox(0,0)[r]{\verb"addr"}}
\put(20,55){\thicklines {\line(1,0){16}}}
\put(20,63){\thicklines {\line(1,0){16}}}
\put(36,55){\thicklines {\line(1,4){2}}}
\put(36,63){\thicklines {\line(1,-4){2}}}
\put(38,63){\thicklines {\line(1,0){24}}}
\put(38,55){\thicklines {\line(1,0){24}}}
\put(62,63){\thicklines {\line(1,-4){2}}}
\put(62,55){\thicklines {\line(1,4){2}}}
\put(64,55){\thicklines {\line(1,0){55}}}
\put(64,63){\thicklines {\line(1,0){55}}}

\end{picture}


\caption{Synchronizing Data Transfer with Handshaking Signals.}
\label{fig-hand}
\end{center}
\end{figure}

Figure~\ref{fig-asyn} shows the interconnections between
external memory and \Tamarack\
when operating in fully asynchronous mode.
In this mode, the \verb"idle" pin is permanently wired to \verb"F".
The acknowledgement signal
\verb"dack" is generated by the external memory (or by peripheral devices
in the case of memory-mapped I/0).

\begin{figure}
\begin{center}
% ===== Asynchronous Mode ==============================================
% ======================================================================
\setlength{\unitlength}{1mm}
\begin{picture}(129,53)(0,0)

  {\thicklines {\put(20,0){\framebox(33,40)[c]{TAMARACK-3}}}}

  {{\put(3,35){\vector(1,0){17}}}}
  {{\put(0,25){\vector(1,0){20}}}}
  {{\put(3,15){\vector(1,0){17}}}}
  {{\put(3,5){\vector(1,0){17}}}}

  \put(18,38){\makebox(0,0)[r]{\small{\verb"datain"}}}
  \put(18,28){\makebox(0,0)[r]{\small{\verb"dack"}}}
  \put(18,18){\makebox(0,0)[r]{\small{\verb"idle"}}}
  \put(18,8){\makebox(0,0)[r]{\small{\verb"ireq"}}}

  \put(2,15){\makebox(0,0)[r]{\small{\verb"F"}}}

  {{\put(53,36){\vector(1,0){30}}}}
  {{\put(53,28){\vector(1,0){30}}}}
  {{\put(53,20){\vector(1,0){30}}}}
  {{\put(53,12){\vector(1,0){30}}}}
  {{\put(53,4){\vector(1,0){17}}}}

  \put(55,39){\makebox(0,0)[l]{\small{\verb"wmem"}}}
  \put(55,31){\makebox(0,0)[l]{\small{\verb"dataout"}}}
  \put(55,23){\makebox(0,0)[l]{\small{\verb"dreq"}}}
  \put(55,15){\makebox(0,0)[l]{\small{\verb"addr"}}}
  \put(55,7){\makebox(0,0)[l]{\small{\verb"iack"}}}

  {\thicklines {\put(83,0){\framebox(33,40)[c]{Memory}}}}

  {{\put(116,30){\line(1,0){10}}}}
  {{\put(126,30){\line(0,1){20}}}}
  {{\put(126,50){\line(-1,0){123}}}}
  {{\put(3,50){\line(0,-1){15}}}}

  {{\put(116,10){\line(1,0){13}}}}
  {{\put(129,10){\line(0,1){43}}}}
  {{\put(129,53){\line(-1,0){129}}}}
  {{\put(0,53){\line(0,-1){28}}}}

\end{picture}


\caption{Fully Asynchronous Operation.}
\label{fig-asyn}
\end{center}
\end{figure}

A memory request is signaled by setting the memory request pin \verb"dreq"
to \verb"T".
The type of request is indicated by the \verb"wmem" pin which is
normally reset to \verb"F" except when writing to memory.
After signaling a memory request,
\verb"dreq" must remain \verb"T" and
the \verb"wmem" flag,
address bus \verb"addr" and dataout bus \verb"dataout" must
remain at stable values
until \verb"dack" becomes \verb"T" signaling
that the request has been satisfied.
In the case of a read request,
incoming data from the external memory will
be stable from when the acknowledgement signal \verb"dack" becomes
\verb"T" until the request signal \verb"dreq" returns to its
original value of \verb"F".
Finally, the microprocessor waits for \verb"dack" to also return to
\verb"F" before starting another memory request.

The use of handshaking signals to synchronize data transfers
between the microprocessor and external memory
requires very little extra circuitry
and no additional control states.
Best case performance by the external memory will
result in exactly the same number of clock cycles as fully
synchronous mode.

\subsection{Extended cycle mode}

A small, but significant variation of fully asynchronous memory
interaction is found in many commercially-available microprocessors
including \Viper\ \cite{Pygott84}.
In \Tamarack, this mode is selected by permanently
wiring the \verb"idle" pin to \verb"T" as shown in Figure~\ref{fig-ext}.
As before,
the acknowledgement signal \verb"dack" is generated by external memory.

The only difference concerns the completion of the memory cycle
after the request signal \verb"dreq" has been reset to \verb"F".
Unlike fully asynchronous mode,
the control logic
does not force the microprocessor to wait until the
acknowledgement signal \verb"dack" also
returns to \verb"F" before starting another memory request.
Instead, it is assumed that
the external memory will always complete the current memory cycle
in time for another request to begin as early as the next clock cycle.
This is a significant difference from fully asynchronous mode because
the speed of external memory is no longer independent of
the microprocessor clock speed.

\begin{figure}
\begin{center}
\input{fig-ext}

\caption{Extended Cycle Operation.}
\label{fig-ext}
\end{center}
\end{figure}

Because extended cycle mode is very similar to fully asynchronous
mode, it is ignored in the rest of this description.
The formal proof of correctness for
\Tamarack\ does not consider this mode of operation.

\section{Internal architecture}
\label{sec-internal}

This section begins with a structural view of the internal architecture
and an overview of how programming level instructions are interpreted
by the hardware.
This is followed by a more detailed view of
the internal architecture as a series of increasingly
concrete interpretation levels.
Finally, we outline some bottom level assumptions which bridge
the gap between our most detailed level of description and actual hardware.

\subsection{Register-transfer level structure}

Figure~\ref{fig-rtl} shows a structural view of
the \Tamarack\ register-transfer level architecture.
It consists of two main parts: a microcoded control unit and
a single-bus datapath.

\begin{figure}
\begin{center}
% ===== Register Transfer Level Architecture ===========================
% ======================================================================
\setlength{\unitlength}{1mm}
% The real height is 224 and origin at 0 but cheat a bit to move the
% caption up.
%\begin{picture}(160,224)(0,0)
\begin{picture}(160,217)(0,3)

% datapath devices
  {\thicklines {\put(90,205){\framebox(30,10)[c]{\large {\verb"interface"}}}}}
  {\thicklines {\put(97.5,190){\framebox(15,10)[c]{\large {\verb"dreq"}}}}}
  {\thicklines {\put(90,175){\framebox(30,10)[c]{\large {\verb"mar"}}}}}
  {\thicklines {\put(90,160){\framebox(30,10)[c]{\large{\verb"addr"}}}}}
  {\thicklines {\put(90,145){\framebox(30,10)[c]{\large {\verb"pc"}}}}}
  {\thicklines {\put(90,130){\framebox(30,10)[c]{\large {\verb"acc"}}}}}
  {\thicklines {\put(90,115){\framebox(30,10)[c]{\large {\verb"zeroflag"}}}}}
  {\thicklines {\put(90,100){\framebox(30,10)[c]{\large {\verb"ir"}}}}}
  {\thicklines {\put(90,85){\framebox(30,10)[c]{\large {\verb"opc"}}}}}
  {\thicklines {\put(90,70){\framebox(30,10)[c]{\large {\verb"rtn"}}}}}
  {\thicklines {\put(97.5,55){\framebox(15,10)[c]{\large {\verb"iack"}}}}}
  {\thicklines {\put(90,40){\framebox(30,10)[c]{\large {\verb"arg"}}}}}
  {\thicklines {\put(90,25){\framebox(30,10)[c]{\large {\verb"alu"}}}}}
  {\thicklines {\put(90,10){\framebox(30,10)[c]{\large {\verb"buf"}}}}}

% datapath outputs
  \put(80,212.5){\line(0,1){7.5}}
  \put(80,220){\vector(1,0){80}}
  \put(120,207.5){\line(1,0){14}}
  \put(136,207.5){\vector(1,0){24}}
  \put(112.5,195){\line(1,0){21.5}}
  \put(136,195){\vector(1,0){24}}
  \put(120,165){\line(1,0){14}}
  \put(136,165){\vector(1,0){24}}
  \put(150,60){\vector(1,0){10}}

  \put(159,222.5){\makebox(0,0)[r]{\small{\verb"wmem"}}}
  \put(159,210){\makebox(0,0)[r]{\small{\verb"dataout"}}}
  \put(159,197.5){\makebox(0,0)[r]{\small{\verb"dreq"}}}
  \put(159,167.5){\makebox(0,0)[r]{\small{\verb"addr"}}}
  \put(159,62.5){\makebox(0,0)[r]{\small{\verb"iack"}}}

% the bus
  \put(135,8){\vector(0,1){209}}
  \put(135,217){\vector(0,-1){209}}
  \put(137,130){\makebox(0,0)[l]{\small{\verb"system"}}}
  \put(137,126){\makebox(0,0)[l]{\small{\verb"bus"}}}

% feedback lines: zeroflag, opc, iack
  \put(120,120){\line(1,0){14}}
  \put(136,120){\line(1,0){13}}
  \put(120,90){\line(1,0){14}}
  \put(136,90){\line(1,0){11}}
  \put(112.5,60){\line(1,0){21.5}}
  \put(136,60){\line(1,0){10}}
  \put(149,120){\line(0,-1){120}}
  \put(147,90){\line(0,-1){88}}
  \put(145,60){\line(0,-1){56}}
  \put(149,0){\line(-1,0){132.5}}
  \put(147,2){\line(-1,0){128.5}}
  \put(145,4){\line(-1,0){124.5}}
  \put(20.5,4){\vector(0,1){76}}
  \put(18.5,2){\vector(0,1){78}}
  \put(16.5,0){\vector(0,1){80}}

% control signals
  {\thicklines {\put(63,10){\framebox(2,192.5){}}}}
  \put(80,210){\vector(1,0){10}}
  \put(80,212.5){\line(0,-1){15}}
  \put(85,207.5){\vector(1,0){5}}
  \put(85,207.5){\line(0,-1){9}}
  \put(85,196.5){\line(0,-1){4}}
  \put(65,197.5){\vector(1,0){32.5}}
  \put(65,192.5){\vector(1,0){32.5}}
  \put(65,180){\vector(1,0){25}}
  \put(65,152.5){\vector(1,0){25}}
  \put(65,147.5){\vector(1,0){25}}
  \put(65,137.5){\vector(1,0){25}}
  \put(65,132.5){\vector(1,0){25}}
  \put(65,107.5){\vector(1,0){25}}
  \put(65,102.5){\vector(1,0){25}}
  \put(65,77.5){\vector(1,0){25}}
  \put(65,72.5){\vector(1,0){25}}
  \put(80,57.5){\vector(1,0){17.5}}
  \put(80,57.5){\line(0,1){15}}
  \put(85,62.5){\vector(1,0){12.5}}
  \put(85,62.5){\line(0,1){9}}
  \put(85,73.5){\line(0,1){4}}
  \put(65,45){\vector(1,0){25}}
  \put(65,32.5){\vector(1,0){25}}
  \put(65,27.5){\vector(1,0){25}}
  \put(65,15){\vector(1,0){25}}

  \put(66,200){\makebox(0,0)[l]{\small{\verb"wmem"}}}
  \put(66,195){\makebox(0,0)[l]{\small{\verb"rmem"}}}
  \put(66,182.5){\makebox(0,0)[l]{\small{\verb"wmar"}}}
  \put(66,155){\makebox(0,0)[l]{\small{\verb"wpc"}}}
  \put(66,150){\makebox(0,0)[l]{\small{\verb"rpc"}}}
  \put(66,140){\makebox(0,0)[l]{\small{\verb"wacc"}}}
  \put(66,135){\makebox(0,0)[l]{\small{\verb"racc"}}}
  \put(66,110){\makebox(0,0)[l]{\small{\verb"wir"}}}
  \put(66,105){\makebox(0,0)[l]{\small{\verb"rir"}}}
  \put(66,80){\makebox(0,0)[l]{\small{\verb"wrtn"}}}
  \put(66,75){\makebox(0,0)[l]{\small{\verb"rrtn"}}}
  \put(66,47.5){\makebox(0,0)[l]{\small{\verb"warg"}}}
  \put(66,35){\makebox(0,0)[l]{\small{\verb"alu0"}}}
  \put(66,30){\makebox(0,0)[l]{\small{\verb"alu1"}}}
  \put(66,17,5){\makebox(0,0)[l]{\small{\verb"rbuf"}}}

% non-bus connections between devices in datapath
  \put(105,175){\vector(0,-1){5}}
  \put(105,130){\vector(0,-1){5}}
  \put(105,100){\vector(0,-1){5}}
  \put(105,40){\vector(0,-1){5}}
  \put(105,25){\vector(0,-1){5}}

% one-way and two-way connections to bus
  \put(135,212.5){\vector(-1,0){15}}
  \put(120,212.5){\vector(1,0){15}}
  \put(135,180){\vector(-1,0){15}}
  \put(135,150){\vector(-1,0){15}}
  \put(120,150){\vector(1,0){15}}
  \put(135,135){\vector(-1,0){15}}
  \put(120,135){\vector(1,0){15}}
  \put(135,105){\vector(-1,0){15}}
  \put(120,105){\vector(1,0){15}}
  \put(135,75){\vector(-1,0){15}}
  \put(120,75){\vector(1,0){15}}
  \put(135,45){\vector(-1,0){15}}
  \put(135,30){\vector(-1,0){15}}
  \put(120,15){\vector(1,0){15}}

  {\thicklines {\put(7.5,115){\framebox(30,30)[c]{\large {\verb"rom"}}}}}
  {\thicklines {\put(7.5,95){\framebox(30,10)[c]{\large {\verb"mpc"}}}}}
  {\thicklines {\put(7.5,80){\framebox(30,10)[c]{\large {\verb"next"}}}}}
  {\thicklines {\put(41.5,115){\framebox(2,30){}}}}

% next to mpc, mpc to rom, mpc to next, rom to decoder
  \put(22.5,90){\vector(0,1){5}}
  \put(22.5,105){\vector(0,1){10}}
  \put(22.5,110){\line(1,0){22.5}}
  \put(45,110){\line(0,-1){41}}
  \put(45,69){\line(-1,0){10.5}}
  \put(34.5,69){\vector(0,1){11}}
  \put(37.5,130){\vector(1,0){4}}

% decoder outputs (right)
  \put(43.5,135){\vector(1,0){19.5}}
  \put(43.5,129){\line(1,0){13.5}}
  \put(43.5,127){\line(1,0){11.5}}
  \put(43.5,125){\line(1,0){9.5}}
  \put(43.5,123){\line(1,0){7.5}}
  \put(43.5,121){\line(1,0){5.5}}
  \put(43.5,119){\line(1,0){3.5}}

% rom and decoder outputs (down)
  \put(57,129){\line(0,-1){72}}
  \put(55,127){\line(0,-1){68}}
  \put(53,125){\line(0,-1){64}}
  \put(51,123){\line(0,-1){60}}
  \put(49,121){\line(0,-1){56}}
  \put(47,119){\line(0,-1){52}}

% rom and decoder outputs (left)
  \put(57,57){\line(-1,0){34.5}}
  \put(55,59){\line(-1,0){30.5}}
  \put(53,61){\line(-1,0){26.5}}
  \put(51,63){\line(-1,0){22.5}}
  \put(49,65){\line(-1,0){18.5}}
  \put(47,67){\line(-1,0){14.5}}

% rom and decoder outputs (up)
  \put(22.5,57){\vector(0,1){23}}
  \put(24.5,59){\vector(0,1){21}}
  \put(26.5,61){\vector(0,1){19}}
  \put(28.5,63){\vector(0,1){17}}
  \put(30.5,65){\vector(0,1){15}}
  \put(32.5,67){\vector(0,1){13}}

% inputs
  \put(14.5,39){\vector(0,1){41}}
  \put(14.5,39){\line(-1,0){14.5}}
  \put(12.5,54){\vector(0,1){26}}
  \put(12.5,54){\line(-1,0){12.5}}
  \put(10.5,69){\vector(0,1){11}}
  \put(10.5,69){\line(-1,0){10.5}}

  \put(0,72){\makebox(0,0)[l]{\small{\verb"dack"}}}
  \put(0,57){\makebox(0,0)[l]{\small{\verb"idle"}}}
  \put(0,42){\makebox(0,0)[l]{\small{\verb"ireq"}}}

  \put(0,212.5){\line(1,0){79}}
  \put(81,212.5){\vector(1,0){9}}

  \put(0,215.5){\makebox(0,0)[l]{\small{\verb"datain"}}}

\end{picture}


\caption{Register-Transfer Level Architecture.}
\label{fig-rtl}
\end{center}
\end{figure}

The control unit is implemented by
the microcode program counter \verb"mpc",
a ROM ({\it Read Only Memory}) for storing microcode,
a decoder which separates
the ROM output into various microinstruction fields,
and combinational logic for
computing the address of the next microinstruction.

The datapath includes
the program counter \verb"pc",
accumulator \verb"acc",
return address register \verb"rtn",
and interrupt acknowledge flag \verb"iack" which are components
of the externally visible state at the programming level.
In addition to these,
several internal registers are needed to interpret programming level
instructions.
These additional, full-size word registers are:

\begin{center}
\begin{tabular}{ll}
\verb"mar"& - memory address register\\
\verb"ir"& - instruction word register\\
\verb"arg"& - argument register for ALU input\\
\verb"buf"& - buffer for ALU output
\end{tabular}
\end{center}

The datapath also includes several functional elements:

\begin{center}
\begin{tabular}{ll}
\verb"alu" & - four functions: \verb"add", \verb"sub", \verb"inc"
and the constant (\verb"wordn 0")\\
\verb"interface" & - switching between system bus and memory data pins\\
\verb"opc" & - implements \verb"opcode" for extracting opcode field\\
\verb"addr" & - implements \verb"address" for extracting address field\\
\verb"zeroflag" & - implements \verb"iszero" to test for zero\\
\verb"dreq"& - two-input OR-gate
\end{tabular}
\end{center}

The storage devices and functional elements of the datapath are
interconnected by a single system bus.
The width of the system bus is exactly one full-size machine word.
The datapath is controlled by signals from the control unit which,
in turn, receives feedback from the datapath.

\subsection{Overview of instruction interpretation}

Each \Tamarack\ instruction is executed by
a sequence of steps which varies depending on the particular
instruction, the machine state and external inputs.
In general, the following actions are taken by the internal
architecture to interpret each programming level instruction.

\begin{itemize}
\item Check for interrupt request, otherwise continue \dots
\item Fetch instruction addressed by program counter \verb"pc".
\item Decode instruction.
\item Fetch operand if needed.
\item Execute instruction.
\item Increment program counter \verb"pc" if necessary.
\end{itemize}

Not all of these actions are performed for each instruction.
In particular, the control instructions JZR, JMP, RFI and NOP
do not require an operand to be fetched from memory.
Similarly, the program counter does not need to be incremented
for the JMP and RFI instructions or for the
JZR instruction when the jump is taken.
There are several opportunities for overlapping some of these
steps (e.g. incrementing the program counter while executing
the current instruction) but this has not been done for the
current design of \Tamarack\
(see suggested exercise in Section~\ref{sec-exer}).

The interpretation of an ADD instruction illustrates with greater detail how
the internal architecture of \Tamarack\
is used to implement its instruction set.
An ADD instruction is interpreted by a sequence of data transfers
over the system bus, interactions with memory, and operations
on data performed by the various functional elements.
The informal notation,

\begin{quote}
$<$destination$>$ $\leftarrow$ $<$expression$>$
\end{quote}

\noindent
is used here
to denote when a value computed from the current machine state
is loaded into a register, flipflop or memory to form a component of the
next machine state.
But unlike its previous use
to describe state changes in the programming level model
between instruction cycles,
this informal notation now describes state changes
at the register-transfer level
between clock cycles.

Exactly eight steps are required to interpret the ADD instruction
but some of these steps might be repeated (i.e., they are repeat-loops)
when the microprocessor is operating in fully asynchronous mode
\footnote{
The \verb"fetch" operation performed by external
memory is not necessarily completed until the last iteration of
the repeat-loop.}.

\begin{center}
\begin{tabular}{ll}

\verb"mar" $\leftarrow$ \verb"pc" &
repeat if $\neg$(\verb"idle" or $\neg$\verb"dack")\\

\verb"ir" $\leftarrow$ \verb"fetch (mem,(address mar))" &
repeat if $\neg$\verb"dack"\\

\verb"mar" $\leftarrow$ \verb"ir" &
repeat if $\neg$(\verb"idle" or $\neg$\verb"dack")\\

\verb"arg" $\leftarrow$ \verb"acc" &
repeat if $\neg$(\verb"idle" or $\neg$\verb"dack")\\

\verb"buf" $\leftarrow$ \verb"add (arg,fetch (mem,(address mar)))" &
repeat if $\neg$\verb"dack"\\

\verb"acc" $\leftarrow$ \verb"buf" & \\

\verb"buf" $\leftarrow$ \verb"inc pc" & \\

\verb"pc" $\leftarrow$ \verb"buf" & 

\end{tabular}
\end{center}

The equation,

\begin{quote}
\verb"clock cycles = 8+n1+n2+n3+n4+n5"
\end{quote}

\noindent
gives the total number of clock cycles
needed to interpret an ADD instruction.
Since every step is executed at least once,
at least eight clock cycles are required.
The variables \verb"n1", \verb"n2", \verb"n3", \verb"n4" and \verb"n5"
denote the number of additional clock cycles spent waiting for the
external memory at various steps in the instruction cycle.

In fully synchronous mode, none of the eight steps are ever repeated
because the pins \verb"idle" and \verb"dack" are both wired to \verb"T".
In this case,
the variables \verb"n1", \verb"n2", \verb"n3", \verb"n4" and \verb"n5"
will all be equal to zero.
Hence, the instruction cycle will be completed in exactly eight
clock cycles.

In fully asynchronous mode,
some of the above steps may be repeated but this will have no
untoward effect except to increase the number of clock cycles.
In this case when the microprocessor is using handshaking signals
to interact with memory,
the variables \verb"n1", \verb"n2", \verb"n3", \verb"n4" and \verb"n5"
will depend on the latency of external memory.
For best case performance,
these variables will all be zero resulting
in exactly the same number of clock cycles as fully
synchronous mode.
More generally,
it is only known that each wait loop will eventually terminate.
This fact depends on the correct implementation of the handshaking protocol
by both the microprocessor and external memory;
establishing this fact is a major step
in the \Tamarack\ proof of correctness.

In either mode of operation,
the cumulative effect of the ADD instruction interpretation sequence
is described by the following updates to the accumulator \verb"acc"
and the program counter \verb"pc".

\begin{quote}
\verb"acc" $\leftarrow$ \verb"add (acc,fetch (mem,(address fetch (mem,(address pc)))))"\\
\verb"pc" $\leftarrow$ \verb"inc pc"
\end{quote}

Simplifying these expressions with the earlier mentioned abbreviations
\verb"inst" and \verb"operand",
yields the programming level description of the ADD instruction semantics
given earlier in Section~\ref{sec-semantics}.

\begin{quote}
\verb"acc" $\leftarrow$ \verb"add (acc,operand)"\\
\verb"pc" $\leftarrow$ \verb"inc pc"
\end{quote}

The interpretation of the other seven programming level instructions
and the processing of a hardware interrupt can be described in
a similar way by a sequence of steps.
Showing by formal proof for each programming level instruction
that the cumulative effect of each sequence
satisfies the semantics of that particular instruction is another
major part
of the \Tamarack\ proof of correctness.

\subsection{Multiple interpretation levels}
\label{sec-multi}

The programming level model of a microprocessor sits at the top
of a hierarchy of interpretation levels
implemented by the internal architecture.
The internal operation of a
microprocessor can generally be
described in terms of the following language levels~\cite{Anceau}.

\begin{center}
\begin{tabular}{ll}
Programming level& - sequential execution of user programs\\
Microprogramming level& - sequential execution of microcode\\
Phase level& - concurrent elementary hardware operations\\
Instant level& - asynchronous sequencing in a clock phase\\
Basic logic components& - circuit level behaviour
\end{tabular}
\end{center}

This description of \Tamarack\ focuses on the three
highest levels
of interpretation (i.e., only down to the phase level).
The programming level has already
been described in terms of
the instruction set semantics and the hardware interrupt facility.
Below the programming level,
increasingly concrete views of
the internal operation of \Tamarack\ are described
at the microprogramming level and at the phase level.

\begin{figure}
\begin{center}
% ===== Hierarchy of Levels ============================================
% ======================================================================
\setlength{\unitlength}{1mm}
\begin{picture}(94,60)(0,0)

  {\thicklines {\put(35,50){\framebox(50,10){programming level model}}}}
  {\thicklines {\put(26,25){\framebox(68,10)
    {microprogramming level interpreter}}}}
  {\thicklines {\put(38,0){\framebox(44,10){phase level interpreter}}}}

  \put(60,50){\vector(0,-1){15}}
  \put(60,25){\vector(0,-1){15}}

  \put(50,42.5){\vector(1,0){10}}
  \put(28,17.5){\vector(1,0){32}}
  \put(0,42.5){\makebox(0,0)[l]{microinstruction sequences}}
  \put(0,17.5){\makebox(0,0)[l]{control signals}}

\end{picture}


\caption{Hierarchy of Interpretation Levels.}
\label{fig-hier}
\end{center}
\end{figure}

At the microprogramming level,
a programming level instruction is interpreted by
executing a sequence of microinstructions.
This sequence of microinstructions
is generated by a FSM ({\it Finite State Machine})
implemented by the control unit.
Microinstructions are
executed by an operational part corresponding to the datapath.

The phase level description decomposes
the interpretation of a single microinstruction
into the parallel execution of a set of elementary operations.
This decomposition reveals the structural organization of the
internal architecture in terms of register-transfer level components.

The concept of multiple interpretation levels
is used by architects to achieve a ``progressive translation
of functions in several stages'' \cite{Anceau}.
We will later describe how this concept
also provides a very effective strategy for controlling proof complexity
in the formal verification of \Tamarack.

\subsubsection{Microprogramming level}
\label{sec-micro}

Every programming level instruction is interpreted by a different
sequence of actions even though
they share many individual steps in common.
This sequence partially depends on the instruction opcode which
is only known part way through the sequence after the instruction word
has been fetched from memory.
The sequence of steps taken may also depend on
the machine state, in particular, on the contents of the
accumulator \verb"acc"
in the case of a jump-if-zero JZR instruction.

The interpretation of each instruction, that is,
the sequence of actions taken for each instruction, and similarly,
the actions taken to process a hardware interrupt are determined
by the control unit.
The control unit FSM generates
datapath commands (represented by microinstructions)
each clock cycle.
Each command causes an action to be performed by the datapath during
the current clock cycle.

The interpretation algorithm implemented by the control unit FSM is
based on {\it conditional branches}, that is,
a Moore machine approach in contrast to a Mealy machine approach
based on {\it conditional instructions}.
Inputs are used to select the next machine state but do not
determine the current output of the state machine.
Some of these inputs consists of
feedback from the datapath, in particular,
the opcode field of the instruction
word register \verb"ir" contents,
a \mbox{test-accumulator-for-zero} flag \verb"zeroflag",
and the current value of the interrupt acknowledge flag \verb"iack".
The FSM also receives external inputs from the \verb"idle",
\verb"dack" and \verb"ireq" pins which determine the behaviour of the FSM.
The FSM for the \Tamarack\ control
unit is described by the flow graph in Figure~\ref{fig-flow}.

\begin{figure}
\begin{center}
\input{fig-flow}

\caption{Control Unit Finite-State Machine Flow Diagram.}
\label{fig-flow}
\end{center}
\end{figure}

The start of an instruction cycle 
occurs when the FSM is in state~0 and about to exit to either state~1
or state~2.
If an interrupt is requested and the \verb"iack" flag is
not already set, then the FSM exits to state~1 to process the interrupt.
Otherwise, the FSM exits to state~2 and causes
the current instruction word to be fetched from memory.
In state~3, the FSM dispatches on the opcode field to
the start of the remaining interpretation sequence for
the current instruction.
For instance, the interpretation of an ADD instruction would cause
a transition from state~3 to state~6.
From this point onwards, the FSM follows a sequence of state
transitions leading back to state~0.
In the case of a JZR instruction, the FSM selects one of two possible exits
from state~4 depending the
\mbox{test-accumulator-for-zero} feedback from the datapath.
Assuming that the FSM never loops indefinitely in a particular state,
then in all cases, including the processing of interrupts,
the FSM always returns to state~0 to begin the next instruction cycle.

For example,
the interpretation of an ADD instruction in fully synchronous mode
results in the following sequence of states.

\begin{quote}
0, 2, 3, 6, 13, 15, 11, 12, and back to 0.
\end{quote}

In fully asynchronous mode, additional clock cycles caused by
delayed handshaking signals from the external memory could result
in a sequence such as:

\begin{quote}
0, 2, 2, 2, 3, 3, 6, 13, 13, 13, 15, 11, 12, and back to 0.
\end{quote}

Each FSM state causes a specific action to be performed by the datapath.
The mapping from FSM states to actions is shown in Figure~\ref{fig-map}.
This mapping,
combined with the flow graph in Figure~\ref{fig-flow},
gives a complete description
of the internal architecture of
\Tamarack\ at one level of abstraction.
This level of description contains no structural details aside
from the conceptual distinction between the
function of the control unit
and the operation of datapath.
Although the components of the internal state are visible in this view,
updates to machine state are described functionally.
This abstract view of the internal architecture is the basis
of an intermediate
step in the formal verification of \Tamarack.

\begin{figure}
\begin{center}
\begin{minipage}{156mm}

\setlength{\unitlength}{1mm}
\newcommand{\actionbox}[2]{
\put(5,10){\thicklines{\circle{10}}}
\put(5,10){\makebox(0,0)[c]{\Large #1}}
\put(15,10){\makebox(0,0)[c]{$\Longrightarrow$}}
\put(47,10){\thicklines{\oval(54,20)}}

\put(91,10){\thicklines{\circle{10}}}
\put(91,10){\makebox(0,0)[c]{\Large #2}}
\put(101,10){\makebox(0,0)[c]{$\Longrightarrow$}}
\put(133,10){\thicklines{\oval(54,20)}}
}

\setlength{\unitlength}{1mm}
\begin{picture}(156,20)

\actionbox{0}{8}

\put(23,10){\makebox(0,0)[l]{\verb"mar" $\leftarrow$ \verb"pc"}}

\put(109,12.5){\makebox(0,0)[l]{\verb"acc" $\leftarrow$ \verb"datain"}}
\put(109,7.5){\makebox(0,0)[l]{\verb"dreq" = \verb"T"}}

\end{picture}

\vskip .15in

\setlength{\unitlength}{1mm}
\begin{picture}(156,20)

\actionbox{1}{9}

\put(23,15){\makebox(0,0)[l]{\verb"rtn" $\leftarrow$ \verb"pc"}}
\put(23,10){\makebox(0,0)[l]{\verb"buf" $\leftarrow$ \verb"wordn 0"}}
\put(23,5){\makebox(0,0)[l]{\verb"iack" $\leftarrow$ \verb"T"}}

\put(109,15){\makebox(0,0)[l]{\verb"dataout" = \verb"acc"}}
\put(109,10){\makebox(0,0)[l]{\verb"wmem" = \verb"T"}}
\put(109,5){\makebox(0,0)[l]{\verb"dreq" = \verb"T"}}

\end{picture}

\vskip .15in

\setlength{\unitlength}{1mm}
\begin{picture}(156,20)

\actionbox{2}{10}

\put(23,12.5){\makebox(0,0)[l]{\verb"ir" $\leftarrow$ \verb"datain"}}
\put(23,7.5){\makebox(0,0)[l]{\verb"dreq" = \verb"T"}}

\put(109,12.5){\makebox(0,0)[l]{\verb"pc" $\leftarrow$ \verb"rtn"}}
\put(109,7.5){\makebox(0,0)[l]{\verb"iack" $\leftarrow$ F}}

\end{picture}

\vskip .15in

\setlength{\unitlength}{1mm}
\begin{picture}(156,20)

\actionbox{3}{11}

\put(23,10){\makebox(0,0)[l]{\verb"mar" $\leftarrow$ \verb"ir"}}

\put(109,10){\makebox(0,0)[l]{\verb"buf" $\leftarrow$ inc \verb"pc"}}

\end{picture}

\vskip .15in

\setlength{\unitlength}{1mm}
\begin{picture}(156,20)

\actionbox{4}{12}

\put(23,10){\makebox(0,0)[l]{$<$nothing$>$}}

\put(109,10){\makebox(0,0)[l]{\verb"pc" $\leftarrow$ \verb"buf"}}

\end{picture}

\vskip .15in

\setlength{\unitlength}{1mm}
\begin{picture}(156,20)

\actionbox{5}{13}

\put(23,10){\makebox(0,0)[l]{\verb"pc" $\leftarrow$ \verb"ir"}}

\put(109,12.5){\makebox(0,0)[l]{\verb"buf" $\leftarrow$ \verb"add (arg,datain)"}}
\put(109,7.5){\makebox(0,0)[l]{\verb"dreq" = \verb"T"}}

\end{picture}

\vskip .15in

\setlength{\unitlength}{1mm}
\begin{picture}(156,20)

\actionbox{6}{14}

\put(23,10){\makebox(0,0)[l]{\verb"arg" $\leftarrow$ \verb"acc"}}

\put(109,12.5){\makebox(0,0)[l]{\verb"buf" $\leftarrow$ \verb"sub (arg,datain)"}}
\put(109,7.5){\makebox(0,0)[l]{\verb"dreq" = \verb"T"}}

\end{picture}

\vskip .15in

\setlength{\unitlength}{1mm}
\begin{picture}(156,20)

\actionbox{7}{15}

\put(23,10){\makebox(0,0)[l]{\verb"arg" $\leftarrow$ \verb"acc"}}

\put(109,10){\makebox(0,0)[l]{\verb"acc" $\leftarrow$ \verb"buf"}}

\end{picture}

\vskip .25in

in all states except where indicated otherwise:
\hspace{3mm}
\begin{minipage}[t]{40mm}
\verb"wmem" = F\\
\verb"dreq" = F\\
\verb"dataout" = \verb"wordn 0"\\
\verb"addr" = \verb"address mar"
\end{minipage}

\end{minipage}


\caption{Mapping from FSM States to Datapath Actions.}
\label{fig-map}
\end{center}
\end{figure}

\subsubsection{Phase level}
\label{sec-phase}

Each command generated by the FSM is a microinstruction which is
interpreted by a set of register-transfer level operations.
The phase level description decomposes
a microprogramming level action such as,

\begin{quote}
\verb"buf" $\leftarrow$ \verb"add (arg,fetch (mem,(address mar)))"
\end{quote}

\noindent
into the parallel execution of a set of elementary operations:

\begin{quote}
\verb"addr" = \verb"address mar"\\
\verb"datain" = \verb"fetch (mem,addr)"\\
\verb"bus" = \verb"datain"\\
\verb"alu" = \verb"add (arg,bus)"\\
\verb"buf" $\leftarrow$ \verb"alu"
\end{quote}

This decomposition reveals the structural organization of the
microprocessor.
Each of the elementary operations shown above corresponds to
a functional element of the register-transfer architecture
or, in the case of \verb"fetch",
to a primitive operation performed by external memory.

The phase level description of \Tamarack\ also refines certain
abstract features of the microprogramming level description.
In particular, it reveals that the FSM is implemented by:

\begin{itemize}
\item A microcode program counter \verb"mpc" to hold the current
state of the FSM.
\item The microcode ROM which encodes the abstract mapping
from FSM states to actions given in Figure~\ref{fig-map}.
\item A decoder which separates the ROM output into various
microinstruction fields.
\item The next address logic which computes the next FSM
state according the flow graph given in Figure~\ref{fig-flow}.
\end{itemize}

The microcode program counter is a register which stores
values belonging to an uninterpreted data type:

\begin{center}
\begin{tabular}{ll}
\verb":*word4"& - FSM state
\end{tabular}
\end{center}

Several uninterpreted primitives for this data type are
needed later on to describe the computation performed by
the next address logic.

\begin{center}
\begin{tabular}{lll}
\verb"word4"& \verb":num"$\rightarrow$\verb"*word4"&
representation of a number\\
\verb"val4"& \verb":*word4"$\rightarrow$\verb"num"&
numerical value of a FSM state
\end{tabular}
\end{center}

Every clock cycle,
the current value of the microcode program counter \verb"mpc"
is used to fetch the current microinstruction
from the microcode ROM.
The microcode is {\it horizontal} which means, in this case, that every
control signal is determined by a unique bit in the microinstruction word.
The format of microinstruction words is shown below.

\begin{center}
% ===== MicroInstruction Word Format ===================================
% ======================================================================
\setlength{\unitlength}{1mm}
\begin{picture}(156.6,26)(0,0)

{\thicklines {\put(0,19){\framebox(156.6,7){}}}}
{\thicklines {\multiput(5.8,19.5)(5.8,0){19}{\line(0,1){6}}}}
{\thicklines {\put(133.4,19.5){\line(0,1){6}}}}

\put(121.8,22.5){\makebox(0,0)[c]{\verb"addr1"}}
\put(145,22.5){\makebox(0,0)[c]{\verb"addr2"}}

\put(43.5,15){\makebox(0,0)[c]{$\underbrace{\rule{84mm}{0mm}}$}}
\put(98.6,15){\makebox(0,0)[c]{$\underbrace{\rule{20.2mm}{0mm}}$}}
\put(133.4,15){\makebox(0,0)[c]{$\underbrace{\rule{43.4mm}{0mm}}$}}

\put(43.5,13){\makebox(0,0)[t]{
\begin{minipage}[t]{84mm}
\begin{center}
datapath\\
control bits
\end{center}
\end{minipage}
}}
\put(98.6,13){\makebox(0,0)[t]{
\begin{minipage}[t]{30mm}
\begin{center}
next\\
address logic\\
control bits
\end{center}
\end{minipage}
}}
\put(133.4,13){\makebox(0,0)[t]{
\begin{minipage}[t]{43.4mm}
\begin{center}
microinstruction\\
addresses
\end{center}
\end{minipage}
}}

\end{picture}

\end{center}

The fifteen datapath
control bits are extracted from the output of the microcode ROM
and sent to the data path over a set of control lines.
These control bits generate individual control signals
including `write' signals for storage elements,
`read' signals for devices which can assert values onto the system bus,
and signals to control the operation of the functional elements such
as the ALU.

In addition to control bits for the datapath,
the microinstruction word contains several other fields
which are used to encode a partial computation of the next FSM state.
The four next address logic
control bits determine how the next address logic
computes the next FSM state.
The two microinstruction address fields, \verb"addr1" and \verb"addr2"
(both of the type \verb":*word4"),
specify destination fields for various kinds of conditional branches
in the microcode which may be
selected by the next address logic as the next FSM state.
The next address logic is just a block of combinational logic
whose function can be derived from the flow graph in Figure~\ref{fig-flow}.

The interpretation of a microinstruction at the phase level during
a single clock cycle results in a sequence of events which includes:

\begin{center}
\begin{itemize}
\item Fetch the current microinstruction.
\item Compute the address of the next microinstruction.
\item Read data onto the system bus.
\item Evaluate operations performed by functional elements.
\item Update storage elements such as memory, registers and flipflops.
\end{itemize}
\end{center}

Some of these events clearly precede other events.
For instance, the current microinstruction has to be fetched
from the microcode ROM before it can be interpreted.
On the other hand,
many of these events take place concurrently:
for example,
the address of the next microinstruction is computed by the
next address logic while the datapath executes the actions specified
by the current microinstruction.
Indeed, much of the activity during a clock cycle is not necessarily
synchronized by an explicit control mechanism:
a change in the inputs to
a functional element such as the ALU might be propagated to its outputs
after a few nanoseconds depending on its implementation.

Even though some constraints on the order of events
during a clock cycle could be described,
this has not be done in this description of \Tamarack.
Instead,
functional elements such as the ALU are modelled without delay
and the update of storage elements is described as an atomic action.
A fundamental feature
of this abstraction
is the assumption that
updates to storage elements do not propagate to their outputs
until the end of the clock cycle,
i.e., there is no possibility of an closed feedback path
resulting in a race condition.

\subsection{Some bottom level assumptions}
\label{sec-bottom}

The relatively abstract view of how a microinstruction is interpreted
during a single clock cycle is the lowest level in this
description of \Tamarack.
The phase level view is also the lowest level of specification in the
formal proof of correctness.
The semantic gap between this abstraction and actual hardware
is bridged by several informally stated assumptions.

First of all,
it is assumed that the design is implemented
by digital circuitry in some kind of synchronous logic design style
(e.g., two-phase, non-overlapping logic).
Depending on the design style,
certain rules must be followed to ensure, for instance,
that updates to storage elements do not propagate to their outputs
until the end of the clock cycle.
Assuming that these rules have been followed and certain other constraints
have been satisfied,
\footnote{
Conventional CAD tools are commonly used to
check that synchronous logic has been correctly implemented
with respect to the rules of
a particular design style
and that other constraints are satisfied
(e.g., no problems with clock skew for a given clock period).}
synchronous logic gives rise to the
abstract view where functional elements are modelled without delay
and the update of storage elements is described as an atomic action.
\footnote{
Although the relatively abstract view of sequential behaviour
in our lowest level of specification is valid,
there is a semantic gap between
this particular level of formal specification and
the view of sequential behaviour supported by
many sophisticated CAD tools.
As it happened, this contributed to
design error
when the formal specification of
\mbox{\scriptsize TAMARACK-2}
was hand-translated
into the design specification language
of the {\scriptsize GENESIL} silicon compiler system
by Shankar et al.\ \cite{SRI}.
Even though the design was correct with respect to the
built-in rules for the two-phase, non-overlapping design style,
there was a mismatch between
the single-phase view of sequential behaviour in the
formal specification and
the two-phase view supported by the silicon compiler.
Although this experiment in verification-driven design
revealed a semantic gap for this particular level
of bottom level specification, this is does {\it not}
represent any kind of inherent limitation of formal methods.}

Synchronous logic only ensures that this abstraction is valid
for the internal logic;
some additional considerations are required to ensure that
this abstraction is valid for
the interface between internal logic and external devices.

Asynchronous input events, such as
a transition from \verb"F" to \verb"T" on the hardware interrupt
request pin \verb"ireq", may occur at any time.
In particular, they may occur at any point during the clock cycle
used to synchronize the internal logic.
Asynchronous input events may or may not be detected during
the clock cycle depending on when they occur with respect to
the minimum set-up time of any internal storage device that samples
(directly or indirectly) this input.
Carefully designed interface circuitry
would be required to
minimize the {\it metastability} problem, that is, the possibility
of unstable equilibria in cross-coupled circuits \cite{Seitz}.

In addition to sampling asynchronous inputs,
the internal logic generates asynchronous outputs such as the
interrupt acknowledge flag \verb"iack".
External outputs may be sampled at any point in continuous time
and every change in their value is significant.
During a clock cycle,
synchronous logic may generate several transient values
before settling on a set of final values.
For signals used internally,
these transient values have no effect,
but additional interface circuitry may be
required to ensure that transient values do not
propagate to asynchronous outputs.

These
%informally stated
informal assumptions underlie the abstract view
of how a microinstruction is interpreted during a single clock cycle.
In the formal theory,
asynchronous input and output events are
modelled with respect to the same discrete time scale
used to model internal synchronous logic.
The model represents the {\it observed}
behaviour of the external environment
as seen by the internal logic.
An asynchronous output event is described by
changing the value of an output signal between adjacent
points of discrete time.
It is assumed that the output signal is
otherwise stable with respect to continuous time.
