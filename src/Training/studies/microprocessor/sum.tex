\section{Summary}

The expressive power of higher-order logic has been
an advantage in several ways for the work described in
this case study.
One of the most important uses of higher-order functions and relations
has been to parameterize definitions by a representation variable
providing a formal basis for our approach to abstract
specification.

We have also found several instances where
the ability to use a relation in place of a function was
very desirable.
Our model of the datapath bus
experiments with relations as an alternative to using
combining functions.
In Section~\ref{sec-asyn} we described some advantages of
a relational style of specification which allows independent
specifications to be given for the microprocessor and
external memory.
Most of the time, functions are more natural and simpler
to use, but occasionally, the well-conceived use of a relation
in place of a function gives a superior result.

It was extremely useful (in fact, much more useful than we
initially expected)
to embed some operators
from temporal logic in our higher-order framework especially
when we compare this approach with our previous attempts
to reason about asynchrony
\cite{Joyce:stirling}.
It is likely that work on embedding other calculi
in higher-order logic will be a particularly interesting
area of future development in the \HOL\ user community.
In addition to Hale's work on Interval Temporal Logic \cite{Hale:thesis},
some other examples include work by Gordon on program logics
\cite{Gordon:banff87},
by Loewenstein on a theory for deterministic and non-deterministic
state machines \cite{Loewenstein:cornell},
and by Camilleri on CSP trace theory \cite{Camilleri:csp}.

A number of errors in the design and specification of \Tamarack\
were discovered in the course of verifying its correctness.
A few of these errors were not simple mistakes and not
the kind of error easily revealed by
simulation (they concerned
subtle aspects of the handshaking protocol specification).
However, many simple mistakes were also found by verification
but they could have been discovered earlier
without incurring the high cost of verification
if the specifications
had been executable.
Camilleri \cite{Camilleri:pisa} points out the advantage of executing
formal specifications as a means of supporting a verification
methodology.
In this respect, the Boyer-Moore approach
\cite{Hunt:thesis,Bevier87,Bevier89}
offers the advantage of directly executable specifications.
The increased expressiveness of higher-order logic results in
terms which do not always have an obvious
interpretation as an executable specification.
However,
the implementation of an executable subset of higher-order logic
is a foreseeable addition to the HOL system.
\footnote{
Camilleri's thesis \cite{Camilleri:thesis}
describes simulation tools for an executable
subset of higher-order logic oriented to hardware simulation.}

There is a very steep learning curve for the \HOL\ system
and many new users must wonder if they can afford to invest
enough effort to receive any return
on their investment.
We can offer some encouragement by observing that
simply getting a large design specification
to be successfully type-checked by the \HOL\ system
is a significant accomplishment.
Fully automated type-checking often reveals errors which
would go unnoticed without the scrutinizing eye of the type-checker.
A typical example would be the interconnection of two ports with
mismatched types (e.g. a 16-bit bus connected to a 13-bit bus).
While type-checking problems require an understanding of higher-order
logic, it does not require any familiarity with theorem-proving.
Of course, there are conventional CAD tools
(as well as ordinary, strongly-typed programming languages) which can be used
to check the consistency of a design.
However, using the \HOL\ system to type-check a design represents
an intermediate goal for new users
along the path towards full-scale verification.

\vskip \baselineskip

{\bf Acknowledgements}.
The work described in this case study owes a great deal to many
people.  A complete set of acknowledgements are given in \cite{Joyce:thesis},
however, I would particularly like to
mention past and present developers of the \HOL\ system
(and its software parent, the \LCF\ system);
this work would not have been possible without the elegant tool they have 
created.
Furthermore, the remarkably painless transition
from an earlier version of \HOL\ to the {\HOL}88 system
is a tribute to the skill of Mike Gordon and Tom Melham.
