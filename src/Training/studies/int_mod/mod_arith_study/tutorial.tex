% =====================================================================
% HOL Case Study: Modular Arithmetic. By Elsa Gunter
% =====================================================================

\documentstyle[12pt,twoside,fleqn,layout]{article}

\sloppy


% ---------------------------------------------------------------------
% Preliminary settings etc.
% ---------------------------------------------------------------------

\renewcommand{\topfraction}{0.8}	  % 0.8 of the top page can be fig.
\renewcommand{\bottomfraction}{0.8}	  % 0.8 of the bottom page can be fig.
\renewcommand{\textfraction}{0.1}	  % 0.1 of the page must contain text
\setcounter{totalnumber}{4}	 	  % max of 4 figures per page
\setcounter{secnumdepth}{3}		  % number sections down to level 3
\setcounter{tocdepth}{3}		  % toc contains numbers to level 3

% ---------------------------------------------------------------------
% Macros
% ---------------------------------------------------------------------
\input{commands}

%\newcommand{\DOC}[1]{\noindent{\LARGE\tt #1}\newline\mbox{}}

\newcommand{\DOC}[1]%
{\bigskip
 \begin{flushleft}\index{#1}
 \begin{tabular}{|c|}\hline
 \begin{minipage}{\minipagewidth}
 \bigskip
 {\LARGE\tt #1}
 \bigskip
 \end{minipage}\\ \hline
 \end{tabular}
 \end{flushleft}
 \bigskip
}

\newcommand{\SYNOPSIS}%
{\bigskip{\noindent\large\bf Synopsis}\newline\mbox{}}

\newcommand{\CATEGORIES}%
{\bigskip{\noindent\large\bf Categories}\newline\mbox{}}

\newcommand{\DESCRIBE}%
{\bigskip{\noindent\large\bf Description}\newline\mbox{}}

\newcommand{\FAILURE}%
{\bigskip{\noindent\large\bf Failure}\newline\mbox{}}

\newcommand{\EXAMPLE}%
{\bigskip{\noindent\large\bf Example}\newline\mbox{}}

\newcommand{\USES}%
{\bigskip{\noindent\large\bf Uses}\newline\mbox{}}

\newcommand{\SEEALSO}%
{\bigskip{\noindent\large\bf See also}}

\newcommand{\ENDDOC}{}


\begin{document}

  \pagestyle{plain}

\begin{titlepage}

\setcounter{page}{1}

\begin{center}
{\Huge{\bf HOL CASE STUDY}}\\
\vskip .3in
{\Large\bf Modular Arithmetic}\\
\end{center}
\vskip .4in


\begin{inset}{Author}
Elsa L.~Gunter\newline
Department of Computer and Information Science\newline
University of Pennsylvania\newline
200 South 33rd Street\newline
Philadelphia, PA 19104, USA\newline
{\bf Telephone:} (1) 215-898-9514\newline
{\bf Email:} \verb+elsa@linc.cis.upenn.edu+\newline
\end{inset}

\begin{inset}{Concepts illustrated}
Application of abstract group theory, use of specialized tactics for group
theory and for the integers, interactive proof, tactics.
\end{inset}

%\begin{inset}{Level of difficulty}
%Intermediate to hard.
%\end{inset}

\begin{inset}{Prerequisites}
At least as much knowledge of \HOL\ as is given in {\sl Getting started with
\HOL}.  Knowledge of the definitions of a group and a subgroup
\end{inset}

\begin{inset}{Co-requisites}
An understanding of the basic concepts of how to do algebra
in \HOL, as explained in {\it Doing Algebra in Simple Type Theory}, 
which can be found in {\small\verb!hol/Training/studies/int_mod!}.
\end{inset}

\begin{inset}{Supporting files}
In the directories {\small\verb%hol/Training/studies/int_mod%} and
{\small\verb%hol/Library/int_mod%}.
\end{inset}

\begin{inset}{Abstract}
The development of the basic group theory of the integers, including
basic modular arithmetic, is described in three sections.  This case
study is intended to give the user extensive hands-on experience proving
facts using theorems and tactics from group theory and the theory of
the integers, applying and developing abstract theories, and working
through large proofs.  The material presented in this case study in
graded in difficulty, with increases in difficulty occurring mainly
from section to section.  This study is intended to be worked through
in an interactive fashion.  While substantial benefit should be obtainable
by simply reading this case study, the full benefit can only be had
by executing the commands here outline as they are read.  At the end
of the second and third sections, projects have been suggested for those
who wish to further test their understanding of the concepts studied
in those sections.
\end{inset}

\thispagestyle{empty}

\end{titlepage}

\newpage
\tableofcontents 
\addtocontents{toc}{\protect\thispagestyle{empty}}
\newpage

\newcommand{\mvdots}{\vspace*{-12pt}\hspace*{2cm}\vdots\vspace*{-12pt}}
\newcommand{\evdots}{\vspace*{-12pt}\hspace*{2cm}\vdots}

\section{Introduction}

This case study is designed to help the intermediate-level user of
\HOL\ to become familiar with proving facts using the group theory and
theory of the integers found in two of the directories in the Library
directory accompanying \HOL.  While it is hoped that this case study will
provide the user with an enhanced capacity for using the \HOL\ theorem
prover for proving theorems in general, it will be assumed here that
the user has a basic familiarity with the inference rules and tactics
found in the \HOL\ System Manual, and has at least some experience using
the goal-stack in \HOL\ for goal-directed theorem proving.  For a basic
introduction to the \HOL\ system, see {\sl Getting started with \HOL}.

The first section, Subgroups of the Integers, is intended to help you
become familiar with the the ideas of subgroup and normal subgroup and
some basic ways of proving theorems with these ideas.  In the next
section, Basic Modular Arithmetic, we will use the notion of a quotient
group to define the integers mod {\small\tt n} together with modular 
addition.  Then we shall use this as an example of how to instantiate
the abstract first order group theory with a concrete case.  The last
section, Subgroups of the Integers, Revisited, focuses heavily on two
important methods for proving facts about the integers.  The first
method is to break the problem into two cases, one where we assume the
integer is non-negative and the other where we assume the integer is
negative and we assume that we already know the result for the
non-negative integers.  In each of these cases we can further reduce
to the natural numbers where we have access to the principle of
induction, should we desire it.  There exists a tactic to carry out
these reductions and we shall have several opportunities to use it.
The other method is to use the fact that any non-empty set which is
bounded above contains its least upper bound (and similarly a greatest
lower bound for sets bounded below).  In this method, you divide the
problem into the subgoals of showing that the set in question is
non-empty and bounded above (and hence has a maximum) and the subgoal
of finishing the principal goal with the consequences of the existence
of such a maximal element added as assumptions.  There exists a tactic
to carry out this reduction (which actually returns the last subgoal
first and the other subgoals after) and there exists a similar tactic
for reasoning about minimal elements.

It will be assumed throughout this case study that the reader is
executing the various commands described herein on a system that is
running \HOL.  The particular environment that will be assumed for
this study is:
\begin{enumerate}
\item \HOL\ version 1.07 (or later)
\item An emacs editor
\end{enumerate}
It will also be assumed that the user has sufficient familiarity with
an emacs editor to be able to create two windows, move between buffers,
copy material from one buffer to another, etc.

While you are working through this case study, you will often need to
use the theorems that have already been proved in the theories
{\small\verb+elt_gp+}, {\small\verb+more_gp+}, and
{\small\verb+integer+}.  A listing of theorems can be found in
Appendix A. They can also be found respectively in the files:
\begin{verbatim}
   hol/Library/group/elt_gp.print
   hol/Library/group/more_gp.print
   hol/Library/integer/integer.print
\end{verbatim}
You will also need to use a variety of tactics, both general-purpose
and subject-specific, that are new and not discussed in the \HOL\
manual.  The various general-purpose tactics are described in Appendix
B, and the subject-specific tactics are described in Appendix C.

Throughout the text below there are a series of boxes which contain
pieces of sessions with the \HOL\ system which illustrate the
concepts that have been discussed immediately before.  The text within
these boxes should be understood to come in sequence, with later work
depending on the earlier work having been done.  The lines beginning
with a {\small\verb+#+} are entered by the user.  The text within
the boxes does not represent an entire session with the \HOL\ system,
nor does it always represent a continuous portion of a session; often
portions of the system response have been omitted.  Omissions are only
of portions of the system response, and they will be indicated by
three vertical dots within the boxes.  All user inputs (with one
stated exception) are included.  Complete shell scripts for each
of the three sessions discussed here may be found in the files:
\begin{verbatim}
  hol/Training/studies/int_mod/int_sbgp.shell1
  hol/Training/studies/int_mod/int_mod.shell
  hol/Training/studies/int_mod/int_sbgp.shell2
\end{verbatim}

As you work through this case study, you will be presented with
problems to be solved, usually of the form ``We need to reduce this
goal to these other subgoals''.  Immediately after, there will follow
(almost always) a box containing a piece of a session with the \HOL\
system which demonstrates a solution to the problem.  At first, you
should enter the solution given and  mainly try to see why the
answer given is an answer.  As you progress you should try more and
more to give your own answer before looking at the one given, and then
afterwards compare.  In general, there is no one right answer to any
of these problems.  You should feel free to experiment, but it will be
necessary that you perform the same reductions (if by other means) as
the tactics that are given do, so that the remainder of the case study
makes sense.  At any time, if you want to try something different out,
remember that you can always use
\begin{verbatim}
   backup();;
\end{verbatim}
or simply
\begin{verbatim}
   b();;
\end{verbatim}
to undo the last thing you did to the goal stack.



\section{Subgroups of the integers}

To begin, create a directory where you will be able to store the files
you will be making.  Enter that directory, start emacs, create two
windows, begin editing a file named
{\small\verb+int_sbgp.show.ml+} in one window and in the other
start a shell.  In the shell start running \HOL.  The file
{\small\verb+int_sbgp.show1.ml+} will be used throughout this
section to record the commands to be executed here.  When you are
finished, this file will be able to create the theory file
{\small\verb+int_sbgp.th+}.  As you create commands to execute in
\HOL, you should write them in this file, and then copy them into
\HOL.  A file which should be essentially the same as the one you are
creating can be found in
\begin{verbatim}
   hol/Library/int_mod/int_sbgp.show1.ml
\end{verbatim}
and, as was mentioned before, a copy of a shell script created by doing
the work in this section can be found in
\begin{verbatim}
   hol/Training/studies/int_mod/int_sbgp.shell1
\end{verbatim}
At any time you should feel free to compare your work with what is
found in these two files.

In this section, we want to create a new theory file, named
{\small\verb+int_sbgp.th+}.  To do so, type
\begin{verbatim}
   new_theory `int_sbgp`;;
\end{verbatim}
into the file {\small\verb+int_sbgp.show.ml+}, and then copy it
into the shell, following it by a carriage return.  This will
initialize the creation of the theory file, and put you in draft mode.
Your shell session should look like the following now:

\setcounter{sessioncount}{1}
\begin{session}
\begin{verbatim}
faulkner+ hol88

       _  _    __    _      __    __
|___   |__|   |  |   |     |__|  |__|
|      |  |   |__|   |__   |__|  |__|

  Version 1.07, built on Jul 13 1989

#new_theory `int_sbgp`;;
() : void
\end{verbatim}
\end{session}

The next thing we want to do is to include the group theory and the
theory of the integers that is already available to \HOL.  This is done
by loading the appropriate Library entries.  To load the group theory,
type
\begin{verbatim}
   load_library `group`;;
\end{verbatim}
into your file and copy it into the shell, followed by a carriage return.

\begin{session}
\begin{verbatim}
#load_library `group`;;
Loading library `group` ...
\end{verbatim}
\evdots
\end{session}

It will respond by typing out a collection of messages informing you
of the progress of the loading of the group theory library.  In the
same manner you need to load the library for the integers.

\begin{session}
\begin{verbatim}
#load_library `integer`;;
Loading library `integer` ...
\end{verbatim}
\evdots
\end{session}

From now on, when I say that you should execute a command in \HOL, I
will mean that you should enter it into the file
{\small\verb+int_sbgp.show.ml+} and then copy it into the shell,
following it by a carriage return.  There is no harm in copying a
command of several lines all at once into the shell (provided that the
command is not more than about 50 lines.  In the last section you will
see a way to deal with commands of arbitrarily many lines.)  Now we
are set up to start proving theorems.

To provide focus for this study, we shall take as our goal the
development of basic facts about subgroups of the integers, and  basic
modular arithmetic.  By modular arithmetic, I am referring to the set
of integers mod {\small\tt n} under addition, where we shall consider
{\small\tt n} as fixed.  In terms of group theory (see
{\small\verb+hol/Library/group/elt_gp.print+} and
{\small\verb+hol/Library/group/more_gp.print+}, or Appendix A for
the theory that has been developed in \HOL), what is meant by the
integers mod {\small\tt n} is the quotient group of the integers by
the subgroup of all multiples of {\small\tt n}.  To make sense of this
statement we need to show the following:
\begin{enumerate}
\item the set all multiples of a fixed integer {\small\tt n} forms a
   subgroup of the integers, and
\item this subgroup is a normal subgroup of the integers.
\end{enumerate}
(We already have the fact that the integers form a group themselves.)
A subgroup {\small\tt S} of the integers is a normal subgroup just
when, for every integer {\small\tt x} in {\small\tt S}, the integer
{\small\verb+((neg x) plus (s plus x))+} is also in {\small\tt S}.
But mbox{\small\verb+((neg x) plus (s plus x))+} is just {\small\tt s}
for any integer {\small\tt s}, so any subgroup of the integers is
normal.  Let's make that the first theorem that we prove.

Our first goal is, for all {\small\tt H}, if {\small\tt H} is a
subgroup of the integers, then {\small\tt H} is a normal subgroup of
the integers.  So, how do we express this in \HOL?  First, the group
associated with the integers is written as the pair
{\small\verb+((\N:integer.T),$plus)+}.  (For more on why this is
how we represent the group of the integers under addition, see
{\it Doing Algebra in Simple Type Theory}.)  Next, we express the
concept that {\small\tt H} is a subgroup of the group
{\small\verb+((\N:integer.T),$plus)+} by
{\small\verb+SUBGROUP((\N.T),$plus)H+}.
Similarly, we express the concept that {\small\tt H} is a normal
subgroup of the group {\small\verb+((\N.T),$plus)+} by
{\small\verb+NORMAL((\N.T),$plus)H+}.  The predicates
{\small\verb+SUBGROUP+} and {\small\verb+NORMAL+} are
defined in the theory {\small\verb+more_gp.th+}.  Putting the
pieces together, the statement of our goal is
\begin{verbatim}
   "!H. SUBGROUP((\N.T),$plus)H ==> NORMAL((\N.T),$plus)H"
\end{verbatim}
We can leave out the type information, because
{\small\verb+$plus+} tells us that we are dealing with the
integers.  Now that we have the statement of our goal, we need to
enter it onto the goal stack.  This is done by entering the command
\begin{verbatim}
   set_goal ([],"!H. SUBGROUP((\N.T),$plus)H ==> NORMAL((\N.T),$plus)H");;
\end{verbatim}
\begin{session}
\begin{verbatim}
#set_goal ([],"!H. SUBGROUP((\N.T),$plus)H ==> NORMAL((\N.T),$plus)H");;
"!H. SUBGROUP((\N. T),$plus)H ==> NORMAL((\N. T),$plus)H"

() : void
\end{verbatim}
\end{session}
Since this goal has no assumptions, it could also be entered onto
the goal stack by
\begin{verbatim}
   g "!H. SUBGROUP((\N.T),$plus)H ==> NORMAL((\N.T),$plus)H";;
\end{verbatim}

The first thing that we want to do in order to prove this theorem,
just as we did more informally above, is focus on a generic {\small\tt H}, 
convert the antecedent of the implication into an assumption, and then
expand {\small\verb+NORMAL+} using the definition of being normal
and the subgroup assumption.  That is, we want to use
{\small\verb+GEN_TAC+} followed by {\small\verb+DISCH_TAC+}
followed by {\small\verb+ASM_REWRITE_TAC [NORMAL_DEF]+}.  We can
carry this out with the single command 

\begin{session}
\begin{verbatim}
#expand (GEN_TAC THEN DISCH_TAC THEN (ASM_REWRITE_TAC [NORMAL_DEF]));;
Definition NORMAL_DEF autoloaded from theory `more_gp`.
\end{verbatim}
\mvdots
\begin{verbatim}
OK..
"!x n. H n ==> H((INV((\N. T),$plus)x) plus (n plus x))"
    [ "SUBGROUP((\N. T),$plus)H" ]

() : void
\end{verbatim}
\end{session}

As a result of executing the previous tactic, we have reduced our goal
to showing that, if {\small\tt n} is in {\small\tt H}, then some
expression involving {\small\tt n} is in {\small\tt H}.  If you look
at the definition of {\small\verb+neg+} found in
\begin{verbatim}
   hol/Library/integer/integer.print
\end{verbatim}
or Appendix A, you will recognize that
{\small\verb+INV((\N. T),$plus)x+} is really just what
{\small\verb+neg x+} is defined to be.  Therefore, what we would
next like to do is to break up the implication, as before, and 
then to rewrite with the definition of {\small\verb+neg+} read
right to left.  To break up the implication exactly as before, we
would first need to deal with the two universal quantifications out
front.  That would require two applications of
{\small\verb+GEN_TAC+}.  However, the universal quantifications
and the implication can each be dealt with by the more general tactic,
{\small\verb+STRIP_TAC+}.  Therefore, we can deal with the two
universal quantifications and the implication all at once with
{\small\verb+(REPEAT STRIP_TAC)+}.  To turn the definition of
{\small\verb+neg+} around we need to rewrite with
{\small\verb+(SYM neg_DEF)+} instead of
{\small\verb+neg_DEF+}.  One way of accomplishing all this is

\begin{session}
\begin{verbatim}
#expand ((REPEAT STRIP_TAC) THEN
#   (PURE_ONCE_REWRITE_TAC[(SYM neg_DEF)]));;
Definition neg_DEF autoloaded from theory `integer`.
neg_DEF = |- neg = INV((\N. T),$plus)

OK..
"H((neg x) plus (n plus x))"
    [ "SUBGROUP((\N. T),$plus)H" ]
    [ "H n" ]

() : void
\end{verbatim}
\end{session}

What remains is to simplify the arithmetic expression down to
{\small\tt n}, and then conclude the result from the assumptions.
The first thing to do to simplify this expression is to use
commutativity to switch {\small\verb+(n plus x)+} around to
{\small\verb+(x plus n)+}.  But we must be a bit careful how we
attempt to do this.  If we use {\small\verb+REWRITE_TAC+} or
{\small\verb+PURE_REWRITE_TAC+} together with
{\small\verb+COMM_PLUS+} (or even
{\small\verb+(SPECL ["n:integer";"x:integer"] COMM_PLUS))+},
we'll end up with a stack overflow error, since these will proceed to
repeatedly switch all applications of {\small\verb+plus+} around
indefinitely.  If we try
\begin{verbatim}
   PURE_ONCE_REWRITE_TAC[(SPECL ["n:integer";"x:integer"] COMM_PLUS)]
\end{verbatim}
we will avoid the stack overflow error, but it won't give us what we
want.  Let's try it and see.
\begin{session}
\begin{verbatim}
#expand (PURE_ONCE_REWRITE_TAC
#          [(SPECL ["n:integer";"x:integer"] COMM_PLUS)]);;
Theorem COMM_PLUS autoloaded from theory `integer`.
COMM_PLUS = |- !M N. M plus N = N plus M

OK..
"H((n plus x) plus (neg x))"
    [ "SUBGROUP((\N. T),$plus)H" ]

() : void
\end{verbatim}
\end{session}

The resultant goal is not what we originally intended to reduce to.
It switched the outermost {\small\verb+plus+} instead of the one
we had intended.  We could proceed from here this particular time, but
sometimes we may end up with a goal that we can not proceed from, and
must back up and find a different way to proceed.  So let us undo this
previous reduction.
\begin{session}
\begin{verbatim}
#backup();;
"H((neg x) plus (n plus x))"
    [ "SUBGROUP((\N. T),$plus)H" ]
    [ "H n" ]

() : void
\end{verbatim}
\end{session}
We could repeat this to undo even more steps if that were desired.

So, how do we deal with rewriting a subterm of a term when
{\small\verb+REWRITE+} is too heavy-handed?  An answer is to use
{\small\verb+SUBST1_TAC+}.  (Actually, throughout this case study,
we shall use a variation, {\small\verb+NEW_SUBST1_TAC+}, that was
added into \HOL\ when you loaded the Library {\small\verb+group+}.
It differs from {\small\verb+SUBST1_TAC+} by dealing with
hypotheses in a bit more subtle manner.  We will need this extra bit
occasionally later on.)  We wish to substitute all occurrences of
{\small\verb+(n plus x)+} by {\small\verb+(x plus n)+}.
This can be accomplished by
\begin{session}
\begin{verbatim}
#expand (NEW_SUBST1_TAC (SPECL ["n:integer";"x:integer"] COMM_PLUS));;
OK..
"H((neg x) plus (x plus n))"
    [ "SUBGROUP((\N. T),$plus)H" ]
    [ "H n" ]

() : void
\end{verbatim}
\end{session}

The next thing we need to do is regroup the addition to get at
{\small\verb+((neg x) plus x)+}.  We need to rewrite with
the associativity of {\small\verb+plus+}.
\begin{session}
\begin{verbatim}
#expand (PURE_ONCE_REWRITE_TAC[ASSOC_PLUS]);;
Theorem ASSOC_PLUS autoloaded from theory `integer`.
ASSOC_PLUS = |- !M N P. M plus (N plus P) = (M plus N) plus P

OK..
"H(((neg x) plus x) plus n)"
    [ "SUBGROUP((\N. T),$plus)H" ]
    [ "H n" ]

() : void
\end{verbatim}
\end{session}

To finish this off, we need to rewrite
{\small\verb+((neg x) plus x)+} to zero 
{\small\verb+(INT 0)+} as it is represented in the integers), and
then use the fact that {\small\verb+(INT 0)+} is the identity of
addition to reduce {\small\verb+((INT 0) plus n)+} to {\small\tt n}.
This would give us a goal of {\small\verb+(H n)+}, which is one of
our assumptions.  We can accomplish these two rewrites and finish off
the goal by
\begin{session}
\begin{verbatim}
#expand (ASM_REWRITE_TAC[PLUS_INV_LEMMA; PLUS_ID_LEMMA]);;
Theorem PLUS_ID_LEMMA autoloaded from theory `integer`.
\end{verbatim}
\mvdots
\begin{verbatim}
OK..
goal proved
. |- H(((neg x) plus x) plus n)
. |- H((neg x) plus (x plus n))
. |- H((neg x) plus (n plus x))
|- !x n. H n ==> H((INV((\N. T),$plus)x) plus (n plus x))
|- !H. SUBGROUP((\N. T),$plus)H ==> NORMAL((\N. T),$plus)H

Previous subproof:
goal proved
() : void
\end{verbatim}
\end{session}

We have finished proving the theorem, but it isn't permanently stored
anywhere.  To do this we'll use {\small\verb+prove_thm+}.  We
want not just to store this theorem, but also to have it available for
use in other proofs in this session.  Therefore for consistency, we
should bind the theorem to the same name that we store it under.  We
can do this by
\begin{session}
\begin{verbatim}
#let INT_SBGP_NORMAL = prove_thm(`INT_SBGP_NORMAL`,
#"!H. SUBGROUP((\N.T),$plus)H ==> NORMAL((\N.T),$plus)H",
#(GEN_TAC THEN DISCH_TAC THEN (ASM_REWRITE_TAC[NORMAL_DEF]) THEN
# (REPEAT STRIP_TAC) THEN (PURE_ONCE_REWRITE_TAC[(SYM neg_DEF)]) THEN
# (NEW_SUBST1_TAC (SPECL ["n:integer";"x:integer"] COMM_PLUS)) THEN
# (PURE_ONCE_REWRITE_TAC[ASSOC_PLUS]) THEN
# (ASM_REWRITE_TAC[PLUS_INV_LEMMA; PLUS_ID_LEMMA])));;
INT_SBGP_NORMAL = 
|- !H. SUBGROUP((\N. T),$plus)H ==> NORMAL((\N. T),$plus)H
\end{verbatim}
\end{session}

The tactic given to {\small\verb+prove_thm+} is the result of
conjoining each of the tactics given to {\small\verb+expand+}.
In general, the composition will be a bit more complicated, since a
tactic may return a list of subgoals, requiring the composition of the
tactic with a list of tactics, using {\small\verb+THENL+}.

In addition to the fact that every subgroup of the integers is normal,
there are a couple of other facts about subgroups that will make our
work later on a little easier.  These two facts are that
{\small\verb+(INT 0)+} is contained in every subgroup of the
integers, and that if {\small\tt N} is in a subgroup of the integers,
then so is {\small\verb+(neg N)+}.  Let's prove each of these
theorems next.

For the first theorem, the goal is to show that for all {\small\tt H},
if {\small\tt H} is a subgroup, then {\small\verb+(INT 0)+} is in
{\small\tt H}, \ie\ we have {\small\verb+H(INT O)+}.
\begin{session}
\begin{verbatim}
#set_goal ([],"!H.SUBGROUP((\N.T),$plus)H ==> H(INT 0)");;
"!H. SUBGROUP((\N. T),$plus)H ==> H(INT 0)"

() : void
\end{verbatim}
\end{session}

As in the previous problem, we want to begin by moving the hypothesis
that {\small\tt H} is a subgroup into the assumptions.
\begin{session}
\begin{verbatim}
#expand (REPEAT STRIP_TAC);;
OK..
"H(INT 0)"
    [ "SUBGROUP((\N. T),$plus)H" ]

() : void
\end{verbatim}
\end{session}

Now, the reason that {\small\verb+(INT 0)+} is in {\small\tt H}
is that {\small\verb+(INT 0)+} is the group identity of the
integers and the group identity of any subgroup of a group is the same
identity as that of the whole group.  This fact is recorded by theorem
{\small\verb+SBGP_ID_GP_ID+}, which can be found in the file
\begin{verbatim}
   hol/Library/group/more_gp.print
\end{verbatim}
or Appendix A.  But to use {\small\verb+SBGP_ID_GP_ID+}, we need
first to rewrite with the fact that {\small\verb+(INT 0)+} is the
group identity of the integers.  This requires using the theorem
{\small\verb+ID_EQ_0+}, which
can be found in
\begin{verbatim}
   hol/Library/integer/integer.print
\end{verbatim}
or Appendix A.
\begin{session}
\begin{verbatim}
#expand (PURE_ONCE_REWRITE_TAC [(SYM ID_EQ_0)]);;
Theorem ID_EQ_0 autoloaded from theory `integer`.
ID_EQ_0 = |- ID((\N. T),$plus) = INT 0

OK..
"H(ID((\N. T),$plus))"
    [ "SUBGROUP((\N. T),$plus)H" ]

() : void
\end{verbatim}
\end{session}

Next, we want to use the fact that the group identity is also the
subgroup identity.  This requires using
\begin{verbatim}
   SBGP_ID_GP_ID  |- SUBGROUP(G,prod)H ==> (ID(H,prod) = ID(G,prod))
\end{verbatim}
Now, {\small\verb+SBGP_ID_GP_ID+} at the outermost level is an
implication, so we will want to use {\small\verb+UNDISCH+} to
expose the equation.  However, once we have used
{\small\verb+UNDISCH+} we will have
{\small\verb+SUBGROUP(G,prod)H+} as a hypothesis.  In order to
rewrite the the equation {\small\verb+ID(G,prod) = ID(H,prod)+}
we need to instantiate {\small\tt G} and {\small\verb+prod+} (and
the type of {\small\tt H}).  But these all occur in the hypothesis of
the equation.  None of the rewrite tactics (for the time being) will
perform an instantiation if the variable to be instantiated occurs
free in the hypotheses of the rewrite theorem.  Thus, none of the
rewrite tactics will do the job for us here.  However, there is a
variation of {\small\verb+NEW_SUBST1_TAC+} that will do the job
for us.  That variation is {\small\verb+SUBST_MATCH_TAC+}.
\begin{session}
\begin{verbatim}
#expand (SUBST_MATCH_TAC (SYM (UNDISCH SBGP_ID_GP_ID)));;
Theorem SBGP_ID_GP_ID autoloaded from theory `more_gp`.
SBGP_ID_GP_ID = |- SUBGROUP(G,prod)H ==> (ID(H,prod) = ID(G,prod))

OK..
"H(ID(H,$plus))"
    [ "SUBGROUP((\N. T),$plus)H" ]

() : void
\end{verbatim}
\end{session}

So now we are confronted with showing that the identity of {\small\tt H}
is contained in {\small\tt H}.  There exists a tactic,
{\small\verb+GROUP_ELT_TAC+}, in group theory that is specially
designed to deal with routine goals of membership in a group.  If we
had as one of our assumptions that {\small\tt H} were a group (we only
have right now that {\small\tt H} is a subgroup), then we could use
{\small\verb+GROUP_ELT_TAC+} to finish this off.  As it is, if we
use {\small\verb+GROUP_ELT_TAC+} now, it will reduce our goal to
showing this.  So let us do that first.
\begin{session}
\begin{verbatim}
#expand GROUP_ELT_TAC;;
OK..
"GROUP(H,$plus)"
    [ "SUBGROUP((\N. T),$plus)H" ]

() : void
\end{verbatim}
\end{session}

Now, in our assumptions we have that {\small\tt H} is a subgroup.  We
need to use this assumption to get that {\small\tt H} is a group.  In
order to do this, we want to take the assumption, rewrite it with the
definition of being a subgroup, take the last conjunct of the result,
and solve the goal with it.  By "take the assumption", I mean use
{{\small\verb+POP_ASSUM \thm. +}\ldots}.
\begin{session}
\begin{verbatim}
#expand (POP_ASSUM \thm. (ACCEPT_TAC (CONJUNCT2 (CONJUNCT2
#    (PURE_ONCE_REWRITE_RULE [SUBGROUP_DEF] thm)))));;
Definition SUBGROUP_DEF autoloaded from theory `more_gp`.
\end{verbatim}
\mvdots
\begin{verbatim}
OK..
goal proved
. |- GROUP(H,$plus)
. |- H(ID(H,$plus))
. |- H(ID((\N. T),$plus))
. |- H(INT 0)
|- !H. SUBGROUP((\N. T),$plus)H ==> H(INT 0)

Previous subproof:
goal proved
() : void
\end{verbatim}
\end{session}

This finishes of the theorem.  Let's give it the name
{\small\verb+INT_SBGP_ZERO+}.  Putting it all together, we have
\begin{session}
\begin{verbatim}
#let INT_SBGP_ZERO = prove_thm (`INT_SBGP_ZERO`,
#"!H.SUBGROUP((\N.T),$plus)H ==> H(INT 0)",
#((REPEAT STRIP_TAC) THEN
# (PURE_ONCE_REWRITE_TAC [(SYM ID_EQ_0)]) THEN
# (SUBST_MATCH_TAC (SYM (UNDISCH SBGP_ID_GP_ID))) THEN
# GROUP_ELT_TAC THEN
# (POP_ASSUM \thm. (ACCEPT_TAC (CONJUNCT2 (CONJUNCT2
#   (PURE_ONCE_REWRITE_RULE [SUBGROUP_DEF] thm)))))));;
INT_SBGP_ZERO = |- !H. SUBGROUP((\N. T),$plus)H ==> H(INT 0)
\end{verbatim}
\end{session}

Let's move on to the other general theorem about subgroups of the
integers.  We want to show that for every subgroup {\small\tt H} of
the integers and for every integer {\small\tt N}, if {\small\tt N} is
in {\small\tt H}, then {\small\verb+(neg N)+} is also in
{\small\tt H}.
\begin{session}
\begin{verbatim}
#set_goal ([],"!H.SUBGROUP((\N.T),$plus)H ==> !N. (H N ==> H (neg N))");;
"!H. SUBGROUP((\N. T),$plus)H ==> (!N. H N ==> H(neg N))"

() : void
\end{verbatim}
\end{session}

As before, let us move all the preconditions into the assumptions to
expose the goal {\small\verb+H(neg N)+}.
\begin{session}
\begin{verbatim}
#expand (REPEAT STRIP_TAC);;
OK..
"H(neg N)"
    [ "SUBGROUP((\N. T),$plus)H" ]
    [ "H N" ]

() : void
\end{verbatim}
\end{session}

Now, the proof of this fact is going to be much the same as the proof
of the previous fact.  In particular, we are going to need to use
{\small\verb+SBGP_INV_GP_INV+} (as {\small\verb+SGBP_ID_GP_ID+}
was used before) to change an inverse in the integers into an inverse
in {\small\tt H}, and then we are going to want to finish up the
{\small\verb+GROUP_ELT_TAC+}.  But for
{\small\verb+GROUP_ELT_TAC+} to finish the problem off we will need
to have among our assumptions that {\small\tt H} is a group.  We can't use
{\small\verb+POP_ASSUM+} to give us access to the assumption that
{\small\tt H} is a subgroup here because we want to access the
assumption that {\small\tt H} is a subgroup and that is not the top
assumption.  There are ways of making use of
{\small\verb+POP_ASSUM+}, but since we want to keep all the
assumptions we currently have, there are easier ways.  Probably the
simplest and easiest way to access the assumption we want is to use
{\small\verb+ASSUME+}.  We can then rewrite the resulting theorem
with the subgroup definition, break it up and put the pieces into the
assumptions.  To break up the result of rewriting with the definition
of being a subgroup and include the results in the assumptions, we
want to use {\small\verb+STRIP_ASSUME_TAC+}.

\begin{session}
\begin{verbatim}
#expand (STRIP_ASSUME_TAC (PURE_ONCE_REWRITE_RULE [SUBGROUP_DEF]
#    (ASSUME "SUBGROUP((\N.T),$plus)H")));;
OK..
"H(neg N)"
    [ "SUBGROUP((\N. T),$plus)H" ]
    [ "H N" ]
    [ "GROUP((\N. T),$plus)" ]
    [ "!x. H x ==> (\N. T)x" ]
    [ "GROUP(H,$plus)" ]

() : void
\end{verbatim}
\end{session}

Now, to get started with the proof in earnest, let us rewrite
{\small\verb+(neg n)+} to the inverse in the integers of
{\small\tt N}.
\begin{session}
\begin{verbatim}
#expand (PURE_ONCE_REWRITE_TAC [neg_DEF]);;
OK..
"H(INV((\N. T),$plus)N)"
    [ "SUBGROUP((\N. T),$plus)H" ]
    [ "H N" ]
    [ "GROUP((\N. T),$plus)" ]
    [ "!x. H x ==> (\N. T)x" ]
    [ "GROUP(H,$plus)" ]

() : void
\end{verbatim}
\end{session}
Then we want to rewrite the inverse in the integers to the inverse in
{\small\tt H}.  (Remember what we did with the identity?)
\begin{session}
\begin{verbatim}
#expand (SUBST_MATCH_TAC
#    (SYM (UNDISCH (SPEC_ALL (UNDISCH SBGP_INV_GP_INV)))));;
Theorem SBGP_INV_GP_INV autoloaded from theory `more_gp`.
\end{verbatim}
\mvdots
\begin{verbatim}
OK..
"H(INV(H,$plus)N)"
    [ "SUBGROUP((\N. T),$plus)H" ]
    [ "H N" ]
    [ "GROUP((\N. T),$plus)" ]
    [ "!x. H x ==> (\N. T)x" ]
    [ "GROUP(H,$plus)" ]

() : void
\end{verbatim}
\end{session}
And finally we want to finish it off with
{\small\verb+GROUP_ELT_TAC+}.
\begin{session}
\begin{verbatim}
#expand GROUP_ELT_TAC;;
OK..
goal proved
.. |- H(INV(H,$plus)N)
... |- H(INV((\N. T),$plus)N)
... |- H(neg N)
.. |- H(neg N)
|- !H. SUBGROUP((\N. T),$plus)H ==> (!N. H N ==> H(neg N))

Previous subproof:
goal proved
() : void
\end{verbatim}
\end{session}

Let's put it all together and call the resulting theorem
{\small\verb+INT_SBGP_neg+}.
\begin{session}
\begin{verbatim}
#let INT_SBGP_neg = prove_thm(`INT_SBGP_neg`,
#"!H.SUBGROUP((\N.T),$plus)H ==> !N. (H N ==> H (neg N))",
#((REPEAT STRIP_TAC) THEN
# (STRIP_ASSUME_TAC (PURE_ONCE_REWRITE_RULE [SUBGROUP_DEF]
#    (ASSUME "SUBGROUP((\N.T),$plus)H"))) THEN
# (PURE_ONCE_REWRITE_TAC [neg_DEF]) THEN
# (SUBST_MATCH_TAC
#    (SYM (UNDISCH (SPEC_ALL (UNDISCH SBGP_INV_GP_INV))))) THEN
# GROUP_ELT_TAC));;
INT_SBGP_neg = 
|- !H. SUBGROUP((\N. T),$plus)H ==> (!N. H N ==> H(neg N))
\end{verbatim}
\end{session}

Since we are going to be working a great deal with the set of
multiples of an integer, let's make a definition to make this easier.
The way we represent a set is with a predicate.   For each {\small\tt n},
we  want to define the predicate saying whether a given {\small\tt m}
is a multiple of {\small\tt n}, that is, whether there exists a
{\small\tt p} such that {\small\tt m} is {\small\verb+p times n+}.
The following command will make such a definition. 

\begin{session}
\begin{verbatim}
#let INT_MULT_SET_DEF = new_definition(`INT_MULT_SET_DEF`,
#        "int_mult_set n = \m. ?p. (m = p times n)");;
INT_MULT_SET_DEF = |- !n. int_mult_set n = (\m. ?p. m = p times n)
\end{verbatim}
\end{session}

The predicate {\small\verb+(int_mult_set n)+} describes (is true
of) those integers which are multiples of the given integer {\small\tt
n}.  We could have given the definition as
\begin{verbatim}
   "int_mult_set n m = ?p. (m = p times n)"
\end{verbatim}
which would have perhaps been more conventional, or equally well as
\begin{verbatim}
   "int_mult_set = \n m. ?p. (m = p times n)"
\end{verbatim}
However, the phrasing chosen emphasizes that we are defining the
entity, the set of multiples of {\small\tt n}, where {\small\tt n} is
considered fixed, or a parameter.

The next thing that we want to show is that the set of multiples of
{\small\tt n} under addition from a normal subgroup of the integers.
\begin{session}
\begin{verbatim}
#set_goal([],"!n.NORMAL((\N.T),$plus)(int_mult_set n)");;
"!n. NORMAL((\N. T),$plus)(int_mult_set n)"

() : void
\end{verbatim}
\end{session}

By the theorem we just proved, if a subset of the integers is a
subgroup, then it is a normal subgroup.  Therefore, it should suffice
to show that the set of multiples of {\small\tt n} is a subgroup in
order to show that it is a normal subgroup.  The tactics
{\small\verb+MP_IMP_TAC+} and {\small\verb+MATCH_MP_IMP_TAC+}
are two more of the tactics in the same package as
{\small\verb+NEW_SUBST1_TAC+}, and they are designed to do deal
with reductions of this kind.  That is, if we need to show {\small\tt A},
and we have a theorem
\begin{verbatim}
   thm = |- B ==> A
\end{verbatim}
then {\small\verb+(MP_IMP_TAC thm)+} will reduce the problem of
showing {\small\tt A} to the problem of showing {\small\tt B}.  More
generally, if we need to show {\small\tt A}, and we have a theorem
\begin{verbatim}
   thm = |- B' ==> A'
\end{verbatim}
where {\small\tt A} is an instance of {\small\verb+A'+}, then
{\small\verb+(MATCH_MP_IMP_TAC thm)+} will reduce the problem of
showing {\small\tt A} to the problem of showing {\small\tt B} where
{\small\tt B} is the corresponding instance of {\small\verb+B'+}.
Thus, remembering first to undo the universal quantification, what we
want is
\begin{session}
\begin{verbatim}
#expand (GEN_TAC THEN (MATCH_MP_IMP_TAC INT_SBGP_NORMAL));;
OK..
"SUBGROUP((\N. T),$plus)(int_mult_set n)"

() : void
\end{verbatim}
\end{session}

From group theory, there is a lemma, {\small\verb+SUBGROUP_LEMMA+},
which says that a set {\small\tt H} is a subgroup of a set {\small\tt
G} provided that {\small\tt G} is a group, that {\small\tt H } is nonempty
and contained in {\small\tt G}, and that {\small\tt H} is closed under
products and inverses.  Therefore, if we rewrite with this lemma, then
we need to show that the integers form a group, that the set of
multiples of a fixed number is non-empty and that it is closed under
addition and additive inverses.  However, we have a theorem from the
integer theory, {\small\verb+integer_as_GROUP+}, that states that
the integers form a group.  So, if we rewrite with that theorem at the
same time as we rewrite with {\small\verb+SUBGROUP_LEMMA+}, this
subgoal of showing the integers form a group would be automatically
dealt with.  For the remaining subgoals, the next step will be to expand
the definition of the {\small\verb+int_mult_set+}.  Therefore, we
might as well rewrite with {\small\verb+INT_MULT_SET_DEF+} at the same
time as the other rewrites.
\begin{session}
\begin{verbatim}
#expand (REWRITE_TAC [SUBGROUP_LEMMA;INT_MULT_SET_DEF;integer_as_GROUP]);;
Theorem integer_as_GROUP autoloaded from theory `integer`.
\end{verbatim}
\mvdots
\begin{verbatim}
OK..
"(?x. (\m. ?p. m = p times n)x) /\
 (!x y.
   (\m. ?p. m = p times n)x /\ (\m. ?p. m = p times n)y ==>
   (\m. ?p. m = p times n)(x plus y)) /\
 (!x.
   (\m. ?p. m = p times n)x ==>
   (\m. ?p. m = p times n)(INV((\N. T),$plus)x))"

() : void
\end{verbatim}
\end{session}

In this subgoal we have a lot of lambda abstractions being applied
to arguments.  Therefore, simplification by beta reduction is called
for.
\begin{session}
\begin{verbatim}
#expand BETA_TAC;;
OK..
"(?x p. x = p times n) /\
 (!x y.
   (?p. x = p times n) /\ (?p. y = p times n) ==>
   (?p. x plus y = p times n)) /\
 (!x. (?p. x = p times n) ==> (?p. INV((\N. T),$plus)x = p times n))"

() : void
\end{verbatim}
\end{session}

We have at this point a goal which is a conjunction of three subgoals:
namely, that there is a multiple of the fixed integer, that the sum of
two multiples is again a multiple, and that the additive inverse of a
multiple is again a multiple.  To solve this compound goal, we want to
solve each of these goals individually.  Therefore, we need to break up
this goal, and {\small\verb+(REPEAT STRIP_TAC)+} will do the job.
\begin{session}
\begin{verbatim}
#expand (REPEAT STRIP_TAC);;
OK..
3 subgoals
"?p. INV((\N. T),$plus)x = p times n"
    [ "x = p times n" ]

"?p. x plus y = p times n"
    [ "x = p times n" ]
    [ "y = p' times n" ]

"?x p. x = p times n"

() : void
\end{verbatim}
\end{session}

This returns three subgoals, and I suggest that you separate out the
work that you do for each of them by putting something like the
following at the start of work on each of the subgoals:
\begin{verbatim}
   % goal 1 -- int_mult_set is non-empty %
\end{verbatim}

The first subgoal to be dealt with is showing that there is a multiple
of {\small\tt n}.  One of the easiest multiples to demonstrate exists is
{\small\verb+(INT 0)+}.
\begin{session}
\begin{verbatim}
#expand (EXISTS_TAC "INT 0");;
OK..
"?p. INT 0 = p times n"

() : void
\end{verbatim}
\end{session}
And, of course, the thing you multiply by to get
{\small\verb+(INT 0)+} is {\small\verb+(INT 0)+}.
\begin{session}
\begin{verbatim}
#expand (EXISTS_TAC "INT 0");;
OK..
"INT 0 = (INT 0) times n"

() : void
\end{verbatim}
\end{session}
If we rewrite this goal with {\small\verb+TIMES_ZERO+}, we will
finish it off.
\begin{session}
\begin{verbatim}
#expand (REWRITE_TAC [TIMES_ZERO]);;
Theorem TIMES_ZERO autoloaded from theory `integer`.
\end{verbatim}
\mvdots
\begin{verbatim}
OK..
goal proved
|- INT 0 = (INT 0) times n
|- ?p. INT 0 = p times n
|- ?x p. x = p times n

Previous subproof:
2 subgoals
"?p. INV((\N. T),$plus)x = p times n"
    [ "x = p times n" ]

"?p. x plus y = p times n"
    [ "x = p times n" ]
    [ "y = p' times n" ]

() : void
\end{verbatim}
\end{session}
This finishes off the first subgoal, leaving two more to be dealt
with.

The next goal is to show that the sum of two multiples of a fixed
integer is again a multiple of that fixed integer.
\begin{verbatim}
   % goal 2 -- int_mult_set closed under addition %
\end{verbatim}
From our assumptions we have that {\small\verb+x = p times n+}
and {\small\verb+y = p' times n+}.  We would like to substitute
these into the equation we are trying to show.  However, first we need
to eliminate the existential quantifier.  Obviously, the particular
multiple of {\small\tt n} that yields {\small\verb+x plus y+} is
{\small\verb+p plus p'+}. 
\begin{session}
\begin{verbatim}
#expand (EXISTS_TAC "p plus p'");;
OK..
"x plus y = (p plus p') times n"
    [ "x = p times n" ]
    [ "y = p' times n" ]

() : void
\end{verbatim}
\end{session}
Now we may rewrite with the assumptions.  However, we also wish to use
distributivity to pull the {\small\tt n} out to the right in the
resulting expression.  Therefore let us rewrite with
\begin{session}
\begin{verbatim}
#expand (ASM_REWRITE_TAC [RIGHT_PLUS_DISTRIB]);;
Theorem RIGHT_PLUS_DISTRIB autoloaded from theory `integer`.
\end{verbatim}
\mvdots
\begin{verbatim}
OK..
goal proved
.. |- x plus y = (p plus p') times n
.. |- ?p. x plus y = p times n

Previous subproof:
"?p. INV((\N. T),$plus)x = p times n"
    [ "x = p times n" ]

() : void
\end{verbatim}
\end{session}
which finishes off the second goal.

We are now left with the third and last subgoal: to show that the
additive inverse of a multiple of {\small\tt n} is again a multiple of
{\small\tt n}. 
\begin{verbatim}
   % goal 3 -- int_mult set closed under additive inverses %
\end{verbatim}
Now, the additive inverse of a number is a fancy way of saying the
negative of a number.  So, let us rewrite the goal to use
{\small\verb+neg+} instead.
\begin{session}
\begin{verbatim}
#expand (PURE_ONCE_REWRITE_TAC[(SYM neg_DEF)]);;
OK..
"?p. neg x = p times n"
    [ "x = p times n" ]

() : void
\end{verbatim}
\end{session}

Obviously, the multiple of {\small\tt n} that gets us
{\small\verb+neg x+} is just the negative of the multiple that
gets us {\small\tt x}.
\begin{session}
\begin{verbatim}
#expand (EXISTS_TAC "neg p");;
OK..
"neg x = (neg p) times n"
    [ "x = p times n" ]

() : void
\end{verbatim}
\end{session}

Finally, if we rewrite with the assumption that
{\small\verb+x = p times n+} we get essentially one of the
clauses of the theorem {\small\verb+TIMES_neg+} from the theory
{\small\verb+integer+}.  If we rewrite with that as well, we
finish off the entire goal.
\begin{session}
\begin{verbatim}
#expand (ASM_REWRITE_TAC[TIMES_neg]);;
Theorem TIMES_neg autoloaded from theory `integer`.
\end{verbatim}
\mvdots
\begin{verbatim}
OK..
goal proved
. |- neg x = (neg p) times n
. |- ?p. neg x = p times n
. |- ?p. INV((\N. T),$plus)x = p times n
|- (?x p. x = p times n) /\
   (!x y.
     (?p. x = p times n) /\ (?p. y = p times n) ==>
     (?p. x plus y = p times n)) /\
   (!x. (?p. x = p times n) ==> (?p. INV((\N. T),$plus)x = p times n))
|- (?x. (\m. ?p. m = p times n)x) /\
   (!x y.
     (\m. ?p. m = p times n)x /\ (\m. ?p. m = p times n)y ==>
     (\m. ?p. m = p times n)(x plus y)) /\
   (!x.
     (\m. ?p. m = p times n)x ==>
     (\m. ?p. m = p times n)(INV((\N. T),$plus)x))
|- SUBGROUP((\N. T),$plus)(int_mult_set n)
|- !n. NORMAL((\N. T),$plus)(int_mult_set n)

Previous subproof:
goal proved
() : void
\end{verbatim}
\end{session}

All the steps put together to save
the theorem looks like the following: 
\begin{session}
\begin{verbatim}
#let INT_MULT_SET_NORMAL = prove_thm(`INT_MULT_SET_NORMAL`,
#"!n. NORMAL((\N. T),$plus)(int_mult_set n)",
#(GEN_TAC THEN (MATCH_MP_IMP_TAC INT_SBGP_NORMAL) THEN
# (REWRITE_TAC[SUBGROUP_LEMMA;INT_MULT_SET_DEF;integer_as_GROUP]) THEN
# BETA_TAC THEN (REPEAT STRIP_TAC) THENL
# [((EXISTS_TAC "INT 0") THEN (EXISTS_TAC "INT 0") THEN
#  (REWRITE_TAC [TIMES_ZERO]));
#  ((EXISTS_TAC "p plus p'") THEN
#   (ASM_REWRITE_TAC [RIGHT_PLUS_DISTRIB]));
#  ((PURE_ONCE_REWRITE_TAC[(SYM neg_DEF)]) THEN
#   (EXISTS_TAC "neg p") THEN
#   (ASM_REWRITE_TAC[TIMES_neg]))]));;
INT_MULT_SET_NORMAL = |- !n. NORMAL((\N. T),$plus)(int_mult_set n)
\end{verbatim}
\end{session}

With this result we conclude this this section.  To shut down, you
should 
execute
\begin{session}
\begin{verbatim}
#close_theory ();;
() : void
\end{verbatim}
\end{session}
followed by
\begin{session}
\begin{verbatim}
#quit();;
faulkner+
\end{verbatim}
\end{session}
if you wish to leave \HOL.

As a closing remark, you might want to create another file
{\small\verb+mk_int_sbgp.ml+} that contains the same work as is
currently in {\small\verb+int_sbgp.show.ml+} minus all the
executions of {\small\verb+set_goal+} and
{\small\verb+expand+}.  These were only used to discover the
proofs, not to save them. 



\section{Basic modular arithmetic}

By working through the previous section, you acquired some practice
proving group-theoretic results within the framework of the integers.
In particular, you proved that the set of multiples of a fixed integer
forms a normal subgroup of the integers.  In this section, we shall
use this result to define the set of integers mod {\small\tt n}, and to
demonstrate that they form a group.  After this, we shall import the
first-order group theory from the group theory library, and thus
create an enriched computational environment for modular arithmetic.

Once again we shall assume that the computing environment in which we
will be working includes an emacs editor, and that you are familiar
with its use.  And once again, I strongly recommend that as you are
working, you write all your commands in a file (this time named
something like {\small\verb+int_mod.show.ml+}) and then copy your work
into the shell window running \HOL.  This will provide you both with a
transcript of your work and a file which you could execute to rebuild
the results we will be proving.  You can find a file containing
essentially the same work as you will be doing in
\begin{verbatim}
   hol/Library/int_mod/int_mod.show.ml
\end{verbatim}
and a copy of the shell script created by doing the work discussed in
this section in
\begin{verbatim}
   hol/Training/studies/int_mod/int_mod.shell
\end{verbatim}
Feel free to examine these as we go along.

In this section, we want to create a new theory file, named
{\small\verb+int_mod.th+}.  As before, once you have started up \HOL,
you initialize this file by executing
\begin{session}
\begin{verbatim}
faulkner% hol88

       _  _    __    _      __    __
|___   |__|   |  |   |     |__|  |__|
|      |  |   |__|   |__   |__|  |__|

  Version 1.07, built on Jul 13 1989

#new_theory `int_mod`;;
() : void
\end{verbatim}
\end{session}

Once again we will be needing the group theory and the integer theory,
so you want to load them.
\begin{session}
\begin{verbatim}
#load_library `group`;;
Loading library `group` ...
\end{verbatim}
\mvdots
\begin{verbatim}
#load_library `integer`;;
Loading library `integer` ...
\end{verbatim}
\end{session}

In addition, we will be needing access to the results from the
previous section.  As a consequence of the work you did in the
previous section, there should be a file named
{\small\verb+int_sbgp.th+} located in the same directory as the one
you are currently working in.  To have access to the results in that
file, you should execute 
\begin{session}
\begin{verbatim}
#new_parent `int_sbgp`;;
Theory int_sbgp loaded
() : void

#include_theory `int_sbgp`;;
() : void
\end{verbatim}
\end{session}

The command {\small\verb+include_theory `int_sbgp`;;+} will make it
possible for you to refer to the theorems and definitions of the
previous section by using the names under which they were stored.

In this section, we wish to develop the arithmetic theory of modular
arithmetic under modular addition.  The first thing we must do is make
a collection of definitions that say what we are talking about.  The
set of integers mod {\small\tt n} is the quotient of the integers by
the normal subgroup consisting of the multiples of {\small\tt n}.
From the last section, we can express the set of multiples of
{\small\tt n} by 
\begin{verbatim}
   int_mult_set n
\end{verbatim}
The group of the integers (under addition) is expressed by
\begin{verbatim}
   (\N:integer.T),plus)
\end{verbatim}
The way we express the quotient set of the integers by the multiples
of {\small\tt n} is
\begin{verbatim}
   quot_set((\N.T),plus)(int_mult_set n)
\end{verbatim}
Therefore, we can define the set of integers mod {\small\tt n} by
\begin{session}
\begin{verbatim}
#let INT_MOD_DEF = new_definition(`INT_MOD_DEF`,
#"int_mod n = quot_set((\N.T),plus)(int_mult_set n)");;
INT_MOD_DEF = |- !n. int_mod n = quot_set((\N. T),$plus)(int_mult_set n)
\end{verbatim}
\end{session}
Similarly, we can express the quotient product (addition mod
{\small\tt n}) by 
\begin{verbatim}
   quot_prod((\N.T),plus)(int_mult_set n)
\end{verbatim}
and we define addition mod {\small\tt n} by
\begin{session}
\begin{verbatim}
#let PLUS_MOD_DEF = new_definition(`PLUS_MOD_DEF`,
#"plus_mod n = quot_prod((\N.T),plus)(int_mult_set n)");;
PLUS_MOD_DEF = 
|- !n. plus_mod n = quot_prod((\N. T),$plus)(int_mult_set n)
\end{verbatim}
\end{session}

Now, an element of the set of integers mod {\small\tt n} is an
equivalence class; it is again a set.  Using group theory we can
identify this set as the (left) coset of the multiples {\small\tt n}
by {\small\tt m}, where {\small\tt m} is a representative of the
equivalence class.  (The quotient set of a group by a normal subgroup
was defined to be the set of left cosets of the normal subgroup.)  Let
us define a function that will take an integer to the associated left
coset.  Such a definition will allow us to phrase theorems in ways in
which we are more accustomed to think.  The left coset in the integers
of the multiples of {\small\tt n} by {\small\tt m} is denoted by 
\begin{verbatim}
   LEFT_COSET((\N.T),$plus)(int_mult_set n)m
\end{verbatim}
and so the definition we want to make is
\begin{session}
\begin{verbatim}
#let MOD_DEF = new_definition(`MOD_DEF`,
#"mod n m = LEFT_COSET((\N.T),$plus)(int_mult_set n)m");;
MOD_DEF = |- !n m. mod n m = LEFT_COSET((\N. T),$plus)(int_mult_set n)m
\end{verbatim}
\end{session}

Let us prove a lemma that says that the elements of the integers mod
{\small\tt n} (as we have defined them) are precisely those sets that
are of the form {\small\verb+mod n m+}, for some {\small\tt m}.
\begin{session}
\begin{verbatim}
#set_goal([],"(!m.int_mod n (mod n m)) /\
              (int_mod n q ==> ?m.(q = mod n m))");;
"(!m. int_mod n(mod n m)) /\ (int_mod n q ==> (?m. q = mod n m))"

() : void
\end{verbatim}
\end{session}

This follows almost immediately from the definitions we just made,
together with the definition of {\small\verb+quot_set+}.  The
definition of {\small\verb+quot_set+} requires that the set that we
are quotienting by be a normal subgroup.  However, from the previous
section, we have that the set of multiples of {\small\tt n} is a
normal subgroup of the integers.  So, let us begin by rewriting with
all of these definitions, together with this fact. 
\begin{session}
\begin{verbatim}
#expand (REWRITE_TAC
#         [INT_MOD_DEF;MOD_DEF;QUOTIENT_SET_DEF;INT_MULT_SET_NORMAL]);;
Theorem INT_MULT_SET_NORMAL autoloaded from theory `int_sbgp`.
\end{verbatim}
\mvdots
\begin{verbatim}
OK..
"!m.
  ?x.
   LEFT_COSET((\N. T),$plus)(int_mult_set n)m =
   LEFT_COSET((\N. T),$plus)(int_mult_set n)x"

() : void
\end{verbatim}
\end{session}

Clearly, for each {\small\tt m} the required {\small\tt x} is just
{\small\tt m}.  So we can finish this off by
\begin{session}
\begin{verbatim}
#expand (GEN_TAC THEN (EXISTS_TAC "m:integer") THEN REFL_TAC);;
OK..
goal proved
|- !m.
    ?x.
     LEFT_COSET((\N. T),$plus)(int_mult_set n)m =
     LEFT_COSET((\N. T),$plus)(int_mult_set n)x
|- (!m. int_mod n(mod n m)) /\ (int_mod n q ==> (?m. q = mod n m))

Previous subproof:
goal proved
() : void
\end{verbatim}
\end{session}

Having finished developing the proof, we want to save the result for
future use.
\begin{session}
\begin{verbatim}
#let INT_MOD_MOD_LEMMA = prove_thm(`INT_MOD_MOD_LEMMA`,
#"(!m.int_mod n (mod n m)) /\ (int_mod n q ==> ?m.(q = mod n m))",
#((REWRITE_TAC
#  [INT_MOD_DEF;MOD_DEF;QUOTIENT_SET_DEF;INT_MULT_SET_NORMAL]) THEN
# GEN_TAC THEN (EXISTS_TAC "m:integer") THEN REFL_TAC));;
INT_MOD_MOD_LEMMA = 
|- (!m. int_mod n(mod n m)) /\ (int_mod n q ==> (?m. q = mod n m))
\end{verbatim}
\end{session}

Another thing that we want to know about the definitions that we've
made is that if we add two integers mod {\small\tt n} the result is
the same as if we add the integers and then compute the result mod
{\small\tt n}.
\begin{session}
\begin{verbatim}
#set_goal([],"!x y. plus_mod n(mod n x)(mod n y) = mod n(x plus y)");;
"!x y. plus_mod n(mod n x)(mod n y) = mod n(x plus y)"

() : void
\end{verbatim}
\end{session}

First, let us expand out the definitions of {\small\verb+plus_mod+} and
{\small\verb+mod+}.
\begin{session}
\begin{verbatim}
#expand (REWRITE_TAC [PLUS_MOD_DEF;MOD_DEF]);;
OK..
"!x y.
  quot_prod
  ((\N. T),$plus)
  (int_mult_set n)
  (LEFT_COSET((\N. T),$plus)(int_mult_set n)x)
  (LEFT_COSET((\N. T),$plus)(int_mult_set n)y) =
  LEFT_COSET((\N. T),$plus)(int_mult_set n)(x plus y)"

() : void
\end{verbatim}
\end{session}

This current goal is a specific instance of the final conclusion of
the theorem {\small\verb+QUOT_PROD+}.  Unfortunately, this theorem is
not simply of the form hypothesis implies final conclusion, so we must
modify it before we can use {\small\verb+MATCH_MP_IMP_TAC+} on it.
That is, we need to undischarge the first hypothesis.
\begin{session}
\begin{verbatim}
#expand (MATCH_MP_IMP_TAC (UNDISCH QUOT_PROD));;
expand (MATCH_MP_IMP_TAC (UNDISCH QUOT_PROD));;
Theorem QUOT_PROD autoloaded from theory `more_gp`.
\end{verbatim}
\mvdots
\begin{verbatim}
OK..
2 subgoals
"NORMAL((\N. T),$plus)(int_mult_set n)"

"(\N. T)x /\ (\N. T)y"
    [ "NORMAL((\N. T),$plus)(int_mult_set n)" ]

() : void
\end{verbatim}
\end{session}

This returns us two subgoals.  The first subgoal 
{\small\verb+(\N. T)x /\ (\N. T)y+} is essentially trivial.
\begin{session}
\begin{verbatim}
#expand (REWRITE_TAC []);;
OK..
goal proved
|- (\N. T)x /\ (\N. T)y

Previous subproof:
"NORMAL((\N. T),$plus)(int_mult_set n)"

() : void
\end{verbatim}
\end{session}
The other goal is almost precisely what
{\small\verb+INT_MULT_SET_NORMAL+} gives us.
\begin{session}
\begin{verbatim}
#expand (ACCEPT_TAC (SPEC_ALL INT_MULT_SET_NORMAL));;
OK..
goal proved
|- NORMAL((\N. T),$plus)(int_mult_set n)
|- !x y.
    quot_prod
    ((\N. T),$plus)
    (int_mult_set n)
    (LEFT_COSET((\N. T),$plus)(int_mult_set n)x)
    (LEFT_COSET((\N. T),$plus)(int_mult_set n)y) =
    LEFT_COSET((\N. T),$plus)(int_mult_set n)(x plus y)
|- !x y. plus_mod n(mod n x)(mod n y) = mod n(x plus y)

Previous subproof:
goal proved
() : void
\end{verbatim}
\end{session}

Let's save our work as {\small\verb+PLUS_MOD_LEMMA+}.
\begin{session}
\begin{verbatim}
#let PLUS_MOD_LEMMA = prove_thm(`PLUS_MOD_LEMMA`,
# "!x y. plus_mod n(mod n x)(mod n y) = mod n(x plus y)",
#((REWRITE_TAC [PLUS_MOD_DEF;MOD_DEF]) THEN
# (MATCH_MP_IMP_TAC (UNDISCH QUOT_PROD)) THENL
# [(REWRITE_TAC []);
#  (ACCEPT_TAC (SPEC_ALL INT_MULT_SET_NORMAL))]));;
PLUS_MOD_LEMMA = |- !x y. plus_mod n(mod n x)(mod n y) = mod n(x plus y)
\end{verbatim}
\end{session}

Since we want to import the first order group theory, we need to show
that {\small\verb+int_mod n+} actually forms a group under
{\small\verb+plus_mod n+}.
\begin{session}
\begin{verbatim}
#set_goal([],"GROUP(int_mod n,plus_mod n)");;
"GROUP(int_mod n,plus_mod n)"

() : void
\end{verbatim}
\end{session}

As usual, let us begin by expanding out the definitions of
{\small\verb+int_mod n+} and {\small\verb+plus_mod n+}.
\begin{session}
\begin{verbatim}
#expand (PURE_ONCE_REWRITE_TAC[INT_MOD_DEF;PLUS_MOD_DEF]);;
OK..
"GROUP
 (quot_set((\N. T),$plus)(int_mult_set n),
  quot_prod((\N. T),$plus)(int_mult_set n))"

() : void
\end{verbatim}
\end{session}

The subgoal that this gives us back is an instance of the consequent
of the {\small\verb+QUOTIENT_GROUP+}.  Therefore, we can once again use
{\small\verb+MATCH_MP_IMP_TAC+} to reduce this goal.
\begin{session}
\begin{verbatim}
#expand (MATCH_MP_IMP_TAC QUOTIENT_GROUP);;
Theorem QUOTIENT_GROUP autoloaded from theory `more_gp`.
\end{verbatim}
\mvdots
\begin{verbatim}
OK..
"NORMAL((\N. T),$plus)(int_mult_set n)"

() : void
\end{verbatim}
\end{session}

This leaves us, once again, with a goal which is essentially the
theorem {\small\verb+INT_MULT_SET_NORMAL+}.
\begin{session}
\begin{verbatim}
#expand (ACCEPT_TAC (SPEC_ALL INT_MULT_SET_NORMAL));;
OK..
goal proved
|- NORMAL((\N. T),$plus)(int_mult_set n)
|- GROUP
   (quot_set((\N. T),$plus)(int_mult_set n),
    quot_prod((\N. T),$plus)(int_mult_set n))
|- GROUP(int_mod n,plus_mod n)

Previous subproof:
goal proved
() : void
\end{verbatim}
\end{session}

And again, let us store our work.
\begin{session}
\begin{verbatim}
#let int_mod_as_GROUP = prove_thm(`int_mod_as_GROUP`,
#"GROUP(int_mod n,plus_mod n)",
#((PURE_ONCE_REWRITE_TAC[INT_MOD_DEF;PLUS_MOD_DEF]) THEN
# (MATCH_MP_IMP_TAC QUOTIENT_GROUP) THEN
# (ACCEPT_TAC (SPEC_ALL INT_MULT_SET_NORMAL))));;
int_mod_as_GROUP = |- GROUP(int_mod n,plus_mod n)
\end{verbatim}
\end{session}

Before we set about importing the group theory, it will be useful to
have some theorems that will allow us to rewrite things expressed in
group-theoretic terms using more familiar terms.  To get these results
it will be useful to note that {\small\verb+mod n+} is a group
homomorphism, the natural homomorphism.  (In fact, we essentially
already have that {\small\verb+mod n+} is a homomorphism by
{\small\verb+PLUS_MOD_LEMMA+}.) 
\begin{session}
\begin{verbatim}
#set_goal([],"mod n = NAT_HOM((\N.T),$plus)(int_mult_set n)");;
"mod n = NAT_HOM((\N. T),$plus)(int_mult_set n)"

() : void
\end{verbatim}
\end{session}

To show that these two functions are the same, we want to show that
they behave the same on all arguments.  The tactic that allows us to
make this reduction is {\small\verb+EXT_TAC+}.
\begin{session}
\begin{verbatim}
#expand (EXT_TAC "m:integer");;
OK..
"!m. mod n m = NAT_HOM((\N. T),$plus)(int_mult_set n)m"

() : void
\end{verbatim}
\end{session}

To reduce this goal, we want to rewrite with the definitions of
{\small\verb+NAT_HOM+} and {\small\verb+mod+}.
\begin{session}
\begin{verbatim}
#expand (REWRITE_TAC [NAT_HOM_DEF;MOD_DEF]);;
Definition NAT_HOM_DEF autoloaded from theory `more_gp`.
\end{verbatim}
\mvdots
\begin{verbatim}
OK..
"!m.
  LEFT_COSET((\N. T),$plus)(int_mult_set n)m =
  (\y.
    GROUP((\N. T),$plus) /\
    NORMAL((\N. T),$plus)(int_mult_set n) /\
    LEFT_COSET((\N. T),$plus)(int_mult_set n)m y)"

() : void
\end{verbatim}
\end{session}

To reduce the resultant subgoal, we need to remove the conditions
that the integers form a group, and that the set of  multiples of
{\small\tt n} is a normal subgroup from the right-hand side of the
equation.  We also will need to rewrite the right-hand side with the
eta axiom.
\begin{session}
\begin{verbatim}
#expand (REWRITE_TAC [integer_as_GROUP;INT_MULT_SET_NORMAL;ETA_AX]);;
Theorem integer_as_GROUP autoloaded from theory `integer`.
integer_as_GROUP = |- GROUP((\N. T),$plus)

OK..
goal proved
|- !m.
    LEFT_COSET((\N. T),$plus)(int_mult_set n)m =
    (\y.
      GROUP((\N. T),$plus) /\
      NORMAL((\N. T),$plus)(int_mult_set n) /\
      LEFT_COSET((\N. T),$plus)(int_mult_set n)m y)
|- !m. mod n m = NAT_HOM((\N. T),$plus)(int_mult_set n)m
|- mod n = NAT_HOM((\N. T),$plus)(int_mult_set n)

Previous subproof:
goal proved
() : void
\end{verbatim}
\end{session}

This finishes off the whole goal.  Putting all this together, we have
\begin{session}
\begin{verbatim}
#let MOD_NAT_HOM_LEMMA = prove_thm(`MOD_NAT_HOM_LEMMA`,
#"mod n = NAT_HOM((\N.T),$plus)(int_mult_set n)",
#((EXT_TAC "m:integer") THEN
# (REWRITE_TAC [NAT_HOM_DEF;MOD_DEF]) THEN
# (REWRITE_TAC [integer_as_GROUP;INT_MULT_SET_NORMAL;ETA_AX])));;
MOD_NAT_HOM_LEMMA = |- mod n = NAT_HOM((\N. T),$plus)(int_mult_set n)
\end{verbatim}
\end{session}

Let us now use the fact that {\small\verb+mod n+} is the natural
homomorphism to show that the group identity is the image of
{\small\verb+(INT 0)+}.
\begin{session}
\begin{verbatim}
#set_goal([],"ID(int_mod n,plus_mod n) = mod n (INT 0)");;
"ID(int_mod n,plus_mod n) = mod n(INT 0)"

() : void
\end{verbatim}
\end{session}

Before we start reducing this goal, let's add to the assumptions the
various theorems that we repeatedly used in the previous work.  In
particular, let us add the fact that the integers form a group, the
fact that the integers mod {\small\tt n} form a group, and the fact
that the set of multiples of {\small\tt n} is a normal subgroup.  The
tactic that allows you to add a list of theorems as assumptions to a
goal is {\small\verb+ASSUME_LIST_TAC+}.
\begin{session}
\begin{verbatim}
#expand (ASSUME_LIST_TAC
#  [integer_as_GROUP;int_mod_as_GROUP;(SPEC_ALL INT_MULT_SET_NORMAL)]);;
OK..
"ID(int_mod n,plus_mod n) = mod n(INT 0)"
    [ "GROUP((\N. T),$plus)" ]
    [ "GROUP(int_mod n,plus_mod n)" ]
    [ "NORMAL((\N. T),$plus)(int_mult_set n)" ]

() : void
\end{verbatim}
\end{session}

Now, let us rewrite with the lemma that we just proved.
\begin{session}
\begin{verbatim}
#expand (REWRITE_TAC [MOD_NAT_HOM_LEMMA]);;
OK..
"ID(int_mod n,plus_mod n) =
 NAT_HOM((\N. T),$plus)(int_mult_set n)(INT 0)"
    [ "GROUP((\N. T),$plus)" ]
    [ "GROUP(int_mod n,plus_mod n)" ]
    [ "NORMAL((\N. T),$plus)(int_mult_set n)" ]

() : void
\end{verbatim}
\end{session}

If we first rewrote {\small\verb+(INT 0)+} as the additive inverse of
the integers, then, by {\small\verb+HOM_ID_INV_LEMMA+}, we should be
able to reduce the subgoal returned to one of showing that
{\small\verb+NAT_HOM+} is a group homomorphism.  However, the tactic
{\small\verb+MATCH_MP_IMP_TAC+} is a little too restrictive to apply
directly.  A tactic that we can use instead, and which is a bit more
general, is {\small\verb+NEW_MATCH_ACCEPT_TAC+}.  To use
{\small\verb+NEW_MATCH_ACCEPT_TAC+} in place of
{\small\verb+MATCH_MP_IMP_TAC+} you need to first undischarge the
antecedent of the implication.  In our case, we want to use
{\small\verb+NEW_MATCH_ACCEPT_TAC+} with the first conjunct of the
result of undischarging the antecedent of
{\small\verb+HOM_ID_INV_LEMMA+}, swapped around using {\small\verb+SYM+}. 
\begin{session}
\begin{verbatim}
#expand ((PURE_ONCE_REWRITE_TAC[(SYM ID_EQ_0)]) THEN
#  (NEW_MATCH_ACCEPT_TAC
#     (SYM (CONJUNCT1 (UNDISCH HOM_ID_INV_LEMMA)))));;
Theorem HOM_ID_INV_LEMMA autoloaded from theory `more_gp`.
\end{verbatim}
\mvdots
\begin{verbatim}
OK..
"GP_HOM
 ((\N. T),$plus)
 (int_mod n,plus_mod n)
 (NAT_HOM((\N. T),$plus)(int_mult_set n))"
    [ "GROUP((\N. T),$plus)" ]
    [ "GROUP(int_mod n,plus_mod n)" ]
    [ "NORMAL((\N. T),$plus)(int_mult_set n)" ]

() : void
\end{verbatim}
\end{session}

Now, we would like to use {\small\verb+NAT_HOM_THM+} to deal with this
goal, but first we must change {\small\verb+int_mod+} and
{\small\verb+plus_mod+} into {\small\verb+quot_set+} and
{\small\verb+quot_prod+}.
\begin{session}
\begin{verbatim}
#expand (PURE_ONCE_REWRITE_TAC[INT_MOD_DEF;PLUS_MOD_DEF]);;
OK..
"GP_HOM
 ((\N. T),$plus)
 (quot_set((\N. T),$plus)(int_mult_set n),
  quot_prod((\N. T),$plus)(int_mult_set n))
 (NAT_HOM((\N. T),$plus)(int_mult_set n))"
    [ "GROUP((\N. T),$plus)" ]
    [ "GROUP(int_mod n,plus_mod n)" ]
    [ "NORMAL((\N. T),$plus)(int_mult_set n)" ]

() : void
\end{verbatim}
\end{session}

The fact that {\small\verb+NAT_HOM+} is a group homomorphism is given to us
by {\small\verb+NAT_HOM_THM+}, but again we have hypotheses to be dealt with
and the conclusion is not in exactly the right form for
{\small\verb+MATCH_MP_IMP_TAC+}.
\begin{session}
\begin{verbatim}
#expand (NEW_MATCH_ACCEPT_TAC (CONJUNCT1 (UNDISCH NAT_HOM_THM)));;
Theorem NAT_HOM_THM autoloaded from theory `more_gp`.
\end{verbatim}
\mvdots
\begin{verbatim}
OK..
"GROUP((\N. T),$plus) /\ NORMAL((\N. T),$plus)(int_mult_set n)"
    [ "GROUP((\N. T),$plus)" ]
    [ "GROUP(int_mod n,plus_mod n)" ]
    [ "NORMAL((\N. T),$plus)(int_mult_set n)" ]

() : void
\end{verbatim}
\end{session}

This last goal follows immediately from the assumptions.
\begin{session}
\begin{verbatim}
#expand (ASM_REWRITE_TAC []);;
OK..
goal proved
.. |- GROUP((\N. T),$plus) /\ NORMAL((\N. T),$plus)(int_mult_set n)
.. |- GP_HOM
      ((\N. T),$plus)
      (quot_set((\N. T),$plus)(int_mult_set n),
       quot_prod((\N. T),$plus)(int_mult_set n))
      (NAT_HOM((\N. T),$plus)(int_mult_set n))
.. |- GP_HOM
      ((\N. T),$plus)
      (int_mod n,plus_mod n)
      (NAT_HOM((\N. T),$plus)(int_mult_set n))
.. |- ID(int_mod n,plus_mod n) =
      NAT_HOM((\N. T),$plus)(int_mult_set n)(INT 0)
.. |- ID(int_mod n,plus_mod n) = mod n(INT 0)
|- ID(int_mod n,plus_mod n) = mod n(INT 0)

Previous subproof:
goal proved
() : void
\end{verbatim}
\end{session}

To store the work,
\begin{session}
\begin{verbatim}
#let ID_EQ_MOD_0 = prove_thm(`ID_EQ_MOD_0`,
#"ID(int_mod n,plus_mod n) = mod n (INT 0)",
#((ASSUME_LIST_TAC
#  [integer_as_GROUP;int_mod_as_GROUP;
#   (SPEC_ALL INT_MULT_SET_NORMAL)]) THEN
# (REWRITE_TAC [MOD_NAT_HOM_LEMMA]) THEN
# (PURE_ONCE_REWRITE_TAC[(SYM ID_EQ_0)]) THEN
# (NEW_MATCH_ACCEPT_TAC
#  (SYM (CONJUNCT1 (UNDISCH HOM_ID_INV_LEMMA)))) THEN
# (PURE_ONCE_REWRITE_TAC[INT_MOD_DEF;PLUS_MOD_DEF]) THEN
# (NEW_MATCH_ACCEPT_TAC (CONJUNCT1 (UNDISCH NAT_HOM_THM))) THEN
# (ASM_REWRITE_TAC [])));;
ID_EQ_MOD_0 = |- ID(int_mod n,plus_mod n) = mod n(INT 0)
\end{verbatim}
\end{session}

The next result we would like to prove along these lines is that the
group inverse of an element in the integers mod {\small\tt n} corresponds to
{\small\verb+mod n+} of the negative of a representative.
\begin{session}
\begin{verbatim}
#set_goal([],"!m.(INV(int_mod n,plus_mod n)(mod n m) = mod n (neg m))");;
"!m. INV(int_mod n,plus_mod n)(mod n m) = mod n(neg m)"

() : void
\end{verbatim}
\end{session}

Try proving this result yourself. The proof is essentially the same as
the one just finished.  (You might want to have a look at
{\small\verb+neg_DEF+} from the theory integer.)  If you get stuck,
remember that you can always take a look in
\begin{verbatim}
   hol/Training/studies/int_mod/int_mod.shell
\end{verbatim}
to see what was done.  The composite result of the work follows.
\begin{session}
\begin{verbatim}
#let INV_EQ_MOD_NEG = prove_thm(`INV_EQ_MOD_NEG`,
#"!m.(INV(int_mod n,plus_mod n)(mod n m) = mod n (neg m))",
#((ASSUME_LIST_TAC
#  [integer_as_GROUP;int_mod_as_GROUP;
#   (SPEC_ALL INT_MULT_SET_NORMAL)]) THEN
# (REWRITE_TAC [MOD_NAT_HOM_LEMMA]) THEN
# (PURE_ONCE_REWRITE_TAC[neg_DEF]) THEN
# (NEW_MATCH_ACCEPT_TAC (SYM (UNDISCH (SPEC_ALL
#    (CONJUNCT2 (UNDISCH HOM_ID_INV_LEMMA)))))) THENL
# [((PURE_ONCE_REWRITE_TAC[INT_MOD_DEF;PLUS_MOD_DEF]) THEN
#   (NEW_MATCH_ACCEPT_TAC (CONJUNCT1 (UNDISCH NAT_HOM_THM))) THEN
#   (ASM_REWRITE_TAC []));
#  (REWRITE_TAC [])]));;
INV_EQ_MOD_NEG = 
|- !m. INV(int_mod n,plus_mod n)(mod n m) = mod n(neg m)
\end{verbatim}
\end{session}

One last definition and theorem before we do the import.  Let us
define {\small\verb+minus_mod n+} as the result of adding
{\small\verb+mod n+} the inverse.
\begin{session}
\begin{verbatim}
#let MINUS_MOD_DEF = new_definition(`MINUS_MOD_DEF`,
#"minus_mod n m p = plus_mod n m (INV(int_mod n,plus_mod n)p)");;
MINUS_MOD_DEF = 
|- !n m p. minus_mod n m p = plus_mod n m(INV(int_mod n,plus_mod n)p)
\end{verbatim}
\end{session}

The obvious theorem we want is
\begin{session}
\begin{verbatim}
#set_goal([], "!m p. minus_mod n (mod n m) (mod n p) = mod n (m minus p)");;
"!m p. minus_mod n(mod n m)(mod n p) = mod n(m minus p)"

() : void
\end{verbatim}
\end{session}
This follows immediately from definitions and results we already have.
\begin{session}
\begin{verbatim}
#expand (REWRITE_TAC [MINUS_MOD_DEF;INV_EQ_MOD_NEG;
#	   MINUS_DEF;PLUS_MOD_LEMMA]);;
Definition MINUS_DEF autoloaded from theory `integer`.
MINUS_DEF = |- !M N. M minus N = M plus (neg N)

OK..
goal proved
|- !m p. minus_mod n(mod n m)(mod n p) = mod n(m minus p)

Previous subproof:
goal proved
() : void

#let MINUS_MOD_LEMMA = prove_thm(`MINUS_MOD_LEMMA`,
#"!m p. minus_mod n (mod n m) (mod n p) = mod n (m minus p)",
#(REWRITE_TAC [MINUS_MOD_DEF;INV_EQ_MOD_NEG;
#              MINUS_DEF;PLUS_MOD_LEMMA]));;
MINUS_MOD_LEMMA = 
|- !m p. minus_mod n(mod n m)(mod n p) = mod n(m minus p)
\end{verbatim}
\end{session}

Now let us turn to the business of importing the group theory.  In the
file
\begin{verbatim}
   hol/Library/group/inst_gp.ml
\end{verbatim}
there is a collection of functions to facilitate incorporating the
first-order group theory into any given example of a group.  They are
also described in Appendix C.  The particular function we will use
here is {\small\verb+return_GROUP_theory+}.  This function takes a
string which acts a prefix for the names of the theorems to be
returned, it takes a theorem stating that the desired example is a
group, and it takes a list of theorems which it will use to rewrite
the returned list of theorems into a more desirable form.  It returns
a list of pairs of strings and theorems.  The strings are intended to
be the names under which the corresponding theorems will be stored.
The function {\small\verb+include_GROUP_theory+} of this same file
behaves much the same as {\small\verb+return_GROUP_theory+}, except
that it actually stores the theorems in the current theory.  We do not 
wish to use this function because we will wish to do a bit more
cleaning up of the returned list of theorems before saving them.

To begin with, let us bind a variable such as {\small\verb+thm_list+}
with the result of using {\small\verb+return_GROUP_theory+} with the
prefix {\small\verb+`INT_MOD`+}, the theorem
{\small\verb+int_mod_as_GROUP+}, and the rewrite theorems
{\small\verb+ID_EQ_MOD_0+} and {\small\verb+(SYM (SPEC_ALL MINUS_MOD_DEF))+}.
\begin{session}
\begin{verbatim}
#let thm_list = return_GROUP_theory `INT_MOD` int_mod_as_GROUP
# [ID_EQ_MOD_0;(SYM (SPEC_ALL MINUS_MOD_DEF))];;
thm_list = 
[(`INT_MOD_CLOSURE`,
  |- !x y. int_mod n x /\ int_mod n y ==> int_mod n(plus_mod n x y));
\end{verbatim}
\evdots
\end{session}

Throughout the theorems that the previous function returned there
is the hypothesis {\small\verb+int_mod n x+}.  In this, {\small\tt x}
is an arbitrary set of integers.  However, for all these theorems, we
are not interested in what they say about arbitrary sets of integers,
but rather only those sets that are of the form {\small\verb+mod n m+}
for some integer {\small\tt m}.  In this instance we know that we have
{\small\verb+int_mod n (mod n m)+} by the first conjunct of
{\small\verb+INT_MOD_MOD_LEMMA+}.  We want to strip each of these
theorems by using {\small\verb+IMP_CANON+}, which will return a list
of theorems.  To each theorem in each of the lists we then want to
instantiate the variables {\small\tt x}, {\small\tt y}, and
{\small\tt z} with {\small\verb+mod n m1+}, {\small\verb+mod n m2+},
and {\small\verb+mod n m3+} respectively, and then rewrite everything
with both the first conjunct of {\small\verb+INT_MOD_MOD_LEMMA+} and
{\small\verb+INV_EQ_MOD_NEG+}.  Now, to do this we will need to create
a function to do all this to the theorem in each of the pairs and then
map this function over {\small\verb+thm_list+}.  The following does
all this and binds the result to {\small\verb+thl1+}. The
{\small\verb+and_then+} is an infix that takes the argument to a
function on the left and the function on the right.  The function
{\small\verb+STRONG_INST+} is the same as {\small\verb+INST+} except that it
instantiates free variables in the hypotheses, rather than failing.
\begin{session}
\begin{verbatim}
#let thl1 = map (\ (name,thm).(name,
#(IMP_CANON thm) and_then
# (map (\thm1.
#  (STRONG_INST
#   [("mod n m1","x:integer -> bool");
#    ("mod n m2","y:integer -> bool");
#    ("mod n m3","z:integer -> bool")] thm1) and_then
#  (REWRITE_RULE[(CONJUNCT1 INT_MOD_MOD_LEMMA);
#                 INV_EQ_MOD_NEG]))))) thm_list;;
thl1 = 
[(`INT_MOD_CLOSURE`, [|- int_mod n(plus_mod n(mod n m1)(mod n m2))]);
\end{verbatim}
\evdots
\end{session}

Next we want to put each of the lists of theorems back together as a
theorem using {\small\verb+LIST_CONJ+}.  We can also remove the superfluous
clause {\small\tt T} that occurs in {\small\verb+INT_MOD_ID_LEMMA+} by doing a
{\small\verb+(REWRITE_RULE [])+} on each of the resultant theorems.
\begin{session}
\begin{verbatim}
#let thl2 = map (\ (name,thl).(name,
# (LIST_CONJ thl) and_then (REWRITE_RULE []))) thl1;;
thl2 = 
[(`INT_MOD_CLOSURE`, |- int_mod n(plus_mod n(mod n m1)(mod n m2)));
\end{verbatim}
\evdots
\end{session}

Before saving these theorems we would like bind all the free variables, 
except for {\small\tt n}, which we wish to think of as a global variable.
The function {\small\verb+GENL+} takes a list of variables and a theorem,
and returns the result of binding each of these variables in the theorem.
The function {\small\verb+frees+} takes a term and returns a list of all
variables occurring free in the term.  So, we want to take
{\small\verb+frees+} of the conclusion ({\small\verb+concl+}) of each
theorem, then use {\small\verb+filter+} to remove any occurrence of
{\small\tt n}, then pass it to {\small\verb+GENL+} and apply the resulting
function to the original theorem.  And we want to  do this to each theorem
in the list. 
\begin{session}
\begin{verbatim}
#let thl3 = map (\ (name,thm).(name,
# GENL (filter (\x.not(x = "n:integer")) (frees (concl thm))) thm))
# thl2;;
thl3 = 
[(`INT_MOD_CLOSURE`,
  |- !m1 m2. int_mod n(plus_mod n(mod n m1)(mod n m2)));
\end{verbatim}
\evdots
\end{session}

Finally, before saving these theorems, we want to filter out all
occurrences of the theorem {\small\verb+|- T+}.  (In this case it is only
{\small\verb+INT_MOD_INV_CLOSURE+} we will be removing .)
\begin{session}
\begin{verbatim}
#let thl4 = filter (\ (name,thm).not((concl thm) = "T")) thl3;;
thl4 = 
[(`INT_MOD_CLOSURE`,
  |- !m1 m2. int_mod n(plus_mod n(mod n m1)(mod n m2)));
\end{verbatim}
\evdots
\end{session}

Now, to save each of these theorems, we can just use {\small\verb+save_thm+}.
If we want to also bind each of them to the name it is paired with
(and to be stored under), we need to the function
{\small\verb+autoload_theory+} with arguments {\small\verb+`theorem`+} and the
name of the current theory.
\begin{session}
\begin{verbatim}
#map (\ (name,thm).
#  (save_thm (name,thm));
#  (autoload_theory (`theorem`,(current_theory()), name))) thl4;;
[(); (); (); (); (); (); (); (); (); (); (); (); (); ()] : void list
\end{verbatim}
\end{session}

We now have at our disposal a fairly considerable collection theorems
that will facilitate doing arithmetic computations when doing further
theorem proving with the integers mod {\small\tt n}.  You might think
about writing functions {\small\verb+return_INT_MOD+} and
{\small\verb+include_INT_MOD+} which behave in a manner similar to
{\small\verb+return_GROUP_THEORY+} and
{\small\verb+include_GROUP_THEORY+}.  The arguments they would take would 
be a bit different.  For example they would probably need to take a
term which would be the value for {\small\tt n}.   You can find one way of
writing such functions in the file
\begin{verbatim}
   hol/Library/int_mod/inst_int_mod.ml
\end{verbatim}

To shut down
\begin{session}
\begin{verbatim}
#close_theory `int_mod`;;
() : void

#quit();;
faulkner%
\end{verbatim}
\end{session}



\section{Subgroups of the integers, revisited}

This section is the last in this study designed to help the user of
\HOL\ become familiar with using the group theory and integer theory
contained in the Library.  In the section Subgroups of the Integers,
you were asked to prove some basic results about subgroups of the
integers.  In the section Basic Modular Arithmetic, you used the
results of the previous section to define the set of integers mod
{\small\tt n}, and to prove that they formed a group.  Then, after
proving a few theorems to be used as rewrites, we imported the first
order group theory theorems, massaged them a bit to get them into a
more usable form, and then saved them.  In this section we shall
return to proving facts about subgroups of the integers.  In
particular, we shall have as our goal to prove that every subgroup of
the integers is cyclic.  The work involved in proving this result is a
goodly piece of the work necessary to prove that greatest common
divisors exist.

The purpose of this section is to familiarize you with two of the
more powerful techniques for proving theorems about the integers.
The first technique is to break a problem about the integers up into
cases, either positive, negative and zero cases, or the cases
corresponding to the natural numbers and corresponding to the negative
of the natural numbers.  By reducing a problem to such cases as these
we can frequently further reduce the problem to one that can be solved
by induction over the natural numbers.  The second technique is to use
maximal or minimal elements of bounded sets of integers to solve
existential problems.  For each of these techniques there are tactics
which we will be learning to use in this section,

As before, we shall be assuming that you are working within an emacs
editor.  For this section, I suggest you store your work in a file
named something like {\small\verb+int_sbgp.show2.ml+}.  You can find a
file containing essentially the same work as you will be doing in the
file
\begin{verbatim}
   hol/Library/int_mod/int_sbgp.show2.ml
\end{verbatim}
and a copy of the shell script in the file
\begin{verbatim}
   hol/Training/study/int_mod/int_sbgp.shell2
\end{verbatim}
Please feel free to consult these as we go along.

In this section, we will not be creating a new theory.  Instead, we
shall be modifying an existing theory.  The theory we wish to modify
is {\small\verb+int_sbgp.th+}, which was created in the section
Subgroups of the Integers.  To be able to modify
{\small\verb+int_sbgp.th+}, it must be our current theory.  We will
not be making any definitions in this section, so we do not need to be
in draft mode.  Hence, we may begin by
\begin{session}
\begin{verbatim}
faulkner% hol88

       _  _    __    _      __    __
|___   |__|   |  |   |     |__|  |__|
|      |  |   |__|   |__   |__|  |__|

  Version 1.07, built on Jul 13 1989

#load_theory `int_sbgp`;;
Theory int_sbgp loaded
() : void
\end{verbatim}
\end{session}

As before, we will be wanting access to the group theory and integer
theory.
\begin{session}
\begin{verbatim}
#load_library `group`;;
Loading library `group` ...
\end{verbatim}
\mvdots
\begin{verbatim}
#load_library `integer`;;
Loading library `integer` ...
\end{verbatim}
\evdots
\end{session}

Also, we would like the results saved in the first tutorial to be
available to us, bound to the names under which they were stored.
\begin{session}
\begin{verbatim}
#include_theory `int_sbgp`;;
() : void

\end{verbatim}
\end{session}

The first result we will work on in this tutorial is that any
subgroup of the integers is closed under multiplication from any
integer.  This result will be useful in proving that every subgroup of
the integers is cyclic, but it is also of interest in its own right.
\begin{session}
\begin{verbatim}
#set_goal([],"!H. SUBGROUP((\N.T),$plus)H ==> 
#                  !m p. H p ==> H (m times p)");;
"!H. SUBGROUP((\N. T),$plus)H ==> (!m p. H p ==> H(m times p))"

() : void
\end{verbatim}
\end{session}

The first thing we wish to do in working to prove this theorem is to
remove the generalization and move the hypothesis that {\small\tt H}
is a subgroup over into the assumptions, leaving us with a goal of
\begin{verbatim}
   "!m p. H p ==> H (m times p)"
\end{verbatim}
We will want to work with this goal in this form.  If we used
{\small\verb+(REPEAT STRIP_TAC)+}, it would go further, removing the
generalizations of {\small\tt m} and {\small\tt p} and moving the
hypothesis that {\small\tt p} is in {\small\tt H} also over into the
assumptions.  Since we don't want to go this far, we will have to do
the two steps we do want individually.
\begin{session}
\begin{verbatim}
#expand (GEN_TAC THEN DISCH_TAC);;
OK..
"!m p. H p ==> H(m times p)"
    [ "SUBGROUP((\N. T),$plus)H" ]

() : void
\end{verbatim}
\end{session}

Some of the theorems that we will need in order to prove this result
will have as hypothesis that we be dealing with a subgroup; others
will require that we be dealing with a group.  Therefore, we would
like to have among our hypotheses both that {\small\tt H} is a
subgroup of the integers, and what this means in terms of the
definition of subgroup.  Therefore, just as we did in the first
tutorial, we want to add to the assumptions the result of rewriting
the clause {\small\verb+SUBGROUP((\N.T).$plus)H+} using the
definition, while still keeping this clause among the assumptions.
\begin{session}
\begin{verbatim}
#expand (FIRST_ASSUM \thm. (STRIP_ASSUME_TAC
#     (PURE_ONCE_REWRITE_RULE [SUBGROUP_DEF] thm)));;
Definition SUBGROUP_DEF autoloaded from theory `more_gp`.
\end{verbatim}
\mvdots
\begin{verbatim}
OK..
"!m p. H p ==> H(m times p)"
    [ "SUBGROUP((\N. T),$plus)H" ]
    [ "GROUP((\N. T),$plus)" ]
    [ "!x. H x ==> (\N. T)x" ]
    [ "GROUP(H,$plus)" ]

() : void
\end{verbatim}
\end{session}

This goal says that for all integers {\small\tt m} and {\small\tt p},
if {\small\tt p} is in the subgroup {\small\tt H}, then so is
{\small\verb+m times p+}.  If we were trying to prove something like
this over the natural numbers, we would proceed by induction, which
would allow us to reduce the multiplication to an addition, which is
the operation of the subgroup {\small\tt H}.  However, we are dealing
with the integers here, and induction is not valid over the integers;
there is no least element to get started.  So how should we proceed?
Well, if we could show this result for all positive integers, zero,
and all negative numbers, then by {\small\verb+TRICHOTOMY+} we would
have the result for all integers. Moreover, by
{\small\verb+NON_NEG_INT_IS_NUM+}, the set of positive integers
together with zero is the image of the set of natural numbers under
{\small\verb+INT+}.  Using this, then, we can reduce half our problem
to one over the natural numbers where we can use induction.  And
frequently the other half of the problem, showing the result for the
negative numbers, follows from the first half.  The tactic which
allows us to reduce our problem to the two problems of showing the
result over the natural numbers, and showing the result for the
negatives, assuming the result for the positives, is
{\small\verb+INT_CASES_TAC+}.
\begin{session}
\begin{verbatim}
#expand INT_CASES_TAC;;
OK..
2 subgoals
"!n2 p. H p ==> H((neg(INT n2)) times p)"
    [ "SUBGROUP((\N. T),$plus)H" ]
\end{verbatim}
\mvdots
\begin{verbatim}
    [ "!n1 p. H p ==> H((INT n1) times p)" ]

"!n1 p. H p ==> H((INT n1) times p)"
    [ "SUBGROUP((\N. T),$plus)H" ]
\end{verbatim}
\evdots
\end{session}

In the subgoals returned, {\small\tt n1} and {\small\tt n2} are both
natural numbers.  For the positive case, we want to induct on
{\small\tt n1}.  (Remember to label the cases in your work so that you
will be able to put everything together when you are done.) 
\begin{session}
\begin{verbatim}
#expand INDUCT_TAC;;
OK..
2 subgoals
"!p. H p ==> H((INT(SUC n1)) times p)"
    [ "SUBGROUP((\N. T),$plus)H" ]
\end{verbatim}
\mvdots
\begin{verbatim}
    [ "!p. H p ==> H((INT n1) times p)" ]

"!p. H p ==> H((INT 0) times p)"
    [ "SUBGROUP((\N. T),$plus)H" ]
\end{verbatim}
\evdots
\end{session}

First, for the {\small\tt 0} case (that is, the case where {\small\tt n1}
is the natural number {\small\tt 0}).  By {\small\verb+TIMES_ZERO+}, we
can rewrite {\small\verb+((INT 0) times p)+} to {\small\verb+(INT 0)+}.
From there the conclusion follows from {\small\verb+INT_SBGP_ZERO+}.
(The hypothesis that {\small\tt p} is in {\small\tt H} is irrelevant.)
\begin{session}
\begin{verbatim}
#expand ((REPEAT STRIP_TAC) THEN
#  (REWRITE_TAC[TIMES_ZERO;(UNDISCH (SPEC_ALL INT_SBGP_ZERO))]));;
Theorem INT_SBGP_ZERO autoloaded from theory `int_sbgp`.
\end{verbatim}
\mvdots
\begin{verbatim}
OK..
goal proved
. |- !p. H p ==> H((INT 0) times p)

Previous subproof:
"!p. H p ==> H((INT(SUC n1)) times p)"
\end{verbatim}
\evdots
\end{session}

Next, for the inductive step of the induction.  We want to show that
{\small\verb+((INT(SUC n1)) times p)+} is in {\small\tt H}, given that
{\small\verb+((INT n1) times p)+} is.  First, let us rework the goal
so that it more precisely says this. 
\begin{session}
\begin{verbatim}
#expand (GEN_TAC THEN DISCH_TAC THEN RES_TAC);;
OK..
"H((INT(SUC n1)) times p)"
    [ "SUBGROUP((\N. T),$plus)H" ]
\end{verbatim}
\mvdots
\begin{verbatim}
    [ "H p" ]
    [ "H((INT n1) times p)" ]
    [ "(\N. T)((INT n1) times p)" ]
    [ "(\N. T)p" ]

() : void
\end{verbatim}
\end{session}

The next thing we want to do is convert {\small\verb+(INT(SUC n1))+}
into an addition in the integers.  (You will probably want to use the
theorem {\small\verb+ADD1+} from the theory {\small\verb+arithmetic+}.)
\begin{session}
\begin{verbatim}
#expand (PURE_REWRITE_TAC[ADD1;(SYM (SPEC_ALL NUM_ADD_IS_INT_ADD))]);;
Theorem NUM_ADD_IS_INT_ADD autoloaded from theory `integer`.
\end{verbatim}
\mvdots
\begin{verbatim}
OK..
"H(((INT n1) plus (INT 1)) times p)"
\end{verbatim}
\evdots
\end{session}

Next we would like to distribute the multiplication by {\small\tt p} over
this sum and simplify the result.  This should give us the term
{\small\verb+(((INT n1) times p) plus p)+}.
\begin{session}
\begin{verbatim}
#expand (PURE_REWRITE_TAC[RIGHT_PLUS_DISTRIB;TIMES_IDENTITY]);;
Theorem TIMES_IDENTITY autoloaded from theory `integer`.
\end{verbatim}
\mvdots
\begin{verbatim}
OK..
"H(((INT n1) times p) plus p)"
\end{verbatim}
\evdots
\end{session}

Now, we want to show that {\small\verb+(((INT n1) times p) plus p)+}
is in {\small\tt H}.  But we already have that
{\small\verb+((INT n1) times p)+} is in {\small\tt H} and that
{\small\tt p} is in {\small\tt H}.  Therefore, this is just 
the kind of problem that {\small\verb+GROUP_ELT_TAC+} is good at. 
\begin{session}
\begin{verbatim}
#expand GROUP_ELT_TAC;;
OK..
goal proved
... |- H(((INT n1) times p) plus p)
... |- H(((INT n1) plus (INT 1)) times p)
... |- H((INT(SUC n1)) times p)
... |- !p. H p ==> H((INT(SUC n1)) times p)
... |- !n1 p. H p ==> H((INT n1) times p)

Previous subproof:
"!n2 p. H p ==> H((neg(INT n2)) times p)"
\end{verbatim}
\evdots
\end{session}
This finishes the induction which finishes the positive case, and
leaves us dealing with the negative case.

To be able to use the positive case to help with the negative case, we
need to reduce the problem of showing that
{\small\verb+((neg(INT n2)) times p)+} is in {\small\tt H} to the
problem of showing that {\small\verb+((INT n2) times p)+} is in
{\small\tt H}, so that the positive case will apply.  Now, the theorem
{\small\verb+TIMES_neg+} will allow us to rewrite
{\small\verb+((neg(INT n2)) times p)+} to
{\small\verb+(neg((INT n2) times p))+}, and the theorem
{\small\verb+INT_SBGP_neg+}, which we proved earlier, allows us to
reduce showing that {\small\verb+neg N+} is in {\small\tt H} to
showing that {\small\tt N} is in {\small\tt H}, for any integer 
{\small\tt N}.
\begin{session}
\begin{verbatim}
#expand ((REPEAT STRIP_TAC) THEN
#  (PURE_ONCE_REWRITE_TAC[TIMES_neg]) THEN
#  (MATCH_MP_IMP_TAC (UNDISCH (SPEC_ALL INT_SBGP_neg))));;
Theorem INT_SBGP_neg autoloaded from theory `int_sbgp`.
\end{verbatim}
\mvdots
\begin{verbatim}
OK..
"H((INT n2) times p)"
\end{\verbatim}
\mvdots
\begin{verbatim}
    [ "!n1 p. H p ==> H((INT n1) times p)" ]
    [ "H p" ]

() : void
\end{verbatim}
\end{session}

Now we have a goal that matches the consequent of the assumption for
the positive case.  So let's use the positive case to finish off our
goal.  We can access the assumption by using ASSUME.
\begin{session}
\begin{verbatim}
#expand (NEW_MATCH_ACCEPT_TAC (UNDISCH (SPEC_ALL
#   (ASSUME "!n1 p. H p ==> H((INT n1) times p)"))));;
OK..
goal proved
.. |- H((INT n2) times p)
.. |- !n2 p. H p ==> H((neg(INT n2)) times p)
... |- !m p. H p ==> H(m times p)
. |- !m p. H p ==> H(m times p)
|- !H. SUBGROUP((\N. T),$plus)H ==> (!m p. H p ==> H(m times p))

Previous subproof:
goal proved
() : void
\end{verbatim}
\end{session}

This finishes off the negative case, which finishes off the theorem.
The composition of the work above is
\begin{session}
\begin{verbatim}
#let INT_SBGP_TIMES_CLOSED = prove_thm(`INT_SBGP_TIMES_CLOSED`,
#"!H. SUBGROUP((\N.T),$plus)H ==> !m p. H p ==> H (m times p)",
#(GEN_TAC THEN DISCH_TAC THEN
# (FIRST_ASSUM \thm.(STRIP_ASSUME_TAC
#   (PURE_ONCE_REWRITE_RULE[SUBGROUP_DEF] thm))) THEN
# INT_CASES_TAC THENL
# [(INDUCT_TAC THENL
#   [((REPEAT STRIP_TAC) THEN
#     (REWRITE_TAC[TIMES_ZERO;(UNDISCH (SPEC_ALL INT_SBGP_ZERO))]));
#    (GEN_TAC THEN DISCH_TAC THEN RES_TAC THEN
#     (PURE_REWRITE_TAC[ADD1;(SYM (SPEC_ALL NUM_ADD_IS_INT_ADD))]) THEN
#     (PURE_REWRITE_TAC[RIGHT_PLUS_DISTRIB;TIMES_IDENTITY]) THEN
#     GROUP_ELT_TAC)]);
#  ((REPEAT STRIP_TAC) THEN
#   (PURE_ONCE_REWRITE_TAC[TIMES_neg]) THEN
#   (MATCH_MP_IMP_TAC (UNDISCH (SPEC_ALL INT_SBGP_neg))) THEN
#   (NEW_MATCH_ACCEPT_TAC (UNDISCH (SPEC_ALL
#     (ASSUME "!n1 p. H p ==> H((INT n1) times p)")))))]));;
INT_SBGP_TIMES_CLOSED = 
|- !H. SUBGROUP((\N. T),$plus)H ==> (!m p. H p ==> H(m times p))
\end{verbatim}
\end{session}

Now let us move on to proving that every subgroup of the integers is
cyclic.  By cyclic we mean that the subgroup is generated by one
element which is repeatedly added to itself.  That is, every element
in the subgroup is a multiple of one fixed element.  We already have a
name for the set integers which are the multiples of one fixed one,
say {\small\tt n}.  That name is {\small\verb+(int_mult_set n)+}.  We
are asking to show that, if {\small\tt H} is a subgroup of the
integers, then there exists an integer {\small\tt n} such that
{\small\verb+H = (int_mult_set n)+}.  But we can show a little bit
more than this.  We can show this where {\small\tt n} is a
non-negative integer.  We can phrase this using the natural numbers.
\begin{session}
\begin{verbatim}
#set_goal([],"!H. SUBGROUP((\N.T),$plus)H ==>
#              ? n.(H = int_mult_set (INT n))");;
"!H. SUBGROUP((\N. T),$plus)H ==> (?n. H = int_mult_set(INT n))"

() : void
\end{verbatim}
\end{session}

To begin with, we will be needing as assumptions both that {\small\tt H} is
a subgroup of the integers and that it is a group in its own right.
Therefore, we may as well begin as we did for the previous theorem.
\begin{session}
\begin{verbatim}
#expand (GEN_TAC THEN DISCH_TAC THEN
#  (FIRST_ASSUM
#   \thm.(STRIP_ASSUME_TAC
#     (PURE_ONCE_REWRITE_RULE[SUBGROUP_DEF] thm))));;
OK..
"?n. H = int_mult_set(INT n)"
    [ "SUBGROUP((\N. T),$plus)H" ]
    [ "GROUP((\N. T),$plus)" ]
    [ "!x. H x ==> (\N. T)x" ]
    [ "GROUP(H,$plus)" ]

() : void
\end{verbatim}
\end{session}
Now, we are faced with proving the existence of a natural number that
will generate {\small\tt H}.  Where are we to get it from?   Well,
there are really two cases, one where {\small\tt H} is just the
singleton {\small\verb+(INT 0)+}, and there other where {\small\tt H}
contains more (and hence contains infinitely many positive and
negative integers).  In the first case our generator is
{\small\verb+(INT 0)+}.  In the  second case, the generator will be
the least positive element of {\small\tt H}.  The cases are different,
since in the first case {\small\tt H} does not contain a least
positive element.  Let us proceed by first expanding out the
definition of {\small\verb+int_mult_set+}, and then assuming either
that the only thing in {\small\tt H} is {\small\verb+(INT 0)+}, or that
{\small\verb+(INT 0)+} isn't the only thing in {\small\tt H}.
\begin{session}
\begin{verbatim}
#expand ((PURE_ONCE_REWRITE_TAC [INT_MULT_SET_DEF]) THEN
# (ASM_CASES_TAC "!m1. (H m1) ==> (m1 = (INT 0))"));;
Definition INT_MULT_SET_DEF autoloaded from theory `int_sbgp`.
INT_MULT_SET_DEF = |- !n. int_mult_set n = (\m. ?p. m = p times n)

OK..
2 subgoals
"?n. H = (\m. ?p. m = p times (INT n))"
\end{verbatim}
\mvdots
\begin{verbatim}
    [ "~(!m1. H m1 ==> (m1 = INT 0))" ]

"?n. H = (\m. ?p. m = p times (INT n))"
\end{verbatim}
\mvdots
\begin{verbatim}
    [ "!m1. H m1 ==> (m1 = INT 0)" ]

() : void
\end{verbatim}
\end{session}

I chose to use an implication in the case split instead of an
equality or logical equivalence, because its negation will be a bit
easier to deal with in the second case.

For the first case, that {\small\tt H} consists only of
{\small\verb+(INT 0)+}.  In this case the natural number we need is
{\small\tt 0}. 
\begin{session}
\begin{verbatim}
#expand (EXISTS_TAC "0");;
OK..
"H = (\m. ?p. m = p times (INT 0))"
\end{verbatim}
\mvdots
\begin{verbatim}
    [ "!m1. H m1 ==> (m1 = INT 0)" ]

() : void
\end{verbatim}
\end{session}

To show that these two sets (functions) are the same, we want to show
they contain exactly the same elements; i.e. they behave the same on
all elements.
\begin{session}
\begin{verbatim}
#expand ((EXT_TAC "m1:integer") THEN BETA_TAC);;
OK..
"!m1. H m1 = (?p. m1 = p times (INT 0))"
\end{verbatim}
\mvdots
\begin{verbatim}
    [ "!m1. H m1 ==> (m1 = INT 0)" ]

() : void
\end{verbatim}
\evdots
\end{session}

Next we would like to reduce the {\small\verb+(p times (INT 0))+} to
simply {\small\verb+(INT 0)+}.  If we use a {\small\verb+REWRITE_TAC+}
instead of a {\small\verb+PURE_REWRITE_TAC+}, it will also remove the
then superfluous existential quantification of {\small\tt p}.

\begin{session}
\begin{verbatim}
#expand (REWRITE_TAC [TIMES_ZERO]);;
OK..
"!m1. H m1 = (m1 = INT 0)"
\end{verbatim}
\mvdots
\begin{verbatim}
    [ "!m1. H m1 ==> (m1 = INT 0)" ]

() : void
\end{verbatim}
\end{session}

To show this resultant goal, we really want to consider two implications:
\begin{verbatim}
   "!m1. H m1 ==> (m1 = INT 0)"
\end{verbatim}
and
\begin{verbatim}
   "!m1.(m1 = INT 0) ==> H m1"
\end{verbatim}
Therefore, let us spilt up the equality.  (In essence, we are showing
that two sets are the same by showing that every element that is in
one is in the other, and vice versa.)
\begin{session}
\begin{verbatim}
#expand (GEN_TAC THEN EQ_TAC);;
OK..
2 subgoals
"(m1 = INT 0) ==> H m1"
\end{verbatim}
\mvdots
\begin{verbatim}
    [ "!m1. H m1 ==> (m1 = INT 0)" ]

"H m1 ==> (m1 = INT 0)"
\end{verbatim}
\mvdots
\begin{verbatim}
    [ "!m1. H m1 ==> (m1 = INT 0)" ]

() : void
\end{verbatim}
\end{session}

The first implication is what we have in the assumptions.
\begin{session}
\begin{verbatim}
#expand (FIRST_ASSUM (\thm.(ACCEPT_TAC (SPEC_ALL thm))));;
OK..
goal proved
. |- H m1 ==> (m1 = INT 0)

Previous subproof:
"(m1 = INT 0) ==> H m1"
    [ "SUBGROUP((\N. T),$plus)H" ]
\end{verbatim}
\mvdots
\begin{verbatim}
    [ "!m1. H m1 ==> (m1 = INT 0)" ]

() : void
\end{verbatim}
\end{session}
(Yes, we could have done {\small\verb+ASM_REWRTIE_TAC+} or a number of other
things.) 

If we move the antecedent of the implication in this goal over to the
assumptions, and then rewrite with the assumptions, the resulting goal,
{\small\verb+H (INT 0)+}, is given to us by {\small\verb+INT_SBGP_ZERO+}.

\newpage % PBHACK

\begin{session}
\begin{verbatim}
#expand (DISCH_TAC THEN
#  (ASM_REWRITE_TAC [(UNDISCH (SPEC_ALL INT_SBGP_ZERO))]));;
OK..
goal proved
. |- (m1 = INT 0) ==> H m1
.. |- !m1. H m1 = (m1 = INT 0)
.. |- !m1. H m1 = (?p. m1 = p times (INT 0))
.. |- H = (\m. ?p. m = p times (INT 0))
.. |- ?n. H = (\m. ?p. m = p times (INT n))

Previous subproof:
"?n. H = (\m. ?p. m = p times (INT n))"
    [ "SUBGROUP((\N. T),$plus)H" ]
    [ "GROUP((\N. T),$plus)" ]
    [ "!x. H x ==> (\N. T)x" ]
    [ "GROUP(H,$plus)" ]
    [ "~(!m1. H m1 ==> (m1 = INT 0))" ]

() : void
\end{verbatim}
\end{session}

Now on to the more substantial case, where {\small\tt H} contains more
than just {\small\verb+(INT 0)+}.  We are still confronted with showing
that a generator exists.  In this case, we said the generator would be
the minimum positive element in {\small\tt H}.  For this to make any
sense we are going need to have that such a minimum element exists.  If
we had that the set of positive elements in {\small\tt H} was non-empty
and bounded below, then by the theorem {\small\verb+DISCRETE+} we would
have that there exists a minimum positive element of {\small\tt H}.  We
could then use {\small\verb+STRIP_ASSUME_TAC+} to introduce a name for
this minimum positive element, and to add to the assumptions the
consequences of its being a minimum positive element of {\small\tt H}.
To make all this a bit easier, there exists a tactic
{\small\verb+INT_MIN_TAC+}, which when given a set predicate, say
{\small\verb+S:integer -> bool+}, takes a goal, introduces a proposed
minimum element, {\small\verb+MIN+}, for {\small\tt S}, and returns
the goal with the added assumptions that {\small\verb+MIN+} is in
{\small\tt S}, and that for every {\small\tt N}, if {\small\tt N} is
below {\small\verb+MIN+}, then {\small\tt N} is not in {\small\tt S},
as well as returning the subgoals of showing that {\small\tt S} is
nonempty and showing that {\small\tt S} is bounded below.  In our
case, we would like to use this tactic with the predicate describing
the positive elements in {\small\tt H}.

\newpage % PBHACK

\begin{session}
\begin{verbatim}
#expand (INT_MIN_TAC "\N. (POS N /\ H N)");;
OK..
3 subgoals
"?LB. !N. N below LB ==> ~(POS N /\ H N)"
 \end{verbatim}
\mvdots
\begin{verbatim}
    [ "~(!m1. H m1 ==> (m1 = INT 0))" ]
    [ "POS M /\ H M" ]

"?M. POS M /\ H M"
\end{verbatim}
\mvdots
\begin{verbatim}
    [ "~(!m1. H m1 ==> (m1 = INT 0))" ]

"?n. H = (\m. ?p. m = p times (INT n))"
\end{verbatim}
\mvdots
\begin{verbatim}
    [ "~(!m1. H m1 ==> (m1 = INT 0))" ]
    [ "!N. N below MIN ==> ~(POS N /\ H N)" ]
    [ "POS MIN /\ H MIN" ]

() : void
\end{verbatim}
\end{session}

The assumption {\small\verb+POS MIN /\ H MIN+} will be more useful to
us as separate assumptions, so let us replace it by each of its conjuncts.
\begin{session}
\begin{verbatim}
#expand (POP_ASSUM STRIP_ASSUME_TAC);;
OK..
"?n. H = (\m. ?p. m = p times (INT n))"
\end{verbatim}
\mvdots
\begin{verbatim}
    [ "!N. N below MIN ==> ~(POS N /\ H N)" ]
    [ "POS MIN" ]
    [ "H MIN" ]

() : void
\end{verbatim}
\end{session}

Now, we want to use {\small\verb+MIN+} to remove the existential
quantifier from our goal.  Unfortunately, the types don't match.  Our
goal needs a natural number, but {\small\verb+MIN+} is an integer.
However, {\small\verb+MIN+} is a positive integer.  Therefore, there
will be a natural number that corresponds to it.  For the time being,
let us suppose that one such exists.  To allow us to make such
suppositions in the middle of proving a theorem, the tactic
{\small\verb+SUPPOSE_TAC+} allows us to add an assumption to our goal,
at the price of having to prove it at the end.  There is also the
tactic {\small\verb+REV_SUPPOSE_TAC+} that will allow you to add an
assumption to the goal, but requires you to prove it first and then
returns you to the original goal after.  (The effect of 
{\small\verb+REV_SUPPOSE_TAC+} could be achieved by using
{\small\verb+SUPPOSE_TAC+} and then doing a {\small\verb+rotate 1;;+}
to switch the goal stack around.)  So, for the time-being, let's use
{\small\verb+SUPPOSE_TAC+} to assume that such a natural number
exists, and show that this gets us the theorem.  Then we will come
back and prove it exists afterwards.  That is, let's use
{\small\verb+SUPPOSE_TAC+} to add {\small\verb+?n. (INT n) = MI+}
as an assumption to our goal.
\begin{session}
\begin{verbatim}
#expand (SUPPOSE_TAC "?n. (INT n) = MIN");;
OK..
2 subgoals
"?n. INT n = MIN"
\end{verbatim}
\mvdots
\begin{verbatim}
    [ "POS MIN" ]
    [ "H MIN" ]

"?n. H = (\m. ?p. m = p times (INT n))"
\end{verbatim}
\mvdots
\begin{verbatim}
    [ "POS MIN" ]
    [ "H MIN" ]
    [ "?n. INT n = MIN" ]

() : void
\end{verbatim}
\end{session}

If we now pop the assumption that there exists a natural number
corresponding to the integer {\small\verb+MIN+} and pass it through
{\small\verb+STRIP_ASSUME_TAC+}, we will introduce {\small\tt n} as a
name for this natural number which we can use to solve our existence
problem, and at the same acquire and equation that we will want to
rewrite with shortly. 
\begin{session}
\begin{verbatim}
#expand (POP_ASSUM STRIP_ASSUME_TAC);;
OK..
"?n. H = (\m. ?p. m = p times (INT n))"
\end{verbatim}
\mvdots
\begin{verbatim}
    [ "POS MIN" ]
    [ "H MIN" ]
    [ "INT n = MIN" ]

() : void
\end{verbatim}
\end{session}

Now we are in a position to eliminate the existential quantifier.  The
{\small\tt n} we want to use is the {\small\tt n} just introduced by
{\small\verb+STRIP_ASSUME_TAC+}, the one that corresponds to
{\small\verb+MIN+}.
\begin{session}
\begin{verbatim}
#expand (EXISTS_TAC "n:num");;
OK..
"H = (\m. ?p. m = p times (INT n))"
\end{verbatim}
\mvdots
\begin{verbatim}
    [ "H MIN" ]
    [ "INT n = MIN" ]

() : void
\end{verbatim}
\end{session}

Since {\small\verb+MIN+} is the value that we really want to be
reasoning about, we would like to replace {\small\verb+(INT n)+} by
{\small\verb+MIN+}.  At the same time, let's remove the assumption
{\small\verb+INT n = MIN+}, for this is the only use we shall have for
it and the list of assumptions will be long enough as is. 
\begin{session}
\begin{verbatim}
#expand (POP_ASSUM \thm. PURE_ONCE_REWRITE_TAC [thm]);;
OK..
"H = (\m. ?p. m = p times MIN)"
\end{verbatim}
\mvdots
\begin{verbatim}
    [ "H MIN" ]

() : void
\end{verbatim}
\end{session}

Now we are faced with trying to prove that two set are equal.  As
usual, we want to reduce this to an elementwise argument.  Also, in
general, when trying to show that two sets are equal, this is done by 
showing two implications: if something is in one then it is in the
other, and vice versa.  That is how we will proceed in this case.
\begin{session}
\begin{verbatim}
#expand ((EXT_TAC "m:integer") THEN BETA_TAC THEN
#  GEN_TAC THEN EQ_TAC);;  
OK..
2 subgoals
"(?p. m = p times MIN) ==> H m"
\end{verbatim}
\mvdots
\begin{verbatim}
    [ "H MIN" ]

"H m ==> (?p. m = p times MIN)"
\end{verbatim}
\mvdots
\begin{verbatim}
    [ "H MIN" ]

() : void
\end{verbatim}
\end{session}

Now we are confronted with two goals: to show that if something is in
{\small\tt H}, then it is a multiple of {\small\verb+MIN+}, and to
show that if something is a multiple of {\small\verb+MIN+}, then it is
in {\small\tt H}.  (You might recognize that the second goal is
closely related to the first theorem we proved in this section.)  For
the first goal, let's break it up into the cases where {\small\tt m}
is positive, where it is negative, and where it is zero.  There is a
tactic, similar to {\small\verb+INT_CASES_TAC+}, for doing just this.
The tactic is {\small\verb+SIMPLE_INT_CASES_TAC+}, and like
{\small\verb+INT_CASES_TAC+}, it must be given a universally quantified
goal. 
\begin{session}
\begin{verbatim}
#expand ((SPEC_TAC ("m:integer","N:integer")) THEN SIMPLE_INT_CASES_TAC);;
OK..
3 subgoals
"H(INT 0) ==> (?p. INT 0 = p times MIN)"
\end{verbatim}
\mvdots
\begin{verbatim}
    [ "H MIN" ]

"!N. NEG N ==> H N ==> (?p. N = p times MIN)"
\end{verbatim}
\mvdots
\begin{verbatim}
    [ "H MIN" ]
    [ "!N. POS N ==> H N ==> (?p. N = p times MIN)" ]

"!N. POS N ==> H N ==> (?p. N = p times MIN)"
\end{verbatim}
\mvdots
\begin{verbatim}
    [ "H MIN" ]

() : void
\end{verbatim}
\end{session}

So what do we do in the positive case?  First let's reduce to the
existence problem.
\begin{session}
\begin{verbatim}
#expand (REPEAT STRIP_TAC);;
OK..
"?p. N = p times MIN"
\end{verbatim}
\mvdots
\begin{verbatim}
    [ "H MIN" ]
    [ "POS N" ]
    [ "H N" ]

() : void
\end{verbatim}
\end{session}

What {\small\tt p} are we looking for?  Well, we want the number of
times {\small\verb+MIN+} divides into {\small\tt N}.  We want the
maximum integer that we can multiply times {\small\verb+MIN+} and get
a value that, when it is subtracted from {\small\tt N}, is not negative.
Just as there is {\small\verb+INT_MIN_TAC+} to help us access minimum
values of sets of integers, there is also the tactic
{\small\verb+INT_MAX_TAC+} to help us access maximum values of sets.
The set whose maximum we want is the set of integers {\small\tt X}
such that {\small\verb+(N minus (X times MIN))+} is not negative. 
\begin{session}
\begin{verbatim}
#expand (INT_MAX_TAC "\X.~(NEG(N minus (X times MIN)))");;
OK..
3 subgoals
"?UB. !N'. UB below N' ==> ~~NEG(N minus (N' times MIN))"
\end{verbatim}
\mvdots
\begin{verbatim}
    [ "H N" ]
    [ "~NEG(N minus (M times MIN))" ]

"?M. ~NEG(N minus (M times MIN))"
\end{verbatim}
\mvdots
\begin{verbatim}
    [ "H N" ]

"?p. N = p times MIN"
\end{verbatim}
\mvdots
\begin{verbatim}
    [ "H N" ]
    [ "!N'. MAX below N' ==> ~~NEG(N minus (N' times MIN))" ]
    [ "~NEG(N minus (MAX times MIN))" ]

() : void
\end{verbatim}
\end{session}

We can eliminate the existential quantifier now.
\begin{session}
\begin{verbatim}
#expand (EXISTS_TAC "MAX:integer");;
OK..
"N = MAX times MIN"
\end{verbatim}
\evdots
\end{session}

To show that {\small\verb+N = (MAX times MIN)+}, it suffices to show
that {\small\verb+(N minus (MAX times MIN)) = (INT 0)+}.  This
requires (backing up to) adding {\small\verb+(neg (MAX times MIN))+}
to both sides.  (You'll probably want to use a partially specialize
version of a theorem about {\small\verb+plus+} in the integers
together with {\small\verb+MATCH_MP_IMP_TAC+}). 
\begin{session}
\begin{verbatim}
#expand ((MATCH_MP_IMP_TAC
#          (SPEC "neg (MAX times MIN)" PLUS_RIGHT_CANCELLATION)) THEN
#  (PURE_REWRITE_TAC [PLUS_INV_LEMMA;(SYM (SPEC_ALL MINUS_DEF))]));;
\end{verbatim}
\mvdots
\begin{verbatim}
OK..
"N minus (MAX times MIN) = INT 0"
\end{verbatim}
\mvdots
\begin{verbatim}
    [ "~NEG(N minus (MAX times MIN))" ]

() : void
\end{verbatim}
\end{session}

To show that this value is {\small\verb+(INT 0)+}, by
{\small\verb+TRICHOTOMY+} it should suffice to show that it is not
negative (which we have among our assumptions) and that it is not
positive.  That is, we want to use {\small\verb+DISJ_CASES_TAC+}
together with the first conjunct of {\small\verb+TRICHOTOMY+}
specialized to this value. 
\begin{session}
\begin{verbatim}
#expand (DISJ_CASES_TAC (CONJUNCT1
#   (SPEC "N minus (MAX times MIN)" TRICHOTOMY)));;
\end{verbatim}
\mvdots
\begin{verbatim}
OK..
2 subgoals
"N minus (MAX times MIN) = INT 0"
\end{verbatim}
\mvdots
\begin{verbatim}
    [ "~NEG(N minus (MAX times MIN))" ]
    [ "NEG(N minus (MAX times MIN)) \/ (N minus (MAX times MIN) = INT 0)" ]

"N minus (MAX times MIN) = INT 0"
\end{verbatim}
\mvdots
\begin{verbatim}
    [ "!N. N below MIN ==> ~(POS N /\ H N)" ]
\end{verbatim}
\mvdots
\begin{verbatim}
    [ "~NEG(N minus (MAX times MIN))" ]
    [ "POS(N minus (MAX times MIN))" ]

() : void
\end{verbatim}
\end{session}

In the case where we assume that our value is positive, we need to
arrive at a contradiction.  So where is this contradiction going to
come from?  If we knew that this value were less than
{\small\verb+MIN+} and that it were in {\small\tt H}, then by one of
our hypotheses, we would have that it was not positive.  So let's
suppose each of these in turn and then get our contradiction (using
{\small\verb+RES_TAC THEN RES_TAC+}).  (Why twice?)
\begin{session}
\begin{verbatim}
#expand (SUPPOSE_TAC "(N minus (MAX times MIN)) below MIN");;
OK..
2 subgoals
"(N minus (MAX times MIN)) below MIN"
\end{verbatim}
\mvdots
\begin{verbatim}
    [ "POS(N minus (MAX times MIN))" ]

"N minus (MAX times MIN) = INT 0"
\end{verbatim}
\mvdots
\begin{verbatim}
    [ "!N. N below MIN ==> ~(POS N /\ H N)" ]
\end{verbatim}
\mvdots
\begin{verbatim}
    [ "POS(N minus (MAX times MIN))" ]
    [ "(N minus (MAX times MIN)) below MIN" ]

() : void
\end{verbatim}
\end{session}
\begin{session}
\begin{verbatim}
#expand (SUPPOSE_TAC "(H (N minus (MAX times MIN))):bool");;
OK..
2 subgoals
"H(N minus (MAX times MIN))"
\end{verbatim}
\mvdots
\begin{verbatim}
    [ "(N minus (MAX times MIN)) below MIN" ]

"N minus (MAX times MIN) = INT 0"
\end{verbatim}
\mvdots
\begin{verbatim}
    [ "!N. N below MIN ==> ~(POS N /\ H N)" ]
\end{verbatim}
\mvdots
\begin{verbatim}
    [ "POS(N minus (MAX times MIN))" ]
    [ "(N minus (MAX times MIN)) below MIN" ]
    [ "H(N minus (MAX times MIN))" ]

() : void
\end{verbatim}
\end{session}
\begin{session}
\begin{verbatim}
#expand (RES_TAC THEN RES_TAC);;
OK..
goal proved
....... |- N minus (MAX times MIN) = INT 0

Previous subproof:
"H(N minus (MAX times MIN))"
\end{verbatim}
\mvdots
\begin{verbatim}
    [ "H N" ]
\end{verbatim}
\mvdots
\begin{verbatim}
    [ "(N minus (MAX times MIN)) below MIN" ]

() : void
\end{verbatim}
\end{session}

Now we begin to pay the piper.  We have made a number of suppositions
as we have gone along and this is the first of them to appear needing
to be proved.  It's the last assumption we made.  Now, we are trying
to show that {\small\verb+N minus (MAX times MIN))+} is in {\small\tt H}.
Since we have that {\small\tt N} is in {\small\tt H}, it would be
beneficial to rewrite {\small\verb+minus+} to {\small\verb+plus neg+}.
\begin{session}
\begin{verbatim}
#expand (PURE_ONCE_REWRITE_TAC [MINUS_DEF]);;
OK..
"H(N plus (neg(MAX times MIN)))"
    [ "SUBGROUP((\N. T),$plus)H" ]
\end{verbatim}
\mvdots
\begin{verbatim}
    [ "H MIN" ]
\end{verbatim}
\mvdots
\begin{verbatim}
    [ "H N" ]
\end{verbatim}
\mvdots
\begin{verbatim}
    [ "(N minus (MAX times MIN)) below MIN" ]

() : void
\end{verbatim}
\end{session}

At this point in time, if we were to use {\small\verb+GROUP_ELT_TAC+}
on this goal, it would reduce it to showing {\small\tt N} is in
{\small\tt H}, which it would then finish off, and to showing that
{\small\verb+(neg(MAX times MIN))+} is in {\small\tt H}.  We could
then use {\small\verb+INT_SBGP_neg+} to reduce the goal of showing
that {\small\verb+MAX times MIN+} is in {\small\tt H}, followed by
using {\small\verb+INT_SBGP_TIMES_CLOSED+} to reduce it further to the
goal of showing that {\small\verb+MIN+} is in {\small\tt H}, which is
one of our assumptions.  There is a tactic that is a bit more general
than {\small\verb+GROUP_ELT_TAC+} which can do all of this for us, if
given the theorems {\small\verb+INT_SBGP_neg+} and
{\small\verb+INT_SGBP_TIMES_CLOSED+}.  That tactic is
{\small\verb+GROUP_TAC+}, and it takes a list of theorems to use to
help reduce group membership goals. 
\begin{session}
\begin{verbatim}
#expand (GROUP_TAC [INT_SBGP_neg;INT_SBGP_TIMES_CLOSED]);;
OK..
goal proved
.... |- H(N plus (neg(MAX times MIN)))
.... |- H(N minus (MAX times MIN))
........ |- N minus (MAX times MIN) = INT 0

Previous subproof:
"(N minus (MAX times MIN)) below MIN"
\end{verbatim}
\mvdots
\begin{verbatim}
    [ "!N'. MAX below N' ==> ~~NEG(N minus (N' times MIN))" ]
\end{verbatim}
\evdots
\end{session}

This puts us back at having to show that
{\small\verb+N minus (MAX times MIN)+} is less than
{\small\verb+MIN+}.  Why should this be so anyway?  Well, 
{\small\verb+(N minus (MAX times MIN))+} must be below
{\small\verb+MIN+} because {\small\verb+(N minus ((MAX plus (INT 1))
times MIN))+} must be negative (or rather not not negative), since
{\small\verb+MAX+} is less than {\small\verb+(MAX plus (INT 1))+}.
And we have as any assumption that any integer that is bigger than
{\small\verb+MAX+} has the property that {\small\tt N} minus the
product of it and {\small\verb+MIN+} is negative.  So, to get started,
let's convert the {\small\verb+below+} relating
{\small\verb+(N minus (MAX times MIN))+} to {\small\verb+MIN+} using the
definition.  

\begin{session}
\begin{verbatim}
#expand (PURE_ONCE_REWRITE_TAC [BELOW_DEF]);;
\end{verbatim}
\mvdots
\begin{verbatim}
OK..
"POS(MIN minus (N minus (MAX times MIN)))"
\end{verbatim}
\mvdots
\begin{verbatim}
    [ "!N'. MAX below N' ==> ~~NEG(N minus (N' times MIN))" ]
\end{verbatim}
\evdots
\end{session}

What we are now confronted with is that
{\small\verb+MIN minus (N minus (MAX times MIN))+} is positive,
whereas we want to be trying to show that
{\small\verb+(N minus ((MAX plus (INT 1)) times MIN))+} must be
negative.  Let us work towards this by first converting the positive
into a negative.  This will require using the fact that
\begin{verbatim}
PLUS_INV_INV_LEMMA = |- !x. neg(neg x) = x
\end{verbatim}
\begin{session}
\begin{verbatim}
#expand ((NEW_SUBST1_TAC 
#     (SYM (SPEC "MIN minus (N minus (MAX times MIN))"
#        PLUS_INV_INV_LEMMA))) THEN
#   (PURE_ONCE_REWRITE_TAC [(SYM (SPEC_ALL(NEG_DEF)))]));;
\end{verbatim}
\mvdots
\begin{verbatim}
OK..
"NEG(neg(MIN minus (N minus (MAX times MIN))))"
\end{verbatim}
\mvdots
\begin{verbatim}
    [ "!N'. MAX below N' ==> ~~NEG(N minus (N' times MIN))" ]
\end{verbatim}
\evdots
\end{session}

Now we have a {\small\verb+neg (MIN minus (N minus (MAX times MIN)))+}.  It
would take us closer to the form we want to change this into
{\small\verb+(N plus (neg (MAX times MIN))) plus (neg MIN)+}.  
\begin{session}
\begin{verbatim}
#expand (PURE_REWRITE_TAC
#    [MINUS_DEF;
#     (SYM (SPEC_ALL (PLUS_DIST_INV_LEMMA)));
#     PLUS_INV_INV_LEMMA]);;
\end{verbatim}
\mvdots
\begin{verbatim}
OK..
"NEG((N plus (neg(MAX times MIN))) plus (neg MIN))"
\end{verbatim}
\mvdots
\begin{verbatim}
    [ "!N'. MAX below N' ==> ~~NEG(N minus (N' times MIN))" ]
\end{verbatim}
\evdots
\end{session}

The next thing we would like to do is regroup and incorporate the
{\small\verb+MIN+} in with the {\small\verb+MAX times MIN+} to get
{\small\verb+(N minus ((MAX times MIN) plus MIN)))+}.  Because
reassociating terms is something that one often has to do when pushing
them into a more desirable form, there are tactics to help make this
task a bit easier.  The tactics {\small\verb+INT_RIGHT_ASSOC_TAC+} and
{\small\verb+INT_LEFT_ASSOC_TAC+} take as argument a term which is either
left associated for the first or right associated for the second and
rewrite the goal with the term reassociated in the opposite fashion.
This is most useful when you have several layers of addition and you
do not want them all pushed to the left or all pushed to the right.
\begin{session}
\begin{verbatim}
#expand ((INT_RIGHT_ASSOC_TAC
#     "(N plus (neg (MAX times MIN))) plus (neg MIN)") THEN
#  (PURE_REWRITE_TAC
#     [(SYM neg_PLUS_DISTRIB);(SYM (SPEC_ALL MINUS_DEF))]));;
\end{verbatim}
\mvdots
\begin{verbatim}
OK..
"NEG(N minus ((MAX times MIN) plus MIN))"
\end{verbatim}
\mvdots
\begin{verbatim}
    [ "!N'. MAX below N' ==> ~~NEG(N minus (N' times MIN))" ]
\end{verbatim}
\evdots
\end{session}

To finish with our rewriting, we would like to convert the
{\small\verb+((MAX times MIN) plus MIN)+} into
{\small\verb+((MAX plus (INT 1)) times MIN)+}.  Rewriting the
{\small\verb+(MAX times MIN) plus MIN)+} requires using distributivity,
which in turn requires that we rewrite the last {\small\verb+MIN+} into
{\small\verb+((INT 1) times MIN)+}.  This is a bit of a pain, since
{\small\verb+MIN+} occurs twice in the goal, and its only the last one
that we wish to rewrite.  Probably the easiest way to deal with this
is to to rewrite with an instantiated and rewritten version of
distributivity.  That is, let's rewrite the rewrite theorem instead of
the goal.
\begin{session}
\begin{verbatim}
#expand (PURE_ONCE_REWRITE_TAC
#    [(PURE_ONCE_REWRITE_RULE [TIMES_IDENTITY] (SYM
#      (SPECL ["MAX:integer";"INT 1";"MIN:integer"]
#        RIGHT_PLUS_DISTRIB)))]);;
OK..
"NEG(N minus ((MAX plus (INT 1)) times MIN))"
\end{verbatim}
\mvdots
\begin{verbatim}
    [ "!N'. MAX below N' ==> ~~NEG(N minus (N' times MIN))" ]
\end{verbatim}
\evdots
\end{session}

All of this rewriting that we have been doing has been to allow us to
use one of our hypotheses to reduce the goal.  The hypothesis to which
I am referring is the one that says that for all {\small\verb+N'+}, if
we have {\small\verb+MAX below N'+}, then
{\small\verb+(N minus (N' times MIN))+} is (not not) negative.  If we
take {\small\verb+N'+} to be {\small\verb+(MAX plus (INT 1))+}, then
the conclusion of this hypothesis is almost exactly our goal.  So
let's use a rewritten version of this hypothesis together with
{\small\verb+MATCH_MP_IMP_TAC+} to reduce this goal. 
\begin{session}
\begin{verbatim}
#expand (MATCH_MP_IMP_TAC (ONCE_REWRITE_RULE []
#    (ASSUME "!N'.MAX below N' ==>~~NEG (N minus (N' times MIN))")));;
OK..
"MAX below (MAX plus (INT 1))"
\end{verbatim}
\evdots
\end{session}

To show this resultant inequality, let's add {\small\verb+(neg MAX)+} to
both sides of it and then simplify the left-hand side to
{\small\verb+(INT 0)+} and the right-hand side to {\small\verb+(INT
1)+}.  We can then translate this inequality into one over the natural
numbers. 
\begin{session}
\begin{verbatim}
#expand ((SUBST_MATCH_TAC
#    (PURE_ONCE_REWRITE_RULE [COMM_PLUS]
#      (SPECL ["A:integer";"B:integer";"neg MAX"]
#        PLUS_BELOW_TRANSL))) THEN
#   (PURE_REWRITE_TAC
#       [(SYM (SPEC_ALL PLUS_GROUP_ASSOC));
#        PLUS_INV_LEMMA;
#        (SYM (SPEC_ALL NUM_LESS_IS_INT_BELOW));
#	PLUS_ID_LEMMA]));;
\end{verbatim}
\mvdots
\begin{verbatim}
OK..
"0 < 1"
\end{verbatim}
\evdots
\end{session}

To show that {\small\verb+0 < 1+}, we want to use the theorem
{\small\verb+LESS_0_0+} from the theory {\small\verb+prim_rec+}, which
says {\small\verb+|- 0 < (SUC 0)+}.  This requires converting
{\small\tt 1} into {\small\verb+(SUC 0)+}.  Probably the easiest way
to do that is with a {\small\verb+CONV_TAC (DEPTH_CONV num_CONV)+}.
\begin{session}
\begin{verbatim}
#expand ((CONV_TAC (DEPTH_CONV num_CONV)) THEN
#  (ACCEPT_TAC (theorem `prim_rec` `LESS_0_0`)));;
OK..
goal proved
|- 0 < 1
|- MAX below (MAX plus (INT 1))
. |- NEG(N minus ((MAX plus (INT 1)) times MIN))
. |- NEG(N minus ((MAX times MIN) plus MIN))
. |- NEG((N plus (neg(MAX times MIN))) plus (neg MIN))
. |- NEG(neg(MIN minus (N minus (MAX times MIN))))
. |- POS(MIN minus (N minus (MAX times MIN)))
. |- (N minus (MAX times MIN)) below MIN
........ |- N minus (MAX times MIN) = INT 0

Previous subproof:
"N minus (MAX times MIN) = INT 0"
\end{verbatim}
\mvdots
\begin{verbatim}
    [ "~NEG(N minus (MAX times MIN))" ]
    [ "NEG(N minus (MAX times MIN)) \/ (N minus (MAX times MIN) = INT 0)" ]

() : void
\end{verbatim}
\end{session}

This finishes off the assumptions we made in showing that
{\small\verb+(N minus (MAX times MIN))+} is not positive.  (Or rather,
more precisely, if it is positive then it is zero.)  Now we are left to
deal with the case where it is either negative or zero.  If we were to
split up these two cases, then {\small\verb+RES_TAC+} would take of the
negative case because it and its negation would both be assumptions.
The zero case would be taken care of by
{\small\verb+(FIRST_ASSUM ACCEPT_TAC)+}.

\newpage % PBHACK

\begin{session}
\begin{verbatim}
#expand ((POP_ASSUM DISJ_CASES_TAC) THENL
#    [RES_TAC;(FIRST_ASSUM ACCEPT_TAC)]);;
OK..
goal proved
.. |- N minus (MAX times MIN) = INT 0
........ |- N minus (MAX times MIN) = INT 0
........ |- N = MAX times MIN
........ |- ?p. N = p times MIN

Previous subproof:
2 subgoals
"?UB. !N'. UB below N' ==> ~~NEG(N minus (N' times MIN))"
\end{verbatim}
\mvdots
\begin{verbatim}

"?M. ~NEG(N minus (M times MIN))"
\end{verbatim}
\mvdots
\begin{verbatim}
    [ "POS N" ]
\end{verbatim}
\evdots
\end{session}

We have just finished showing that {\small\tt N}, an arbitrary
positive element of our subgroup {\small\tt H}, is a multiple of the
minimum positive element in {\small\tt H}, given certain assumptions:
that there exists a minimum positive element, that this minimum
positive element corresponds to a natural number, and that there
exists a maximum number of times that this minimum element can be
subtracted from {\small\tt N} leaving a non-negative result.  We now
find ourselves confronted with having to have to prove the last of
these: that there exists a maximum number of times that the minimum
positive element can be subtracted from {\small\tt N}.  (It was to be
able to prove this that we needed to consider cases where {\small\tt N}
was positive, negative, or zero.)  The tactic we used to introduce
{\small\verb+MAX+}, {\small\verb+INT_MAX_TAC+}, has already reduced
the problem of showing that {\small\verb+MAX+} exists to the subgoals
of showing that there is a multiple of {\small\verb+MIN+} which when
subtracted from {\small\tt N} gives a non-negative result (the set is
non-empty), and of showing that there is a value {\small\verb+UB+}
such that any value greater than {\small\verb+UB+}, if multiplied
times {\small\verb+MIN+} and then subtracted from {\small\tt N},
yields a negative result (the set is bounded above).  So, first we
must show that there exists some number of times that we can subtract
{\small\verb+MIN+} from {\small\tt N} and still get a non-negative
result.  For this, zero will do very nicely (since we are in the case
where {\small\tt N} is positive).  Moreover, we can then simplify this
down to the goal of showing that {\small\tt N} is not negative.

\newpage % PBHACK

\begin{session}
\begin{verbatim}
#expand ((EXISTS_TAC "INT 0") THEN 
#  (PURE_REWRITE_TAC
#    [MINUS_DEF;TIMES_ZERO;PLUS_INV_ID_LEMMA;PLUS_ID_LEMMA]));;
#    [MINUS_DEF;TIMES_ZERO;PLUS_INV_ID_LEMMA;PLUS_ID_LEMMA]));;
\end{verbatim}
\mvdots
\begin{verbatim}
OK..
"~NEG N"
\end{verbatim}
\mvdots
\begin{verbatim}
    [ "POS N" ]
\end{verbatim}
\evdots
\end{session}

We know that {\small\tt N} is not negative since we have the
assumption that it is positive, and we know from
{\small\verb+TRICHOTOMY+} that an integer is not both positive and
negative.
\begin{session}
\begin{verbatim}
#expand (ACCEPT_TAC (REWRITE_RULE [(ASSUME "POS N")]
#    (CONJUNCT1 (CONJUNCT2 (SPEC "N:integer" TRICHOTOMY)))));;
OK..
goal proved
. |- ~NEG N
. |- ?M. ~NEG(N minus (M times MIN))

Previous subproof:
"?UB. !N'. UB below N' ==> ~~NEG(N minus (N' times MIN))"
\end{verbatim}
\mvdots
\begin{verbatim}
    [ "POS MIN" ]
\end{verbatim}
\evdots
\end{session}

Next, for the second half of showing that {\small\verb+MAX+} exists,
we have to show that there is some value {\small\verb+UB+} that when
we multiply anything bigger than it times {\small\verb+MIN+} and
subtract the result from {\small\tt N}, we get a negative number.
(You might recognize this as essentially the Archemidian Principle).
So what value shall we take?  Well, since {\small\verb+MIN+} is a
positive number, about the only thing we know about it is that it is
at least as big as {\small\verb+(INT 1)+}.  (In fact, it could be
{\small\verb+(INT 1)+}.)  Therefore, to be on the safe side, the value
had better be at least {\small\tt N}.
\begin{session}
\begin{verbatim}
#expand (EXISTS_TAC "N:integer");;
OK..
"!N'. N below N' ==> ~~NEG(N minus (N' times MIN))"
\end{verbatim}
\mvdots
\begin{verbatim}
    [ "POS MIN" ]
\end{verbatim}
\evdots
\end{session}

Now, the antecedent of our current goal is {\small\verb+N below N'+},
whereas the conclusion is that {\small\verb+N minus (N' times MIN)+}
is negative (or rather not not negative).  It would be helpful if we
were to put these to into the same terms, \ie, either both in terms of
something is below something else, or something minus something else
is either positive or negative.  Let's put them both in terms of the
order relation, {\small\verb+below+} (mainly for the practice of using
facts about {\small\verb+below+}). 
\begin{session}
\begin{verbatim}
#expand ((REWRITE_TAC
#     [NEG_DEF;
#      MINUS_DEF;
#      (SYM (SPEC_ALL PLUS_DIST_INV_LEMMA));
#      PLUS_INV_INV_LEMMA]) THEN
#  (PURE_REWRITE_TAC[(SYM (SPEC_ALL MINUS_DEF));
#      (SYM (SPEC_ALL BELOW_DEF))]));;
OK..
"!N'. N below N' ==> N below (N' times MIN)"
\end{verbatim}
\mvdots
\begin{verbatim}
    [ "POS MIN" ]
\end{verbatim}
\evdots
\end{session}

We have as one of our assumptions that {\small\verb+MIN+} is positive.
By the definition of positive there exists some natural number
{\small\tt n} such that {\small\verb+MIN = INT (SUC n)+}.  Let's add
this into our assumptions and then rewrite our goal with it (with an
eye to working a {\small\tt 1} into things). 
\begin{session}
\begin{verbatim}
#expand ((STRIP_ASSUME_TAC
#    (PURE_ONCE_REWRITE_RULE [POS_DEF] (ASSUME "POS MIN"))) THEN
#  (PURE_ONCE_ASM_REWRITE_TAC[]));;
\end{verbatim}
\mvdots
\begin{verbatim}
OK..
"!N'. N below N' ==> N below (N' times (INT(SUC n)))"
\end{verbatim}
\mvdots
\begin{verbatim}
    [ "POS MIN" ]
\end{verbatim}
\mvdots
\begin{verbatim}
    [ "MIN = INT(SUC n)" ]

() : void
\end{verbatim}
\end{session}

Now, there are two possibilities to be considered here.  Either
{\small\verb+O = n+}, in which case {\small\verb+N' times (INT(SUC n))+} 
simplifies to {\small\verb+N'+} and we are done, or {\small\verb+0 < n+}.
(The theorem giving us these cases is {\small\verb+LESS_0_CASES+} from
the theory {\small\verb+arithmetic+}.) 
\begin{session}
\begin{verbatim}
#expand ((DISJ_CASES_TAC (SPEC "n:num" LESS_0_CASES)) THENL
#    [(POP_ASSUM \thm. (ASM_REWRITE_TAC
#        [(SYM thm);(SYM (num_CONV "1"));TIMES_IDENTITY]));
#     ALL_TAC]);;
\end{verbatim}
\mvdots
\begin{verbatim}
OK..
"!N'. N below N' ==> N below (N' times (INT(SUC n)))"
\end{verbatim}
\mvdots
\begin{verbatim}
    [ "MIN = INT(SUC n)" ]
    [ "0 < n" ]

() : void
\end{verbatim}
\end{session}

For the case of {\small\verb+0 < n+}, by transitivity, given the
assumption {\small\verb+N below N'+}, it suffices to show
{\small\verb+N' below (N' times (INT(SUC n)))+}.
\begin{session}
\begin{verbatim}
#expand ((REPEAT STRIP_TAC) THEN
#  (MP_IMP_TAC
#    (SPECL ["N:integer";"N':integer";"N' times (INT(SUC n))"]
#     TRANSIT)) THEN
#  (ASM_REWRITE_TAC []));;
\end{verbatim}
\mvdots
\begin{verbatim}
OK..
"N' below (N' times (INT(SUC n)))"
\end{verbatim}
\mvdots
\begin{verbatim}
    [ "MIN = INT(SUC n)" ]
    [ "0 < n" ]
    [ "N below N'" ]

() : void
\end{verbatim}
\end{session}

The next thing to do is to convert {\small\verb+(N' times (INT(SUC n)))+}
into {\small\verb+(N' times (INT n)) plus N'+} and then reduce the
problem to showing {\small\verb+(INT 0) below (N' times (INT n))+}.
\begin{session}
\begin{verbatim}
#expand ((PURE_REWRITE_TAC
#     [ADD1; (SYM (SPEC_ALL NUM_ADD_IS_INT_ADD));
#      LEFT_PLUS_DISTRIB; TIMES_IDENTITY]) THEN
#  (NEW_SUBST1_TAC
#    (SPECL ["N':integer";"(N' times (INT n)) plus N'";"neg N'"]
#      PLUS_BELOW_TRANSL)) THEN
#  (PURE_REWRITE_TAC [PLUS_GROUP_ASSOC;PLUS_INV_LEMMA;PLUS_ID_LEMMA]));;
\end{verbatim}
\mvdots
\begin{verbatim}
OK..
"(INT 0) below (N' times (INT n))"
\end{verbatim}
\mvdots
\begin{verbatim}
    [ "POS N" ]
\end{verbatim}
\mvdots
\begin{verbatim}
    [ "0 < n" ]
    [ "N below N'" ]

() : void
\end{verbatim}
\end{session}

Using {\small\verb+POS_MULT_PRES_BELOW+}, to show that
{\small\verb+(INT 0)+} is below {\small\verb+(N' times (INT n))+}, it
suffices to show that {\small\verb+N'+} is positive and that
{\small\verb+(INT 0)+} is below {\small\verb+(INT n)+}.
\begin{session}
\begin{verbatim}
#expand ((NEW_SUBST1_TAC 
#    (SYM (CONJUNCT1 (SPEC "N':integer" TIMES_ZERO)))) THEN
#  (NEW_SUBST1_TAC
#    (SYM (UNDISCH (SPECL ["N':integer"; "INT 0"; "INT n"]
#      POS_MULT_PRES_BELOW)))));;
\end{verbatim}
\mvdots
\begin{verbatim}
OK..
2 subgoals
"POS N'"
\end{verbatim}
\mvdots
\begin{verbatim}
    [ "POS N" ]
\end{verbatim}
\mvdots
\begin{verbatim}
    [ "0 < n" ]
    [ "N below N'" ]

"(INT 0) below (INT n)"
\end{verbatim}
\mvdots
\begin{verbatim}
    [ "0 < n" ]
    [ "N below N'" ]
    [ "POS N'" ]

() : void
\end{verbatim}
\end{session}

That {\small\verb+(INT 0)+} is below {\small\verb+(INT n)+} follows
from {\small\verb+NUM_LESS_IS_INT_BELOW+} and the assumption that
{\small\verb+0 < n+}.
\begin{session}
\begin{verbatim}
#expand (ASM_REWRITE_TAC [(SYM (SPEC_ALL (NUM_LESS_IS_INT_BELOW)))]);;
OK..
goal proved
. |- (INT 0) below (INT n)

Previous subproof:
"POS N'"
\end{verbatim}
\mvdots
\begin{verbatim}
    [ "POS N" ]
\end{verbatim}
\mvdots
\begin{verbatim}
    [ "N below N'" ]

() : void
\end{verbatim}
\end{session}

By {\small\verb+POS_IS_ZERO_BELOW+}, to show that {\small\verb+N'+} is
positive is to show that {\small\verb+(INT 0)+} is below
{\small\verb+N'+}.  And since we have that {\small\tt N} is below
{\small\verb+N'+}, by transitivity it suffices to show that
{\small\verb+(INT 0)+} is below {\small\tt N}.  But this we already
have, since {\small\tt N} is positive.
\begin{session}
\begin{verbatim}
#expand ((PURE_ONCE_REWRITE_TAC [POS_IS_ZERO_BELOW]) THEN
#  (MP_IMP_TAC (SPECL ["INT 0";"N:integer";"N':integer"] TRANSIT)) THEN
#  (ASM_REWRITE_TAC[(SYM (SPEC_ALL POS_IS_ZERO_BELOW))]));;
\end{verbatim}
\mvdots
\begin{verbatim}
OK..
goal proved
.. |- POS N'
... |- (INT 0) below (N' times (INT n))
... |- N' below (N' times (INT(SUC n)))
.. |- !N'. N below N' ==> N below (N' times (INT(SUC n)))
. |- !N'. N below N' ==> N below (N' times (INT(SUC n)))
.. |- !N'. N below N' ==> N below (N' times MIN)
.. |- !N'. N below N' ==> ~~NEG(N minus (N' times MIN))
.. |- ?UB. !N'. UB below N' ==> ~~NEG(N minus (N' times MIN))
........ |- ?p. N = p times MIN
...... |- !N. POS N ==> H N ==> (?p. N = p times MIN)

Previous subproof:
2 subgoals
"H(INT 0) ==> (?p. INT 0 = p times MIN)"
\end{verbatim}
\mvdots
\begin{verbatim}

"!N. NEG N ==> H N ==> (?p. N = p times MIN)"
    [ "SUBGROUP((\N. T),$plus)H" ]
\end{verbatim}
\mvdots
\begin{verbatim}
    [ "!N. POS N ==> H N ==> (?p. N = p times MIN)" ]

() : void

\end{verbatim}
\end{session}

With this, we have finished showing that there exists some value
such that, if any larger value is multiplied times {\small\verb+MIN+}
and then subtracted from {\small\tt N}, the result is negative.  This,
in turn, finishes showing that there exists a maximum number of times
that {\small\verb+MIN+} can be subtracted from {\small\tt N}.  And
this, in turn, finishes showing that if {\small\tt N} is positive,
then it is a multiple of {\small\verb+MIN+}.  (All of this presupposes
that {\small\verb+MIN+} exists, of course.)  This leaves us back at
needing to deal with the case where {\small\tt N} is negative and the
case where {\small\tt N} is zero.

To show the negative case, we want to reduce it to the positive case
which we have just finished.  To begin this, we need to change the
hypothesis of {\small\verb+NEG N+} into an assumption in terms of
{\small\verb+POS+}.  Also, the particular instance of the positive
case that we are reducing to is that of {\small\verb+(neg N)+}.
Therefore, it would be helpful, as well, to express {\small\tt N} in
the conclusion as {\small\verb+(neg (neg N))+}. 
\begin{session}
\begin{verbatim}
#expand ((PURE_ONCE_REWRITE_TAC [NEG_DEF]) THEN
#  (REPEAT STRIP_TAC) THEN
#  (NEW_SUBST1_TAC (SYM (SPEC "N:integer" PLUS_INV_INV_LEMMA))));;
OK..
"?p. neg(neg N) = p times MIN"
    [ "SUBGROUP((\N. T),$plus)H" ]
\end{verbatim}
\mvdots
\begin{verbatim}
    [ "!N. POS N ==> H N ==> (?p. N = p times MIN)" ]
    [ "POS(neg N)" ]
    [ "H N" ]

() : void
\end{verbatim}
\end{session}

Let us work on the assumptions a bit.  We know that we have the
positive case in our assumptions.  We also know that we are interested
in the particular instance of {\small\verb+(neg N)+}.  In the instance
of {\small\verb+(neg N)+} the positive case has the hypothesis that
{\small\verb+(neg N)+} is in {\small\tt H}.  Using
{\small\verb+INT_SBGP_neg+}, we can satisfy that hypothesis, since we
have that {\small\tt N} is {\small\tt H}.  Therefore, using the
positive case and {\small\verb+INT_SBGP_neg+}, let's use
{\small\verb+STRIP_ASSUME_TAC+} to introduce a {\small\tt p} and add
to the assumptions that {\small\verb+N = p times MIN+}.
\begin{session}
\begin{verbatim}
#expand (STRIP_ASSUME_TAC (MP
#    (UNDISCH (SPEC "neg N"
#      (ASSUME "!N. POS N ==> H N ==> (?p. N = p times MIN)")))
#    (UNDISCH (SPEC "N:integer" (UNDISCH
#      (SPEC "H:integer -> bool" INT_SBGP_neg))))));;
OK..
"?p. neg(neg N) = p times MIN"
\end{verbatim}
\mvdots
\begin{verbatim}
    [ "!N. POS N ==> H N ==> (?p. N = p times MIN)" ]
    [ "POS(neg N)" ]
    [ "H N" ]
    [ "neg N = p times MIN" ]

() : void
\end{verbatim}
\end{session}

The multiple of {\small\verb+MIN+} we need in order to get
{\small\verb+(neg(neg N))+} is the negative of the multiple needed to
get {\small\verb+(neg N)+}.  If we then rewrite with the assumptions
and {\small\verb+TIME_neg+}, we will finish of this goal. 
\begin{session}
\begin{verbatim}
#expand ((EXISTS_TAC "neg p") THEN (ASM_REWRITE_TAC [TIMES_neg]));;
OK..
goal proved
. |- ?p. neg(neg N) = p times MIN
.... |- ?p. neg(neg N) = p times MIN
.. |- !N. NEG N ==> H N ==> (?p. N = p times MIN)

Previous subproof:
"H(INT 0) ==> (?p. INT 0 = p times MIN)"
\end{verbatim}
\evdots
\end{session}

We've done the positive case and the negative case, which leaves us
back at doing the zero case.  The multiple of {\small\verb+MIN+} we
need to get zero is zero, since zero times anything is zero.
\begin{session}
\begin{verbatim}
#expand (DISCH_TAC THEN (EXISTS_TAC "INT 0") THEN
#  (REWRITE_TAC [TIMES_ZERO]));;
OK..
goal proved
|- H(INT 0) ==> (?p. INT 0 = p times MIN)
...... |- H m ==> (?p. m = p times MIN)

Previous subproof:
"(?p. m = p times MIN) ==> H m"
    [ "SUBGROUP((\N. T),$plus)H" ]
\end{verbatim}
\mvdots
\begin{verbatim}
    [ "H MIN" ]

() : void
\end{verbatim}
\end{session}

With finishing off the zero case, we have finished showing that, if
{\small\tt N} is an element in {\small\tt H}, then {\small\tt N} is a
multiple of {\small\verb+MIN+}, and hence is in the subgroup generated
by {\small\verb+MIN+}.  To show that {\small\tt H} and the subgroup
generated by {\small\verb+MIN+} are the same, we still have to show
that if {\small\tt N} is a multiple of {\small\verb+MIN+}, then
{\small\tt N} is in {\small\tt H}.  (We also have to show that
{\small\verb+MIN+} exists and that it corresponds to an integer, but
we'll get to that in a bit.)  Since we have that {\small\verb+MIN+} is
in {\small\tt H}, by {\small\verb+INT_SBGP_TIMES_CLOSED+} we have that
all multiples of {\small\verb+MIN+} are in {\small\tt H}. 
\begin{session}
\begin{verbatim}
#expand (STRIP_TAC THEN (ASM_REWRITE_TAC []) THEN
#  (NEW_MATCH_ACCEPT_TAC
#    (UNDISCH (SPEC_ALL (UNDISCH (SPEC_ALL INT_SBGP_TIMES_CLOSED))))));;
OK..
goal proved
.. |- (?p. m = p times MIN) ==> H m
...... |- H = (\m. ?p. m = p times MIN)
....... |- H = (\m. ?p. m = p times (INT n))
....... |- ?n. H = (\m. ?p. m = p times (INT n))
....... |- ?n. H = (\m. ?p. m = p times (INT n))

Previous subproof:
Previous subproof:
"?n. INT n = MIN"
\end{verbatim}
\mvdots
\begin{verbatim}
    [ "POS MIN" ]
\end{verbatim}
\evdots
\end{session}

To show that {\small\verb+MIN+} corresponds to an integer, let's first
add to our assumptions what it means for {\small\verb+MIN+} to be
positive.
\begin{session}
\begin{verbatim}
#expand (STRIP_ASSUME_TAC
#    (PURE_ONCE_REWRITE_RULE [POS_DEF] (ASSUME "POS MIN")));;
OK..
"?n. INT n = MIN"
\end{verbatim}
\mvdots
\begin{verbatim}
    [ "POS MIN" ]
\end{verbatim}
\mvdots
\begin{verbatim}
    [ "MIN = INT(SUC n)" ]

() : void
\end{verbatim}
\end{session}

Then we see that the integer we need is just {\small\verb+(SUC n)+},
given to us by the fact that {\small\verb+MIN+} is positive.
\begin{session}
\begin{verbatim}
#expand ((EXISTS_TAC "SUC n") THEN (ASM_REWRITE_TAC []));;
OK..
goal proved
. |- ?n. INT n = MIN
. |- ?n. INT n = MIN
...... |- ?n. H = (\m. ?p. m = p times (INT n))
..... |- ?n. H = (\m. ?p. m = p times (INT n))

Previous subproof:
2 subgoals
"?LB. !N. N below LB ==> ~(POS N /\ H N)"
\end{verbatim}
\mvdots
\begin{verbatim}

"?M. POS M /\ H M"
    [ "SUBGROUP((\N. T),$plus)H" ]
\end{verbatim}
\mvdots
\begin{verbatim}
    [ "~(!m1. H m1 ==> (m1 = INT 0))" ]

() : void
\end{verbatim}
\end{session}

This puts us back at showing that {\small\verb+MIN+} exists.  Recall
that {\small\verb+MIN+} was the minimum positive element in {\small\tt H}.
Just as  before with {\small\verb+INT_MAX_TAC+} and {\small\verb+MAX+},
the tactic {\small\verb+INT_MIN_TAC+} has already broken this problem
down into the two subgoals of showing that there is a positive element
in {\small\tt H}, and showing that there is an integer, such that
anything less than that is not a positive element of {\small\tt H}.

First, for the goal of showing that there exists a positive element in
{\small\tt H}.  The top assumption states that it is not the case that
every element of {\small\tt H} is equal to zero.  It would be more
convenient for us if we could turn this around into the declaration
that for some particular {\small\verb+m1+}, {\small\verb+m1+} is in
{\small\tt H} and {\small\verb+m1+} is not equal to zero.  We can
convert the {\small\verb+~(!m1.+\ldots\small\tt)} into a
{\small\verb+?m1.~(+\ldots\small\tt)} by the use of
{\small\verb+NOT_FORALL_CONV+}.  And we can further break up the
{\small\verb+~(+\ldots\small\verb+==>+\ldots\small\tt)} by using the
theorems {\small\verb+IMP_DISJ_THM+} and {\small\verb+DE_MORGAN_THM+}.
\begin{session}
\begin{verbatim}
#expand (POP_ASSUM \thm. STRIP_ASSUME_TAC 
#    (REWRITE_RULE [IMP_DISJ_THM;DE_MORGAN_THM]
#     (CONV_RULE NOT_FORALL_CONV thm)));;
\end{verbatim}
\mvdots
\begin{verbatim}
OK..
"?M. POS M /\ H M"
    [ "SUBGROUP((\N. T),$plus)H" ]
\end{verbatim}
\mvdots
\begin{verbatim}
    [ "H m1" ]
    [ "~(m1 = INT 0)" ]

() : void
\end{verbatim}
\end{session}

So far, we have an element {\small\verb+m1+} of {\small\tt H}, and we
know its not zero, but we don't know whether it positive or negative.
It sounds like another cases argument.  Furthermore, we know how to
handle the positive case and the zero case.
\begin{session}
\begin{verbatim}
#expand ((DISJ_CASES_TAC
#    (CONJUNCT1 (SPEC "m1:integer" TRICHOTOMY))) THENL
#  [((EXISTS_TAC "m1:integer") THEN (ASM_REWRITE_TAC []));
#   ((POP_ASSUM DISJ_CASES_TAC) THENL
#    [ALL_TAC; RES_TAC])]);;
OK..
"?M. POS M /\ H M"
    [ "SUBGROUP((\N. T),$plus)H" ]
\end{verbatim}
\mvdots
\begin{verbatim}
    [ "H m1" ]
    [ "~(m1 = INT 0)" ]
    [ "NEG m1" ]

() : void
\end{verbatim}
\end{session}

For the negative case, the {\small\tt M} we want is {\small\verb+(neg m1)+}.
The goal then follows from the definition of {\small\verb+NEG+} and
the theorem {\small\verb+INT_SBGP_neg+}.
\begin{session}
\begin{verbatim}
#expand ((EXISTS_TAC "neg m1") THEN
#  (ASM_REWRITE_TAC
#    [(SYM (SPEC_ALL NEG_DEF));
#     (UNDISCH (SPEC "m1:integer"
#      (UNDISCH (SPEC_ALL INT_SBGP_neg))))]));;
OK..
goal proved
... |- ?M. POS M /\ H M
... |- ?M. POS M /\ H M
.. |- ?M. POS M /\ H M


Previous subproof:
"?LB. !N. N below LB ==> ~(POS N /\ H N)"
\end{verbatim}
\evdots
\end{session}

Lastly to show that the set of positive elements of {\small\tt H} is
bounded below.  The obvious lower bound to the set of positive
elements is zero.
\begin{session}
\begin{verbatim}
#expand (EXISTS_TAC "INT 0");;
OK..
"!N. N below (INT 0) ==> ~(POS N /\ H N)"
\end{verbatim}
\evdots
\end{session}

By {\small\verb+NEG_IS_BELOW_ZERO+}, that {\small\tt N} is below
{\small\verb+(INT 0)+} is the same thing as that {\small\tt N} is
negative.
\begin{session}
\begin{verbatim}
#expand (PURE_ONCE_REWRITE_TAC [(SYM (SPEC_ALL NEG_IS_BELOW_ZERO))]);;
\end{verbatim}
\mvdots
\begin{verbatim}
OK..
"!N. NEG N ==> ~(POS N /\ H N)"
\end{verbatim}
\evdots
\end{session}

Given that {\small\tt N} is negative, by the second conjunct of
{\small\verb+TRICHOTOMY+}, we have that {\small\tt N} is not positive,
and hence that it is not a positive element of {\small\tt H}.
\begin{session}
\begin{verbatim}
#expand (GEN_TAC THEN DISCH_TAC THEN
#  (REWRITE_TAC
#    [(REWRITE_RULE[(ASSUME "NEG N")]
#    (CONJUNCT1 (CONJUNCT2 (SPEC "N:integer" TRICHOTOMY))))]));;
OK..
goal proved
|- !N. NEG N ==> ~(POS N /\ H N)
|- !N. N below (INT 0) ==> ~(POS N /\ H N)
|- ?LB. !N. N below LB ==> ~(POS N /\ H N)
.... |- ?n. H = (\m. ?p. m = p times (INT n))
... |- ?n. H = int_mult_set(INT n)
|- !H. SUBGROUP((\N. T),$plus)H ==> (?n. H = int_mult_set(INT n))

Previous subproof:
goal proved
() : void
\end{verbatim}
\end{session}

With this we finish showing that the set of positive elements in
{\small\tt H} is bounded below, which finishes showing that
{\small\verb+MIN+} exits, which finishes showing the {\small\tt H} is
a cyclic subgroup  (generated by {\small\verb+MIN+}).  For the last
time, let us compose the above work and save it.   Saving it is a
little more difficult than previously.  This time the composite tactic
is physically so large that on most emacs editors you can not enter it
directly into a shell.  What you should do instead is to put the
composed work into a separate file named something like
{\small\verb+temp.ml+}, save this file, and then enter into the shell:
\begin{session}
\begin{verbatim}
#loadt `temp`;;

	INT_SBGP_CYCLIC = 
	|- !H. SUBGROUP((\N. T),$plus)H ==> (?n. H = int_mult_set(INT n))


	File temp loaded
	() : void

\end{verbatim}
\end{session}

Here is what the composite work yields.
\begin{verbatim}
let INT_SBGP_CYCLIC = prove_thm(`INT_SBGP_CYCLIC`,
"!H. SUBGROUP((\N.T),$plus)H ==> ? n.(H = int_mult_set (INT n))",
(GEN_TAC THEN DISCH_TAC THEN
 (FIRST_ASSUM
   \thm.(STRIP_ASSUME_TAC
     (PURE_ONCE_REWRITE_RULE[SUBGROUP_DEF] thm))) THEN
 (PURE_ONCE_REWRITE_TAC [INT_MULT_SET_DEF]) THEN
 (ASM_CASES_TAC "!m1. (H m1) ==> (m1 = (INT 0))") THENL
 [((EXISTS_TAC "0") THEN
   (EXT_TAC "m1:integer") THEN BETA_TAC THEN
   (REWRITE_TAC [TIMES_ZERO]) THEN
   GEN_TAC THEN EQ_TAC THENL
   [(FIRST_ASSUM (\thm.(ACCEPT_TAC (SPEC_ALL thm))));
    (DISCH_TAC THEN
     (ASM_REWRITE_TAC [(UNDISCH (SPEC_ALL INT_SBGP_ZERO))]))]);
  ((INT_MIN_TAC "\N. (POS N /\ H N)") THENL
   [((POP_ASSUM STRIP_ASSUME_TAC) THEN
     (SUPPOSE_TAC "?n. (INT n) = MIN") THENL
     [((POP_ASSUM STRIP_ASSUME_TAC) THEN
       (EXISTS_TAC "n:num") THEN
       (POP_ASSUM \thm. PURE_ONCE_REWRITE_TAC [thm]) THEN
       (EXT_TAC "m:integer") THEN BETA_TAC THEN
       GEN_TAC THEN EQ_TAC THENL
       [((SPEC_TAC ("m:integer","N:integer")) THEN
         SIMPLE_INT_CASES_TAC THENL
         [((REPEAT STRIP_TAC) THEN
           (INT_MAX_TAC "\X.~(NEG(N minus (X times MIN)))") THENL
           [((EXISTS_TAC "MAX:integer") THEN
             (MATCH_MP_IMP_TAC
               (SPEC "neg (MAX times MIN)" PLUS_RIGHT_CANCELLATION)) THEN
             (PURE_REWRITE_TAC
               [PLUS_INV_LEMMA;(SYM (SPEC_ALL MINUS_DEF))]) THEN
             (DISJ_CASES_TAC (CONJUNCT1
               (SPEC "N minus (MAX times MIN)" TRICHOTOMY))) THENL
             [((SUPPOSE_TAC "(N minus (MAX times MIN)) below MIN") THENL
               [((SUPPOSE_TAC "(H (N minus (MAX times MIN))):bool") THENL
                 [(RES_TAC THEN RES_TAC);
                  ((PURE_ONCE_REWRITE_TAC [MINUS_DEF]) THEN
                   (GROUP_TAC [INT_SBGP_neg;INT_SBGP_TIMES_CLOSED]))]);
                ((PURE_ONCE_REWRITE_TAC [BELOW_DEF]) THEN
                 (NEW_SUBST1_TAC 
                   (SYM (SPEC "MIN minus (N minus (MAX times MIN))"
                     PLUS_INV_INV_LEMMA))) THEN
                 (PURE_ONCE_REWRITE_TAC [(SYM (SPEC_ALL(NEG_DEF)))]) THEN
                 (PURE_REWRITE_TAC
                    [MINUS_DEF;
                     (SYM (SPEC_ALL (PLUS_DIST_INV_LEMMA)));
                     PLUS_INV_INV_LEMMA]) THEN
                 (INT_RIGHT_ASSOC_TAC
                   "(N plus (neg (MAX times MIN))) plus (neg MIN)") THEN
                 (PURE_REWRITE_TAC
                    [(SYM neg_PLUS_DISTRIB);(SYM (SPEC_ALL MINUS_DEF))]) THEN
                 (PURE_ONCE_REWRITE_TAC
                   [(PURE_ONCE_REWRITE_RULE [TIMES_IDENTITY] (SYM
                     (SPECL ["MAX:integer";"INT 1";"MIN:integer"]
                       RIGHT_PLUS_DISTRIB)))]) THEN
                 (MATCH_MP_IMP_TAC (ONCE_REWRITE_RULE []
                   (ASSUME
                     "!N'.MAX below N' ==>
                       ~~NEG (N minus (N' times MIN))"))) THEN
                 (SUBST_MATCH_TAC
                   (PURE_ONCE_REWRITE_RULE [COMM_PLUS]
                     (SPECL ["A:integer";"B:integer";"neg MAX"]
                       PLUS_BELOW_TRANSL))) THEN
                 (PURE_REWRITE_TAC
                    [(SYM (SPEC_ALL PLUS_GROUP_ASSOC));
                     PLUS_INV_LEMMA;
                     (SYM (SPEC_ALL NUM_LESS_IS_INT_BELOW));
                     PLUS_ID_LEMMA]) THEN
                 (CONV_TAC (DEPTH_CONV num_CONV)) THEN
                 (ACCEPT_TAC (theorem `prim_rec` `LESS_0_0`)))]);
              ((POP_ASSUM DISJ_CASES_TAC) THENL
               [RES_TAC;(FIRST_ASSUM ACCEPT_TAC)])]);
            ((EXISTS_TAC "INT 0") THEN 
             (PURE_REWRITE_TAC
               [MINUS_DEF;TIMES_ZERO;PLUS_INV_ID_LEMMA;PLUS_ID_LEMMA]) THEN
             (ACCEPT_TAC (REWRITE_RULE [(ASSUME "POS N")]
               (CONJUNCT1 (CONJUNCT2 (SPEC "N:integer" TRICHOTOMY))))));
            ((EXISTS_TAC "N:integer") THEN
             (REWRITE_TAC
               [NEG_DEF;
                MINUS_DEF;
                (SYM (SPEC_ALL PLUS_DIST_INV_LEMMA));
                PLUS_INV_INV_LEMMA]) THEN
             (PURE_REWRITE_TAC[(SYM (SPEC_ALL MINUS_DEF));
                (SYM (SPEC_ALL BELOW_DEF))]) THEN
             (STRIP_ASSUME_TAC
               (PURE_ONCE_REWRITE_RULE [POS_DEF] (ASSUME "POS MIN"))) THEN
             (PURE_ONCE_ASM_REWRITE_TAC[]) THEN
             ((DISJ_CASES_TAC (SPEC "n:num" LESS_0_CASES)) THENL
              [(POP_ASSUM \thm. (ASM_REWRITE_TAC
                 [(SYM thm);(SYM (num_CONV "1"));TIMES_IDENTITY]));
               ALL_TAC]) THEN
             (REPEAT STRIP_TAC) THEN
             (MP_IMP_TAC
               (SPECL ["N:integer";"N':integer";"N' times (INT(SUC n))"]
                TRANSIT)) THEN
             (ASM_REWRITE_TAC []) THEN
             (PURE_REWRITE_TAC
               [ADD1; (SYM (SPEC_ALL NUM_ADD_IS_INT_ADD));
                LEFT_PLUS_DISTRIB; TIMES_IDENTITY]) THEN
             (NEW_SUBST1_TAC
               (SPECL ["N':integer";"(N' times (INT n)) plus N'";"neg N'"]
                 PLUS_BELOW_TRANSL)) THEN
             (PURE_REWRITE_TAC
               [PLUS_GROUP_ASSOC;PLUS_INV_LEMMA;PLUS_ID_LEMMA]) THEN
             (NEW_SUBST1_TAC 
              (SYM (CONJUNCT1 (SPEC "N':integer" TIMES_ZERO)))) THEN
             (NEW_SUBST1_TAC
               (SYM (UNDISCH (SPECL ["N':integer"; "INT 0"; "INT n"]
                 POS_MULT_PRES_BELOW)))) THENL
             [(ASM_REWRITE_TAC [(SYM (SPEC_ALL (NUM_LESS_IS_INT_BELOW)))]);
              ((PURE_ONCE_REWRITE_TAC [POS_IS_ZERO_BELOW]) THEN
               (MP_IMP_TAC
                 (SPECL ["INT 0";"N:integer";"N':integer"] TRANSIT)) THEN
               (ASM_REWRITE_TAC[(SYM (SPEC_ALL POS_IS_ZERO_BELOW))]))])]);
          ((PURE_ONCE_REWRITE_TAC [NEG_DEF]) THEN
           (REPEAT STRIP_TAC) THEN
           (NEW_SUBST1_TAC (SYM (SPEC "N:integer" PLUS_INV_INV_LEMMA))) THEN
           (STRIP_ASSUME_TAC (MP
             (UNDISCH (SPEC "neg N"
               (ASSUME "!N. POS N ==> H N ==> (?p. N = p times MIN)")))
             (UNDISCH (SPEC "N:integer" (UNDISCH
               (SPEC "H:integer -> bool" INT_SBGP_neg)))))) THEN
           (EXISTS_TAC "neg p") THEN (ASM_REWRITE_TAC [TIMES_neg]));
          (DISCH_TAC THEN (EXISTS_TAC "INT 0") THEN
           (REWRITE_TAC [TIMES_ZERO]))]);
        (STRIP_TAC THEN (ASM_REWRITE_TAC []) THEN
         (NEW_MATCH_ACCEPT_TAC
           (UNDISCH (SPEC_ALL (UNDISCH (SPEC_ALL INT_SBGP_TIMES_CLOSED))))))]);
      ((STRIP_ASSUME_TAC
         (PURE_ONCE_REWRITE_RULE [POS_DEF] (ASSUME "POS MIN"))) THEN
       (EXISTS_TAC "SUC n") THEN (ASM_REWRITE_TAC []))]);
    ((POP_ASSUM \thm. STRIP_ASSUME_TAC 
       (REWRITE_RULE [IMP_DISJ_THM;DE_MORGAN_THM]
         (CONV_RULE NOT_FORALL_CONV thm))) THEN
     ((DISJ_CASES_TAC
        (CONJUNCT1 (SPEC "m1:integer" TRICHOTOMY))) THENL
      [((EXISTS_TAC "m1:integer") THEN (ASM_REWRITE_TAC []));
       ((POP_ASSUM DISJ_CASES_TAC) THENL
        [ALL_TAC; RES_TAC])]) THEN
     (EXISTS_TAC "neg m1") THEN
     (ASM_REWRITE_TAC
       [(SYM (SPEC_ALL NEG_DEF));
        (UNDISCH (SPEC "m1:integer"
          (UNDISCH (SPEC_ALL INT_SBGP_neg))))]));
    ((EXISTS_TAC "INT 0") THEN
     (PURE_ONCE_REWRITE_TAC [(SYM (SPEC_ALL NEG_IS_BELOW_ZERO))]) THEN
     GEN_TAC THEN DISCH_TAC THEN
     (REWRITE_TAC
       [(REWRITE_RULE[(ASSUME "NEG N")]
        (CONJUNCT1 (CONJUNCT2 (SPEC "N:integer" TRICHOTOMY))))]))])]));;
\end{verbatim}

Since we used {\small\verb+load_theory+} to enter
{\small\verb+int_sbgp+}, and thus were not in draft mode, we do not
need to use {\small\verb+close_theory ()+}, but only
{\small\verb+quit()+} to close down. 
\begin{session}
\begin{verbatim}
#quit();;
faulkner%
\end{verbatim}
\end{session}

Over the course of this study, you have gained enough familiarity with
the definitions and facts in the group theory library to be able to
use them in a concrete setting, namely, the setting of the integers.
You have practiced importing (or instantiating) an abstract theory,
that of elementary group theory, in a different theory, that of
modular arithmetic, which in turn may be viewed as an abstract theory
waiting to be instantiated by a given integer.  Finally, you have had
the experience of working through a fairly long and complicated proof
of the fact that every subgroup of the integers is cyclic, \ie, is the
set of multiples of a given element.  Along the way, you gained
experience with reasoning by the cases of positive, negative or zero,
with reducing a problem over the integers to one over the natural
numbers, and with using maximal and minimal elements of bounded sets
of integers to solve problems of existence.  It is hoped that all this
experience will stand you in good stead as you go on to use the group
theory and the integers in stating and solving your own individual
problems.

For those of you who would like a parting project, at the beginning of
this section, it was said that the fact that every subgroup of the
integers is cyclic is a goodly piece of the work necessary to show
that greatest common divisors exist for the integers.  An outline of
how to see this will follow.  The project is to implement this in \HOL.

For each pair of integers {\small\tt M} and {\small\tt N}, consider
the set of sums of multiples of {\small\tt M} and {\small\tt N}
decribed by the predicate 
\begin{verbatim}
   \X:integer. (?A B. (X = (A times M) plus (B times N)))
\end{verbatim}
In number theory this set is sometimes called the {\it modulus} of
{\small\tt M} and {\small\tt N}.  Show that for all {\small\tt M} and
{\small\tt N}, this set always forms a subgroup of the integers.
Since this set is a subgroup of the integers, by the previous theorem,
it is cyclic.  Define the {\small\verb+gcd(M,N)+} to be the generator
of this subgroup.  That {\small\tt M} and {\small\tt N} are both
multiples of {\small\verb+gcd(M,N)+} follows from the fact that both
{\small\tt M} and {\small\tt N} are in this subgroup.  To see that the
{\small\verb+gcd(M,N)+} is a multiple of any {\small\tt C} which
divides both {\small\tt M} and {\small\tt N}, recall that the
{\small\verb+gcd(M,N)+} must be of the form
{\small\verb+(A times M) plus (B times N)+} for some integers
{\small\tt A} and {\small\tt B}.  Since {\small\tt M} is a multiple of
{\small\tt C} and {\small\tt N} is a multiple of {\small\tt C}, by
distributivity the {\small\verb+gcd(M,N)+} being
{\small\verb+(A times M) plus (B times N)+} is a multiple of
{\small\tt C}.  If you want to see that the {\small\verb+gcd(M,N)+} is
actually the greast divisor of both {\small\tt M} and {\small\tt N} in
order theoretic terms (provided at least one of {\small\tt M} or
{\small\tt N} is not {\small\verb+(INT 0)+}) show that, if {\small\tt X}
and {\small\tt Y} are positive integers and {\small\verb+Z = X times Y+},
then {\small\tt X} is below {\small\tt Z}. 

Happy Theorem Proving and Best Wishes, Elsa L. Gunter



\appendix

\section{Related pre-proven theorems}
\subsection{First order group theory}
\begin{verbatim}
The Theory elt_gp
Parents --  HOL     
Constants --
  GROUP ":(* -> bool) # (* -> (* -> *)) -> bool"
  ID ":(* -> bool) # (* -> (* -> *)) -> *"
  INV ":(* -> bool) # (* -> (* -> *)) -> (* -> *)"

Definitions --
  GROUP_DEF
    |- !G prod.
        GROUP(G,prod) =
        (!x y. G x /\ G y ==> G(prod x y)) /\
        (!x y z. G x /\ G y /\ G z ==> 
          (prod(prod x y)z = prod x(prod y z))) /\
        (?e. G e /\
             (!x. G x ==> (prod e x = x)) /\
             (!x. G x ==> (?y. G y /\ (prod y x = e))))
  ID_DEF
    |- !G prod.
        ID(G,prod) =
        (@e. G e /\
             (!x. G x ==> (prod e x = x)) /\
             (!x. G x ==> (?y. G y /\ (prod y x = e))))
  INV_DEF
    |- !G prod x. INV(G,prod)x = (@y. G y /\ (prod y x = ID(G,prod)))

Theorems --
\end{verbatim}
A group is closed under multiplication.
\begin{verbatim}
  CLOSURE 
   |- GROUP(G,prod) ==> (!x y. G x /\ G y ==> G(prod x y))

\end{verbatim}
The multiplication in a group is associative.
\begin{verbatim}
  GROUP_ASSOC
    |- GROUP(G,prod) ==>
       (!x y z. G x /\ G y /\ G z ==> (prod(prod x y)z = prod x(prod y z)))

\end{verbatim}
{\tt ID} is both a left and a right identity in {\tt G}.
\begin{verbatim}
  ID_LEMMA
    |- GROUP(G,prod) ==>
       G(ID(G,prod)) /\
       (!x. G x ==> (prod(ID(G,prod))x = x)) /\
       (!x. G x ==> (prod x(ID(G,prod)) = x)) /\
       (!x. G x ==> (?y. G y /\ (prod y x = ID(G,prod))))

\end{verbatim}
{\tt G} is closed under the taking of inverses.
\begin{verbatim}
  INV_CLOSURE
    |- GROUP(G,prod) ==> (!x. G x ==> G(INV(G,prod)x))

\end{verbatim}
A left inverse for {\tt x} in {\tt G} with respect to {\tt ID} is also a right
inverse for {\tt x} in {\tt G} with respect to {\tt ID}.
\begin{verbatim}
  LEFT_RIGHT_INV
    |- GROUP(G,prod) ==>
       (!x y. G x /\ G y ==>
             (prod y x = ID(G,prod)) ==> (prod x y = ID(G,prod)))

\end{verbatim}
{\tt INV x} is both a left inverse for {\tt x} and a right inverse for {\tt x}
in {\tt G}.
\begin{verbatim}
  INV_LEMMA
    |- GROUP(G,prod) ==>
       (!x. G x ==>
            (prod(INV(G,prod)x)x = ID(G,prod)) /\
            (prod x(INV(G,prod)x) = ID(G,prod)))

\end{verbatim}
We have right and left cancelation in {\tt G}.
\begin{verbatim}
  LEFT_CANCELLATION
    |- GROUP(G,prod) ==>
       (!x y z. G x /\ G y /\ G z ==> (prod x y = prod x z) ==> (y = z))

  RIGHT_CANCELLATION
    |- GROUP(G,prod) ==>
       (!x y z. G x /\ G y /\ G z ==> (prod y x = prod z x) ==> (y = z))

\end{verbatim}
Given elements {\tt x} and {\tt y} in {\tt G}, there exist a unique
element {\tt z} in {\tt G} such that \mbox{\tt (prod x z = y)}.
\begin{verbatim}
  RIGHT_ONE_ONE_ONTO
    |- GROUP(G,prod) ==>
       (!x y. G x /\ G y ==>
        (?z. G z /\ (prod x z = y) /\ (!u. G u /\ (prod x u = y) ==> (u = z))))

\end{verbatim}
Given elements {\tt x} and {\tt y} in {\tt G}, there exist a unique
element {\tt z} in {\tt G} such that \mbox{\tt (prod z x = y)}.
\begin{verbatim}
  LEFT_ONE_ONE_ONTO
    |- GROUP(G,prod) ==>
       (!x y. G x /\ G y ==>
        (?z. G z /\ (prod z x = y) /\ (!u. G u /\ (prod u x = y) ==> (u = z))))

\end{verbatim}
{\tt ID} is the unique left identity and the unique right identity in {\tt G}.
\begin{verbatim}
  UNIQUE_ID
    |- GROUP(G,prod) ==>
       (!e. G e /\
            ((?x. G x /\ (prod e x = x)) \/ (?x. G x /\ (prod x e = x))) ==>
            (e = ID(G,prod)))

\end{verbatim}
{\tt INV} is the unique left inverse for {\tt x}.
\begin{verbatim}
  UNIQUE_INV
    |- GROUP(G,prod) ==>
       (!x. G x ==>
            (!u. G u /\ (prod u x = ID(G,prod)) ==> (u = INV(G,prod)x)))

\end{verbatim}
The inverse of the identity is the identity.
\begin{verbatim}
  INV_ID_LEMMA
    |- GROUP(G,prod) ==> (INV(G,prod)(ID(G,prod)) = ID(G,prod))
\end{verbatim}
The inverse of the inverse of {\tt x} is {\tt x}.
\begin{verbatim}
  INV_INV_LEMMA
    |- GROUP(G,prod) ==> (!x. G x ==> (INV(G,prod)(INV(G,prod)x) = x))

\end{verbatim}
The group product anti-distributes over the inverse.
\begin{verbatim}
  DIST_INV_LEMMA
    |- GROUP(G,prod) ==>
       (!x y. G x /\ G y ==>
              (prod(INV(G,prod)x)(INV(G,prod)y) = INV(G,prod)(prod y x)))
\end{verbatim}

\subsection{Higher order group theory}

Included here is the result of a development of the some basic higher-order
group theory in HOL.

\begin{verbatim}
The Theory more_gp
Parents --  HOL     elt_gp     
Constants --
  SUBGROUP ":(* -> bool) # (* -> (* -> *)) -> ((* -> bool) -> bool)"
  LEFT_COSET
    ":(* -> bool) # (* -> (* -> *)) ->
      ((* -> bool) -> (* -> (* -> bool)))"
  EQUIV_REL ":(* -> bool) -> ((* -> (* -> bool)) -> bool)"
  NORMAL ":(* -> bool) # (* -> (* -> *)) -> ((* -> bool) -> bool)"
  set_prod
    ":(* -> bool) # (* -> (* -> *)) ->
      ((* -> bool) -> ((* -> bool) -> (* -> bool)))"
  quot_set
    ":(* -> bool) # (* -> (* -> *)) ->
      ((* -> bool) -> ((* -> bool) -> bool))"
  quot_prod
    ":(* -> bool) # (* -> (* -> *)) ->
      ((* -> bool) -> ((* -> bool) -> ((* -> bool) -> (* -> bool))))"
  GP_HOM
    ":(* -> bool) # (* -> (* -> *)) ->
      ((** -> bool) # (** -> (** -> **)) -> ((* -> **) -> bool))"
  IM ":(* -> bool) -> ((* -> **) -> (** -> bool))"
  KERNEL
    ":(* -> bool) # (* -> (* -> *)) ->
      ((** -> bool) # (** -> (** -> **)) -> ((* -> **) -> (* -> bool)))"
  INV_IM ":(* -> bool) -> ((** -> bool) -> ((* -> **) -> (* -> bool)))"
  NAT_HOM
    ":(* -> bool) # (* -> (* -> *)) ->
      ((* -> bool) -> (* -> (* -> bool)))"
  quot_hom
    ":(* -> bool) # (* -> (* -> *)) ->
      ((** -> bool) # (** -> (** -> **)) ->
       ((* -> bool) -> ((* -> **) -> ((* -> bool) -> **))))"
  GP_ISO
    ":(* -> bool) # (* -> (* -> *)) ->
      ((** -> bool) # (** -> (** -> **)) -> ((* -> **) -> bool))"     
Definitions --
  SUBGROUP_DEF
    |- !G prod H.
        SUBGROUP(G,prod)H =
        GROUP(G,prod) /\ (!x. H x ==> G x) /\ GROUP(H,prod)
  LEFT_COSET_DEF
    |- !G prod H x y.
        LEFT_COSET(G,prod)H x y =
        GROUP(G,prod) /\
        SUBGROUP(G,prod)H /\
        G x /\
        G y /\
        (?h. H h /\ (y = prod x h))
  EQUIV_REL_DEF
    |- !G R.
        EQUIV_REL G R =
        (!x. G x ==> R x x) /\
        (!x y. G x /\ G y ==> R x y ==> R y x) /\
        (!x y z. G x /\ G y /\ G z ==> R x y /\ R y z ==> R x z)
  NORMAL_DEF
    |- !G prod N.
        NORMAL(G,prod)N =
        SUBGROUP(G,prod)N /\
        (!x n. G x /\ N n ==> N(prod(INV(G,prod)x)(prod n x)))
  SET_PROD_DEF
    |- !G prod A B.
        set_prod(G,prod)A B =
        (\x.
          GROUP(G,prod) /\
          (!a. A a ==> G a) /\
          (!b. B b ==> G b) /\
          (?a. A a /\ (?b. B b /\ (x = prod a b))))
  QUOTIENT_SET_DEF
    |- !G prod N q.
        quot_set(G,prod)N q =
        NORMAL(G,prod)N /\ (?x. G x /\ (q = LEFT_COSET(G,prod)N x))
  QUOTIENT_PROD_DEF
    |- !G prod N q r.
        quot_prod(G,prod)N q r =
        LEFT_COSET
        (G,prod)
        N
        (prod
         (@x. G x /\ (q = LEFT_COSET(G,prod)N x))
         (@y. G y /\ (r = LEFT_COSET(G,prod)N y)))
  GP_HOM_DEF
    |- !G1 prod1 G2 prod2 f.
        GP_HOM(G1,prod1)(G2,prod2)f =
        GROUP(G1,prod1) /\
        GROUP(G2,prod2) /\
        (!x. G1 x ==> G2(f x)) /\
        (!x y. G1 x /\ G1 y ==> (f(prod1 x y) = prod2(f x)(f y)))
  IM_DEF  |- !G f. IM G f = (\y. ?x. G x /\ (y = f x))
  KERNEL_DEF
    |- !G1 prod1 G2 prod2 f.
        KERNEL(G1,prod1)(G2,prod2)f =
        (\x.
          GP_HOM(G1,prod1)(G2,prod2)f /\ G1 x /\ (f x = ID(G2,prod2)))
  INV_IM_DEF  |- !G1 G2 f. INV_IM G1 G2 f = (\x. G1 x /\ G2(f x))
  NAT_HOM_DEF
    |- !G prod N x.
        NAT_HOM(G,prod)N x =
        (\y.
          GROUP(G,prod) /\ NORMAL(G,prod)N /\ LEFT_COSET(G,prod)N x y)
  QUOTIENT_HOM_DEF
    |- !G1 prod1 G2 prod2 N f.
        quot_hom(G1,prod1)(G2,prod2)N f =
        (\q.
          f
          (@w.
            GP_HOM(G1,prod1)(G2,prod2)f /\
            NORMAL(G1,prod1)N /\
            (!n. N n ==> KERNEL(G1,prod1)(G2,prod2)f n) /\
            (?x. G1 x /\ (q = LEFT_COSET(G1,prod1)N x)) ==>
            G1 w /\ (q = LEFT_COSET(G1,prod1)N w)))
  GP_ISO_DEF
    |- !G1 prod1 G2 prod2 f.
        GP_ISO(G1,prod1)(G2,prod2)f =
        GP_HOM(G1,prod1)(G2,prod2)f /\
        (?g.
          GP_HOM(G2,prod2)(G1,prod1)g /\
          (!x. G1 x ==> (g(f x) = x)) /\
          (!y. G2 y ==> (f(g y) = y)))
  
Theorems --
  SBGP_ID_GP_ID  |- SUBGROUP(G,prod)H ==> (ID(H,prod) = ID(G,prod))
  SBGP_INV_GP_INV
    |- SUBGROUP(G,prod)H ==> (!x. H x ==> (INV(H,prod)x = INV(G,prod)x))
  SBGP_SBGP_LEMMA
    |- SUBGROUP(G,prod)H /\ SUBGROUP(H,prod)K1 ==> SUBGROUP(G,prod)K1
  GROUP_IS_SBGP  |- GROUP(G,prod) ==> SUBGROUP(G,prod)G
  ID_IS_SBGP  |- GROUP(G,prod) ==> SUBGROUP(G,prod)(\x. x = ID(G,prod))
  SUBGROUP_LEMMA
    |- SUBGROUP(G,prod)H =
       GROUP(G,prod) /\
       (?x. H x) /\
       (!x. H x ==> G x) /\
       (!x y. H x /\ H y ==> H(prod x y)) /\
       (!x. H x ==> H(INV(G,prod)x))
  SBGP_INTERSECTION
    |- (?j. Ind j) ==>
       (!i. Ind i ==> SUBGROUP(G,prod)(H i)) ==>
       SUBGROUP(G,prod)(\x. !i. Ind i ==> H i x)
  COR_SBGP_INT
    |- SUBGROUP(G,prod)H /\ SUBGROUP(G,prod)K1 ==>
       SUBGROUP(G,prod)(\x. H x /\ K1 x)
  LEFT_COSETS_COVER
    |- SUBGROUP(G,prod)H ==> (!x. G x ==> LEFT_COSET(G,prod)H x x)
  LEFT_COSET_DISJOINT_LEMMA
    |- SUBGROUP(G,prod)H ==>
       (!x y.
         G x /\ G y ==>
         (?w. LEFT_COSET(G,prod)H x w /\ LEFT_COSET(G,prod)H y w) ==>
         (!z. LEFT_COSET(G,prod)H x z ==> LEFT_COSET(G,prod)H y z))
  LEFT_COSET_DISJOINT_UNION
    |- SUBGROUP(G,prod)H ==>
       (!x. G x ==> (?y. G y /\ LEFT_COSET(G,prod)H y x)) /\
       (!x y.
         G x /\ G y ==>
         (LEFT_COSET(G,prod)H x = LEFT_COSET(G,prod)H y) \/
         ((\z. LEFT_COSET(G,prod)H x z /\ LEFT_COSET(G,prod)H y z) =
          (\z. F)))
  LEFT_COSET_EQUIV_REL
    |- SUBGROUP(G,prod)H ==> EQUIV_REL G(LEFT_COSET(G,prod)H)
  LEFT_COSETS_SAME_SIZE
    |- SUBGROUP(G,prod)H ==>
       (!x y.
         G x /\ G y ==>
         (?f g.
           (!u. LEFT_COSET(G,prod)H x u ==> LEFT_COSET(G,prod)H y(f u)) /\
           (!v. LEFT_COSET(G,prod)H y v ==> LEFT_COSET(G,prod)H x(g v)) /\
           (!u. LEFT_COSET(G,prod)H x u ==> (g(f u) = u)) /\
           (!v. LEFT_COSET(G,prod)H y v ==> (f(g v) = v))))
  GROUP_IS_NORMAL  |- GROUP(G,prod) ==> NORMAL(G,prod)G
  ID_IS_NORMAL  |- GROUP(G,prod) ==> NORMAL(G,prod)(\x. x = ID(G,prod))
  NORMAL_INTERSECTION
    |- SUBGROUP(G,prod)H /\ NORMAL(G,prod)N ==>
       NORMAL(H,prod)(\x. H x /\ N x)
  NORM_NORM_INT
    |- NORMAL(G,prod)N1 /\ NORMAL(G,prod)N2 ==>
       NORMAL(G,prod)(\n. N1 n /\ N2 n)
  NORMAL_PROD
    |- NORMAL(G,prod)N /\ SUBGROUP(G,prod)H ==>
       SUBGROUP(G,prod)(set_prod(G,prod)H N)
  QUOT_PROD
    |- NORMAL(G,prod)N ==>
       (!x y.
         G x /\ G y ==>
         (quot_prod
          (G,prod)
          N
          (LEFT_COSET(G,prod)N x)
          (LEFT_COSET(G,prod)N y) =
          LEFT_COSET(G,prod)N(prod x y)))
  QUOTIENT_GROUP
    |- NORMAL(G,prod)N ==> GROUP(quot_set(G,prod)N,quot_prod(G,prod)N)
  GP_HOM_COMP
    |- GP_HOM(G1,prod1)(G2,prod2)f /\ GP_HOM(G2,prod2)(G3,prod3)g ==>
       GP_HOM(G1,prod1)(G3,prod3)(\x. g(f x))
  HOM_ID_INV_LEMMA
    |- GP_HOM(G1,prod1)(G2,prod2)f ==>
       (f(ID(G1,prod1)) = ID(G2,prod2)) /\
       (!x. G1 x ==> (f(INV(G1,prod1)x) = INV(G2,prod2)(f x)))
  Id_GP_HOM  |- GROUP(G1,prod1) ==> GP_HOM(G1,prod1)(G1,prod1)(\x. x)
  Triv_GP_HOM
    |- GROUP(G1,prod1) /\ GROUP(G2,prod2) ==>
       GP_HOM(G1,prod1)(G2,prod2)(\x. ID(G2,prod2))
  GP_HOM_RESTRICT_DOM
    |- GP_HOM(G1,prod1)(G2,prod2)f /\ SUBGROUP(G1,prod1)H1 ==>
       GP_HOM(H1,prod1)(G2,prod2)f
  SUBGROUP_HOM_IM
    |- GP_HOM(G1,prod1)(G2,prod2)f ==>
       (!H. SUBGROUP(G1,prod1)H ==> SUBGROUP(G2,prod2)(IM H f))
  GROUP_HOM_IM
    |- GP_HOM(G1,prod1)(G2,prod2)f ==> SUBGROUP(G2,prod2)(IM G1 f)
  GP_HOM_RESTRICT_RANGE
    |- GP_HOM(G1,prod1)(G2,prod2)f /\
       SUBGROUP(G2,prod2)H2 /\
       (!y. IM G1 f y ==> H2 y) ==>
       GP_HOM(G1,prod1)(H2,prod2)f
  GP_HOM_RES_TO_IM
    |- GP_HOM(G1,prod1)(G2,prod2)f ==> GP_HOM(G1,prod1)(IM G1 f,prod2)f
  GP_HOM_RES_TO_SBGP
    |- GP_HOM(G1,prod1)(G2,prod2)f /\ SUBGROUP(G1,prod1)H ==>
       GP_HOM(H,prod1)(G2,prod2)f /\
       (KERNEL(H,prod1)(G2,prod2)f =
        (\x. H x /\ KERNEL(G1,prod1)(G2,prod2)f x)) /\
       (!y. IM H f y ==> IM G1 f y)
  KERNEL_NORMAL
    |- GP_HOM(G1,prod1)(G2,prod2)f ==>
       NORMAL(G1,prod1)(KERNEL(G1,prod1)(G2,prod2)f)
  KERNEL_IM_LEMMA
    |- GP_HOM(G1,prod1)(G2,prod2)f ==>
       (IM(KERNEL(G1,prod1)(G2,prod2)f)f = (\y. y = ID(G2,prod2)))
  KERNEL_INV_IM_LEMMA
    |- GP_HOM(G1,prod1)(G2,prod2)f ==>
       (KERNEL(G1,prod1)(G2,prod2)f = INV_IM G1(\y. y = ID(G2,prod2))f)
  SUBGROUP_INV_IM
    |- GP_HOM(G1,prod1)(G2,prod2)f /\ SUBGROUP(G2,prod2)H ==>
       SUBGROUP(G1,prod1)(INV_IM G1 G2 f) /\
       (!x. KERNEL(G1,prod1)(G2,prod2)f x ==> INV_IM G1 G2 f x)
  NORMAL_INV_IM
    |- GP_HOM(G1,prod1)(G2,prod2)f /\ NORMAL(G2,prod2)H ==>
       NORMAL(G1,prod1)(INV_IM G1 G2 f)
  NAT_HOM_THM
    |- GROUP(G,prod) /\ NORMAL(G,prod)N ==>
       GP_HOM
       (G,prod)
       (quot_set(G,prod)N,quot_prod(G,prod)N)
       (NAT_HOM(G,prod)N) /\
       (!q.
         (?x. G x /\ (q = LEFT_COSET(G,prod)N x)) ==>
         (?x. G x /\ (q = NAT_HOM(G,prod)N x))) /\
       (KERNEL
        (G,prod)
        (quot_set(G,prod)N,quot_prod(G,prod)N)
        (NAT_HOM(G,prod)N) =
        N)
  QUOTIENT_HOM_LEMMA
    |- GP_HOM(G1,prod1)(G2,prod2)f /\
       SUBGROUP(G1,prod1)H /\
       (!h. H h ==> KERNEL(G1,prod1)(G2,prod2)f h) ==>
       (!x y. LEFT_COSET(G1,prod1)H x y ==> (f x = f y))
  QUOT_HOM_THM
    |- GP_HOM(G1,prod1)(G2,prod2)f /\
       NORMAL(G1,prod1)N /\
       (!n. N n ==> KERNEL(G1,prod1)(G2,prod2)f n) ==>
       GP_HOM
       (quot_set(G1,prod1)N,quot_prod(G1,prod1)N)
       (G2,prod2)
       (quot_hom(G1,prod1)(G2,prod2)N f) /\
       (!x.
         G1 x ==>
         (quot_hom(G1,prod1)(G2,prod2)N f(NAT_HOM(G1,prod1)N x) = f x)) /\
       (IM(quot_set(G1,prod1)N)(quot_hom(G1,prod1)(G2,prod2)N f) =
        IM G1 f) /\
       (KERNEL
        (quot_set(G1,prod1)N,quot_prod(G1,prod1)N)
        (G2,prod2)
        (quot_hom(G1,prod1)(G2,prod2)N f) =
        IM(KERNEL(G1,prod1)(G2,prod2)f)(NAT_HOM(G1,prod1)N)) /\
       (!g.
         GP_HOM(quot_set(G1,prod1)N,quot_prod(G1,prod1)N)(G2,prod2)g /\
         (!x. G1 x ==> (g(NAT_HOM(G1,prod1)N x) = f x)) ==>
         (!q.
           quot_set(G1,prod1)N q ==>
           (g q = quot_hom(G1,prod1)(G2,prod2)N f q)))
  QUOTIENT_IM_LEMMA
    |- SUBGROUP(G,prod)H /\ NORMAL(G,prod)N /\ (!n. N n ==> H n) ==>
       (IM H(NAT_HOM(G,prod)N) = quot_set(H,prod)N)
  GP_ISO_COMP
    |- GP_ISO(G1,prod1)(G2,prod2)f /\ GP_ISO(G2,prod2)(G3,prod3)g ==>
       GP_ISO(G1,prod1)(G3,prod3)(\x. g(f x))
  Id_GP_ISO  |- GROUP(G1,prod1) ==> GP_ISO(G1,prod1)(G1,prod1)(\x. x)
  GP_ISO_INV
    |- GP_ISO(G1,prod1)(G2,prod2)f ==>
       (?g.
         (!x. G1 x ==> (g(f x) = x)) /\
         (!y. G2 y ==> (f(g y) = y)) /\
         GP_ISO(G2,prod2)(G1,prod1)g)
  GP_ISO_IM_LEMMA  |- GP_ISO(G1,prod1)(G2,prod2)f ==> (IM G1 f = G2)
  GP_ISO_KERNEL
    |- GP_HOM(G1,prod1)(G2,prod2)f ==>
       (GP_ISO(G1,prod1)(IM G1 f,prod2)f =
        (KERNEL(G1,prod1)(G2,prod2)f = (\x. x = ID(G1,prod1))))
  GP_ISO_CHAR
    |- GP_ISO(G1,prod1)(G2,prod2)f =
       GP_HOM(G1,prod1)(G2,prod2)f /\
       (IM G1 f = G2) /\
       (KERNEL(G1,prod1)(G2,prod2)f = (\x. x = ID(G1,prod1)))
  FIRST_ISO_THM
    |- GP_HOM(G1,prod1)(G2,prod2)f ==>
       GP_ISO
       (quot_set(G1,prod1)(KERNEL(G1,prod1)(G2,prod2)f),
        quot_prod(G1,prod1)(KERNEL(G1,prod1)(G2,prod2)f))
       (IM G1 f,prod2)
       (quot_hom(G1,prod1)(G2,prod2)(KERNEL(G1,prod1)(G2,prod2)f)f) /\
       (!x.
         G1 x ==>
         (quot_hom
          (G1,prod1)
          (G2,prod2)
          (KERNEL(G1,prod1)(G2,prod2)f)
          f
          (NAT_HOM(G1,prod1)(KERNEL(G1,prod1)(G2,prod2)f)x) =
          f x))
  CLASSICAL_FIRST_ISO_THM
    |- GP_HOM(G1,prod1)(G2,prod2)f ==>
       (?f_bar.
         GP_ISO
         (quot_set(G1,prod1)(KERNEL(G1,prod1)(G2,prod2)f),
          quot_prod(G1,prod1)(KERNEL(G1,prod1)(G2,prod2)f))
         (IM G1 f,prod2)
         f_bar /\
         (!x.
           G1 x ==>
           (f_bar(NAT_HOM(G1,prod1)(KERNEL(G1,prod1)(G2,prod2)f)x) =
            f x)))
  SND_ISO_THM
    |- NORMAL(G,prod)N /\ NORMAL(G,prod)M /\ (!n. N n ==> M n) ==>
       GP_ISO
       (quot_set
        (quot_set(G,prod)N,quot_prod(G,prod)N)
        (quot_set(M,prod)N),
        quot_prod
        (quot_set(G,prod)N,quot_prod(G,prod)N)
        (quot_set(M,prod)N))
       (quot_set(G,prod)M,quot_prod(G,prod)M)
       (quot_hom
        (quot_set(G,prod)N,quot_prod(G,prod)N)
        (quot_set(G,prod)M,quot_prod(G,prod)M)
        (quot_set(M,prod)N)
        (quot_hom
         (G,prod)
         (quot_set(G,prod)M,quot_prod(G,prod)M)
         N
         (NAT_HOM(G,prod)M)))
  CLASSICAL_SND_ISO_THM
    |- NORMAL(G,prod)N /\ NORMAL(G,prod)M /\ (!n. N n ==> M n) ==>
       (?f.
         GP_ISO
         (quot_set
          (quot_set(G,prod)N,quot_prod(G,prod)N)
          (quot_set(M,prod)N),
          quot_prod
          (quot_set(G,prod)N,quot_prod(G,prod)N)
          (quot_set(M,prod)N))
         (quot_set(G,prod)M,quot_prod(G,prod)M)
         f)
  THIRD_ISO_THM
    |- SUBGROUP(G,prod)H /\ NORMAL(G,prod)N ==>
       GP_ISO
       (quot_set(H,prod)(\x. H x /\ N x),
        quot_prod(H,prod)(\x. H x /\ N x))
       (quot_set(set_prod(G,prod)H N,prod)N,
        quot_prod(set_prod(G,prod)H N,prod)N)
       (quot_hom
        (H,prod)
        (quot_set(set_prod(G,prod)H N,prod)N,
         quot_prod(set_prod(G,prod)H N,prod)N)
        (\x. H x /\ N x)
        (NAT_HOM(set_prod(G,prod)H N,prod)N))
  CLASSICAL_THIRD_ISO_THM
    |- SUBGROUP(G,prod)H /\ NORMAL(G,prod)N ==>
       (?f.
         GP_ISO
         (quot_set(H,prod)(\x. H x /\ N x),
          quot_prod(H,prod)(\x. H x /\ N x))
         (quot_set(set_prod(G,prod)H N,prod)N,
          quot_prod(set_prod(G,prod)H N,prod)N)
         f)
  
******************** more_gp ********************

\end{verbatim}

\subsection{The integers as a group}
Included here is the result of a development of the theory of the integers
in HOL.

\begin{verbatim}
print_theory `integer`;;
The Theory integer
Parents --  HOL     more_arith     elt_gp     
Types --  ":integer"     
Constants --
  plus ":integer -> (integer -> integer)"
  minus ":integer -> (integer -> integer)"
  times ":integer -> (integer -> integer)"
  below ":integer -> (integer -> bool)"
  is_integer ":num # num -> bool"
  REP_integer ":integer -> num # num"
  ABS_integer ":num # num -> integer"     INT ":num -> integer"
  proj ":num # num -> integer"     neg ":integer -> integer"
  POS ":integer -> bool"     NEG ":integer -> bool"     
Curried Infixes --
  plus ":integer -> (integer -> integer)"
  minus ":integer -> (integer -> integer)"
  times ":integer -> (integer -> integer)"
  below ":integer -> (integer -> bool)"     
Definitions --
  IS_INTEGER_DEF  |- !X. is_integer X = (?p. X = p,0) \/ (?n. X = 0,n)
  integer_AXIOM  |- ?rep. TYPE_DEFINITION is_integer rep
  REP_integer
    |- REP_integer =
       (@rep.
         (!x' x''. (rep x' = rep x'') ==> (x' = x'')) /\
         (!x. is_integer x = (?x'. x = rep x')))
  ABS_integer  |- !x. ABS_integer x = (@x'. x = REP_integer x')
  INT_DEF  |- !p. INT p = (@N. p,0 = REP_integer N)
  PROJ_DEF
    |- !p n.
        proj(p,n) =
        (n < p => 
         (@K1. REP_integer K1 = p - n,0) | 
         (@K1. REP_integer K1 = 0,n - p))
  PLUS_DEF
    |- !M N.
        M plus N =
        proj
        ((FST(REP_integer M)) + (FST(REP_integer N)),
         (SND(REP_integer M)) + (SND(REP_integer N)))
  neg_DEF  |- neg = INV((\N. T),$plus)
  MINUS_DEF  |- !M N. M minus N = M plus (neg N)
  TIMES_DEF
    |- !M N.
        M times N =
        proj
        (((FST(REP_integer M)) * (FST(REP_integer N))) +
         ((SND(REP_integer M)) * (SND(REP_integer N))),
         ((FST(REP_integer M)) * (SND(REP_integer N))) +
         ((SND(REP_integer M)) * (FST(REP_integer N))))
  POS_DEF  |- !M. POS M = (?n. M = INT(SUC n))
  NEG_DEF  |- !M. NEG M = POS(neg M)
  BELOW_DEF  |- !M N. M below N = POS(N minus M)

Theorems --
  INT_ONE_ONE  |- !m n. (INT m = INT n) = (m = n)
  NUM_ADD_IS_INT_ADD  |- !m n. (INT m) plus (INT n) = INT(m + n)
  ASSOC_PLUS  |- !M N P. M plus (N plus P) = (M plus N) plus P
  COMM_PLUS  |- !M N. M plus N = N plus M
  PLUS_ID  |- !M. (INT 0) plus M = M
  PLUS_INV  |- !M. ?N. N plus M = INT 0
  integer_as_GROUP  |- GROUP((\N. T),$plus)
  ID_EQ_0  |- ID((\N. T),$plus) = INT 0
\end{verbatim}
With these definitions and theorems, we are now in a position to instantiate
the theory of groups with the particular example of the integers.  The theorems
{\tt PLUS\_ID} and {\tt PLUS\_INV} allow us to automatically rewrite the
instantiated theory in a form that is more traditional.  The resulting theory
is listed below.
\begin{verbatim}
  PLUS_GROUP_ASSOC  |- !x y z. (x plus y) plus z = x plus (y plus z)
  PLUS_ID_LEMMA
    |- (!x. (INT 0) plus x = x) /\
       (!x. x plus (INT 0) = x) /\
       (!x. ?y. y plus x = INT 0)
  PLUS_LEFT_RIGHT_INV
    |- !x y. (y plus x = INT 0) ==> (x plus y = INT 0)
  PLUS_INV_LEMMA
    |- !x. ((neg x) plus x = INT 0) /\ (x plus (neg x) = INT 0)
  PLUS_LEFT_CANCELLATION  |- !x y z. (x plus y = x plus z) ==> (y = z)
  PLUS_RIGHT_CANCELLATION  |- !x y z. (y plus x = z plus x) ==> (y = z)
  PLUS_RIGHT_ONE_ONE_ONTO
    |- !x y. ?z. (x plus z = y) /\ (!u. (x plus u = y) ==> (u = z))
  PLUS_LEFT_ONE_ONE_ONTO
    |- !x y. ?z. (z plus x = y) /\ (!u. (u plus x = y) ==> (u = z))
  PLUS_UNIQUE_ID
    |- !e. (?x. e plus x = x) \/ (?x. x plus e = x) ==> (e = INT 0)
  PLUS_UNIQUE_INV  |- !x u. (u plus x = INT 0) ==> (u = neg x)
  PLUS_INV_INV_LEMMA  |- !x. neg(neg x) = x
  PLUS_DIST_INV_LEMMA  |- !x y. (neg x) plus (neg y) = neg(y plus x)
\end{verbatim}  
Using the computational theory inherited from the first order group
theory, we can more readily proceed to develop more of the standard theory
of the integers.  Below is listed the theorems that were proven to extend
the theory to include various order theoretic facts about the integers.
\begin{verbatim}
  neg_PLUS_DISTRIB  |- neg(M plus N) = (neg M) plus (neg N)
  PLUS_IDENTITY  |- !M. (M plus (INT 0) = M) /\ ((INT 0) plus M = M)
  PLUS_INVERSE
    |- !M. (M plus (neg M) = INT 0) /\ ((neg M) plus M = INT 0)
  NEG_NEG_IS_IDENTITY  |- !M. neg(neg M) = M
  NUM_MULT_IS_INT_MULT  |- !m n. (INT m) times (INT n) = INT(m * n)
  ASSOC_TIMES  |- !M N P. M times (N times P) = (M times N) times P
  COMM_TIMES  |- !M N. M times N = N times M
  TIMES_IDENTITY  |- !M. (M times (INT 1) = M) /\ ((INT 1) times M = M)
  RIGHT_PLUS_DISTRIB
    |- !M N P. (M plus N) times P = (M times P) plus (N times P)
  LEFT_PLUS_DISTRIB
    |- !M N P. M times (N plus P) = (M times N) plus (M times P)
  TIMES_ZERO
    |- !M. (M times (INT 0) = INT 0) /\ ((INT 0) times M = INT 0)
  TIMES_neg
    |- (!M N. M times (neg N) = neg(M times N)) /\
       (!M N. (neg M) times N = neg(M times N))
  neg_IS_TIMES_neg1  |- !M. neg M = M times (neg(INT 1))
  TRICHOTOMY
    |- !M.
        (POS M \/ NEG M \/ (M = INT 0)) /\
        ~(POS M /\ NEG M) /\
        ~(POS M /\ (M = INT 0)) /\
        ~(NEG M /\ (M = INT 0))
  NON_NEG_INT_IS_NUM  |- !N. ~NEG N = (?n. N = INT n)
  INT_CASES  |- !P. (!m. P(INT m)) /\ (!m. P(neg(INT m))) ==> (!M. P M)
  NUM_LESS_IS_INT_BELOW  |- !m n. m < n = (INT m) below (INT n)
  ANTISYM  |- !M. ~M below M
  TRANSIT  |- !M N P. M below N /\ N below P ==> M below P
  COMPAR  |- !M N. M below N \/ N below M \/ (M = N)
  DOUBLE_INF  |- !M. (?N. N below M) /\ (?P. M below P)
  PLUS_BELOW_TRANSL  |- !M N P. M below N = (M plus P) below (N plus P)
  neg_REV_BELOW  |- !M N. (neg M) below (neg N) = N below M
  DISCRETE
    |- !S1.
        (?M. S1 M) ==>
        ((?K1. !N. N below K1 ==> ~S1 N) ==>
         (?M1. S1 M1 /\ (!N1. N1 below M1 ==> ~S1 N1))) /\
        ((?K1. !N. K1 below N ==> ~S1 N) ==>
         (?M1. S1 M1 /\ (!N1. M1 below N1 ==> ~S1 N1)))
         (?M1. S1 M1 /\ (!N1. M1 below N1 ==> ~S1 N1)))
  POS_MULT_PRES_BELOW
    |- !M N P. POS M ==> (N below P = (M times N) below (M times P))
  NEG_MULT_REV_BELOW
    |- !M N P. NEG M ==> (N below P = (M times P) below (M times N))
  POS_IS_ZERO_BELOW  |- !N. POS N = (INT 0) below N
  NEG_IS_BELOW_ZERO  |- !N. NEG N = N below (INT 0)
  neg_ONE_ONE  |- !x y. (neg x = neg y) = (x = y)
  neg_ZERO  |- neg(INT 0) = INT 0
  INT_INTEGRAL_DOMAIN
    |- !x y. (x times y = INT 0) ==> (x = INT 0) \/ (y = INT 0)
  TIMES_LEFT_CANCELLATION
    |- !x y z. ~(x = INT 0) ==> (x times y = x times z) ==> (y = z)
  TIMES_RIGHT_CANCELLATION
    |- !x y z. ~(x = INT 0) ==> (y times x = z times x) ==> (y = z)
  
******************** integer ********************

() : void

\end{verbatim}

Although the theory of the integers was developed through a particular
representation, the set of definitions and theorems whose statements
do not mention this representation are sufficient to characterize the
integers.


\section{General-purpose tactics}

\DOC{SUPPOSE\_TAC}
\vspace*{-3mm}{\small\begin{verbatim}
SUPPOSE_TAC : term -> tactic
\end{verbatim}}\vspace*{-3mm}

\SYNOPSIS
A tactic for adding an assumption to a goal, with the condition
that the assumption be proved later.



\DESCRIBE
The tactic {\small\verb%SUPPOSE_TAC t%} when applied to a goal {\small\verb%([a1;...;an],Goal)%}
returns the two subgoals
\vspace*{-3mm}{\small\begin{verbatim}
   ([t;a1;...;an],Goal)

   ([a1;...;an],t)
\end{verbatim}}\vspace*{-3mm}


\FAILURE
The tactic {\small\verb%SUPPOSE_TAC%} fails if it is not given a term of type {\small\verb%":bool"%}.


\EXAMPLE
\vspace*{-3mm}{\small\begin{verbatim}
   SUPPOSE_TAC "?n. INT n = MIN"
\end{verbatim}}\vspace*{-3mm}
\noindent when applied to the goal
\vspace*{-3mm}{\small\begin{verbatim}
   (["POS MIN"], "?x:num. Y = ((MAX times (INT x)) plus REM)")
\end{verbatim}}\vspace*{-3mm}
returns the two subgoals
\vspace*{-3mm}{\small\begin{verbatim}
   (["?n. INT n = MIN"; "POS MIN"], "?x. Y = ((MAX times (INT x)) plus REM)")

   (["POS MIN"], "?n. INT n = MIN")
\end{verbatim}}\vspace*{-3mm}

\USES
Adding new assumptions to a goal to assists in its proof.

\SEEALSO
\vspace*{-3mm}{\small\begin{verbatim}
REV_SUPPOSE_TAC, ASSUME_TAC
\end{verbatim}}\vspace*{-3mm}

\ENDDOC

\DOC{REV\_SUPPOSE\_TAC}
\vspace*{-3mm}{\small\begin{verbatim}
REV_SUPPOSE_TAC : term -> tactic
\end{verbatim}}\vspace*{-3mm}

\SYNOPSIS
A tactic for adding an assumption to a goal, with the condition
that the assumption be proved first.

\DESCRIBE
The tactic {\small\verb%REV_SUPPOSE_TAC t%} when applied to a goal {\small\verb%([a1;...;an],Goal)%}
returns the two subgoals
\vspace*{-3mm}{\small\begin{verbatim}
   ([a1;...;an],t)

   ([t;a1;...;an],Goal)
\end{verbatim}}\vspace*{-3mm}
\noindent The differnce between {\small\verb%REV_SUPPOSE_TAC%} and {\small\verb%SUPPOSE_TAC%} is the
order in which they reurn the subgoals.

\FAILURE
The tactic {\small\verb%REV_SUPPOSE_TAC%} fails if it is not given a term of type
{\small\verb%":bool"%}.


\EXAMPLE
\vspace*{-3mm}{\small\begin{verbatim}
   REV_SUPPOSE_TAC "?n. INT n = MIN"
\end{verbatim}}\vspace*{-3mm}
\noindent when applied to the goal
\vspace*{-3mm}{\small\begin{verbatim}
   (["POS MIN"], "?x:num. Y = ((MAX times x) plus REM)")
\end{verbatim}}\vspace*{-3mm}
returns the two subgoals
\vspace*{-3mm}{\small\begin{verbatim}
   (["POS MIN"], "?n. INT n = MIN")

   (["?n. INT n = MIN"; "POS MIN"], "?x. Y = ((MAX times x) plus REM)")
\end{verbatim}}\vspace*{-3mm}

\USES
Adding new assumptions to a goal to assists in its proof.

\SEEALSO
\vspace*{-3mm}{\small\begin{verbatim}
SUPPOSE_TAC, ASSUME_TAC
\end{verbatim}}\vspace*{-3mm}

\DOC{ASSUME\_LIST\_TAC}
\vspace*{-3mm}{\small\begin{verbatim}
ASSUME_LIST_TAC : thm list -> tactic
\end{verbatim}}\vspace*{-3mm}

\SYNOPSIS
A tactic for addin a list of theorems as assumptions to a goal.

\DESCRIBE
The tactic {\small\verb%ASSUME_LIST_TAC [thm1;...;thmn]%} when applied to a goal
{\small\verb%([a1;...;am],Goal)%} returns the subgoal {\small\verb%([a1;...;am;thm1;...;thmn],Goal)%}.
If {\small\verb%hyp%} is a hypothesis of one of the theorems {\small\verb%thmi%}, and {\small\verb%hyp%} is
not among the assumptions {\small\verb%[a1;...;am]%} then the subgoal
{\small\verb%([a1;...;am;...],hyp)%} is also returned.

\EXAMPLE
\vspace*{-3mm}{\small\begin{verbatim}
   ASSUME_LIST_TAC [integer_as_GROUP;int_mod_as_GROUP]
\end{verbatim}}\vspace*{-3mm}
where
\vspace*{-3mm}{\small\begin{verbatim}
   integer_as_GROUP = |- GROUP((\N. T),$plus)
   int_mod_as_GROUP = |- GROUP(int_mod n,plus_mod n)
\end{verbatim}}\vspace*{-3mm}
when applied to the goal
\vspace*{-3mm}{\small\begin{verbatim}
   ([],"ID(int_mod n,plus_mod n) = mod n(INT 0)")
\end{verbatim}}\vspace*{-3mm}
returns the subgoal
\vspace*{-3mm}{\small\begin{verbatim}
   (["GROUP(int_mod n,plus_mod n)"; "GROUP((\N. T),$plus)"],
    "ID(int_mod n,plus_mod n) = mod n(INT 0)")
\end{verbatim}}\vspace*{-3mm}
\USES
Adding a collection of standard facts to the assumptions of a goal
so that hypotheses of theorems used in proving the goal will already
be among the assumptions.

\SEEALSO
\vspace*{-3mm}{\small\begin{verbatim}
ASSUME_TAC, SUPPOSE_TAC, REV_SUPPOSE_TAC
\end{verbatim}}\vspace*{-3mm}

\ENDDOC

\DOC{NEW\_SUBST1\_TAC}
\vspace*{-3mm}{\small\begin{verbatim}
NEW_SUBST1_TAC : thm -> tactic
\end{verbatim}}\vspace*{-3mm}

\SYNOPSIS
Substitutes all occurrences of an expression within a goal for an
equal expression.

\DESCRIBE
The tactic {\small\verb%NEW_SUBST1_TAC thm%} where {\small\verb%thm = |- exp1 = exp2%},
when applied to a goal {\small\verb%([a1;...;an],P(exp1))%} where {\small\verb%exp1%} does not
occur in {\small\verb%P%}, returns the subgoal {\small\verb%([a1;...;an],P(exp2))%}.  If {\small\verb%hyp%}
is a hypothesis of {\small\verb%thm%} which is not among the assumptions
{\small\verb%[a1,...,an]%}, then the subgoal {\small\verb%([a1,...,an],hyp)%} is also returned.

\FAILURE
The tactic {\small\verb%NEW_SUBST1_TAC%} fails if the conclusionof the theorem it
is given is not an equation.


\EXAMPLE
\vspace*{-3mm}{\small\begin{verbatim}
   NEW_SUBST1_TAC
    (UNDISCH (SPECL ["neg((Y times QUOT) plus REM)"; "X:integer"]
      PLUS_UNIQUE_INV))
\end{verbatim}}\vspace*{-3mm}
where
\vspace*{-3mm}{\small\begin{verbatim}
   PLUS_UNIQUE_INV = |- !x u. (u plus x = INT 0) ==> (u = neg x)
\end{verbatim}}\vspace*{-3mm}
when applied to the goal
\vspace*{-3mm}{\small\begin{verbatim}
   (["X minus ((Y times QUOT) plus REM) = INT 0"],
    "MIN times X = TOP plus (MIN times REM)")
\end{verbatim}}\vspace*{-3mm}
returns the subgoals
\vspace*{-3mm}{\small\begin{verbatim}
   (["X plus (neg((Y times QUOT) plus REM)) = INT 0";
     "X minus ((Y times QUOT) plus REM) = INT 0"],
    "MIN times (neg(neg((Y times QUOT) plus REM))) =
     TOP plus (MIN times REM)");

   (["X minus ((Y times QUOT) plus REM) = INT 0"],
    "X plus (neg((Y times QUOT) plus REM)) = INT 0")
\end{verbatim}}\vspace*{-3mm}

\USES
For use when you need to carefully control the rewritng of a goal.

\SEEALSO
\vspace*{-3mm}{\small\begin{verbatim}
SUBST1_TAC, SUBST_TAC, PURE_ONCE_REWRITE_TAC, REWRIET_TAC, SUBST_MATCH_TAC
\end{verbatim}}\vspace*{-3mm}

\ENDDOC

\DOC{SUBST\_MATCH\_TAC}
\vspace*{-3mm}{\small\begin{verbatim}
SUBST_MATCH_TAC : thm -> tactic
\end{verbatim}}\vspace*{-3mm}

\SYNOPSIS
Rewriting with a single theorem, particularly when the rewrite theorem
has hypotheses.

\DESCRIBE
The tactic {\small\verb%SUBST_MATCH_TAC thm%} strips the theorem {\small\verb%thm%} to find an
equation {\small\verb%lhs = rhs%} then looks for a match bewteen {\small\verb%lhs%} and the
subterms of the goal.  Once a match is found the thmeorem is
instantiated to the particular instance found, and {\small\verb%NEW_SUBST1_TAC%}
is used to write the goal with the result.


\FAILURE
The tactic {\small\verb%SUBST_MATCH_TAC thm%} will fail if either {\small\verb%thm%} does not
strip to and equation, or if no match is found with the left hand side
of the equation.

\EXAMPLE
The tactic
\vspace*{-3mm}{\small\begin{verbatim}
   SUBST_MATCH_TAC (SYM (UNDISCH SBGP_ID_GP_ID))
\end{verbatim}}\vspace*{-3mm}
where 
\vspace*{-3mm}{\small\begin{verbatim}
   SBGP_ID_GP_ID = |- SUBGROUP(G,prod)H ==> (ID(H,prod) = ID(G,prod))
\end{verbatim}}\vspace*{-3mm}
when applied to the goal
\vspace*{-3mm}{\small\begin{verbatim}
   (["SUBGROUP((\N. T),$plus)H"],"H(ID((\N. T),$plus))")
\end{verbatim}}\vspace*{-3mm}
returns the subgoal
\vspace*{-3mm}{\small\begin{verbatim}
   (["SUBGROUP((\N. T),$plus)H"],"H(ID(H,$plus))")
\end{verbatim}}\vspace*{-3mm}

\USES
Rewriting with a theorem that has hypotheses tha need to be
instantiated by the matching of the rewriting.

\SEEALSO
\vspace*{-3mm}{\small\begin{verbatim}
NEW_SUBST1_TAC, PURE_ONCE_REWRITE_TAC
\end{verbatim}}\vspace*{-3mm}

\ENDDOC

\DOC{MP\_IMP\_TAC}
\vspace*{-3mm}{\small\begin{verbatim}
MP_IMP_TAC
\end{verbatim}}\vspace*{-3mm}

\SYNOPSIS
The tactic {\small\verb%MP_IMP_TAC thm%} uses modus ponens with the theorem {\small\verb%thm%}
to reduce a goal.

\DESCRIBE
The tactic
\vspace*{-3mm}{\small\begin{verbatim}
   MP_IMP_TAC thm
\end{verbatim}}\vspace*{-3mm}
where {\small\verb%thm%} is a theorem of the form {\small\verb%|- A ==> B%}, when applied to the
goal {\small\verb%([a1;...;an],B)%} returns the subgoal {\small\verb%([a1;...;an],A)%}.  If {\small\verb%hyp%}
is a hypothesis of {\small\verb%thm%} which does not occor among the assumptions
{\small\verb%[a1;...;an]%}, then the subgoal {\small\verb%([a1;...;an],hyp)%} is also returned.


\FAILURE
The tactic {\small\verb%MP_IMP_TAC thm%} fails if either the conclusion of {\small\verb%thm%} is
not an implication, or if the consequent of the implication is not the
statment of the goal.


\EXAMPLE
The tactic
\vspace*{-3mm}{\small\begin{verbatim}
   MP_IMP_TAC 
    (SPECL ["N:integer";"N':integer";"N' times (INT(SUC n))"] TRANSIT)
\end{verbatim}}\vspace*{-3mm}
where
\vspace*{-3mm}{\small\begin{verbatim}
   TRANSIT = |- !M N P. M below N /\ N below P ==> M below P
\end{verbatim}}\vspace*{-3mm}
when applied to the goal
\vspace*{-3mm}{\small\begin{verbatim}
   (["N below N'"],"N below (N' times (INT(SUC n)))")
\end{verbatim}}\vspace*{-3mm}
returns the subgoal
\vspace*{-3mm}{\small\begin{verbatim}
   (["N below N'"], "N below N' /\ N' below (N' times (INT(SUC n)))")
\end{verbatim}}\vspace*{-3mm}

\USES
Using a theorem to reduce a goal using modus ponens.

\SEEALSO
\vspace*{-3mm}{\small\begin{verbatim}
MATCH_MP_IMP_TAC, REDUCE_TAC, REST_TAC
\end{verbatim}}\vspace*{-3mm}

\ENDDOC

\DOC{MATCH\_MP\_IMP\_TAC}
\vspace*{-3mm}{\small\begin{verbatim}
MATCH_MP_IMP_TAC : thm -> tactic
\end{verbatim}}\vspace*{-3mm}

\SYNOPSIS
Reduces a goal using modus ponens and a theorem which is an implication
whose antecedent matches the statement of the goal.

\DESCRIBE
The tactic {\small\verb%MATCH_IMP_IMP_TAC thm%} where the statement of {\small\verb%thm%} is a
(possibly universally quantified) implication {\small\verb%A ==> B%}, when applied to
a goal {\small\verb%([a1;...;an],B')%}, where {\small\verb%B'%} is an instance of {\small\verb%B%}, returns the
subgoal {\small\verb%([a1;...;an],A')%} where {\small\verb%A'%} is the corresponding instance of {\small\verb%A%}.
If {\small\verb%thm%} has a hypothesis {\small\verb%hyp%} which does not occur among the assumptions
{\small\verb%[a1;...;an]%}, then the subgoal {\small\verb%([a1;...;an],hyp)%} is also returned.

\FAILURE
The tactic {\small\verb%MATCH_MP_IMP_TAC thm%} fails if either the conclusion of
{\small\verb%thm%} is not a (possibliy universally quantified) implication, or if
the consequent of the implication does not match the statment of the goal.


\EXAMPLE
The tactic
\vspace*{-3mm}{\small\begin{verbatim}
   MATCH_MP_IMP_TAC INT_SBGP_NORMAL
\end{verbatim}}\vspace*{-3mm}
where
\vspace*{-3mm}{\small\begin{verbatim}
   INT_SUBGROUP_NORMAL =
     |- !H. SUBGROUP((\N. T),$plus)H ==> NORMAL((\N. T),$plus)H
\end{verbatim}}\vspace*{-3mm}
when applied to the goal
\vspace*{-3mm}{\small\begin{verbatim}
   ([],"NORMAL((\N. T),$plus)(int_mult_set n)")
\end{verbatim}}\vspace*{-3mm}
returns the subgoal
\vspace*{-3mm}{\small\begin{verbatim}
   ([],"SUBGROUP((\N. T),$plus)(int_mult_set n)")
\end{verbatim}}\vspace*{-3mm}


\USES
Reducing goals using modus ponens with a theorem as a template.
{\small\verb%MATCH_MP_IMP_TAC%} allows theorems to act as specialized tactics.

\SEEALSO
\vspace*{-3mm}{\small\begin{verbatim}
MP_IMP_TAC, REDUCE_TAC, RES_TAC
\end{verbatim}}\vspace*{-3mm}

\ENDDOC

\DOC{REDUCE\_TAC}
\vspace*{-3mm}{\small\begin{verbatim}
REDUCE_TAC : thm list -> tactic
\end{verbatim}}\vspace*{-3mm}

\SYNOPSIS
Repeated reduces a goal using modus ponens and the given list of theorems.

\DESCRIBE
The tactic {\small\verb%REDUCE_TAC thm_list%} repeatedly reduces a goal by
stripping and using modus ponens with any theorem from {\small\verb%thm_list%}
which is an implication and whose implication conclusion matches the
goal statement.  It also solves those subgoals which match any of the
given theorems, or which are included among the assumptions.

\EXAMPLE
Then tactic
\vspace*{-3mm}{\small\begin{verbatim}
   REDUCE_TAC [CLOSURE; INV_CLOSURE]
\end{verbatim}}\vspace*{-3mm}
where
\vspace*{-3mm}{\small\begin{verbatim}
  CLOSURE = |- GROUP(G,prod) ==> (!x y. G x /\ G y ==> G(prod x y))
  INV_LEMMA =
    |- GROUP(G,prod) ==>
        (!x. G x ==>
             (prod(INV(G,prod)x)x = ID(G,prod)) /\
             (prod x(INV(G,prod)x) = ID(G,prod)))
\end{verbatim}}\vspace*{-3mm}
when applied to the goal
\vspace*{-3mm}{\small\begin{verbatim}
   (["GROUP(Mat,comb)"; "Mat Y"; "Mat X"],
    "Mat(INV(Mat,comb)(comb X(INV(Mat,comb)Y)))")
\end{verbatim}}\vspace*{-3mm}
solves the goal.

\USES
The tactic {\small\verb%REDUCE_TAC%} when used with a collection theorems that
act as reductions for a theory can finish routine goals for the thoery.

\SEEALSO
\vspace*{-3mm}{\small\begin{verbatim}
MP_IMP_TAC, MATCH_MP_IMP_TAC, ASM_REWRITE_TAC
\end{verbatim}}\vspace*{-3mm}

\ENDDOC

\DOC{EXT\_TAC}
\vspace*{-3mm}{\small\begin{verbatim}
EXT_TAC : term -> tactic
\end{verbatim}}\vspace*{-3mm}

\SYNOPSIS
The tactic {\small\verb%EXT_TAC%} reduces a goal of showing that two functions are
equal to showing that they are equal on all values.

\DESCRIBE
The tactic {\small\verb%EXT_TAC "x"%} when aplied to a goal {\small\verb%[a1;...;an],"f = g"%},
where {\small\verb%x%} is a variable of the same type as the common domain of {\small\verb%f%}
and {\small\verb%g%}, returns the subgoal {\small\verb%([a1;...;an],"!x. f x = g x")%}.

\FAILURE
The tactic {\small\verb%EXT_TAC var%} fails if {\small\verb%var%} is not a term variable, or if
the tactic is applied to a goal that is not the equation of two functions
with common domain type the same as the type of {\small\verb%var%}.

\EXAMPLE
The tactic
\vspace*{-3mm}{\small\begin{verbatim}
   EXT_TAC "N:integer"
\end{verbatim}}\vspace*{-3mm}
when applied to the goal
\vspace*{-3mm}{\small\begin{verbatim}
   ([],"Mult = \x. ?m. (x = m times c)")
\end{verbatim}}\vspace*{-3mm}
returns the subgoal
\vspace*{-3mm}{\small\begin{verbatim}
([], "!N. Mult N = (\x. ?m. x = m times c)N")
\end{verbatim}}\vspace*{-3mm}
which then can be further reduced by applications of {\small\verb%BETA_TAC%},
{\small\verb%GEN_TAC%}, and {\small\verb%EQ_TAC%}.

\USES
Reducing the problem of showing that two functions are equal to an
elementwise argument.

\ENDDOC





\section{Specialized tactics and functions}
\subsection{Group theory}
\DOC{GROUP\_TAC}
\vspace*{-3mm}{\small\begin{verbatim}
GROUP_TAC : thm list -> tactic
\end{verbatim}}\vspace*{-3mm}

\SYNOPSIS
A tactic for solving or reducing goals of group membership.

\DESCRIBE
The tactic
\vspace*{-3mm}{\small\begin{verbatim}
   GROUP_TAC thm_list
\end{verbatim}}\vspace*{-3mm}
\noindent repeatedly reduces the goal using matching and reverse modus
ponens with the assumptions, the theorems from {\small\verb%elt_gp.th%} which express
closure facts, and the theorem in {\small\verb%thm_list%}.  It returns those subgoal
which cannot be further reduced with the given facts.

\EXAMPLE
\vspace*{-3mm}{\small\begin{verbatim}
   GROUP_TAC [INT_SBGP_neg;INT_SBGP_TIMES_CLOSED]
\end{verbatim}}\vspace*{-3mm}
\noindent applied to the goal
\vspace*{-3mm}{\small\begin{verbatim}
   (["GROUP(H,$plus)"; "H N"; "H MIN"], "H(N plus (neg(MAX times MIN)))")
\end{verbatim}}\vspace*{-3mm}
\noindent where
\vspace*{-3mm}{\small\begin{verbatim}
   INT_SBGP_neg =
     |- !H. SUBGROUP((\N. T),$plus)H ==> (!N. H N ==> H(neg N))

   INT_SBGP_TIMES_CLOSED =
     |- !H. SUBGROUP((\N. T),$plus)H ==> (!m p. H p ==> H(m times p))
\end{verbatim}}\vspace*{-3mm}
\noindent solves the goal, whereas
\vspace*{-3mm}{\small\begin{verbatim}
   GROUP_ELT_TAC
\end{verbatim}}\vspace*{-3mm}
\noindent returns the subgoal
\vspace*{-3mm}{\small\begin{verbatim}
   (["GROUP(H,$plus)"; "H N"; "H MIN"], "H(neg(MAX times MIN))")
\end{verbatim}}\vspace*{-3mm}

\USES
Reducing a goal of showing that a compound element is in a group using
standard group theory closure facts, and any others added by the user.


\SEEALSO
\vspace*{-3mm}{\small\begin{verbatim}
REDUCE_TAC, GROUP_ELT_TAC
\end{verbatim}}\vspace*{-3mm}

\DOC{GROUP\_ELT\_TAC}
\vspace*{-3mm}{\small\begin{verbatim}
GROUP_ELT_TAC : tactic
\end{verbatim}}\vspace*{-3mm}

\SYNOPSIS
A tactic to solve or reduce routine goals of group membership.

\DESCRIBE
The tactic
\vspace*{-3mm}{\small\begin{verbatim}
   GROUP_ELT_TAC
\end{verbatim}}\vspace*{-3mm}
\noindent has the same effect as
\vspace*{-3mm}{\small\begin{verbatim}
   GROUP_TAC []
\end{verbatim}}\vspace*{-3mm}

\EXAMPLE
\vspace*{-3mm}{\small\begin{verbatim}
   GROUP_ELT_TAC
\end{verbatim}}\vspace*{-3mm}
\noindent applied to the goal
\vspace*{-3mm}{\small\begin{verbatim}
   (["GROUP(H,$plus)"; "H N"; "H MIN"],
    "H(N plus (INV(H,$plus)(MAX times MIN)))")
\end{verbatim}}\vspace*{-3mm}
\noindent returns the goal
\vspace*{-3mm}{\small\begin{verbatim}
   (["GROUP(H,$plus)"; "H N"; "H MIN"], "H(MAX times MIN)")
\end{verbatim}}\vspace*{-3mm}

\USES
Solving or reducing routine goals of group membership.

\SEEALSO
\vspace*{-3mm}{\small\begin{verbatim}
REDUCE_TAC, GROUP_TAC
\end{verbatim}}\vspace*{-3mm}

\ENDDOC

\DOC{GROUP\_RIGHT\_ASSOC\_TAC}
\vspace*{-3mm}{\small\begin{verbatim}
GROUP_RIGHT_ASSOC_TAC : term -> tactic
\end{verbatim}}\vspace*{-3mm}

\SYNOPSIS
Reassociate a subterm from left to right.

\DESCRIBE
The tactic
\vspace*{-3mm}{\small\begin{verbatim}
   GROUP_RIGHT_ASSOC_TAC tm
\end{verbatim}}\vspace*{-3mm}
\noindent rewrites a goal {\small\verb%P(tm)%} into {\small\verb%P(tm')%} where {\small\verb%tm%} is of the form
{\small\verb%(prod (prod a b) c)%} for some product {\small\verb%prod%} and terms {\small\verb%a%}, {\small\verb%b%} and {\small\verb%c%},
and {\small\verb%tm'%} is {\small\verb%(prod a (prod b c))%} provided the goal has an assumption
of the form {\small\verb%GROUP(G,prod)%}.  {\small\verb%GROUP_RIGHT_ASSOC_TAC%} uses {\small\verb%GROUP_ELT_TAC%}
to handle the group membership requirements which arise.

\FAILURE
The tactic {\small\verb%GROUP_RIGHT_ASSOC_TAC%} fails if it is not given a term of
the form {\small\verb%(prod (prod a b) c)%}, or if it does not find an assumption
of the form {\small\verb%GROUP(G,prod)%}.

\EXAMPLE
\vspace*{-3mm}{\small\begin{verbatim}
   GROUP_RIGHT_ASSOC_TAC "comb (comb (comb u v) s) t)"
\end{verbatim}}\vspace*{-3mm}
\noindent applied to the goal
\vspace*{-3mm}{\small\begin{verbatim}
   (["GROUP(M,comb)";"M s"; "M t"; "M u"; "M v"],
    "comb(comb(comb(comb u v)s)t)(INV(M,comb)(comb s t)) = comb u v")
\end{verbatim}}\vspace*{-3mm}
\noindent returns the subgoal
\vspace*{-3mm}{\small\begin{verbatim}
   (["M(comb u v)"; "GROUP(M,comb)"; "M s"; "M t"; "M u"; "M v"],
    "comb(comb(comb u v)(comb s t))(INV(M,comb)(comb s t)) = comb u v")
\end{verbatim}}\vspace*{-3mm}

\USES
Careful rewriting of computational expressions.

\SEEALSO
\vspace*{-3mm}{\small\begin{verbatim}
GROUP_LEFT_ASSOC_TAC, INT_RIGHT_ASSOC_TAC, INT_LEFT_ASSOC_TAC
\end{verbatim}}\vspace*{-3mm}

\ENDDOC

\DOC{GROUP\_LEFT\_ASSOC\_TAC}
\vspace*{-3mm}{\small\begin{verbatim}
GROUP_LEFT_ASSOC_TAC : term -> tactic
\end{verbatim}}\vspace*{-3mm}

\SYNOPSIS
Reassociate a subterm from right to left.

\DESCRIBE
The tactic
\vspace*{-3mm}{\small\begin{verbatim}
  GROUP_LEFT_ASSOC_TAC tm
\end{verbatim}}\vspace*{-3mm}
\noindent rewrites a goal {\small\verb%P(tm)%} into {\small\verb%P(tm')%} where {\small\verb%tm%} is of the form
{\small\verb%(prod a (prod b c))%} for some product {\small\verb%prod%} and terms {\small\verb%a%}, {\small\verb%b%} and {\small\verb%c%},
and {\small\verb%tm'%} is {\small\verb%(prod (prod a b) c)%} provided the goal has an assumption
of the form {\small\verb%GROUP(G,prod)%}.  {\small\verb%GROUP_LEFT_ASSOC_TAC%} uses {\small\verb%GROUP_ELT_TAC%}
to handle the group membership requirements which arise.

\FAILURE
The tactic {\small\verb%GROUP_LEFT_ASSOC_TAC%} fails if it is not given a term of
the form {\small\verb%(prod a (prod b c))%}, or if it does not find an assumption
of the form {\small\verb%GROUP(G,prod)%}.

\EXAMPLE
\vspace*{-3mm}{\small\begin{verbatim}
   GROUP_LEFT_ASSOC_TAC "comb u (comb v (comb s t))"
\end{verbatim}}\vspace*{-3mm}
\noindent applied to the goal
\vspace*{-3mm}{\small\begin{verbatim}
   (["GROUP(M,comb)"; "M s"; "M t"; "M u"; "M v"],
    "comb(INV(M,comb)(comb u v))(comb u(comb v(comb s t))) = comb s t")
\end{verbatim}}\vspace*{-3mm}
\noindent returns the subgoal
\vspace*{-3mm}{\small\begin{verbatim}
   (["M(comb s t)"; "GROUP(M,comb)"; "M s"; "M t"; "M u"; "M v"],
    "comb(INV(M,comb)(comb u v))(comb(comb u v)(comb s t)) = comb s t")
\end{verbatim}}\vspace*{-3mm}

\USES
Careful rewriting of computational expressions.

\SEEALSO
\vspace*{-3mm}{\small\begin{verbatim}
GROUP_RIGHT_ASSOC_TAC, INT_RIGHT_ASSOC_TAC, INT_LEFT_ASSOC_TAC
\end{verbatim}}\vspace*{-3mm}

\ENDDOC

\DOC{return\_GROUP\_thm}
\vspace*{-3mm}{\small\begin{verbatim}
return_GROUP_thm : string -> thm -> thm list -> thm
return_GROUP_thm : string -> thm -> proof
\end{verbatim}}\vspace*{-3mm}

\SYNOPSIS
A function for instantiating and simplifying a theorem from {\small\verb%elt_gp.th%}.

\DESCRIBE
The function
\vspace*{-3mm}{\small\begin{verbatim}
   return_GROUP_thm thm_name is_group_thm rewrite_list
\end{verbatim}}\vspace*{-3mm}
\noindent returns the result of instantiating the theorem named {\small\verb%thm_name%}
in the theory {\small\verb%elt_gp.th%} with the group and product given in the theorem
{\small\verb%is_group_thm%}, removing the {\small\verb%GROUP(G,prod)%} hypothesis from it using this
theorem, and rewriting it with {\small\verb%rewrite_list%}.

\FAILURE
The function {\small\verb%return_GROUP_thm%} fails if it is not given a theorem
of the form {\small\verb%|- GROUP(G,prod)%}.

\EXAMPLE
\vspace*{-3mm}{\small\begin{verbatim}
  return_GROUP_thm
    `INV_LEMMA`
    (theorem `integer` `integer_as_GROUP`)
    [(SYM (definition `integer` `neg_DEF`)); (theorem `integer` `ID_EQ_0`)];;
\end{verbatim}}\vspace*{-3mm}
\noindent returns
\vspace*{-3mm}{\small\begin{verbatim}
  |- !x. ((neg x) plus x = INT 0) /\ (x plus (neg x) = INT 0)
\end{verbatim}}\vspace*{-3mm}
\noindent which is what the theorem {\small\verb%INV_LEMMA%} from {\small\verb%elt_gp.th%} says
in the case of the integers.

\USES
Accessing a specific theorem from {\small\verb%elt_gp.th%} reworded in a theory which is an
instance of a group.

\SEEALSO
\vspace*{-3mm}{\small\begin{verbatim}
include_GROUP_thm, return_GROUP_theory, include_GROUP_theory
\end{verbatim}}\vspace*{-3mm}

\ENDDOC

\DOC{include\_GROUP\_thm}
\vspace*{-3mm}{\small\begin{verbatim}
include_GROUP_thm : string -> string -> thm -> thm list -> thm
include_GROUP_thm : string -> string -> thm -> proof
\end{verbatim}}\vspace*{-3mm}

\SYNOPSIS
A function for instantiating and simplifying, and then storing a theorem
from {\small\verb%elt_gp.th%}.

\DESCRIBE
The function
\vspace*{-3mm}{\small\begin{verbatim}
   include_GROUP_thm elt_gp_name new_thm_name is_group_thm rewrite_list
\end{verbatim}}\vspace*{-3mm}
\noindent has the effect of
\vspace*{-3mm}{\small\begin{verbatim}
   save_thm (new_thm_name,
             (return_GROUP_thm elt_gp_name is_group_thm rewrite_list));;
\end{verbatim}}\vspace*{-3mm}

\FAILURE
The function {\small\verb%include_GROUP_thm%} fails either if it is not given a
theorem of the form {\small\verb%|- GROUP(G,prod)%}, causing {\small\verb%return_GROUP_thm%} to
fail, or if it is given a string that is the same as the name of a
previously saved theorem, causing {\small\verb%save_thm%} to fail.

\EXAMPLE
\vspace*{-3mm}{\small\begin{verbatim}
   include_GROUP_thm
     `INV_LEMMA`
     `PLUS_neg_LEMMA`
     (theorem `integer` `integer_as_GROUP`)
     [(SYM (definition `integer` `neg_DEF`)); (theorem `integer` `ID_EQ_0`)];;
\end{verbatim}}\vspace*{-3mm}
\noindent saves the theorem
\vspace*{-3mm}{\small\begin{verbatim}
   |- !x. ((neg x) plus x = INT 0) /\ (x plus (neg x) = INT 0)
\end{verbatim}}\vspace*{-3mm}
\noindent under the name {\small\verb%PLUS_neg_LEMMA%} in the current theory.

\USES
Adding to the current theory a specific theorem for group theory in a
theory which is an instance of a group.

\SEEALSO
\vspace*{-3mm}{\small\begin{verbatim}
return_GROUP_thm, return_GROUP_theory, include_GROUP_theory
\end{verbatim}}\vspace*{-3mm}

\ENDDOC

\DOC{return\_GROUP\_theory}
\vspace*{-3mm}{\small\begin{verbatim}
return_GROUP_theory : string -> thm -> thm list -> (string # thm)list
\end{verbatim}}\vspace*{-3mm}

\SYNOPSIS
A function for instantiating and simplifying all the theorems from
{\small\verb%elt_gp.th%}.


\DESCRIBE
The function
\vspace*{-3mm}{\small\begin{verbatim}
   return_GROUP_theory prefix is_group_thm rewrite_list
\end{verbatim}}\vspace*{-3mm}
\noindent returns the list resulting from of instantiating the each
theorem in the theory {\small\verb%elt_gp.th%} with the group and product given in
the theorem {\small\verb%is_group_thm%}, removing the {\small\verb%GROUP(G,prod)%} hypothesis
from it using this theorem, rewriting it with {\small\verb%rewrite_list%},
and pairing it with its original name prefixed by {\small\verb%prefix%}.

\FAILURE
The function {\small\verb%return_GROUP_theory%} fails if it is not given a theorem
of the form {\small\verb%|- GROUP(G,prod)%}.

\USES
Accessing all the theorems from {\small\verb%elt_gp.th%} reworded in a theory which is an
instance of a group, in order to further modify some of the theorems before
storing the collection in the current theory.

\SEEALSO
\vspace*{-3mm}{\small\begin{verbatim}
return_GROUP_thm, include_GROUP_thm, include_GROUP_theory
\end{verbatim}}\vspace*{-3mm}

\ENDDOC

\DOC{include\_GROUP\_theory}
\vspace*{-3mm}{\small\begin{verbatim}
include_GROUP_theory : string -> thm -> thm list -> thm list
\end{verbatim}}\vspace*{-3mm}

\SYNOPSIS
A function for instantiating and simplifying, and then storing all
the theorems from {\small\verb%elt_gp.th%}.

\DESCRIBE
The function
\vspace*{-3mm}{\small\begin{verbatim}
   include_GROUP_theory prefix is_group_thm rewrite_list
\end{verbatim}}\vspace*{-3mm}
\noindent has the effect of mapping {\small\verb%save_thm%} over the result of
\vspace*{-3mm}{\small\begin{verbatim}
   return_GROUP_theory prefix is_group_thm rewrite_list
\end{verbatim}}\vspace*{-3mm}
\noindent after removing all trivial theorems ({\small\verb%|- T%}) from the list.

\FAILURE
The function {\small\verb%include_GROUP_theory%} fails either if it is not given a
theorem of the form {\small\verb%|- GROUP(G,prod)%}, causing {\small\verb%return_GROUP_theory%}
to fail, or if one of the names in the list returned by
{\small\verb%return_GROUP_theory%} is the same as the name of a previously saved
theorem, causing {\small\verb%save_thm%} to fail.

\USES
Adding to the current theory all the theorems from {\small\verb%elt_gp.th%}, in a form
that is compatible with the current theory.

\SEEALSO
\vspace*{-3mm}{\small\begin{verbatim}
return_GROUP_thm, include_GROUP_thm, return_GROUP_theory
\end{verbatim}}\vspace*{-3mm}

\ENDDOC


\subsection{The integers}

\DOC{INT\_CASES\_TAC}
\vspace*{-3mm}{\small\begin{verbatim}
INT_CASES_TAC : tactic
\end{verbatim}}\vspace*{-3mm}

\SYNOPSIS
A tactic for turning a problem over the integers into two cases over
the natural numbers.

\DESCRIBE
The tactic
\vspace*{-3mm}{\small\begin{verbatim}
   INT_CASES_TAC
\end{verbatim}}\vspace*{-3mm}
\noindent reduces a universally quantified goal over the integers,
\vspace*{-3mm}{\small\begin{verbatim}
   ([a1;...;an], "!n:integer. P n")
\end{verbatim}}\vspace*{-3mm}
\noindent to two subgoals:
\vspace*{-3mm}{\small\begin{verbatim}
   ([a1;...;an], "!n1:num. P (INT n1)")

   (["!n1:num. P (INT n1)";a1;...;an], "!n2:num. P (neg(INT n2))")
\end{verbatim}}\vspace*{-3mm}
\noindent After making this reduction one can proceed by induction on the
natural numbers.


\FAILURE
The tactic {\small\verb%INT_CASES_TAC%} fails if it is applied to a goal that is
not a universal quantification over the type {\small\verb%":integer"%}.

\EXAMPLE
\vspace*{-3mm}{\small\begin{verbatim}
   INT_CASES_TAC
\end{verbatim}}\vspace*{-3mm}
\noindent when applied to the goal
\vspace*{-3mm}{\small\begin{verbatim}
   (["SUBGROUP((\N. T),$plus)H"], "!m p. H p ==> H(m times p)")
\end{verbatim}}\vspace*{-3mm}
\noindent returns the subgoals
\vspace*{-3mm}{\small\begin{verbatim}
   (["SUBGROUP((\N. T),$plus)H"], "!n1 p. H p ==> H((INT n1) times p)")
\end{verbatim}}\vspace*{-3mm}
\noindent and
\vspace*{-3mm}{\small\begin{verbatim}
   (["!n1 p. H p ==> H((INT n1) times p)"; "SUBGROUP((\N. T),$plus)H"],
    "!n2 p. H p ==> H((neg(INT n2)) times p)")
\end{verbatim}}\vspace*{-3mm}

\USES
Reducing a goal over the integers to ones over the natural numbers,
where induction can be applied.

\SEEALSO
\vspace*{-3mm}{\small\begin{verbatim}
SIMPLE_INT_CASES_TAC
\end{verbatim}}\vspace*{-3mm}

\ENDDOC

\DOC{SIMPLE\_INT\_CASES\_TAC}
\vspace*{-3mm}{\small\begin{verbatim}
SIMPLE_INT_CASES_TAC : tactic
\end{verbatim}}\vspace*{-3mm}

\SYNOPSIS
A tactic for splitting a problem over the integers into the three cases
of positive integers, negative integers, and zero.

\DESCRIBE
The tactic
\vspace*{-3mm}{\small\begin{verbatim}
   SIMPLE_INT_CASES_TAC
\end{verbatim}}\vspace*{-3mm}
\noindent reduces a universally quantified goal over the integers,
\vspace*{-3mm}{\small\begin{verbatim}
   ([a1;...;an], "!N:integer. P N")
\end{verbatim}}
\noindent to three subgoals:
\vspace*{-3mm}{\small\begin{verbatim}
   ([a1;...;an], "!N. POS N ==> P N")

   (["!N. POS N ==> P N";a1;...;an], "!N. NEG N ==> P N")

   ([a1;...;an], "P (INT 0)")
\end{verbatim}}\vspace*{-3mm}

\FAILURE
The tactic {\small\verb%SIMPLE_INT_CASES_TAC%} fails if it is applied to a goal
that is not a universal quantification over the type {\small\verb%":integer"%}.

\EXAMPLE
\vspace*{-3mm}{\small\begin{verbatim}
   SIMPLE_INT_CASES_TAC
\end{verbatim}}\vspace*{-3mm}
\noindent when applied to the goal
\vspace*{-3mm}{\small\begin{verbatim}
   (["SUBGROUP((\N. T),$plus)H";
     "~(!m1. H m1 ==> (m1 = INT 0))";
     "!N. N below MIN ==> ~(POS N /\ H N)";
     "POS MIN";
     "H MIN"],
    "!N. H N ==> ?p. N = p times MIN")
\end{verbatim}}\vspace*{-3mm}
\noindent returns the subgoals
\vspace*{-3mm}{\small\begin{verbatim}
   (["SUBGROUP((\N. T),$plus)H";
     "~(!m1. H m1 ==> (m1 = INT 0))";
     "!N. N below MIN ==> ~(POS N /\ H N)";
     "POS MIN";
     "H MIN"],
    "!N. POS N ==> H N ==> ?p. N = p times MIN")

   (["!N. POS N ==> H N ==> ?p. N = p times MIN";
     "SUBGROUP((\N. T),$plus)H";
     "~(!m1. H m1 ==> (m1 = INT 0))";
     "!N. N below MIN ==> ~(POS N /\ H N)";
     "POS MIN";
     "H MIN"],
    "!N. NEG N ==> H N ==> ?p. N = p times MIN")

   (["SUBGROUP((\N. T),$plus)H";
     "~(!m1. H m1 ==> (m1 = INT 0))";
     "!N. N below MIN ==> ~(POS N /\ H N)";
     "POS MIN";
     "H MIN"],
    "H (INT 0) ==> ?p. INT 0 = p times MIN"
\end{verbatim}}\vspace*{-3mm}
The value for {\small\verb%p%} that is needed for the first of these subgoals is
the greatest integer that can be multiplied times {\small\verb%MIN%} and subtracted
from {\small\verb%N%} leaving a non-negative result.  (This value is not found
by induction (at least not in any direct fashion.)  The value for
{\small\verb%p%} needed for the second subgoal is the negative of the one given
for {\small\verb%neg N%} by the first case.  The value for {\small\verb%p%} in the last case is
(INT 0).

\USES
Reducing a goal over the integers to the cases of positive integers,
negative integers, and the zero case, particularly when you do not want
to reduce the problem to one of induction over the natural numbers.

\SEEALSO
\vspace*{-3mm}{\small\begin{verbatim}
INT_CASES_TAC
\end{verbatim}}\vspace*{-3mm}

\ENDDOC

\DOC{INT\_MIN\_TAC}
\vspace*{-3mm}{\small\begin{verbatim}
INT_MIN_TAC : term -> tactic
\end{verbatim}}\vspace*{-3mm}

\SYNOPSIS
Tactic to add the assumption that {\small\verb%MIN%} is the least element 
given set of integers, and add the additional subgoals to prove
that such a least element exists.

\DESCRIBE
The tactic
\vspace*{-3mm}{\small\begin{verbatim}
   INT_MIN_TAC "S:integer -> bool"
\end{verbatim}}\vspace*{-3mm}
\noindent adds to a goal {\small\verb%([a1;...;an], Goal)%} the assumptions
\vspace*{-3mm}{\small\begin{verbatim}
   S MIN
\end{verbatim}}\vspace*{-3mm}
\noindent and
\vspace*{-3mm}{\small\begin{verbatim}
   !N. N below MIN ==> ~S N
\end{verbatim}}\vspace*{-3mm}
\noindent and returns the additional subgoals:
\vspace*{-3mm}{\small\begin{verbatim}
   ([a1;...;an], "?M:integer. S M")

   (["S M";a1;...;an], "?LB. !N. N below LB ==> ~S N")
\end{verbatim}}\vspace*{-3mm}
\noindent to show {\small\verb%S%} is not empty, and {\small\verb%S%} has a lower bound.

\FAILURE
The tactic {\small\verb%NT_MIN_TAC%} fails if it is not given a term of type
{\small\verb%":integer -> bool"%}.


\EXAMPLE
The tactic
\vspace*{-3mm}{\small\begin{verbatim}
   INT_MIN_TAC "\q. ~NEG (Y minus (q times X))";;
\end{verbatim}}\vspace*{-3mm}
\noindent when applied to the goal
\vspace*{-3mm}{\small\begin{verbatim}
   (["POS X"; "POS Y"],
    "?q r. (Y = (q times X) plus r) /\ r below Y /\ ~NEG r")
\end{verbatim}}\vspace*{-3mm}
\noindent yields the following three subgoals:
\vspace*{-3mm}{\small\begin{verbatim}
   (["~NEG(Y minus (MIN times X))";
     "!N. N below MIN ==> ~~NEG(Y minus (N times X))";
     "POS X";
     "POS Y"],
    "?q r. (Y = (q times X) plus r) /\ r below Y /\ ~NEG r")

   (["POS X"; "POS Y"], "?M. ~NEG(Y minus (M times X))")

   (["~NEG(Y minus (M times X))"; "POS X"; "POS Y"],
    "?LB. !N. N below LB ==> ~~NEG(Y minus (N times X))");;
\end{verbatim}}\vspace*{-3mm}
\noindent The next step would then be to use {\small\verb%EXISTS_TAC "MIN:integer"%} on the
first subgoal.


\USES
Setting up to solve an existence goal, where the solution will be
given by the least element satisfying some property.


\SEEALSO
\vspace*{-3mm}{\small\begin{verbatim}
INT_MAX_TAC
\end{verbatim}}\vspace*{-3mm}

\ENDDOC

\DOC{INT\_MAX\_TAC}
\vspace*{-3mm}{\small\begin{verbatim}
INT_MAX_TAC : term -> tactic
\end{verbatim}}\vspace*{-3mm}

\SYNOPSIS
Tactic to add the assumption that {\small\verb%MAX%} is the greatest element 
given set of integers, and add the additional subgoals to prove
that such a greatest element exists.

\DESCRIBE
The tactic
\vspace*{-3mm}{\small\begin{verbatim}
   INT_MAX_TAC "S:integer -> bool"
\end{verbatim}}\vspace*{-3mm}
\noindent adds to a goal {\small\verb%([a1;...;an], Goal)%} the assumptions
\vspace*{-3mm}{\small\begin{verbatim}
   S MAX
\end{verbatim}}\vspace*{-3mm}
\noindent and
\vspace*{-3mm}{\small\begin{verbatim}
   !N. MAX below N ==> ~S N
\end{verbatim}}\vspace*{-3mm}
\noindent and returns the additional subgoals:
\vspace*{-3mm}{\small\begin{verbatim}
   ([a1;...;an], "?M:integer. S M")

   (["S M";a1;...;an], "?UB. !N. UB below N ==> ~S N")
\end{verbatim}}\vspace*{-3mm}
\noindent to show {\small\verb%S%} is not empty, and {\small\verb%S%} has an upper bound.

\FAILURE
The tactic {\small\verb%NT_MAX_TAC%} fails if it is not given a term of type
{\small\verb%":integer -> bool"%}.


\EXAMPLE
The tactic
\vspace*{-3mm}{\small\begin{verbatim}
   INT_MAX_TAC "\q. ~POS (Y minus (q times X))";;
\end{verbatim}}\vspace*{-3mm}
\noindent when applied to the goal
\vspace*{-3mm}{\small\begin{verbatim}
   (["POS X"; "POS Y"],
    "?q r. (Y = (q times X) minus r) /\ r below Y /\ ~NEG r")
\end{verbatim}}\vspace*{-3mm}
\noindent yields the following three subgoals:
\vspace*{-3mm}{\small\begin{verbatim}
   (["~POS(Y minus (MAX times X))";
     "!N. MAX below N ==> ~~POS(Y minus (N times X))";
     "POS X";
     "POS Y"],
    "?q r. (Y = (q times X) minus r) /\ r below Y /\ ~NEG r")

   (["POS X"; "POS Y"], "?M. ~POS(Y minus (M times X))")

   (["~POS(Y minus (M times X))"; "POS X"; "POS Y"],
    "?UB. !N. UB below N ==> ~~POS(Y minus (N times X))");;
\end{verbatim}}\vspace*{-3mm}
The next step would then be to use {\small\verb%EXISTS_TAC "MAX:integer"%} on the
first subgoal.


\USES
Setting up to solve an existence goal, where the solution will be
given by the greatest element satisfying some property.


\SEEALSO
\vspace*{-3mm}{\small\begin{verbatim}
INT_MIN_TAC
\end{verbatim}}\vspace*{-3mm}

\ENDDOC

\DOC{INT\_RIGHT\_ASSOC\_TAC}
\vspace*{-3mm}{\small\begin{verbatim}
INT_RIGHT_ASSOC_TAC : term -> tactic
\end{verbatim}}\vspace*{-3mm}

\DESCRIBE
The tactic
\vspace*{-3mm}{\small\begin{verbatim}
   INT_RIGHT_ASSOC_TAC tm
\end{verbatim}}\vspace*{-3mm}
\noindent rewrites a goal {\small\verb%P(tm)%} into {\small\verb%P(tm')%} where {\small\verb%tm%} is of the form
{\small\verb%(a plus b) plus c)%} and {\small\verb%tm'%} is {\small\verb%(a plus (b plus c))%}.

\FAILURE
The tactic {\small\verb%INT_RIGHT_ASSOC_TAC%} fails if it is not given a term of
the form {\small\verb%((a plus b) plus c)%}.

\EXAMPLE
\vspace*{-3mm}{\small\begin{verbatim}
   INT_RIGHT_ASSOC_TAC " ((u plus v) plus s) plus t"
\end{verbatim}}\vspace*{-3mm}
\noindent applied to the goal
\vspace*{-3mm}{\small\begin{verbatim}
   ([], "(((u plus v) plus s) plus t) plus (neg(s plus t)) = u plus v")
\end{verbatim}}\vspace*{-3mm}
\noindent returns the subgoal
\vspace*{-3mm}{\small\begin{verbatim}
   ([],"((u plus v) plus (s plus t)) plus (neg(s plus t)) = u plus v")
\end{verbatim}}\vspace*{-3mm}

\USES
Careful rewriting of computational expressions over the integers.

\SEEALSO
\vspace*{-3mm}{\small\begin{verbatim}
GROUP_RIGHT_ASSOC_TAC, GROUP_LEFT_ASSOC_TAC, INT_LEFT_ASSOC_TAC
\end{verbatim}}\vspace*{-3mm}

\ENDDOC

\DOC{INT\_LEFT\_ASSOC\_TAC}
\vspace*{-3mm}{\small\begin{verbatim}
INT_LEFT_ASSOC_TAC : term -> tactic
\end{verbatim}}\vspace*{-3mm}

\DESCRIBE
The tactic
\vspace*{-3mm}{\small\begin{verbatim}
   INT_LEFT_ASSOC_TAC tm
\end{verbatim}}\vspace*{-3mm}
\noindent rewrites a goal {\small\verb%P(tm)%} into {\small\verb%P(tm')%} where {\small\verb%tm%} is of the form
{\small\verb%(a plus (b plus c))%} and {\small\verb%tm'%} is {\small\verb%(a plus b) plus c)%}.

\FAILURE
The tactic {\small\verb%INT_LEFT_ASSOC_TAC%} fails if it is not given a term of
the form {\small\verb%(a plus (b plus c))%}.

\EXAMPLE
\vspace*{-3mm}{\small\begin{verbatim}
   INT_LEFT_ASSOC_TAC " u plus (v plus (s plus t))"
\end{verbatim}}\vspace*{-3mm}
\noindent applied to the goal
\vspace*{-3mm}{\small\begin{verbatim}
   ([], "(neg (u plus v)) plus (u plus (v plus (s plus t))) = s plus t")
\end{verbatim}}\vspace*{-3mm}
\noindent returns the subgoal
\vspace*{-3mm}{\small\begin{verbatim}
   ([],"(neg(u plus v)) plus ((u plus v) plus (s plus t)) = s plus t")
\end{verbatim}}\vspace*{-3mm}

\USES
Careful rewriting of computational expressions over the integers.

\SEEALSO
\vspace*{-3mm}{\small\begin{verbatim}
GROUP_RIGHT_ASSOC_TAC, GROUP_LEFT_ASSOC_TAC, INT_LEFT_ASSOC_TAC
\end{verbatim}}\vspace*{-3mm}

\ENDDOC


\end{document}
